\section{Alcances y limitaciones}

\subsection{Alcances}

Este proyecto pretende diseñar e implementar un prototipo de SNS orientado al deporte bajo dispositivos móviles ANDROID, utilizando una arquitectura orientada a servicios que facilite el desarrollo y la interoperabilidad con diferentes sistemas que existan actualmente en el mercado. Para esto, se utilizará un entorno de desarrollo que brinda Android a los desarrolladores en conjunto con los diferentes dispositivos disponibles para el desarrollo del proyecto.

Debido a la escogencia de tecnologías móviles para el desarrollo del trabajo, se ha decidido incluir funcionalidades de geolocalización y demás de las que dependa ésta. Las funcionalidades de que utilizan el componente de geolocalización serán:

\begin{itemize}
  \item Ubicación de lugares deportivos por parte de usuarios del SNS
  \item Cercanía a eventos deportivos por parte de un usuario del SNS
  \item Cercanía entre usuarios del SNS que compartan una relación (sea simétrica o asimétrica)
\end{itemize}

Otras funcionalidad que se hace interesante (y que será implementada) a la hora de revisar los hallazgos en otras redes sociales, son los reportes estadísticos sobre densidad de población úbicada en cierto espacio deportivo en cada hora del día.

Una última funcionalidad que, para un “usuario deportista” de la red social deportiva, sería muy atractiva es aquella que maneje contenidos de salud y una base de conocimiento de los deportes a implementar sobre la base de datos.

En cuanto a los deportes, se ha decidido realizar (en la etapa de análisis), encuestas a deportistas para averiguar que deportes pueden ser los candidatos a implementar sobre el SNS a desarrollar, teniendo como pauta la siguiente aseveración: Los deportes, resultado de la encuesta, elegidos, serán aquellos que en su participación sean los de menor población practicante.

\subsection{Limitaciones}

Entre las diferentes limitaciones que se pueden encontrar en el desarrollo del actual proyecto, se encuentran las siguientes:

\begin{itemize}
  \item \textbf{Disponibilidad de dispositivos de prueba:} Ya que en el mercado existe una gran cantidad de dispositivos móviles, todos con diferentes especificaciones, es imposible garantizar que la aplicación a diseñar sea soportada por todos los dispositivos del mercado. Sin embargo, se tienen diferentes dispositivos, entre tablets y celulares, en donde se pueden realizar las pruebas (referenciados en los recursos de hardware, capítulo \ref{cap:costos}), limitando los dispositivos soportados oficialmente por el prototipo.

  \item \textbf{Disponibilidad de equipos a usar como servidores:} Ya que el proyecto se basa en la creación de un prototipo, se utilizarán los computadores personales disponibles para proveer los servidores que se necesiten, limitando el rendimiento que de los mismos.
  
  \item \textbf{Recolección de información}: La búsqueda de información se hará sobre la ciudad de Bogotá, haciendo énfasis en la comunidad universitaria.
  
  \item \textbf{Utilización de software libre y con fines académicos:} Será utilizado, en su mayoría, software libre para la realización del proyecto, así como también software que preste licencia con fines académicos, debido a que no se cuenta con el presupuesto necesario para probar herramientas privativas (a parte de versiones de prueba) que pudieran llegar a ser mejores que sus homólogos libres.
  
  \item \textbf{Etapas del ciclo de vida del software no contempladas:} No se llevará acabo una etapa de implantación del software debido a que éste prototipo, aunque funcional, no estará direccionado de inmediato al mercado próximo ya que, debido a las limitaciones de tiempo de los autores, no será posible implementar todos los requerimientos no funcionales que se llegaran a dar al SNS. Por supuesto, al no haber una etapa de implantación, para este trabajo tampoco será presentada la etapa de mantenimiento.
\end{itemize}
