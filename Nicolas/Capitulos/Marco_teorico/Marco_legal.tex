En cuanto al trabajo con datos, las leyes creadas en Colombia para la protección y manejo de estos son:

\begin{itemize}
  \item Constitución Nacional, en el artículo 15, del derecho a la intimidad y al buen nombre. Debido al carácter que tiene un SNS, es posible que una persona pueda degradar el buen nombre de otra, así como violar su intimidad.
  \item Código penal, artículos 220 y 221, acerca de la injuria y la calumnia. Similar al anterior punto, casos de injuria y calumnia corresponden a una mala utilización de un SNS, en aras de provocar daño a otra persona.
  \item Ley 23 de 1982, de los derechos de autor. Debido a los espacios de difusión de información que se presentan en un SNS, esta ley entra a escena. Además, las normas de derechos de autor también califican dentro de cualquier trabajo de software que se haga.
  \item Ley 527 de 1999, la cual reglamenta el manejo de mercancías en el comercio electrónico, la utilización de firmas digitales, la reglamentación para certificados expedidos de forma electrónica con firma digital, el manejo de los mensajes de datos y las disposiciones de la Superintendencia de Industria y Comercio. Debido a que este SNS puede llegar a ser utilizado con carácter comercial, esta ley aplica en la creación del SNS.
  \item Ley 663 de 2000, artículo 91, de las obligaciones de presentar registro mercantíl para los sitios web que tengan carácter comercial. De igual manera que el anterior, algunas de las funcionalidades que se pudieran llegar a implementar, pudiesen llegar a ser con un carácter netamente comercial, por lo que es necesario estar al tanto de esta ley y artículo en particular.
  \item Ley 1266 de 2008, el cual reglamenta el tratamiento de datos personales en bases de datos personales, haciendo énfasis en las financieras y comerciales. Debido a que los SNS manejan datos personales de sus usuarios y, a su vez, puede llegar a incluir funcionalidades con carácter comercial, es debido tener en cuenta la ley.
  \item Ley 1273 de 2009, la cual reglamenta el uso de la información y los sistemas de información en contra de la violación de la confidencialidad, la integridad y la disponibilidad de los datos y los sistemas de información, así como también hurtos informáticos. Debido a que ésta ley reglamenta la utilización de la información que tengan en su poder los manejadores de un sistema informático, es menester conocer dicha ley.
  \item Ley 1480 de 2011, la cual reglamenta los derechos y deberes tanto de consumidores como de productores en todos los sectores económicos, aplicándose ésta a los productos tanto importados como nacionales. Debido a que el SNS puede tener funcionalidades de carácter comercial, es necesario saber que leyes aplican a consumidores y productores o prestadores de servicio que la pudieren utilizar.
  \item Resolución 3066 de 2011, la cual busca proteger los derechos de los usuarios de servicios de comunicaciones en los cuales se establece también los derechos sobre los servicios adquiridos en telecomunicaciones. Debido a que el usuario del SNS pudiera llegar a obtener servicios en telecomunicaciones (por ejemplo, la conexión del SNS con la línea telefónica para algún tipo de acción), es necesario conocerla para saber cuando si y cuando no se debe juzgar una posible violación de los derechos de los usuarios de parte de los creadores y de aquellos que pudieran llegar a mantener el SNS.
  \item Ley 1581 de 2012, de disposiciones acerca del tratamiento de datos personales, haciendo énfasis en las políticas de manejo y en las auditorías. Debido a que el SNS tratará con algunos datos personales de sus usuarios, se debe prestar atención a que políticas de manejo de datos personales existen, una de ellas, en esta ley.
  \item Decreto 1377 de 2013, el cual dictamina las políticas de protección y tratamiento de datos personales. Igual que el anterior, se hace necesario conocer que decretos dictaminan la utilización de datos personales por parte de los usuarios del SNS.
\end{itemize}

Lo anterior es tomado de \textit{Legislación en internet} \cite{leg_int}.

Además, se deben tener en cuenta las condiciones de servicio que Google ha impuesto para las aplicaciones desarrolladas para Android, así como también las condiciones aplicadas a la utilización de dichas aplicaciones. Cada una de estas políticas desarrolladas para Android involucra el análisis del entorno legal que enmarca al SNS, debido a que la tecnología a utilizar será principalmente basa en Android. Entonces, se han de tener en cuenta las siguientes condiciones de servicio:

\begin{itemize}
  \item Google Play Terms of Service, el cual dictamina las pautas de uso de Google Play por parte del usuario final, así como también las facultades que tiene Google sobre la información y las aplicaciones instaladas en el dispositivo de un usuario.
  \item Developer Distribution Agreement, el cual reglamenta el uso que el desarrollador o distribuidor de aplicaciones hace de Google Play. Habla acerca del licenciamiento, el manejo de precios y pagos, el manejo de marcas y publicidad y la dada de baja de las aplicaciones de Google Play.
  \item Google Play Business and Program Policies, el cual reglamenta cómo deben ser utilizadas las aplicaciones en cuanto a la información publicada en las mismas y además quien puede utilizar Google Play. Además, reglamenta la devolución, compra, descarga y soporte de productos (aplicaciones) en Google Play.
  \item Developer Content Policy, el cual establece las políticas de contenido y publicidad que puede poner un desarrollador en sus aplicaciones.
\end{itemize}
