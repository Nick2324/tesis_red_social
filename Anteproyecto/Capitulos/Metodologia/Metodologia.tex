
Equipo SCRUM:

\begin{itemize}
  \item Product Owner \\
	Doctor Carlos Enrique Montenegro

  \item Developement Team
	\begin{itemize}
	  \item Nicolás Mauricio García Garzón
	  \item Luis Felipe Gonzalez Moreno
	\end{itemize}

  \item Scrum Master \\
	Profesor Alejandro Daza
\end{itemize}

Actividades de cada Sprint (15 días máximo):
  
\begin{itemize}
  \item Sprint Planning (4 horas máximo)\\
  Se planea una reunión con el Scrum Master al inicio de cada iteración. En esta reunión se discutirá el desempeño del Sprint anterior, los servicios/funcionalidades que hagan falta para completar el prototipo, y que servicios deben ser realizados como parte de la actual iteración.
	\begin{itemize}
	  \item ¿Qué puede hacerse en este sprint? \\
	    Con base a los servicios y casos de uso de negocio que han sido encontrado, se determina que puede ser desarrollado por el developement team en la próxima iteración del prototipo.
	  \item ¿Como se llevará a cabo este trabajo? \\
	    Se dividen los servicios propuestos como parte del Sprint entre los desarrolladores y se comparten las expectativas que debe cumplir cada servicio para que sea aceptado en el desarrollo.
	\end{itemize}
  \item Daily Scrum (15 minutos máximo)[Solo participa el developement team] \\
  Partiendo de los items asignados a cada desarrollador, se dividen en tareas aún mas pequeñas que, de cumplirse satisfactoriamente, sirven como ruta para cumplir con el Sprint Goal. Se responden las siguientes preguntas para llevar un seguimiento continuo del desarrollo del prototipo:
	\begin{itemize}
	  \item ¿Qué se hizo ayer que ayudó al developement team para cumplir el Sprint goal?
	  \item ¿Qué se va a hacer hoy para ayudar al developement team a cumplir el Sprint goal?
	  \item ¿Hay algún impedimento para que el developement team cumpla el Sprint goal?
	\end{itemize}
  \item Sprint Review (2 horas máximo) \\
  Se realiza al final de cada Sprint. La idea de esta actividad es mostrar que se hizo en el sprint con respecto a las tareas propuestas desde el inicio del mismo. Se determina si se alcanzó el Sprint goal y se discute que servicios/tareas proponer para el siguiente sprint. Las actividades básicas son:
	\begin{itemize}
	  \item Se socializa la experiencia en el sprint, que problemas ocurrieron y cómo se solucionaron.
	  \item Se exponen los diferentes elementos que fueron construidos y se resuelven preguntas acerca de los mismos.
	  \item Se propone que puede hacerse en el siguiente Sprint basados en la experiencia del actual.
	  \item Se revisa como el cambio en el entorno puede cambiar las prioridades en el trabajo del equipo.
	\end{itemize}
  \item Sprint Retrospective
  \begin{itemize}
    \item Se toman las experiencias del actual sprint para formular sugerencias que ayuden a mejorar los sprint futuros

  \end{itemize}
\end{itemize}

A continuación, se presenta un product backlig inicial y tentativo para iniciar el proyecto.

\begin{table}[h]
  \caption{Product backlog inicial}
  \label{tab:backlog}

  \begin{center}
    \resizebox{15cm}{!}{
      \begin{tabular}{|l|l|l|}
        \hline
        Tarea & Días & Condición de aprobación \\ 
        \hline
        \hline
        Levantamiento de requerimientos & 2 & Satisfacción de todos los \\ 
        
         &  & requerimientos para la red social \\ 
        
        Definición de requerimientos funcionales y no funcionales & 2 & Modularización y descripción total de  \\ 
        
         &  & todos los requerimientos \\ 
        
        Investigación de tecnologías existentes & 6 & Escogencia de las tecnologías a utilizar \\ 
        
         &  & para implementar la red social \\ 
        
        Diseño de casos de uso & 5 & Cubrimiento de todos los requerimientos identificados \\ 
        
        Refinamiento de requerimientos & 1 & Trazabilidad entre casos de uso y requerimientos \\ 
        
        Identificación de servicios candidatos & 1 & Cubrimiento de todos los requerimientos identificados \\ 
        
        Diseño de blueprints de servicios & 4 & Cubrimiento de todos los servicios candidatos \\ 
        
        Escogencia de servicios a ser implementados & 1 & Viabilidad de un prototipo funcional \\ 
        
        Composición estática de servicios & 5 & Concordancia entre los casos de uso  \\ 
        
         &  & y la composición de servicios \\ 
        
        Diseño de diagrama de clases & 5 & Estructuración completa de los servicios a ser implementados \\ 
        
         &  & y concordancia con requerimientos no funcionales \\ 
        
        Diseño de base de datos & 6 & Diseño que cubra los servicios a ser implementados \\ 
        
         &  &  y requerimientos no funcionales \\ 
        
        Diseño de interfaz gráfica de usuario & 4 & Cubrimiento de los servicios y casos de uso a ser implementados \\ 
        
         &  &  a ser implementados \\ 
        
        Refinamiento de casos de uso & 1 & Trazabilidad \\ 
        
        Refinamiento de blueprints de servicios & 1 & Trazabilidad \\ 
        
        Refinamiento de servicios a ser implementados & 1 & Trazabilidad \\ 
        
        Refinamiento de composición estática de servicios & 1 & Trazabilidad \\ 
        
        Construcción de la interfaz de usuario & 20 & Navegabilidad sobre las funcionalidades a ser implementadas \\ 
        
        y refinamiento de interfaz de usuario &  &  \\ 
        
        Construcción de los servicios a ser  & 35 & Construcción de prototipo funcional sin fallas \\ 
        
        implementados y refinamiento de modelos &  & detectadas en tiempo de desarrollo \\ 
        
        Pruebas del prototipo por parte del equipo de desarrollo & 2 & Prototipo sin fallas detectables en sus funcionalidades \\ 
        
        Pruebas del prototipo por parte del usuario final & 7 & Prototipo aceptado por el usuario final en al menos un 70\% \\
        \hline

        \end{tabular}
    }
  \end{center}
\end{table}
