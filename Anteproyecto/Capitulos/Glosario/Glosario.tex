\begin{itemize}
  \item OSN : Online Social Network – Red Social En-línea, es una red social que crece dentro del ámbito web.
  \item SNS : Social Network Services – Servicios de Redes Sociales.
  \item Offline Social Network : Red social que crece en el ámbito real, no en el virtual como lo es en las OSN.
  \item Sistemas transversales : Sistemas que interactúan entre sí, hechos o no con tecnologías diferentes sobre paradigmas diferentes, con un fin común.
  \item API : Application Programming Interface – Interfaz de programas de aplicación, es el conjunto de funciones y procedimientos de un sistema que pueden ser utilizados en otro sistema. Es el puente de conexión entre ambos sistemas.
  \item Interoperabilidad : Capacidad de un sistema de software para trabajar con otro sistema con la característica de que esta cooperación sea hecha de la manera más transparente posible.
  \item SOA : Service-Oriented Architecture – Arquitectura Orientada a Servicios.
  \item TI : Tecnologías de la Información
  \item ROI : Return on investment – Retorno en inversión.
  \item Product Backlog : Productos que están pendientes por realizar para finalizar el proyecto.
  \item Sprint Goal : El objetivo, que de cumplirse, determina si el Sprint fue exitoso.
  \item Nodo : Punto de intersección de varios elementos
  \item Arista : Uniones entre nodos
  \item StakeHolders : Partes interesadas en un tema específico.
  \item UX (User eXperiencie): Se refiere a la percepción que adquiere el usuario de un dispositivo (software, hardware) con respecto a las tareas realizadas con este, basándose en las sensaciones/reacciones que este le provoca.
\end{itemize}
