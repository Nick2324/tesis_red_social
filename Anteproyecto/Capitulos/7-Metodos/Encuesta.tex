\section{Encuesta inicial}

\subsection{Información necesaria}

El proposito de esta encuesta, de carácter exploratorio, es determinar, en la poblacion bogotana, como se relaciona la practica de algún deporte con el uso del internet.
Para esto, la encuesta está enfocada para obtener los siguientes datos:
\begin{itemize}
  \item ¿Desde qué lugar suelen conectarse a internet?
  \item ¿Cuales son los deportes mas practicados?
  \item ¿Cómo buscan los temas relacionados a la practica de un deporte?
\end{itemize}


\subsection{Naturaleza de la encuesta}

El tipo de encuesta elegido para realizar el estudio fue la encuesta electrónica por internet, valiéndonos de la herramienta de generación de encuestas de Google Drive (antes Google Docs).

\subsection{Técnicas de esacalamiento utilizadas}

Teniendo en cuenta que la encuesta debe ser lo más corta y sencilla posible, la técnica a utilizar será la de realizar preguntas con única y múltiple respuesta. Esto nos permitirá también analizar la distribución de las respuestas entre los usuarios.


\subsection{Trabajo de campo}

El trabajo de campo se realizó mediante una encuesta, de manera virtual, donde se obtienen las diferentes preferencias de los encuestados al momento de practicar un deporte y su relacion con el uso de internet para este proposito.
Los resultados de las encuestas se presentan en el anexo \ref{anexo1}

\subsection{Análisis de datos}
