\section{Encuesta inicial}

\subsection{Información necesaria}

El proposito de esta encuesta, de carácter exploratorio, es determinar, en la poblacion bogotana, como se relaciona la practica de algún deporte con el uso del internet.
Para esto, la encuesta está enfocada para obtener los siguientes datos:
\begin{itemize}
  \item ¿Desde qué lugar suelen conectarse a internet?
  \item ¿Cuales son los deportes mas practicados?
  \item ¿Cómo buscan los temas relacionados a la practica de un deporte?
\end{itemize}

\subsection{Naturaleza de la encuesta}

El tipo de encuesta elegido para realizar el estudio fue la encuesta electrónica por internet, valiéndonos de la herramienta de generación de encuestas de Google Drive (antes Google Docs).

\subsection{Técnicas de esacalamiento utilizadas}

Teniendo en cuenta que la encuesta debe ser lo más corta y sencilla posible, la técnica a utilizar será la de realizar preguntas con única y múltiple respuesta. Esto nos permitirá también analizar la distribución de las respuestas entre los usuarios.


\subsection{Trabajo de campo}

El trabajo de campo se realizó mediante una encuesta, de manera virtual, donde se obtienen las diferentes preferencias de los encuestados al momento de practicar un deporte y su relacion con el uso de internet para este proposito.
Los resultados de las encuestas se presentan en el anexo \ref{anexo1}

\subsection{Análisis de resultados}

La encuesta se practicó a 155 personas, a travez de internet, y valiendonos de grupos y sitios web que frecuentan los jovenes de Bogotá, en su mayoria estudiantes universitarios.\\

\begin{itemize}
  \item Edad \\
  Los rangos de edad de los encuestados varian de 13 hasta 60 años, concentrandose en el rango de 20 a 27 años. Aún cuando esta pregunta no refleja ningún comportamiento de analisis, refleja que la encuesta fue practicada, en su gran mayoria, a los jovenes Bogotanos.
  \item Ocupación \\
  Se puede apreciar como el 66\% de los encuestados son estudiantes. Los estudiantes suelen tener grupos de amigos/conocidos en su lugar de estudio con quienes pasan tiempo por fuera de sus lugares de estudio, son jovenes que, en su mayoria, estan disfrutando de su etapa de estudiantes universitarios, concentrandose mayoritariamente en sus estudios. Por otra parte, un 25\% de los encuestados dicen ser Empleados, y teniendo en cuenta como y a quien se le realiza la encuesta, se puede suponer que son estudiantes que, a parte de estudiar, tambien tienen que trabajar.
  \item Elementos electronicos \\
  El elemento que mas  dicen tener los encuestados es el computador portatil (37\%) , lo cual tiene sentido teniendo en cuenta las necesidades de un estudiante universitario, seguido muy de cerca del computador de escritorio (26\%) y el smartphone (26\%). En la mayoria de los casos, aseguran tener tanto computador portatil como smarthpone. De allí se puede deducir que a los jóvenes les gusta estar en constante conexion con el mundo digital y la internet.
  \item Lugar de acceso a internet \\
  El lugar desde el que se accede a internet con mayor frecuencia es el hogar con un 43\%, seguido de el lugar de estudio (23\%) y del internet movil(13\%). Adicionalmente, los encuestados aseguran que los lugares en los que duran mas tiempo conectados son el hogar (72\%) y el internet móvil (15\%). Esto muestra que hay preferencia en conectarse desde lugares y dispositivos en los que se sienten mas en privado (o en control) de quienes tienen acceso a la informacion contenida por estos dispositivos.
  \item ¿Practica deporte? \\
  El 63\% de los encuestados asegura practicar algún deporte, mientras el 37\% no. Las razones por las que este importante porcentaje de la población no practican algun deporte sale del alcance de esta primera encuesta.
  \item Deportes practicados \\
  En los deportes practicados, resaltan el Futbol (27\%), Baloncesto (13\%) y Ciclismo (13\%), mientras que los deportes en los que se requieren implementos o lugares especializados no son tan comunes (Tenis 7\%, Escalada deportiva 3\%, Patinaje 1\% y no se practican deportes como Rugby, Futbol Americano o Golf)
  \item Métodos de búsqueda \\
  Para analizar los métodos de busqueda, se realizaron preguntas enfocadas a la busqueda de nuevos deportes, implementos, lugares y grupos o equipos para practicar estos deportes. El comun denominador para cada una de ellas fue consultar con los amigos, en donde siempre fue de las opciones mas populares, solo superada por la consulta de tiendas deportivas (en el caso de la busqueda de implementos) y el CouchSurfing (en el caso de la busqueda de un nuevo deporte), demostrando que las opiniones de los amigos/conocidos tienen mayor importancia que cualquier otra forma de busqueda.
\end{itemize}


\subsection{Formato de encuesta}

\begin{enumerate}
  \item ¿Qué edad tiene?
  \item Sexo
  \begin{itemize}
    \item Hombre
    \item Mujer
  \end{itemize}
  \item Cual es su ocupación
  \begin{itemize}
    \item Estudiante
    \item Empleado
    \item Desempleado
    \item Otro
  \end{itemize}
  \item Cual de los siguientes elementos posee usted en la actualidad (Seleccione todos los que apliquen)
  \begin{itemize}
    \item Tablet
    \item Smartphone
    \item Computador de escritorio
    \item Computador portatil
    \item Ninguno
  \end{itemize}
  \item ¿Desde qué lugar suele usted conectarse a internet? (Seleccione todos los que apliquen)
  \begin{itemize}
    \item Hogar
    \item Casa de un amigo/conocido
    \item Café internet
    \item Trabajo
    \item Lugar de estuio
    \item Cualquier lugar (internet movil)
  \end{itemize}
  \item ¿Cual es el lugar desde el cual usted dura mas tiempo navegando por internet?
  \begin{itemize}
    \item Hogar
    \item Casa de un amigo/conocido
    \item Café internet
    \item Trabajo
    \item Lugar de estuio
    \item Cualquier lugar (internet movil)
  \end{itemize}
  \item En promedio, ¿Cuanto tiempo utiliza usted el internet por dia?
  \begin{itemize}
    \item Menos de una hora
    \item De una a tres horas
    \item De tres a seis horas
    \item Mas de seis horas
  \end{itemize}
  \item ¿Practica usted algún deporte?
  \begin{itemize}
    \item Si
    \item No
  \end{itemize}
  \item En caso de practicar algún deporte, ¿Qué deporte practica?
  \begin{itemize}
    \item Fútbol
    \item Voleyball
    \item Tennis
    \item Golf
    \item Rugby
    \item Fútbol americano
    \item Patinaje
    \item Ciclismo
    \item Escalada deportiva
    \item Baloncesto
    \item Otro
  \end{itemize}
  \item Cuando quiere buscar personas con quien practicar deporte ¿Por qué medio lo hace?
  \begin{itemize}
    \item Internet
    \item Compañeros cercanos
    \item Equipos consolidados
    \item No sabe donde buscar
  \end{itemize}
  \item Cuando quiere buscar un lugar donde practicar deporte ¿Por qué me dio lo hace?
  \begin{itemize}
    \item Internet
    \item Consulta con amigos/conocidos
    \item Centros especializados en su deporte
    \item No sabe donde buscar
  \end{itemize}
  \item Cuando quiere buscar implementos deportivos, ¿Por qué medio lo hace?
  \begin{itemize}
    \item Tiendas deportivas
    \item Tiendas en linea (Mercadolibre, olx, Amazon)
    \item Redes sociales (Facebook, twitter)
    \item Le pregunta a un conocido
    \item Tiendas de cadena (Jumbo, exito, Makro)
    \item Outlets
    \item Television
    \item Radio
  \end{itemize}
  \item Cuando quiere buscar un nuevo deporte para practicar, ¿Por qué medio lo hace?
  \begin{itemize}
    \item Le pregunta a un conocido
    \item Va a complejos deportivos
    \item Internet (CouchSurfing)
    \item Publicaciones deportivas
    \item Television (Canales deportivos)
    \item Radio
  \end{itemize}
  \item Usted quiere practicar un deporte nuevo, ¿Cómo prefiere hacerlo?
  \begin{itemize}
    \item Solo
    \item Con un grupo de amigos
    \item Con grupos previamente consolidados en ese deporte
    \item Desconocidos con intereses comunes en ese deporte
  \end{itemize}
  \item BONUS: ¿Considera usted que los videojuegos sean un deporte?
  \begin{itemize}
    \item Si
    \item No
  \end{itemize}
\end{enumerate}
