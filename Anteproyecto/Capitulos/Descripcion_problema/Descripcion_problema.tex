
El hombre, en su estado de ser social, ha creado diversos medios de comunicación. Pasó del lenguaje a aparatos que constituyen un canal para difundir dicho lenguaje, tales como el teléfono, la radio, la televisión, la creación del internet y el computador personal y, en última instancia, la creación de los dispositivos móviles inteligentes (smartphones y tablets) . Hoy, gracias a la evolución de la web 1.0 estática en contenido a una web 2.0 dinámica en cuanto a interacción social, las redes sociales se han convertido en un medio de difusión con gran afluencia de información, de informantes y de informados. Redes sociales como facebook, twitter o youtube se han vuelto famosas para compartir \textbf{cualquier tipo de información}: Desde lo último ocurrido en la vida de un ciudadano del común hasta la difusión de los problemas por los que pasa el mundo en general. La gran cantidad de información diversa y dispersa sobre un solo medio hace que haya demasiado ruido para aquellas personas que usan las redes sociales con objetivos informativos particulares, ya sea de carácter científico, pasando por carácter artístico o, inclusive, de carácter deportivo.

Uno de los problemas con los que se ha encontrado uno de los autores del presente trabajo (Nicolás García, vicepresidente de la Asociación Colombiana de Lacrosse) en la integración del internet con temas de carácter deportivo, ha sido que los deportistas son bombardeados de información (consejos deportivos, nutricionales o de preparación física; historia de los deportes y sus reglas, etc) en diversos espacios de dicha red. Este problema no es nuevo: éste se ha presentado con la masificación de la utilización de internet desde que éste fuera creado. Como dice en \cite{judging}, hay algunos ítems que deberá cumplir una fuente para ser confiable (autor o patrocinador, información de contacto proveída, propósitos para los que fue creado el sitio donde se despliega el recurso, verificación de la información). Reducir la búsqueda de los deportistas a un solo lugar confiable (que de, de alguna manera, el soporte a las pautas sugeridas en \cite{judging}) con información deportiva desembocaría en la confiabilidad de la información que se encuentra allí y convertiría a dicho lugar en herramienta más corriente para búsquedas en temas de deporte.

Otro problema de la integración de internet y temas de carácter deportivo es la falta de utilización, de parte de los deportistas, de las redes sociales dedicadas al deporte. De nuevo, debido a la experiencia del vicepresidente de la Asociación Colombiana de Lacrosse, se ha visto que los deportistas utilizan medios como twitter, facebook o couchsurfing (que añaden ruido a la información que ellos necesitan) para difundir información deportiva sin pensar en las bondades que puede ofrecer un sitio dedicado a deportistas. Las estrategias que ha desplegado la Asociación (no solo en cabeza de él, sino que en conjunto con otras personas de las ciudades de Medellín, Tunja, Pereira y Bogotá) han sido sobre las redes sociales antes mencionadas y, además, sobre la red social Instagram. Un SNS deportivo tendrá como prioridad la gestión de una red social fuera de línea de carácter deportivo y, por tanto, debe proveer mecanismos que ayuden a facilitar tareas que se hagan en el mundo real (por ejemplo, búsqueda de personas interesadas en la participación de un nuevo deporte como lo es el lacrosse en Colombia).

Un fenómeno particular que se dio en el crecimiento de la Asociación Colombiana de Lacrosse fue la utilización de las redes sociales más conocidas como medio de comunicación y difusión del deporte, presentándose un problema: Las personas usuario de esas redes no buscaban por deportes nuevos, sino que tenían sus propios propósitos que no siempre se acoplaban a los deportivos. Aún con el esfuerzo de difusión por poco más de 2 años (la Asociación funciona desde el 15 de Julio del año 2012), el deporte aún es desconocido, teniendo personas (en su mayoría) extranjeras asociadas a cada red social (nuevamente, Facebook, Twitter e Instagram, e incluso Youtube o Google+).

Los problemas presentados aquí pueden ser resueltos por medio de la implementación de SNS. Ya que el componente Geolocalización en Tiempo Real es importante debido a que la ubicación de los deportistas es de gran ayuda a otros deportistas para salir a interactuar con ellos, así como el hecho de tener a mano la información deseada en cualquier momento que se requiera (por ejemplo, consejos para tratar una lesión al instante), se puede abordar ambos problemas con la utilización de tecnologías móviles.

Así, con la evolución de la comunicación humana trasladándose a los espacios virtuales por medio de las redes sociales y la falta de aplicaciones, en el campo de los smartphones, que soporten interacciones sociales enfocadas a los deportes en general, en este trabajo se creará un servicio de red social centrado en los deportes sobre tecnologías móviles para la administración de las redes sociales de cada persona en un ámbito deportivo desde su dispositivo móvil.

Haciendo uso de las TIC, ¿cómo se puede facilitar la comunicación y el acceso a la información a quienes son parte de la comunidad deportiva?
