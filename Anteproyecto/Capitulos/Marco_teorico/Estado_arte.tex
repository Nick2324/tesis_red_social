\section{Estado del arte} \label{cap:estado_arte}
\subsection{UX - Análisis}
Los servicios de redes sociales (SNS por sus siglas en ingles: Social Network Services) como Facebook, LinkedIn, Twitter, SportTracker o Xportia, ofrecen servicios para la gestión de la OSN de cada usuario que acceda a estas aplicaciones. Según un estudio hecho para medir la experiencia de usuario (UX por sus siglas en ingles: User eXperience) en los SNS, se encontraron 8 categorías que son críticas a la hora de diseñar una SNS y son:

\begin{enumerate}
  \item Self-expresion: Capacidad que tengan las OSN de compartir contenido relacionado a la vida real de los usuarios tal como lo pueden ser las fotos, los videos, los comentarios o las comunicaciones directas.
  \item Reciprocity: Interacción bilateral en tiempo real, es decir, interacción instantánea con uno o varios individuales en la OSN (por ejemplo, por medio de los servicios de mensajería instantánea).
  \item Learning: La información recibida por medio de la OSN debe poder ser utilizada en pro del desarrollo cognitivo del individual; debe existir información útil al individual que usa la OSN.
  \item Curiosity: El contenido de la OSN debe ser interesante para quien la utiliza.
  \item Suitability of functionality: Se refiere a cuán ``utilizable'' es una funcionalidad.
  \item Suitability of content: La calidad y exactitud de la información que en la OSN reside debe ser suficiente para el individual perteneciente a ella.
  \item Completeness of the user network: Los individuales deben querer pertenecer a la red social y buscar eficientemente a otros individuales para poder formar lazos con ellos y hacer crecer su red social.
  \item Trust and privacy: Confianza en los servicios de las OSN, así como también la capacidad que tiene el usuario de gestionar la privacidad del contenido que comparte en dicha OSN. \cite{social_experience}

Cada uno de las categorías nombradas hace parte de los factores que impulsan la utilización de los SNS para la gestión de las OSN de las personas. 
\end{enumerate}

\subsection{Factor distancia en la formación de las redes sociales}

La formación de redes sociales (tanto fuera de línea como en línea) es afecta por la distancia entre cada individual y el posible tipo de enlace que los uniría. En \cite{evolution} se hizo un estudio acerca de la formación de lazos, la formación de triadas entre individuales de una red social basada en la inscripción localizaciones recomendadas y frecuentadas por los usuarios, teniendo como parámetros ``la edad'' o tiempo de vinculación del individual a la red social, el grado de cada individual (número de conexiones que tiene un individual a otro) y la localización de cada individual en la red social. También se analizó cómo afectaba la creación de nuevos lazos con la movilidad del usuario (el desplazamiento por lugares geográficos distintos). En conclusión, se verificó que la formación de lazos depende proporcionalmente de la edad y del grado del individual y es inversamente proporcional a la distancia que a cada individual y que la formación de lazos puede modelarse con solo dos de los tres factores (el grado y la distancia); en cuanto a la formación de triadas, se verificó que ésta depende de las características sociales de la red, tomando énfasis en los individuales compartidos entre los posibles individuales formadores de triadas. Además, en cuanto a la creación de nuevos lazos teniendo en cuenta los lugares visitados por cada usuario de la red social, se presenta un patrón: Los usuarios escogen un lugar popular para visitar y, posteriormente, dirimen con que usuario crear un lazo teniendo en cuenta su popularidad y que frecuente los mismos lugares siempre.

\begin{landscape}
  
\begin{table}
  \caption{Comparacion de redes, parte 1}
  \label{tab:comparacion_redes_1}

  \begin{center}
  
  \resizebox{20cm}{!}{
  \begin{tabular}{|p{5cm}|llll|}
    \hline
    Fun\textbackslash Red social & \multicolumn{1}{c}{Sportfactor} & \multicolumn{1}{c}{Deportesreunidos} & \multicolumn{1}{c}{Mybestplay} & \multicolumn{1}{c|}{Subetudeporte} \\ 
    \hline
    Gestión de foros & \multicolumn{1}{c}{} & \multicolumn{1}{c}{Si} & \multicolumn{1}{c}{} & \multicolumn{1}{c|}{Si} \\ 
    \hline
    Gestión de encuentros deportivos & \multicolumn{1}{c}{} & \multicolumn{1}{c}{- Organización de eventos} & \multicolumn{1}{c}{} & \multicolumn{1}{c|}{} \\ 
     & \multicolumn{1}{c}{} & \multicolumn{1}{c}{-Encuentros deportivos informales} & \multicolumn{1}{c}{} & \multicolumn{1}{c|}{} \\ 
     & \multicolumn{1}{c}{} & \multicolumn{1}{c}{} & \multicolumn{1}{c}{} & \multicolumn{1}{c|}{} \\ 
    \hline
    Creación de grupos & \multicolumn{1}{c}{} & \multicolumn{1}{c}{Si} & \multicolumn{1}{c}{} & \multicolumn{1}{c|}{} \\ 
    \hline
    Manejo de torneos & \multicolumn{1}{c}{} & \multicolumn{1}{c}{- Organización y difusión} & \multicolumn{1}{c}{} & \multicolumn{1}{c|}{} \\ 
    \hline
    Difusión info. Deportiva & \multicolumn{1}{c}{-RSS de noticias} & \multicolumn{1}{c}{- Blog propio} & \multicolumn{1}{c}{-Difusión de eventos} & \multicolumn{1}{c|}{- Gestión de blogs} \\ 
     & \multicolumn{1}{c}{} & \multicolumn{1}{c}{} & \multicolumn{1}{c}{-Blog propio} & \multicolumn{1}{c|}{} \\ 
    \hline
    Serv. self-expression & \multicolumn{1}{c}{} & \multicolumn{1}{c}{-Difusión de multimedia} & \multicolumn{1}{c}{-Difusión de multimedia } & \multicolumn{1}{c|}{-Difusión de multimedia} \\ 
     & \multicolumn{1}{c}{} & \multicolumn{1}{c}{} & \multicolumn{1}{c}{} & \multicolumn{1}{c|}{} \\ 
    \hline
    Sistema estadístico & \multicolumn{1}{c}{-Medición de avance en} & \multicolumn{1}{c}{- Sistemas de estadísticas para cada servicio} & \multicolumn{1}{c}{} & \multicolumn{1}{c|}{} \\ 
     & \multicolumn{1}{c}{ estadísticas del deporte practicado} & \multicolumn{1}{c}{} & \multicolumn{1}{c}{} & \multicolumn{1}{c|}{} \\ 
    \hline
    Gestión de transversales & \multicolumn{1}{c}{-Trainner personales} & \multicolumn{1}{c}{} & \multicolumn{1}{c}{} & \multicolumn{1}{c|}{} \\ 
     & \multicolumn{1}{c}{-Guías de nutrición} & \multicolumn{1}{c}{} & \multicolumn{1}{c}{} & \multicolumn{1}{c|}{} \\ 
     & \multicolumn{1}{c}{- Catalogo de lesiones y fisioterapia} & \multicolumn{1}{c}{} & \multicolumn{1}{c}{} & \multicolumn{1}{c|}{} \\ 
    \hline
    Servicios deportivos & \multicolumn{1}{c}{-Guía deportiva (shops, restaurantes, etc.)} & \multicolumn{1}{c}{} & \multicolumn{1}{c}{} & \multicolumn{1}{c|}{} \\ 
     & \multicolumn{1}{c}{} & \multicolumn{1}{c}{} & \multicolumn{1}{c}{} & \multicolumn{1}{c|}{} \\ 
    \hline
    Soporte multi-deporte & \multicolumn{1}{c}{Si} & \multicolumn{1}{c}{Si} & \multicolumn{1}{c}{Solo deportes en equipo} & \multicolumn{1}{c|}{Si} \\ 
    \hline
    Gestión de tipos de usu. & \multicolumn{1}{c}{} & \multicolumn{1}{c}{- Equipos } & \multicolumn{1}{c}{Si} & \multicolumn{1}{c|}{} \\ 
     & \multicolumn{1}{c}{} & \multicolumn{1}{c}{- Clubes} & \multicolumn{1}{c}{} & \multicolumn{1}{c|}{} \\ 
     & \multicolumn{1}{c}{} & \multicolumn{1}{c}{-Centros deportivos} & \multicolumn{1}{c}{} & \multicolumn{1}{c|}{} \\ 
    \hline
    Gestión de sponsors & \multicolumn{1}{c}{} & \multicolumn{1}{c}{} & \multicolumn{1}{c}{Si} & \multicolumn{1}{c|}{} \\ 
    \hline
    Gestión del conocimiento & \multicolumn{1}{c}{} & \multicolumn{1}{c}{} & \multicolumn{1}{c}{} & \multicolumn{1}{c|}{} \\ 
    \hline
    Gestión de geolocaliza. & \multicolumn{1}{c}{} & \multicolumn{1}{c}{} & \multicolumn{1}{c}{} & \multicolumn{1}{c|}{} \\ 
     & \multicolumn{1}{c}{} & \multicolumn{1}{c}{} & \multicolumn{1}{c}{} & \multicolumn{1}{c|}{} \\ 
    \hline
    Soporte móvil & \multicolumn{1}{c}{} & \multicolumn{1}{c}{} & \multicolumn{1}{c}{} & \multicolumn{1}{c|}{} \\ 
     & \multicolumn{1}{c}{} & \multicolumn{1}{c}{} & \multicolumn{1}{c}{} & \multicolumn{1}{c|}{} \\ 
    \hline
    Conexión con otros SNS & \multicolumn{1}{c}{} & \multicolumn{1}{c}{} & \multicolumn{1}{c}{} & \multicolumn{1}{c|}{} \\ 
     & \multicolumn{1}{c}{} & \multicolumn{1}{c}{} & \multicolumn{1}{c}{} & \multicolumn{1}{c|}{} \\ 
    \hline
  \end{tabular}
  }
    \end{center}
\end{table}
  
  \newpage
  
  \begin{table}
  \caption{Comparacion de redes, parte 2}
  \label{tab:comparacion_redes_2}

  \begin{center}
  
  \resizebox{20cm}{!}{
    \begin{tabular}{|p{4cm}|p{9cm}p{7cm}p{7cm}|}
\hline
Fun\textbackslash Red social & Sporttia & Amatteur & Fitivity  \\ 
\hline
Gestión de foros &  &  &  \\ 
\hline
Gestión de encuentros deportivos & - Organización de eventos en centros deportivos & - Publicación o búsqueda de eventos deportivos & -Basado en geolocalización \\ 
 & - Gestión de jugadores &  &  \\ 
 & -Gestión de características del partido &  &  \\ 
\hline
Creación de grupos &  &  &  \\ 
\hline
Manejo de torneos &  &  &  \\ 
\hline
Difusión info. Deportiva &  &  &  \\ 
 &  &  &  \\ 
\hline
Serv. self-expression &  & -Difusión de multimedia &  \\ 
 &  &  &  \\ 
\hline
Sistema estadístico &  &  &  \\ 
 &  &  &  \\ 
\hline
Gestión de transversales &  &  &  \\ 
 &  &  &  \\ 
 &  &  &  \\ 
\hline
Servicios deportivos & -Alquiler de centros deportivos & - Servicios de compra y venta de artículos deportivos &  \\ 
 &  &  &  \\ 
\hline
Soporte multi-deporte & Si & Si & Si \\ 
\hline
Gestión de tipos de usu. & -Deportista -Centro deportivo & -Deportista  &  \\ 
 &  & -Equipo &  \\ 
 &  &  -Organización &  \\ 
\hline
Gestión de sponsors &  & -Promoción como deportista, equipo u organización &  \\ 
\hline
Gestión del conocimiento & - Clases virtuales &  &  \\ 
\hline
Gestión de geolocaliza. &  & Si & Si \\ 
 &  &  &  \\ 
\hline
Soporte móvil &  &  & -Android \\ 
 &  &  & -IOS \\ 
\hline
Conexión con otros SNS &  &  &  \\ 
\hline
\multicolumn{1}{l}{} &  &  & \multicolumn{1}{l}{} \\ 
\end{tabular}
  }
      \end{center}
\end{table}

\newpage

\begin{table}
  \caption{Comparacion de redes, parte 3}
  \label{tab:comparacion_redes_3}

  \begin{center}
  \resizebox{20cm}{!}{
  \begin{tabular}{|p{4cm}|p{7cm}p{6cm}p{9cm}|}
\hline
Fun\textbackslash Red social & Bkool & Deportmeet & Sportsnak \\ 
\hline
Gestión de foros &  &  & - Foros con profesionales (managers, coaches, teams) \\ 
 &  &  & - Ofrece posibilidad al usuario de ser moderador de foros \\ 
\hline
Gestión de encuentros deportivos & - Creación de eventos deportivos (solo o con amigos) &  - Gestión de eventos deportivos & - Manejo de eventos deportivos \\ 
 & - Gestión de ``retos'' &  &  \\ 
\hline
Gestión de grupos & Si &  &  \\ 
\hline
Manejo de torneos &  &  &  \\ 
\hline
Difusión info. Deportiva & - Gestión de información de ligas & - Artículos de profesionales & -Asociación con blogs deportivos \\ 
 &  &  & - Manejo de ``live scores'' \\ 
\hline
Serv. self-expression & -Subida de texto plano & -Difusión de multimedia & - Manejo contenido plano y multimedia \\ 
 & -Difusión de multimedia &  & - Uso de mensajería instantánea \\ 
\hline
Sistema estadístico & - Estadísticas de deportista & - Gestión del nivel del deportista &  \\ 
 &  & -Manejo de perfiles de usuario &  \\ 
\hline
Gestión de transversales &  & -Foros de nutrición &  \\ 
\hline
Servicios deportivos &  & - Venta de artículos deportivos & - Módulos para negociantes en temas de deporte \\ 
 &  &  & - Manejo de ofertas en ofrecimiento de instalaciones deportivas \\ 
 &  &  & -- Herramientas para hacer ``boost'' a negociantes (bussiness member) \\ 
\hline
Soporte multi-deporte & Deportes de ruta & Si & Si \\ 
 &  &  &  \\ 
\hline
Gestión de tipos de usu. &  &  & -Public member \\ 
 &  &  & -Club member \\ 
 &  &  & -Bussiness member \\ 
\hline
Gestión de sponsors &  &  & - Manejo de ``sponsorship'' \\ 
\hline
Gestión del conocimiento &  &  &  \\ 
 &  &  &  \\ 
\hline
Gestión de geolocaliza. & - Posibilidad de grabar trazados & - Localización de eventos & - Geolocalización de actividad deportiva cercana a un punto \\ 
 & (deportes de ruta) &  &  \\ 
\hline
Soporte móvil & -Android &  &  \\ 
 & -IOS &  &  \\ 
\hline
Conexión con otros SNS & -Facebook &  &  \\ 
 & -twitter &  &  \\ 
\hline
\end{tabular}
}
  
  \end{center}
\end{table}

\newpage

\begin{table}
  \caption{Comparacion de redes, parte 4}
  \label{tab:comparacion_redes_4}

  \begin{center}
    \resizebox{20cm}{!}{
    \begin{tabular}{|p{5cm}|lll|}
\hline
Fun\textbackslash Red social & Huddlers & Yoyde & Timpik \\ 
\hline
Gestión de foros &  &  &  \\ 
 &  &  &  \\ 
\hline
Gestión de encuentros deportivos & - Organización de eventos deportivos & - Manejo de eventos deportivos & - Manejo de eventos deportivos \\ 
 &  &  &  \\ 
\hline
Gestión de grupos &  &  &  \\ 
 &  &  &  \\ 
\hline
Manejo de torneos &  & Si &  \\ 
\hline
Difusión info. Deportiva &  & - Manejo de blogs &  \\ 
 &  &  &  \\ 
\hline
Serv. self-expression &  & -Manejo de ``muro'' & - Manejo de ``muro''  \\ 
 &  &  & -Gestión de mensajería \\ 
\hline
Sistema estadístico &  &  &  \\ 
 &  &  &  \\ 
 &  &  &  \\ 
\hline
Gestión de transversales &  &  &  \\ 
\hline
Servicios deportivos &  &  &  \\ 
\hline
Soporte multi-deporte & Si & Si & Si \\ 
 &  &  &  \\ 
\hline
Gestión de tipos de usu. &  & -Club deportivo & - Manejo de perfil deportivo \\ 
 &  & -Deportista &  \\ 
\hline
Gestión de sponsors &  &  &  \\ 
\hline
Gestión del conocimiento &  &  &  \\ 
 &  &  &  \\ 
\hline
Gestión de geolocaliza. & - Funcionalidad ``jugando en'' & - Manejo de escenarios deportivos &  \\ 
 &  & - Manejo de ``rutas'' &  \\ 
\hline
Soporte móvil & -IOS &  & -Android \\ 
\hline
Conexión con otros SNS &  &  &  \\ 
\hline
\end{tabular}
}
  
  \end{center}
\end{table}

\begin{table}
  \caption{Comparacion de redes, parte 5}
  \label{tab:comparacion_redes_5}

  \begin{center}
  \resizebox{20cm}{!}{
    \begin{tabular}{|p{4cm}|p{7cm}p{7cm}p{8cm}|}
\hline
Fun\textbackslash Red social & Socialsports & Strava & Ineftos \\ 
\hline
Gestión de foros &  &  & Si \\ 
\hline
Gestión de encuentros deportivos & - Organizador de eventos deportivos & - Manejo de desafíos (challenges) & - Organización de eventos \\ 
\hline
Gestión de grupos &  &  & Si \\ 
\hline
Manejo de torneos &  &  &  \\ 
\hline
Difusión info. Deportiva &  &  & - Manejo de blogs para estudiantes \\ 
\hline
Serv. self-expression & - Manejo de multimedia &  & - Manejo de mensajería \\ 
 &  &  & - Manejo de ``muro'' \\ 
 &  &  & - Manejo de multimedia \\ 
\hline
Sistema estadístico &  & - Gestión de estadísticas del atleta & - Utiliza mecanismo de encuestas para autorregularse \\ 
 &  & - Gestión de ``follows'' a otros deportistas para comparación de estadísticas (competencia) & - Gestión de foros: Estadísticas de foro \\ 
\hline
Gestión de transversales &  &  &  \\ 
\hline
Servicios deportivos & - Evaluación de la comunidad sobre los prestadores de servicio &  &  \\ 
\hline
Soporte multi-deporte & Si & Monodeporte (ciclomontañismo) & Si \\ 
\hline
Gestión de tipos de usu. & - Manejo de perfil de deportista (deportes practicados, lugares frecuentados, horarios frecuentados) &  & - Manejo de usuarios (profesores, alumnos, entidades sin ánimo de lucro) \\ 
 & - Manejo de usuarios (prestadores de servicio y deportistas) &  &  \\ 
\hline
Gestión de sponsors &  &  &  \\ 
\hline
Gestión del conocimiento &  & - Encuentro de consejos deportivos & - ``Social learning'' \\ 
\hline
Gestión de geolocaliza. &  & - Gestión de trazados logrados &  \\ 
 &  & - Gestión de trazados &  \\ 
\hline
Soporte móvil &  & -Android &  \\ 
\hline
Conexión con otros SNS &  &  &  \\ 
\hline
\end{tabular}
  }
  \end{center}
\end{table}

\end{landscape}
