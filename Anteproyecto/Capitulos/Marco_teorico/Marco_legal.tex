En cuanto al trabajo con datos, las leyes creadas en Colombia para la protección y manejo de estos son:
\begin{itemize}
  \item Constitución Nacional
  \item Ley 527 de 1999, la cual reglamenta el manejo de mercancías en el comercio electrónico, la utilización de firmas digitales, la
reglamentación para certificados expedidos de forma electrónica con firma digital, el manejo de los mensajes de datos y las
disposiciones de la Superintendencia de Industria y Comercio.
  \item Ley 1266 de 2008, el cual reglamenta el tratamiento de datos personales en bases de datos personales, haciendo énfasis en las
financieras y comerciales.
  \item Ley 1273 de 2009, la cual reglamenta el uso de la información y los sistemas de información en contra de la violación de la
confidencialidad, la integridad y la disponibilidad de los datos y los sistemas de información, así como también hurtos informáticos.
  \item Ley 1480 de 2011, la cual reglamenta los derechos y deberes tanto de consumidores como de productores en todos los sectores
económicos, aplicándose ésta a los productos tanto importados como nacionales.
  \item Resolución 3066 de 2011, la cual busca proteger los derechos de los usuarios de servicios de comunicaciones en los cuales se
establece también los derechos sobre los servicios adquiridos en telecomunicaciones.
  \item Decreto 1377 de 2013, el cual dictamina las políticas de protección y tratamiento de datos personales.
Además, se deben tener en cuenta las condiciones de servicio que Google ha impuesto para las aplicaciones desarrolladas para Android, así
como también las condiciones aplicadas a la utilización de dichas aplicaciones. Entonces, se han de tener en cuenta las siguientes
condiciones de servicio:
  \item Google Play Terms of Service, el cual dictamina las pautas de uso de Google Play por parte del usuario final, así como también las
facultades que tiene Google sobre la información y las aplicaciones instaladas en el dispositivo de un usuario.
  \item Developer Distribution Agreement, el cual reglamenta el uso que el desarrollador o distribuidor de aplicaciones hace de Google Play.
Habla acerca del licenciamiento, el manejo de precios y pagos, el manejo de marcas y publicidad y la dada de baja de las aplicaciones
de Google Play.
  \item Google Play Business and Program Policies, el cual reglamenta cómo deben ser utilizadas las aplicaciones en cuanto a la información
publicada en las mismas y además quien puede utilizar Google Play. Además, reglamenta la devolución, compra, descarga y soporte
de productos (aplicaciones) en Google Play.
  \item Developer Content Policy, el cual establece las políticas de contenido y publicidad que puede poner un desarrollador en sus
aplicaciones.
\end{itemize}
