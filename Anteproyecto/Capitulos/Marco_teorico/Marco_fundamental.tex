\section{UX - Análisis}
Los servicios de redes sociales (SNS por sus siglas en ingles: Social Network Services) como Facebook, LinkedIn, Twitter, SportTracker o Xportia, ofrecen servicios para la gestión de la OSN de cada usuario que acceda a estas aplicaciones. Según un estudio hecho para medir la experiencia de usuario \cite{user_behavior_online} (UX por sus siglas en ingles: User eXperience) en los SNS, se encontraron 8 categorías que son críticas a la hora de diseñar una SNS y son:

\begin{enumerate}
  \item Self-expresion: Capacidad que tengan las OSN de compartir contenido relacionado a la vida real de los usuarios tal como lo pueden ser las fotos, los videos, los comentarios o las comunicaciones directas.
  \item Reciprocity: Interacción bilateral en tiempo real, es decir, interacción instantánea con uno o varios individuales en la OSN (por ejemplo, por medio de los servicios de mensajería instantánea).
  \item Learning: La información recibida por medio de la OSN debe poder ser utilizada en pro del desarrollo cognitivo del individual; debe existir información útil al individual que usa la OSN.
  \item Curiosity: El contenido de la OSN debe ser interesante para quien la utiliza.
  \item Suitability of functionality: Se refiere a cuán ``utilizable'' es una funcionalidad.
  \item Suitability of content: La calidad y exactitud de la información que en la OSN reside debe ser suficiente para el individual perteneciente a ella.
  \item Completeness of the user network: Los individuales deben querer pertenecer a la red social y buscar eficientemente a otros individuales para poder formar lazos con ellos y hacer crecer su red social.
  \item Trust and privacy: Confianza en los servicios de las OSN, así como también la capacidad que tiene el usuario de gestionar la privacidad del contenido que comparte en dicha OSN. \cite{social_experience}
\end{enumerate}

Cada uno de las categorías nombradas hace parte de los factores que impulsan la utilización de los SNS para la gestión de las OSN de las personas.
