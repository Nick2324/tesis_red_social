
Equipo SCRUM:

\begin{itemize}
  \item Product Owner \\
	Doctor Carlos Enrique Montenegro

  \item Developement Team
	\begin{itemize}
	  \item Nicolás Mauricio García Garzón
	  \item Luis Felipe Gonzalez Moreno
	\end{itemize}

  \item Scrum Master \\
	Profesor Alejandro Daza
\end{itemize}

Actividades de cada Sprint (15 días máximo):
  
\begin{itemize}
  \item Sprint Planning (4 horas máximo)\\
  Se realizó una reunión con el Scrum Master al inicio de cada iteración. En esta reunión se discutió el desempeño del Sprint anterior, los servicios/funcionalidades que hacían falta para completar el prototipo, y que servicios debían ser realizados en la iteración.
	\begin{itemize}
	  \item ¿Qué se hizo en cada sprint? \\
	    Con base a los servicios y casos de uso de negocio encontrados, se determino que podía ser desarrollado por el developement team en cada iteración del prototipo.
	  \item ¿Como se llevará a cabo este trabajo? \\
	    Se dividieron los servicios propuestos como parte del Sprint entre los desarrolladores y se compartieron las expectativas que debe cumplir cada servicio para que sea aceptado en el desarrollo.
	\end{itemize}
  \item Daily Scrum (15 minutos máximo)[Solo participa el developement team] \\
  Partiendo de los items asignados a cada desarrollador, se dividieron en tareas aún mas pequeñas que sirvieron como ruta para cumplir con el Sprint Goal. Se respondieron las siguientes preguntas para llevar un seguimiento continuo del desarrollo del prototipo:
	\begin{itemize}
	  \item ¿Qué se hizo ayer que ayudó al developement team para cumplir el Sprint goal?
	  \item ¿Qué se va a hacer hoy para ayudar al developement team a cumplir el Sprint goal?
	  \item ¿Hay algún impedimento para que el developement team cumpla el Sprint goal?
	\end{itemize}
  \item Sprint Review (2 horas máximo) \\
  Se realizó al final de cada Sprint. Se mostro qué se hizo en el sprint con respecto a las tareas propuestas desde el inicio del mismo. Las actividades básicas que se llevaron a cabo fueron:
	\begin{itemize}
	  \item Socializar la experiencia en el sprint, que problemas ocurrieron y cómo se solucionaron.
	  \item Exponer los diferentes elementos que fueron construidos y se resuelven preguntas acerca de los mismos.
	  \item Proponer que puede hacerse en el siguiente Sprint basados en la experiencia del actual.
	  \item Revisar cómo el cambio en el entorno pudo cambiar las prioridades en el trabajo del equipo.
	\end{itemize}
  \item Sprint Retrospective
  \begin{itemize}
    \item Se tomaron las experiencias de cada sprint para formular sugerencias que ayudaron a mejorar los siguientes.

  \end{itemize}
\end{itemize}

A continuación, se presenta el product backlog inicial del proyecto.

\begin{table}[h]
  \caption{Product backlog inicial}
  \label{tab:backlog}

  \begin{center}
    \resizebox{15cm}{!}{
      \begin{tabular}{|l|l|l|}
        \hline
        Tarea & Días & Condición de aprobación \\ 
        \hline
        \hline
        Levantamiento de requerimientos & 2 & Satisfacción de todos los \\ 
        
         &  & requerimientos para la red social \\ 
        
        Definición de requerimientos funcionales y no funcionales & 2 & Modularización y descripción total de  \\ 
        
         &  & todos los requerimientos \\ 
        
        Investigación de tecnologías existentes & 6 & Escogencia de las tecnologías a utilizar \\ 
        
         &  & para implementar la red social \\ 
        
        Diseño de casos de uso & 5 & Cubrimiento de todos los requerimientos identificados \\ 
        
        Refinamiento de requerimientos & 1 & Trazabilidad entre casos de uso y requerimientos \\ 
        
        Identificación de servicios candidatos & 1 & Cubrimiento de todos los requerimientos identificados \\ 
        
        Diseño de blueprints de servicios & 4 & Cubrimiento de todos los servicios candidatos \\ 
        
        Escogencia de servicios a ser implementados & 1 & Viabilidad de un prototipo funcional \\ 
        
        Composición estática de servicios & 5 & Concordancia entre los casos de uso  \\ 
        
         &  & y la composición de servicios \\ 
        
        
        Diseño de base de datos & 6 & Diseño que cubra los servicios a ser implementados \\ 
        
         &  &  y requerimientos no funcionales \\ 
        
        Diseño de interfaz gráfica de usuario & 4 & Cubrimiento de los servicios y casos de uso a ser implementados \\ 
        
         &  &  a ser implementados \\ 
        
        Refinamiento de casos de uso & 1 & Trazabilidad \\ 
        
        Refinamiento de blueprints de servicios & 1 & Trazabilidad \\ 
        
        Refinamiento de servicios a ser implementados & 1 & Trazabilidad \\ 
        
        Refinamiento de composición estática de servicios & 1 & Trazabilidad \\ 
        
        Construcción de la interfaz de usuario & 20 & Navegabilidad sobre las funcionalidades a ser implementadas \\ 
        
        y refinamiento de interfaz de usuario &  &  \\ 
        
        Construcción de los servicios a ser  & 35 & Construcción de prototipo funcional sin fallas \\ 
        
        implementados y refinamiento de modelos &  & detectadas en tiempo de desarrollo \\ 
        
        Pruebas del prototipo por parte del equipo de desarrollo & 2 & Prototipo sin fallas detectables en sus funcionalidades \\ 
        
        Pruebas del prototipo por parte del usuario final & 7 & Prototipo aceptado por el usuario final en al menos un 70\% \\
        \hline

        \end{tabular}
    }
  \end{center}
\end{table}
