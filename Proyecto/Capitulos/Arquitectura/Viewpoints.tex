\section{Vistas archimate}

A continuación se muestran las vistas generadas por los autores para la muestra de la arquitectura modelada sobre Archimate 2.0. En el capítulo de anexos, en la sección \ref{sec:anexo_artefactos}, se puede consultar la descripción de los artefactos que en cada una de ellas aparece.

\subsubsection{Actores y Roles}

\begin{figure}[!htb]
  \begin{center}
    \includegraphics[width=11cm]{./imagenes/actores.png}
    \caption{Actores}
    \label{fig:Actores}
    \textbf{Fuente:}  Autores
  \end{center}
\end{figure}

Aunque esta no es una vista archimate, haciendo uso de esta es posible observar los actores tenidos en cuenta en la creación del SNS. Un actor deportivo es quien desencadena a los demás actores que pueden desenvolverse en la red social debido a las funcionalidades pensadas para ella. El único actor que no ha sido propiamente pensado para ser soportado por la red social, pero que se hace necesario a la hora de tener en claro el negocio, es el periodista. El resto, en su totalidad, son soportados por la red social.

\begin{figure}[!htb]
  \begin{center}
    \includegraphics[width=11cm]{./imagenes/roles.png}
    \caption{Roles}
    \label{fig:Roles}
    \textbf{Fuente:}  Autores
  \end{center}
\end{figure}

Es una versión pequeña de un punto de vista organizacional en el que solo se intenta mostrar la relación de asignación entre actores y roles identificados en el diseño de la arquitectura. Debido a la complejidad de una red social deportiva, es posible que un actor adquiera todos los roles mostrados en la vista, sin embargo, especializandose en unos o accediendo principalmente a un rol. Tal es el caso de los roles de agrupación deportiva, entrenador deportivo, prestador de servicios deportivos y patrocinador deportivo.

\subsubsection{Introductory viewpoint}

\begin{figure}[!htb]
  \begin{center}
    \includegraphics[width=11cm]{./imagenes/introductory.png}
    \caption{Punto de vista introductorio}
    \label{fig:introductory}
    \textbf{Fuente:}  Autores
  \end{center}
\end{figure}

Esta vista muestra las principales características ofrecidas por la red social. Así, es visto que el SNS buscará, en específico, cumplir las funciones de geolocalización, del soporte de información acerca de un deporte, del manejo de social media (esto es, el despliegue de funciones para interactuar con otros en la red social) y el módulo de gestión del conocimiento (similar a la gestión de una comunidad/foro en internet).

\subsubsection{Layered viewpoint}

\begin{figure}[!htb]
  \begin{center}
    \includegraphics[width=11cm]{./imagenes/generallayered.png}
    \caption{Punto de Vista General por Capas}
    \label{fig:general_layered}
    \textbf{Fuente:}  Autores
  \end{center}
\end{figure}

Este punto de vista es una ampliación del punto de vista introductorio. En este punto de vista es posible observar la visión que han tenido los autores para mostrar una red social deportiva real, todas las funciones que cumple y servicios que realiza/consume un actor deportivo sobre la red. A su vez, es posible observar la ampliación y la aparición de nuevas funcionalidades que soportaría el SNS, teniendo como base la orientación de la red social al deporte amateur o en camino de ser profesional. Entre las nuevas funcionalidades puede observarse la gestión de patrocinios deportivos, la gestión de compra/venta sobre el SNS, la realización de predicciones estadísticas (y, por ende, la producción de las estadísticas mismas) de eventos deportivos y actores deportivos en la red social y la gestión de eventos deportivos. En cuanto al elemento ampliado, se puede observar la ampliación de la geolocalización al ambito de los actores deportivos en la red social así como también de los eventos que en ella se produzcan.

En la capa tecnológica puede observarse la arquitectura cliente/servidor sobre internet, teniendo como terminales los dispositivos móbiles que, en este caso, serán dispositivos Android.

\subsection{Business Functions Viewpoints}

\subsubsection{Patrocinios Deportivos}

\begin{figure}[!htb]
  \begin{center}
    \includegraphics[width=11cm]{./imagenes/business_functions/patrociniosdeportivos.png}
    \caption{Patrocinios Deportivos}
    \label{fig:bf_patrocinios_deportivos}
    \textbf{Fuente:}  Autores
  \end{center}
\end{figure}

En este punto de vista se puede observar cómo los patrocinadores pueden patrocinar diferentes actores en la red social. Más a fondo, se puede ver como un patrocinador apoya los proyectos desempeñados por un actor que desempeña un rol específico en la red social. Así, por ejemplo, un gestor de encuentros deportivos puede pedir un patrocinio o recibir una petición de un patrocinador para ser patrocinado por el evento que va a realizar. También se puede observar que un patrocinador deportivo patrocina proyectos de otros actores sobre la red social con su financiamiento, con la prestación de uno o varios servicios o con la provisión de uno o varios recursos materiales.

\subsubsection{Organización de Eventos Deportivos}

\begin{figure}[!htb]
  \begin{center}
    \includegraphics[width=11cm]{./imagenes/business_functions/organizacioneventosdeportivos.png}
    \caption{Organización de Eventos Deportivos}
    \label{fig:bf_organizacion_eventos_deportivos}
    \textbf{Fuente:}  Autores
  \end{center}
\end{figure}

En este punto de vista se puede observar el paso de información a través de los diferentes roles que interactúan en la creación y ejecución de un evento deportivo (patrocinadores, agrupaciones deportivas y gestores de encuentros deportivos) quienes intercambian la información de cada uno para ser patrocinado, para pedir participación o para participar en el evento deportivo.

\subsubsection{Formación de Grupos Deportivos}

\begin{figure}[!htb]
  \begin{center}
    \includegraphics[width=11cm]{./imagenes/business_functions/formaciongruposdeportivos.png}
    \caption{Formación de Grupos Deportivos}
    \label{fig:bf_formacion_grupos_deportivos}
    \textbf{Fuente:}  Autores
  \end{center}
\end{figure}

En cuanto a la formación de grupos se refiere, entre roles intercambian la información acerca de sus peticiones a unirse a la red social, así como la respuesta a la misma. Se puede ver que los roles que interactuan en la formación de grupos deportivos son solo los de entrenadores y agrupaciones deportivas y que ellos mismos conforman los grupos deportivos.

\subsubsection{Estadísticas Deportivas}

\begin{figure}[!htb]
  \begin{center}
    \includegraphics[width=11cm]{./imagenes/business_functions/estadisticasdeportivas.png}
    \caption{Estadísticas Deportivas}
    \label{fig:bf_estadisticas_deportivas}
    \textbf{Fuente:}  Autores
  \end{center}
\end{figure}

Sobre este punto de vista se puede observar que, en específico, entrenadores, patrocinadores deportivos y agrupaciones deportivas producen y utilizan estadísticas. Estas estadísticas son traducidas en calificaciones a los demás en, por ejemplo, el préstamo de servicios deportivos. En el punto de vista de proceso de negocio de estadísticas deportivas es más preciso ver cuales son los tipos de estadística con los cuales se trabaja.

\subsubsection{Entrenamiento Deportivo}

\begin{figure}[!htb]
  \begin{center}
    \includegraphics[width=11cm]{./imagenes/business_functions/entrenamientodeportivo.png}
    \caption{Entrenamiento Deportivo}
    \label{fig:bf_entrenamiento_deportivo}
    \textbf{Fuente:}  Autores
  \end{center}
\end{figure}

Sobre este punto de vista se puede observar el flujo de información entre los roles participantes en un entrenamiento deportivo. Todos están basados en la valoración del trabajo producido en la práctica según los eventos producidos en la misma y la valoración que a estos de cada rol para, así, producir ejercicios deportivos para el entrenamiento.

\subsubsection{Buscador Deportivo}

\begin{figure}[!htb]
  \begin{center}
    \includegraphics[width=11cm]{./imagenes/business_functions/buscadordeportivo.png}
    \caption{Buscador deportivo}
    \label{fig:BF_BuscadorDeportivo}
    \textbf{Fuente:}  Autores
  \end{center}
\end{figure}

\subsubsection{Consejos}

\begin{figure}[!htb]
  \begin{center}
    \includegraphics[width=11cm]{./imagenes/business_functions/Consejos.png}
    \caption{Consejos}
    \label{fig:BF_Consejos}
    \textbf{Fuente:}  Autores
  \end{center}
\end{figure}

\subsubsection{Periodismo}

\begin{figure}[!htb]
  \begin{center}
    \includegraphics[width=11cm]{./imagenes/business_functions/Periodismo.png}
    \caption{Periodismo}
    \label{fig:BF_Periodismo}
    \textbf{Fuente:}  Autores
  \end{center}
\end{figure}

\subsubsection{Servicios}

\begin{figure}[!htb]
  \begin{center}
    \includegraphics[width=11cm]{./imagenes/business_functions/Servicios.png}
    \caption{Servicios}
    \label{fig:BF_Servicios}
    \textbf{Fuente:}  Autores
  \end{center}
\end{figure}

\subsection{Business Process Viewpoints}

\subsubsection{Patrocinios Deportivos}

\begin{figure}[!htb]
  \begin{center}
    \includegraphics[width=11cm]{./imagenes/business_process/patrociniosdeportivos.png}
    \caption{Patrocinios Deportivos}
    \label{fig:bp_patrocinios_deportivos}
    \textbf{Fuente:}  Autores
  \end{center}
\end{figure}

En el proceso de patrocinios, los autores han orientado el proceso de negocio partiendo desde el patrocinado hacia el patrocinador (con la petición de patrocinios) y desde el patrocinador al patrocinado (dejando al patrocinador decidir si patrocinar o no un proyecto según el seguimiento que él le haga a éste). Se puede ver también que los servicios ofrecidos soportan todos los procesos de negocio ilustrados.

\subsubsection{Organización de Eventos Deportivos}

\begin{figure}[!htb]
  \begin{center}
    \includegraphics[width=11cm]{./imagenes/business_process/organizacioneventosdeportivos.png}
    \caption{Organización de Eventos Deportivos}
    \label{fig:bp_organizacion_eventos_deportivos}
    \textbf{Fuente:}  Autores
  \end{center}
\end{figure}

En cuanto al proceso de organización de eventos deportivos, los autores han decidido dividir éste en tres procesos grandes: Planear el proyecto, ejecutar el proyecto y participación en el evento deportivo. Se puede ver también que un gestor de eventos deportivos puede trabajar a la par con un patrocinador deportivo para organizar el evento deportivo, con lo cual se puede discernir la conexión entre un patrocinio al evento deportivo y la organización del mismo. El primer gran proceso es soportado por servicios que proporcionan capacidades para tratar con los datos del evento; del segundo gran proceso se soportan algunos de los subprocesos pertenecientes a este, los que son considerados valiosos para el desarrollo del SNS como la invitación y notificación a participantes, así como también la clausura de un evento y la gestión de formatos deportivos; en el tercer gran proceso solo se tienen servicios para la petición a participar en el evento o la participación en el evento mismo.

\subsubsection{Formación de Grupos Deportivos}

\begin{figure}[!htb]
  \begin{center}
    \includegraphics[width=11cm]{./imagenes/business_process/formaciongruposdeportivos.png}
    \caption{Formación de Grupos Deportivos}
    \label{fig:bp_formacion_grupos_deportivos}
    \textbf{Fuente:}  Autores
  \end{center}
\end{figure}

Este punto de vista de proceso de negocio está basado en dos procesos principales: La creación de un grupo deportivo y la interacción con el grupo deportivo. Lo que los arquitectos transmiten es el cómo, luego de la creación del grupo, se desempeñan procesos que luego serán recurrentes a través de la vida del grupo deportivo: La interacción grupal y la asignación de roles sobre integrantes del grupo deportivo. Según el punto de vista, solo aquellos con rol de entrenadores deportivos y de agrupaciones deportivas pueden formar un grupo deportivo. Hay un tercer proceso que es la lectura de información del grupo deportivo, en el caso de un actor deportivo que esté interesado en unirse/participar en el grupo deportivo. Este punto de vista es utilizado tanto para grupos deportivos como para equipos deportivos, no permitiendo al primero participar o tener características del segundo.

Los procesos de negocio representados son cubiertos por un servicio grande específico para prestar funcionalidades a los grupos deportivos y éste, a su vez, hace uso del servicio de social media para poder implementar la interacción grupal.

\subsubsection{Estadísticas Deportivas}

\begin{figure}[!htb]
  \begin{center}
    \includegraphics[width=11cm]{./imagenes/business_process/estadisticasdeportivas.png}
    \caption{Estadísticas Deportivas}
    \label{fig:bp_estadisticas_deportivo}
    \textbf{Fuente:}  Autores
  \end{center}
\end{figure}

Esta vista se ha dedicado a expresar el proceso de generación de estadísticas. El diferenciador de este proceso es el tipo de estadísticas que se generan y, por ende, los objetos de negocio que estarán asociados a dichas estadísticas. Las estadísticas generadas serán las de servicios deportivos ofrecidos sobre la red social, sobre estadísticas deportivas de agrupaciones deportivas, sobre estadísticas geoespaciales teniendo en cuenta aspectos de niveles de juego, deportes practicados y eventos generados sobre dicha ubicación geoespacial, y estadísticas propias de un evento deportivo.

En la vista, sobre la capa de aplicación, pueden verse los servicios que soportarán el proceso de generación de estadísticas.

\subsubsection{Entrenamiento Deportivo}

\begin{figure}[!htb]
  \begin{center}
    \includegraphics[width=11cm]{./imagenes/business_process/entrenamientodeportivo.png}
    \caption{Entrenamiento Deportivo}
    \label{fig:bp_entrenamiento_deportivo}
    \textbf{Fuente:}  Autores
  \end{center}
\end{figure}

El proceso de entrenamiento deportivo ha sido dividido en dos procesos grandes, uno atado al entrenador solamente (entrenar grupo deportivo) y otro atado tanto a la agrupación deportiva como al entrenador deportivo (ejecutar plan de entrenamiento). Sobre estos dos grandes procesos se destacan las funciones expresadas en el punto de vista de funciones de negocio de entrenamiento deportivo. Sobre esta vista se manejan dos objetos de negocio principales: evaluaciones deportivas y planes de entrenamiento.

En función de soportar los procesos expresados por los autores, se generaron tres servicios de aplicación principales: La ejecución de un plan de entrenamiento, otros servicios de gestión del plan de entrenamiento y un servicio para el soporte de la evaluación deportiva, el cual se nombró "servicio de comunicación colaborativa".

\subsubsection{Buscador de consejos deportivos}

\begin{figure}[!htb]
  \begin{center}
    \includegraphics[width=11cm]{./imagenes/business_process/buscadorconsejosdeportivos.png}
    \caption{Buscador de consejos deportivos}
    \label{fig:BP_BuscadorConsejosDeportivos}
    \textbf{Fuente:}  Autores
  \end{center}
\end{figure}

\subsubsection{Buscador deportivo}

\begin{figure}[!htb]
  \begin{center}
    \includegraphics[width=11cm]{./imagenes/business_process/buscadordeportivo.png}
    \caption{Buscador deportivo}
    \label{fig:BP_BuscadorDeportivo}
    \textbf{Fuente:}  Autores
  \end{center}
\end{figure}

\subsubsection{Evaluador consejos de salud}

\begin{figure}[!htb]
  \begin{center}
    \includegraphics[width=11cm]{./imagenes/business_process/evaluadorconsejossalud.png}
    \caption{Evaluador consejos de salud}
    \label{fig:BP_EvaluadorConsejosSalud}
    \textbf{Fuente:}  Autores
  \end{center}
\end{figure}

\subsubsection{Generador de consejos deportivos}

\begin{figure}[!htb]
  \begin{center}
    \includegraphics[width=11cm]{./imagenes/business_process/generadorconsejosdeportivos.png}
    \caption{Generador de consejos deportivos}
    \label{fig:BP_GeneradorConsejosDeportivos}
    \textbf{Fuente:}  Autores
  \end{center}
\end{figure}

\subsubsection{Generador de contenidos de salud}

\begin{figure}[!htb]
  \begin{center}
    \includegraphics[width=11cm]{./imagenes/business_process/generadorcontenidossalud.png}
    \caption{Generador de contenidos de salud}
    \label{fig:BP_GeneradorContenidosSalud}
    \textbf{Fuente:}  Autores
  \end{center}
\end{figure}

\subsection{Application Usage Viewpoints}

\subsubsection{Patrocinios Deportivos}

\begin{figure}[!htb]
  \begin{center}
    \includegraphics[width=11cm]{./imagenes/application_usage/patrociniosdeportivos.png}
    \caption{Patrocinios Deportivos}
    \label{fig:au_patrocinios_deportivos}
    \textbf{Fuente:}  Autores
  \end{center}
\end{figure}

A parte de lo ya observado en el punto de vista de proceso de negocio de patrocinios deportivos, en este punto de vista puede observarse con los objetos de negocio, en particular, que en la aplicación el objeto de negocio "proyecto" no va a ser soportado más que como un posible paso de mensajes entre patrocinador y patrocinado, puesto que el servicio que soporta el proceso de negocio de petición de patrocinios sólo envía la intención de ser patrocinado como un mensaje, en otras palabras, el SNS no soporta la creación de proyectos aunque si facilita la visualización de factores que influyen en el patrocinio de un actor deportivo (como, por ejemplo, las estadísticas deportivas generadas para un actor deportivo).

Se hace la creación de componentes de negocio para la gestión de cada facilidad ofrecida por un patrocinador, así como también uno para el patrocinio mismo, gestión de patrocinios, soportando los servicios de patrocinio ofrecidos. La utilización de los demás servicios se ve reflejada en componentes relacionados con estadísticas, con grupos deportivos y con eventos deportivos. También se observa un componente de comunicación, el cual realiza el servicio de mensajería para el envío de mensajes (proyecto) entre patrocinador y patrocinado.

\subsubsection{Organización de Eventos Deportivos}

\begin{figure}[!htb]
  \begin{center}
    \includegraphics[width=11cm]{./imagenes/application_usage/organizacioneventosdeportivos.png}
    \caption{Organización de Eventos Deportivos}
    \label{fig:au_organizacion_eventos_deportivos}
    \textbf{Fuente:}  Autores
  \end{center}
\end{figure}

Para la organización de eventos deportivos, a parte de lo visto en el punto de vista de proceso de negocio, se puede observar que el SNS deportivo, en una fase final de desarrollo (más allá del prototipo que se alcanza en este proyecto), está dedicado al soporte de funcionalidades para eventos deportivos tales como torneos, clínicas, eventos informativos (como ejemplo, una conferencia deportiva) y, el elemento principal, las prácticas deportivas (prácticas informales realizadas por jugadores amateur).

En soporte de los servicios mostrados, está el componente de gestión de eventos que, junto con la gestión de patrocinio, de geolocalización, de estadísticas y de usuarios.

\subsubsection{Formación de Grupos Deportivos}

\begin{figure}[!htb]
  \begin{center}
    \includegraphics[width=11cm]{./imagenes/application_usage/formaciongruposdeportivos.png}
    \caption{Formación de Grupos Deportivos}
    \label{fig:au_formacion_grupos_deportivos}
    \textbf{Fuente:}  Autores
  \end{center}
\end{figure}

A parte de lo ya dicho en el punto de vista de proceso de negocio de formación de grupos deportivos, se puede observar los componentes en soporte de los servicios de aplicación: Uno para la gestión del grupo deportivo, otro para la gestión de usuarios y el otro para la gestión de la comunicación.

\subsubsection{Estadísticas Deportivas}

\begin{figure}[!htb]
  \begin{center}
    \includegraphics[width=11cm]{./imagenes/application_usage/estadisticasdeportivas.png}
    \caption{Estadísticas Deportivas}
    \label{fig:au_estadisticas_deportivas}
    \textbf{Fuente:}  Autores
  \end{center}
\end{figure}

A parte de lo expresado en el punto de vista de proceso de negocio de estadísticas deportivas, hay múltiples componentes que interactuarán por medio de los servicios que soportan debido al amplio espectro que ocupan las estadísticas sobre el SNS. A parte del componente de gestión de estadísticas, el de usuarios, el de eventos, el de servicios y el de geolocalización ocupan un lugar por ser estos evaluados por medio de estadísticas. Se encuentra también uno de manejo de contenidos que es usado para el compartir las estadísticas y comentar sobre ellas como se haría en la red social deportiva real.

\subsubsection{Entrenamiento Deportivo}

\begin{figure}[!htb]
  \begin{center}
    \includegraphics[width=11cm]{./imagenes/application_usage/entrenamientodeportivo.png}
    \caption{Entrenamiento Deportivo}
    \label{fig:au_entrenamiento_deportivo}
    \textbf{Fuente:}  Autores
  \end{center}
\end{figure}

Para el soporte de los servicios proporcionados por el SNS, son utilizados componentes de estadística, de gestión de usuarios y de grupos deportivos, así como también de comunicación y de manejo de contenidos para la evaluación de los entrenados y la ejecución del plan de entrenamiento por parte de los mismos. Hay un último componente de aplicación que es el de gestión de servicio de coaching, el cual realiza los servicios relacionados con el plan de entrenamiento y que, a su vez, está compuesto en el componente de gestión de servicios ya que éste entrenamiento deportivo es visto en la red social como otro servicio prestado a los actores que ella interactúan.

\subsubsection{Buscador de consejos deportivos}

\begin{figure}[!htb]
  \begin{center}
    \includegraphics[width=11cm]{./imagenes/application_usage/buscadorconsejosdeportivos.png}
    \caption{Buscador de consejos deportivos}
    \label{fig:BP_BuscadorConsejosDeportivos}
    \textbf{Fuente:}  Autores
  \end{center}
\end{figure}

\subsubsection{Buscador deportivo}

\begin{figure}[!htb]
  \begin{center}
    \includegraphics[width=11cm]{./imagenes/application_usage/buscadordeportivo.png}
    \caption{Buscador deportivo}
    \label{fig:BP_BuscadorDeportivo}
    \textbf{Fuente:}  Autores
  \end{center}
\end{figure}

\subsubsection{Evaluador consejos de salud}

\begin{figure}[!htb]
  \begin{center}
    \includegraphics[width=11cm]{./imagenes/application_usage/evaluadorconsejossalud.png}
    \caption{Evaluador consejos de salud}
    \label{fig:BP_EvaluadorConsejosSalud}
    \textbf{Fuente:}  Autores
  \end{center}
\end{figure}

\subsubsection{Generador de consejos deportivos}

\begin{figure}[!htb]
  \begin{center}
    \includegraphics[width=11cm]{./imagenes/application_usage/generadorconsejosdeportivos.png}
    \caption{Generador de consejos deportivos}
    \label{fig:BP_GeneradorConsejosDeportivos}
    \textbf{Fuente:}  Autores
  \end{center}
\end{figure}

\subsubsection{Generador de contenidos de salud}

\begin{figure}[!htb]
  \begin{center}
    \includegraphics[width=11cm]{./imagenes/application_usage/generadorcontenidossalud.png}
    \caption{Generador de contenidos de salud}
    \label{fig:BP_GeneradorContenidosSalud}
    \textbf{Fuente:}  Autores
  \end{center}
\end{figure}

\subsection{Product Viewpoint}

\subsubsection{Product}

\begin{figure}[!htb]
  \begin{center}
    \includegraphics[width=11cm]{./imagenes/Product.png}
    \caption{Producto}
    \label{fig:Product}
    \textbf{Fuente:}  Autores
  \end{center}
\end{figure}

\subsection{Punto de vista de infraestructura}

\begin{figure}[!htb]
  \begin{center}
    \includegraphics[width=11cm]{./imagenes/infrastructure.png}
    \caption{Punto de vista de infraestructura}
    \label{fig:infrastructure}
    \textbf{Fuente:}  Autores
  \end{center}
\end{figure}

Sobre esta capa se puede observar que los arquitectos han decidido utilizar una arquitectura cliente/servidor para el soporte del desarrollo del SNS. A su vez, se puede observar que el software utilizado por parte del cliente deberá ser un sistema android con el cliente desarrollado específicamente para dar la GUI del SNS. Por parte del servidor se puede ver que se utilizarán entornos Java para el servidor de aplicaciones, Neo4j como la base de datos y sistema operativo Debian. Todo estará soportado sobre conexiones por medio de la red internet.