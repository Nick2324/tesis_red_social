% Please add the following required packages to your document preamble:
% \usepackage[table,xcdraw]{xcolor}
% If you use beamer only pass "xcolor=table" option, i.e. \documentclass[xcolor=table]{beamer}
\begin{table}[!htb]
	\caption{Entidad E\_Usuario}
	\label{tab:entidad_usuario}
	\begin{center}
		\resizebox{16cm}{!}{
\begin{tabular}{|l|l|l|l|}
\hline
\multicolumn{4}{|c|}{\textbf{Entidad}} \\ \hline
\textbf{Nombre} & \multicolumn{3}{l|}{E\_Usuario} \\ \hline
\textbf{Descripcion} & \multicolumn{3}{l|}{Representa a una persona u organización que tiene una cuenta en la aplicación.} \\ \hline
\multicolumn{4}{|c|}{\textbf{Atributos}} \\ \hline
Nombre & Tipo & Longitud & Descripción \\
			\hline
usuario{[}P{]} & String & 20 &  Representa el nombre de usuario que va a utilizar para iniciar sesión en la aplicación. \\ \hline
contraseña & String & 40 &  Se guarda la contraseña del usuario encriptada con el algoritmo SHA-1 \\ \hline
E-mail{[}U{]} & String & 40 &  \\ \hline
Genero & String & 1 & M|F \\ \hline
numeroContacto & String & 10 &  \\ \hline
primerNombre & String & 40 &  \\ \hline
segundoNombre & String & 40 &  \\ \hline
primerApellido & String & 40 &  \\ \hline
segundoApellido & String & 40 &  \\ \hline
fechaNacimiento & String & 10 & Formato DD/MM/YYYY \\ \hline
fechaRegistro & String & 10 & Formato DD/MM/YYYY \\ \hline
Genero & String & 1 & M/F \\ \hline
estado & boolean & N/A &  Representa el estado del usuario (Activo/inactivo) \\ \hline
\multicolumn{4}{|c|}{\textbf{Relaciones a entidades}} \\ \hline
\multicolumn{2}{|l|}{\textbf{Relacion}} & \multicolumn{2}{l|}{\textbf{Entidad}} \\ \hline
\multicolumn{2}{|l|}{EnviaMensaje} & \multicolumn{2}{l|}{Mensaje} \\ \hline
\multicolumn{2}{|l|}{RecibeMensaje} & \multicolumn{2}{l|}{Mensaje} \\ \hline
\multicolumn{2}{|l|}{PublicaMultimedia} & \multicolumn{2}{l|}{Multimedia} \\ \hline
\multicolumn{2}{|l|}{AsumePosicionDeporte} & \multicolumn{2}{l|}{Posicion} \\ \hline
\multicolumn{2}{|l|}{PracticaDeporte} & \multicolumn{2}{l|}{Deporte} \\ \hline
\multicolumn{2}{|l|}{ParticipaEvento} & \multicolumn{2}{l|}{Evento} \\ \hline
\multicolumn{2}{|l|}{ConformaGrupo} & \multicolumn{2}{l|}{Grupo} \\ \hline
\multicolumn{2}{|l|}{AsumeRol} & \multicolumn{2}{l|}{Rol} \\ \hline
\multicolumn{2}{|l|}{DueñoTimeline} & \multicolumn{2}{l|}{Timeline} \\ \hline
\multicolumn{2}{|l|}{PublicaPost} & \multicolumn{2}{l|}{Post} \\ \hline
\multicolumn{2}{|l|}{Comenta} & \multicolumn{2}{l|}{Comentario} \\ \hline
\multicolumn{2}{|l|}{GeneraContenido} & \multicolumn{2}{l|}{Contenido} \\ \hline
\multicolumn{2}{|l|}{Amigo} & \multicolumn{2}{l|}{Usuario} \\ \hline
\end{tabular}
		} \\
		\textbf{Fuente}: Autores
		\end{center}
	\end{table}