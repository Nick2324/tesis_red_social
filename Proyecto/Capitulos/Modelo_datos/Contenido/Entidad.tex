% Please add the following required packages to your document preamble:
% \usepackage[table,xcdraw]{xcolor}
% If you use beamer only pass "xcolor=table" option, i.e. \documentclass[xcolor=table]{beamer}
\begin{table}[h]
\begin{tabular}{|l|l|l|l|l|}
\hline
\multicolumn{5}{|c|}{\cellcolor[HTML]{9BC2E6}\textbf{Entidad}} \\ \hline
\textbf{Nombre} & \multicolumn{4}{l|}{E_Contenido} \\ \hline
\textbf{Descripcion} & \multicolumn{4}{l|}{\begin{tabular}[c]{@{}l@{}}Representa un contenido registrado en la base de\\   conocimientos de la aplicación.\end{tabular}} \\ \hline
\multicolumn{5}{|c|}{\textbf{Atributos}} \\ \hline
\multicolumn{2}{|l|}{tipoContenido{[}P{]}} & int & N/A & \begin{tabular}[c]{@{}l@{}}Representa\\   el tipo de contenido que se referencia (Deportivo/De salud)\end{tabular} \\ \hline
\multicolumn{2}{|l|}{fechaCreacion} & String & 10 & \begin{tabular}[c]{@{}l@{}}Formato\\   DD/MM/YYYY\end{tabular} \\ \hline
\multicolumn{2}{|l|}{cuerpo} & String & 4000 &  \\ \hline
\multicolumn{2}{|l|}{titulo} & String & 100 &  \\ \hline
\multicolumn{2}{|l|}{tags} & String & 100 &  \\ \hline
\multicolumn{2}{|l|}{consecutivoContenido{[}P{]}} & long &  &  \\ \hline
\multicolumn{5}{|c|}{\textbf{Relaciones a entidades}} \\ \hline
\multicolumn{2}{|l|}{\textbf{Relacion}} & \multicolumn{3}{l|}{\textbf{Entidad}} \\ \hline
\multicolumn{2}{|l|}{Comentario} & \multicolumn{3}{l|}{Comentario} \\ \hline
\multicolumn{2}{|l|}{RelacionadoCon} & \multicolumn{3}{l|}{Deporte} \\ \hline
\multicolumn{2}{|l|}{ExplicadoCon} & \multicolumn{3}{l|}{Multimedia} \\ \hline
\end{tabular}
\end{table}