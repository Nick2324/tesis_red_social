La identificación de los requerimientos funcionales consignados en éste capítulo fue la base para realizar la arquitectura del software a implementar.

Se identificaron, en el análisis de requerimientos, 14 posibles módulos enunciados a continuación:

\begin{enumerate}
	\item \textbf{Gestión de usuarios*}: Módulo que controla características inherentes a todos los tipos de usuario de la red social en cuanto al manejo de su información personal y roles que cumplen
	\item \textbf{Gestión de deportes*}: Módulo por medio del cual se controla la información detallada de un deporte
	\item \textbf{Gestión de equipos}: Módulo que ayuda a la gestión de datos competentes a equipos deportivos
	\item \textbf{Gestión de torneos}: Módulo que suple las necesidades de un organizador de eventos cuando éste desea trabajar con la información de un torneo deportivo
	\item \textbf{Gestión de eventos deportivos*}: Módulo que brinda funcionalidades de gestión de eventos deportivos
	\item \textbf{Gestión de patrocinadores}: Módulo que brinda funcionalidades al patrocinador que lo use, para patrocinar y controlar patrocinios, así como para seguir actividad de posibles patrocinados.
	\item \textbf{Gestión de organizaciones}: Módulo que ofrece funciones de gestión de organizaciones
	\item \textbf{Gestión de self-expression}: Módulo que es utilizado para el manejo de contenido propio generado por un actor en la red social o un evento que uno o más actores manejen en la red social
	\item \textbf{Gestión del conocimiento}: Módulo que gestiona artículos/post relacionados con tips en campos de salud y deportivos en si
	\item \textbf{Gestión de geolocalización*}: Módulo que ayuda al control de todas las funcionalidades de geolocalización
	\item \textbf{Gestión de estadísticas}: Módulo que permite la generación y visualización de estadísticas diversas acerca de deportistas, organizaciones, ubicaciones o cualquier otro concepto que maneje estadísticas en el SNS
	\item \textbf{Gestión de entrenadores}: Módulo que permite la gestión de opcionalidades ofrecidas a entrenadores deportivos, tal como el seguimiento de entrenados o la asignación de planes deportivos a los mismos
	\item \textbf{Gestión de canales de difusión}: Módulo que refiere a todo lo relacionado con noticias deportivas
	\item \textbf{Gestión de grupos deportivos}:  Módulo de gestión de funcionalidades ofrecidas a grupos deportivos informales (diferentes a los equipos deportivos, caso especial de los grupos deportivos)
\end{enumerate}

Para el desarrollo del prototipo, los autores se concentran en los módulos marcados con * en la anterior lista. Para saber los criterios por los cuales se han escogido éstos módulos, el lector puede dirigirse a \ref{chap:alcances_limitaciones}.

La lista de requerimientos puede encontrarse en \ref{app:req_funcionales}.