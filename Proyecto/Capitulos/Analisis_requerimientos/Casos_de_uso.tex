Luego de haber consignado los requerimientos funcionales en la sección \ref{subsec:requerimientos_funcionales}, se retrataron estos en casos de uso. En las figuras \ref{fig:cu1} a \ref{fig:cu70} se puede observar gráficamente la configuración de los casos de uso. La descripción de los casos de uso se hace en tres fases: La primera fase, describe aspectos no-dinámicos del caso de uso; la segunda fase comprende un flujo de hechos para el caso de uso; la tercera fase comprende la descripción de las excepciones que podría causar el caso de uso, haciendo referencia también a cual flujo es afectado con la excepción descrita.

Los diagramas estarán dividos por cada módulo de gestión identificado basado cada uno en la especificación de requerimientos.

\begin{table}[!htb]
	\caption{CU001-NOMBRE: Descripción}
	\label{tab:cu001_desc}
	\begin{center}
		\resizebox{11cm}{!}{
		\begin{tabular}{|p{4cm}|p{7cm}|}
			\hline
			\multicolumn{2}{|c|}{Descripción de caso de uso} \\
			\hline
			Nombre & \\
			\hline
			Identificador & \\
			\hline
			Descripción & \\
			\hline
			Actor & \\
			\hline
			Disparador & \\
			\hline
			Inclusiones & \\
			\hline
			Puntos de extensión & \\
			\hline
			Precondiciones & \\
			\hline
			Postcondiciones & \\
			\hline
			Notas & \\
			\hline
		\end{tabular}
		} \\
		\textbf{Fuente}: Autores
	\end{center}
\end{table}

\begin{table}[!htb]
	\caption{CU001-NOMBRE: Flujos de hechos}
	\label{tab:cu001_flujo}
	\begin{center}
		\resizebox{13cm}{!}{
		\begin{tabular}{|p{2cm}||p{4cm}|p{7cm}|}
			\hline
			\multicolumn{3}{|c|}{Detalle de flujo de hechos de caso de uso}
			Nombre & \multicolumn{2}{|p|}{}
			Paso & Acción del actor & Respuesta del sistema \\
			\hline
			 & & \\
			\hline
		\end{tabular}
		} \\
		\textbf{Fuente}: Autores
	\end{center}
\end{table}

\subsection{Módulo de administración de entrenadores}

A continuación se muestran los casos de uso del módulo de administración de entrenadores

\begin{table}[!htb]
	\caption{CU001-Gestión de entrenadores: Descripción}
	\label{tab:cu001_desc}
	\begin{center}
		\resizebox{11cm}{!}{
		\begin{tabular}{|p{4cm}|p{7cm}|}
			\hline
			\multicolumn{2}{|c|}{Descripción de caso de uso} \\
			\hline
			Nombre & Gestión de entrenadores \\
			\hline
			Identificador & CU001 \\
			\hline
			Descripción & Gestiona las opciones dadas a los entrenadores deportivos en el SNS \\
			\hline
			Actor & !*!*!*!*!*! \\
			\hline
			Disparador & Actor entrenador elige la opción de gestión de entrenadores \\
			\hline
			Inclusiones & \\
			\hline
			Puntos de extensión & \\
			\hline
			Precondiciones &  \\
			\hline
			Postcondiciones & \\
			\hline
			Notas & \\
			\hline
		\end{tabular}
		} \\
		\textbf{Fuente}: Autores
	\end{center}
\end{table}

\begin{table}[!htb]
	\caption{CU001-Gestión de entrenadores: Flujos de hechos}
	\label{tab:cu001_flujo}
	\begin{center}
		\resizebox{13cm}{!}{
		\begin{tabular}{|p{2cm}||p{4cm}|p{7cm}|}
			\hline
			\multicolumn{3}{|c|}{Detalle de flujo de hechos de caso de uso}
			Nombre & \multicolumn{2}{|p|}{}
			Paso & Acción del actor & Respuesta del sistema \\
			\hline
			 & & \\
			\hline
		\end{tabular}
		} \\
		\textbf{Fuente}: Autores
	\end{center}
\end{table}

\subsection{Módulo de administración de eventos deportivos}

A continuación se muestran los casos de uso del módulo de administración de eventos deportivos

\subsection{Módulo de administración de torneos}

A continuación se muestran los casos de uso del módulo de administración de torneos

\subsection{Módulo de administración de equipos}

A continuación se muestran los casos de uso del módulo de administración de equipos

\subsection{Módulo de estadísticas}

A continuación se muestran los casos de uso del módulo de estadísticas

\subsection{Módulo de gestión de sponsors}

A continuación se muestran los casos de uso del módulo de gestión de sponsors

