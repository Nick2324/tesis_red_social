Luego de haber consignado los requerimientos funcionales en la sección \ref{subsec:requerimientos_funcionales} y de haber formado, con ellos, la arquitectura tratada en el capítulo \ref{chap:arquitectura}, se retrató cada funcionalidad encontrada y que sería, a final de cuentas, soportada por el SNS en los casos de uso a continuación. En las figuras \ref{fig:cu1} a \ref{fig:cu70} se puede observar gráficamente la configuración de los casos de uso. La descripción de los casos de uso se hace en tres fases: La primera fase, describe aspectos no-dinámicos del caso de uso; la segunda fase comprende un flujo de hechos para el caso de uso; la tercera fase comprende la descripción de las excepciones que podría causar el caso de uso, haciendo referencia también a cual flujo es afectado con la excepción descrita.

Los diagramas estarán dividos por cada módulo de gestión identificado basado cada uno en la especificación de requerimientos.

\begin{table}[!htb]
	\caption{CU001-NOMBRE: Descripción}
	\label{tab:cu001_desc}
	\begin{center}
		\resizebox{15cm}{!}{
		\begin{tabular}{|p{4cm}|p{11cm}|}
			\hline
			\multicolumn{2}{|c|}{Descripción de caso de uso} \\
			\hline
			Nombre & \\
			\hline
			Identificador & \\
			\hline
			Descripción & \\
			\hline
			Actor & \\
			\hline
			Disparador & \\
			\hline
			Inclusiones & \\
			\hline
			Puntos de extensión & \\
			\hline
			Precondiciones & \\
			\hline
			Postcondiciones & \\
			\hline
			Notas & \\
			\hline
		\end{tabular}
		} \\
		\textbf{Fuente}: Autores
	\end{center}
\end{table}

\begin{table}[!htb]
	\caption{CU001-NOMBRE: Flujos de hechos}
	\label{tab:cu001_flujo}
	\begin{center}
		\resizebox{15cm}{!}{
		\begin{tabular}{|p{1.5cm}|p{6cm}|p{6.5cm}|}
			\hline
			\multicolumn{3}{|c|}{Detalle de flujo de hechos de caso de uso} \\
			\hline
			Nombre & \multicolumn{2}{|c|}{Nombre del flujo} \\
			\hline
			Paso & Acción del actor & Respuesta del sistema \\
			\hline
			 & & \\
			\hline
		\end{tabular}
		} \\
		\textbf{Fuente}: Autores
	\end{center}
\end{table}

\subsection{Módulo de administración de entrenadores}

A continuación se muestran los casos de uso del módulo de administración de entrenadores

\begin{table}[!htb]
	\caption{CU001-Gestión de entrenadores: Descripción}
	\label{tab:cu001_desc}
	\begin{center}
		\resizebox{15cm}{!}{
		\begin{tabular}{|p{4cm}|p{11cm}|}
			\hline
			\multicolumn{2}{|c|}{Descripción de caso de uso} \\
			\hline
			Nombre & Gestión de entrenadores \\
			\hline
			Identificador & CU001 \\
			\hline
			Descripción & Gestiona las opciones dadas a los entrenadores deportivos en el SNS \\
			\hline
			Actor &
			\begin{itemize}
				\item Entrenador
				\item Jugador
				\item Organización
			\end{itemize}			 \\
			\hline
			Disparador & Actor con rol de entrenador deportivo elige la opción de gestión de entrenadores \\
			\hline
			Inclusiones & N/A \\
			\hline
			Puntos de extensión & N/A \\
			\hline
			Precondiciones &  
			\begin{itemize}
				\item La aplicación ha sido cargada por un actor con rol de entrenador deportivo
			\end{itemize}
			\\
			\hline
			Postcondiciones & 
			\begin{itemize}
				\item El usuario puede ver el menú de gestión de entrenadores
			\end{itemize}
			\\
			\hline
			Notas & 
			\begin{itemize}
				\item Generalización de:
				\begin{itemize}
					\item Convertirse en entrenador de jugador/equipo
					\item Dejar de entrenar a jugador/equipo
					\item Llevar seguimiento de entrenamientos
					\item Promocionar servicios de entrenamiento
				\end{itemize}
			\end{itemize}
			\\
			\hline
		\end{tabular}
		} \\
		\textbf{Fuente}: Autores
	\end{center}
\end{table}

\begin{table}[!htb]
	\caption{CU001-Gestión de entrenadores: Flujos de hechos}
	\label{tab:cu001_flujo}
	\begin{center}
		\resizebox{15cm}{!}{
		\begin{tabular}{|p{1.5cm}|p{6cm}|p{6.5cm}|}
			\hline
			\multicolumn{3}{|c|}{Detalle de flujo de hechos de caso de uso} \\
			\hline
			Nombre & \multicolumn{2}{|c|}{Nombre del flujo} \\
			\hline
			Paso & Acción del actor & Respuesta del sistema \\
			\hline
			 & & \\
			\hline
		\end{tabular}
		} \\
		\textbf{Fuente}: Autores
	\end{center}
\end{table}

\begin{table}[!htb]
	\caption{CU002-Convertirse en entrenador de jugador/equipo: Descripción}
	\label{tab:cu002_desc}
	\begin{center}
		\resizebox{15cm}{!}{
		\begin{tabular}{|p{4cm}|p{11cm}|}
			\hline
			\multicolumn{2}{|c|}{Descripción de caso de uso} \\
			\hline
			Nombre & Convertirse en entrenador de jugador/equipo \\
			\hline
			Identificador & CU002 \\
			\hline
			Descripción & Ayuda a pedir/aceptar la petición de ser entrenador de un jugador o equipo en el SNS \\
			\hline
			Actor &
			\begin{itemize}
				\item Entrenador
				\item Jugador
				\item Organización
			\end{itemize}			 \\
			\hline
			Disparador & Actor con rol de entrenador deportivo elige la opción de entrenar a equipo o jugador \\
			\hline
			Inclusiones & N/A \\
			\hline
			Puntos de extensión & N/A \\
			\hline
			Precondiciones &  
			\begin{itemize}
				\item La aplicación ha sido cargada por un actor con rol de entrenador deportivo
			\end{itemize}
			\\
			\hline
			Postcondiciones & 
			\begin{itemize}
				\item El usuario ha elegido o no ser entrenador de un jugador o equipo deportivo.
				\item El usuario sigue gestionando su rol como entrenador de otros usuarios de la red social.
				\item El usuario regresa al menú de gestión de entrenamiento deportivo
			\end{itemize}
			\\
			\hline
			Notas & N/A
			\\
			\hline
		\end{tabular}
		} \\
		\textbf{Fuente}: Autores
	\end{center}
\end{table}

\begin{table}[!htb]
	\caption{CU002-Gestión de entrenadores: Flujos de hechos}
	\label{tab:cu002_flujo}
	\begin{center}
		\resizebox{15cm}{!}{
		\begin{tabular}{|p{1.5cm}|p{6cm}|p{6.5cm}|}
			\hline
			\multicolumn{3}{|c|}{Detalle de flujo de hechos de caso de uso} \\
			\hline
			Nombre & \multicolumn{2}{|c|}{Nombre del flujo} \\
			\hline
			Paso & Acción del actor & Respuesta del sistema \\
			\hline
			 & & \\
			\hline
		\end{tabular}
		} \\
		\textbf{Fuente}: Autores
	\end{center}
\end{table}

\begin{table}[!htb]
	\caption{CU003-Dejar de entrenar a jugador/equipo: Descripción}
	\label{tab:cu003_desc}
	\begin{center}
		\resizebox{15cm}{!}{
		\begin{tabular}{|p{4cm}|p{11cm}|}
			\hline
			\multicolumn{2}{|c|}{Descripción de caso de uso} \\
			\hline
			Nombre & Dejar de entrenar a jugador/equipo \\
			\hline
			Identificador & CU003 \\
			\hline
			Descripción & Permite dejar de ser entrenador de un jugador o equipo en el SNS \\
			\hline
			Actor &
			\begin{itemize}
				\item Entrenador
				\item Jugador
				\item Organización
			\end{itemize}			 
			\\
			\hline
			Disparador & Actor con rol de entrenador deportivo elige la opción de dejar de ser entrenador de equipo o jugador \\
			\hline
			Inclusiones & N/A \\
			\hline
			Puntos de extensión & N/A \\
			\hline
			Precondiciones &  
			\begin{itemize}
				\item La aplicación ha sido cargada por un actor con rol de entrenador deportivo
			\end{itemize}
			\\
			\hline
			Postcondiciones & 
			\begin{itemize}
				\item El usuario ha elegido o no dejar de ser entrenador de un jugador o equipo deportivo.
				\item El usuario sigue gestionando su rol como entrenador de otros usuarios de la red social.
				\item El usuario regresa al menú de gestión de entrenamiento deportivo.
			\end{itemize}
			\\
			\hline
			Notas & N/A
			\\
			\hline
		\end{tabular}
		} \\
		\textbf{Fuente}: Autores
	\end{center}
\end{table}

\begin{table}[!htb]
	\caption{CU003-Dejar de entrenar a jugador/equipo: Flujos de hechos}
	\label{tab:cu003_flujo}
	\begin{center}
		\resizebox{15cm}{!}{
		\begin{tabular}{|p{1.5cm}|p{6cm}|p{6.5cm}|}
			\hline
			\multicolumn{3}{|c|}{Detalle de flujo de hechos de caso de uso} \\
			\hline
			Nombre & \multicolumn{2}{|c|}{Nombre del flujo} \\
			\hline
			Paso & Acción del actor & Respuesta del sistema \\
			\hline
			 & & \\
			\hline
		\end{tabular}
		} \\
		\textbf{Fuente}: Autores
	\end{center}
\end{table}


\begin{table}[!htb]
	\caption{CU004-Llevar seguimiento de entrenamientos: Descripción}
	\label{tab:cu004_desc}
	\begin{center}
		\resizebox{15cm}{!}{
		\begin{tabular}{|p{4cm}|p{11cm}|}
			\hline
			\multicolumn{2}{|c|}{Descripción de caso de uso} \\
			\hline
			Nombre & Llevar seguimiento de entrenamientos \\
			\hline
			Identificador & CU004 \\
			\hline
			Descripción & Permite administrar los planes de entrenamientos dados a los entrenados en cuanto a su cumplimiento, a su actualización y en cuanto a su creación \\
			\hline
			Actor &
			\begin{itemize}
				\item Entrenador
				\item Jugador
				\item Organización
			\end{itemize}			 
			\\
			\hline
			Disparador & Actor con rol de entrenador deportivo elige la opción de seguimiento de planes de entrenamiento \\
			\hline
			Inclusiones & N/A \\
			\hline
			Puntos de extensión & N/A \\
			\hline
			Precondiciones &  
			\begin{itemize}
				\item La aplicación ha sido cargada por un actor con rol de entrenador deportivo
			\end{itemize}
			\\
			\hline
			Postcondiciones & 
			\begin{itemize}
				\item El usuario se encuentra en la administración de planes deportivos
			\end{itemize}
			\\
			\hline
			Notas & N/A
			\\
			\hline
		\end{tabular}
		} \\
		\textbf{Fuente}: Autores
	\end{center}
\end{table}

\begin{table}[!htb]
	\caption{CU004-Llevar seguimiento de entrenamientos: Flujos de hechos}
	\label{tab:cu004_flujo}
	\begin{center}
		\resizebox{15cm}{!}{
		\begin{tabular}{|p{1.5cm}|p{6cm}|p{6.5cm}|}
			\hline
			\multicolumn{3}{|c|}{Detalle de flujo de hechos de caso de uso} \\
			\hline
			Nombre & \multicolumn{2}{|c|}{Nombre del flujo} \\
			\hline
			Paso & Acción del actor & Respuesta del sistema \\
			\hline
			 & & \\
			\hline
		\end{tabular}
		} \\
		\textbf{Fuente}: Autores
	\end{center}
\end{table}

\begin{table}[!htb]
	\caption{CU005-Promocionar servicios de entrenamiento: Descripción}
	\label{tab:cu005_desc}
	\begin{center}
		\resizebox{15cm}{!}{
		\begin{tabular}{|p{4cm}|p{11cm}|}
			\hline
			\multicolumn{2}{|c|}{Descripción de caso de uso} \\
			\hline
			Nombre & Promocionar servicios de entrenamiento \\
			\hline
			Identificador & CU005 \\
			\hline
			Descripción & Permite la promoción de servicios de entrenamiento hacia los equipos/jugadores posiblemente interesados, así como también sobre las noticias nuevas en la red social \\
			\hline
			Actor &
			\begin{itemize}
				\item Entrenador
				\item Jugador
				\item Organización
			\end{itemize}			 
			\\
			\hline
			Disparador & Actor con rol de entrenador deportivo elige la opción de promoción de sus servicios de entrenamiento deportivo \\
			\hline
			Inclusiones & N/A \\
			\hline
			Puntos de extensión & N/A \\
			\hline
			Precondiciones &  
			\begin{itemize}
				\item La aplicación ha sido cargada por un actor con rol de entrenador deportivo
			\end{itemize}
			\\
			\hline
			Postcondiciones & 
			\begin{itemize}
				\item El usuario ha promocionado sus servicios y decide seguir promocionando.
				\item El usuario vuelve a la opción de gestión de entrenamiento deportivo
			\end{itemize}
			\\
			\hline
			Notas & N/A
			\\
			\hline
		\end{tabular}
		} \\
		\textbf{Fuente}: Autores
	\end{center}
\end{table}

\begin{table}[!htb]
	\caption{CU005-Promocionar servicios de entrenamiento: Flujos de hechos}
	\label{tab:cu005_flujo}
	\begin{center}
		\resizebox{15cm}{!}{
		\begin{tabular}{|p{1.5cm}|p{6cm}|p{6.5cm}|}
			\hline
			\multicolumn{3}{|c|}{Detalle de flujo de hechos de caso de uso} \\
			\hline
			Nombre & \multicolumn{2}{|c|}{Nombre del flujo} \\
			\hline
			Paso & Acción del actor & Respuesta del sistema \\
			\hline
			 & & \\
			\hline
		\end{tabular}
		} \\
		\textbf{Fuente}: Autores
	\end{center}
\end{table}

\subsection{Módulo de administración de eventos deportivos}

A continuación se muestran los casos de uso del módulo de administración de eventos deportivos

\begin{table}[!htb]
	\caption{CU006-Administrar eventos deportivos: Descripción}
	\label{tab:cu006_desc}
	\begin{center}
		\resizebox{15cm}{!}{
		\begin{tabular}{|p{4cm}|p{11cm}|}
			\hline
			\multicolumn{2}{|c|}{Descripción de caso de uso} \\
			\hline
			Nombre & Administrar eventos deportivos \\
			\hline
			Identificador & CU006 \\
			\hline
			Descripción & Permite la administración de eventos deportivos realizados \\
			\hline
			Actor & Todo actor de la red social	 
			\\
			\hline
			Disparador & Se elige la opción de administrar un evento deportivo \\
			\hline
			Inclusiones & N/A \\
			\hline
			Puntos de extensión & N/A \\
			\hline
			Precondiciones &  
			\begin{itemize}
				\item La aplicación ha sido cargada por un actor con rol de organizador de eventos deportivos
			\end{itemize}
			\\
			\hline
			Postcondiciones & 
			\begin{itemize}
				\item El usuario está en la pantalla de administración de eventos deportivos
			\end{itemize}
			\\
			\hline
			Notas & 
			\begin{itemize}
				\item Generalización de:
				\begin{itemize}
					\item Crear evento deportivo
					\item Cancelar evento deportivo
					\item Actualizar información de evento deportivo
					\item Administrar involucrados
				\end{itemize}
			\end{itemize}
			\\
			\hline
		\end{tabular}
		} \\
		\textbf{Fuente}: Autores
	\end{center}
\end{table}

\begin{table}[!htb]
	\caption{CU006-Administrar eventos deportivos: Flujos de hechos}
	\label{tab:cu006_flujo}
	\begin{center}
		\resizebox{15cm}{!}{
		\begin{tabular}{|p{1.5cm}|p{6cm}|p{6.5cm}|}
			\hline
			\multicolumn{3}{|c|}{Detalle de flujo de hechos de caso de uso} \\
			\hline
			Nombre & \multicolumn{2}{|c|}{Nombre del flujo} \\
			\hline
			Paso & Acción del actor & Respuesta del sistema \\
			\hline
			 & & \\
			\hline
		\end{tabular}
		} \\
		\textbf{Fuente}: Autores
	\end{center}
\end{table}

\begin{table}[!htb]
	\caption{CU007-Crear evento deportivo: Descripción}
	\label{tab:cu007_desc}
	\begin{center}
		\resizebox{15cm}{!}{
		\begin{tabular}{|p{4cm}|p{11cm}|}
			\hline
			\multicolumn{2}{|c|}{Descripción de caso de uso} \\
			\hline
			Nombre & Crear evento deportivo \\
			\hline
			Identificador & CU007 \\
			\hline
			Descripción & Permite la creación de un evento deportivo en la red social deportiva \\
			\hline
			Actor & Todo actor de la red social	 
			\\
			\hline
			Disparador & Se elige la opción de crear un evento deportivo \\
			\hline
			Inclusiones & N/A \\
			\hline
			Puntos de extensión & N/A \\
			\hline
			Precondiciones &  
			\begin{itemize}
				\item La aplicación ha sido cargada por un actor con rol de organizador de eventos deportivos
				\item Se ha elegido la opción de administrar eventos deportivos
			\end{itemize}
			\\
			\hline
			Postcondiciones & 
			\begin{itemize}
				\item El usuario crea un evento deportivo
				\item El usuario está en la pantalla de administración de eventos deportivos
			\end{itemize}
			\\
			\hline
			Notas & N/A
			\\
			\hline
		\end{tabular}
		} \\
		\textbf{Fuente}: Autores
	\end{center}
\end{table}

\begin{table}[!htb]
	\caption{CU007-Crear evento deportivo: Flujos de hechos}
	\label{tab:cu007_flujo}
	\begin{center}
		\resizebox{15cm}{!}{
		\begin{tabular}{|p{1.5cm}|p{6cm}|p{6.5cm}|}
			\hline
			\multicolumn{3}{|c|}{Detalle de flujo de hechos de caso de uso} \\
			\hline
			Nombre & \multicolumn{2}{|c|}{Nombre del flujo} \\
			\hline
			Paso & Acción del actor & Respuesta del sistema \\
			\hline
			 & & \\
			\hline
		\end{tabular}
		} \\
		\textbf{Fuente}: Autores
	\end{center}
\end{table}

\begin{table}[!htb]
	\caption{CU008-Cancelar evento deportivo: Descripción}
	\label{tab:cu008_desc}
	\begin{center}
		\resizebox{15cm}{!}{
		\begin{tabular}{|p{4cm}|p{11cm}|}
			\hline
			\multicolumn{2}{|c|}{Descripción de caso de uso} \\
			\hline
			Nombre & Cancelar evento deportivo \\
			\hline
			Identificador & CU008 \\
			\hline
			Descripción & Permite la cancelación de un evento deportivo en la red social deportiva \\
			\hline
			Actor & Todo actor de la red social	 
			\\
			\hline
			Disparador & Se elige la opción de cancelar un evento deportivo \\
			\hline
			Inclusiones & N/A \\
			\hline
			Puntos de extensión & N/A \\
			\hline
			Precondiciones &  
			\begin{itemize}
				\item La aplicación ha sido cargada por un actor con rol de organizador de eventos deportivos
				\item Se ha elegido la opción de administrar eventos deportivos
			\end{itemize}
			\\
			\hline
			Postcondiciones & 
			\begin{itemize}
				\item El usuario cancela un evento deportivo
				\item El usuario está en la pantalla de administración de eventos deportivos
			\end{itemize}
			\\
			\hline
			Notas & N/A
			\\
			\hline
		\end{tabular}
		} \\
		\textbf{Fuente}: Autores
	\end{center}
\end{table}

\begin{table}[!htb]
	\caption{CU008-Cancelar evento deportivo: Flujos de hechos}
	\label{tab:cu008_flujo}
	\begin{center}
		\resizebox{15cm}{!}{
		\begin{tabular}{|p{1.5cm}|p{6cm}|p{6.5cm}|}
			\hline
			\multicolumn{3}{|c|}{Detalle de flujo de hechos de caso de uso} \\
			\hline
			Nombre & \multicolumn{2}{|c|}{Nombre del flujo} \\
			\hline
			Paso & Acción del actor & Respuesta del sistema \\
			\hline
			 & & \\
			\hline
		\end{tabular}
		} \\
		\textbf{Fuente}: Autores
	\end{center}
\end{table}

\begin{table}[!htb]
	\caption{CU009-Actualizar información de evento deportivo: Descripción}
	\label{tab:cu009_desc}
	\begin{center}
		\resizebox{15cm}{!}{
		\begin{tabular}{|p{4cm}|p{11cm}|}
			\hline
			\multicolumn{2}{|c|}{Descripción de caso de uso} \\
			\hline
			Nombre & Actualizar información de evento deportivo \\
			\hline
			Identificador & CU009 \\
			\hline
			Descripción & Permite la actualizar la información de un evento deportivo en la red social deportiva \\
			\hline
			Actor & Todo actor de la red social	 
			\\
			\hline
			Disparador & Se elige la opción de actualizar la información de un evento deportivo \\
			\hline
			Inclusiones & N/A \\
			\hline
			Puntos de extensión & N/A \\
			\hline
			Precondiciones &  
			\begin{itemize}
				\item La aplicación ha sido cargada por un actor con rol de organizador de eventos deportivos
				\item Se ha elegido la opción de administrar eventos deportivos
			\end{itemize}
			\\
			\hline
			Postcondiciones & 
			\begin{itemize}
				\item El usuario cambia la información de un evento deportivo
				\item El usuario está en la pantalla de administración de eventos deportivos
			\end{itemize}
			\\
			\hline
			Notas & N/A
			\\
			\hline
		\end{tabular}
		} \\
		\textbf{Fuente}: Autores
	\end{center}
\end{table}

\begin{table}[!htb]
	\caption{CU009-Actualizar información de evento deportivo: Flujos de hechos}
	\label{tab:cu009_flujo}
	\begin{center}
		\resizebox{15cm}{!}{
		\begin{tabular}{|p{1.5cm}|p{6cm}|p{6.5cm}|}
			\hline
			\multicolumn{3}{|c|}{Detalle de flujo de hechos de caso de uso} \\
			\hline
			Nombre & \multicolumn{2}{|c|}{Nombre del flujo} \\
			\hline
			Paso & Acción del actor & Respuesta del sistema \\
			\hline
			 & & \\
			\hline
		\end{tabular}
		} \\
		\textbf{Fuente}: Autores
	\end{center}
\end{table}

\begin{table}[!htb]
	\caption{CU010-Administrar involucrados: Descripción}
	\label{tab:cu010_desc}
	\begin{center}
		\resizebox{15cm}{!}{
		\begin{tabular}{|p{4cm}|p{11cm}|}
			\hline
			\multicolumn{2}{|c|}{Descripción de caso de uso} \\
			\hline
			Nombre & Administrar involucrados \\
			\hline
			Identificador & CU010 \\
			\hline
			Descripción & Permite la administración de los stakeholder asociados a alguno de los eventos creados \\
			\hline
			Actor & Todo actor de la red social	 
			\\
			\hline
			Disparador & Se elige la opción de administración de involucrados en el evento \\
			\hline
			Inclusiones & N/A \\
			\hline
			Puntos de extensión & N/A \\
			\hline
			Precondiciones &  
			\begin{itemize}
				\item La aplicación ha sido cargada por un actor con rol de organizador de eventos deportivos
				\item Se ha elegido un evento al cual administrar involucrados
			\end{itemize}
			\\
			\hline
			Postcondiciones & 
			\begin{itemize}
				\item El usuario está en la pantalla de administración de involucrados en un evento deportivo
			\end{itemize}
			\\
			\hline
			Notas & 
			\begin{itemize}
				\item Generalización de:
				\begin{itemize}
					\item Añadir colaborador
					\item Retirar colaborador
					\item Añadir participante
					\item Retirar participante
					\item Añadir espectador
					\item Retirar espectador
				\end{itemize}
			\end{itemize}
			\\
			\hline
		\end{tabular}
		} \\
		\textbf{Fuente}: Autores
	\end{center}
\end{table}

\begin{table}[!htb]
	\caption{CU010-Administrar involucrados: Flujos de hechos}
	\label{tab:cu010_flujo}
	\begin{center}
		\resizebox{15cm}{!}{
		\begin{tabular}{|p{1.5cm}|p{6cm}|p{6.5cm}|}
			\hline
			\multicolumn{3}{|c|}{Detalle de flujo de hechos de caso de uso} \\
			\hline
			Nombre & \multicolumn{2}{|c|}{Nombre del flujo} \\
			\hline
			Paso & Acción del actor & Respuesta del sistema \\
			\hline
			 & & \\
			\hline
		\end{tabular}
		} \\
		\textbf{Fuente}: Autores
	\end{center}
\end{table}

\begin{table}[!htb]
	\caption{CU011-Añadir colaborador: Descripción}
	\label{tab:cu011_desc}
	\begin{center}
		\resizebox{15cm}{!}{
		\begin{tabular}{|p{4cm}|p{11cm}|}
			\hline
			\multicolumn{2}{|c|}{Descripción de caso de uso} \\
			\hline
			Nombre & Añadir colaborador\\
			\hline
			Identificador & CU011 \\
			\hline
			Descripción & Permite añadir un colaborador en la organización de un evento deportivo en el evento escogido \\
			\hline
			Actor & Todo actor de la red social	 
			\\
			\hline
			Disparador & Se elige la opción de añadir colaborador \\
			\hline
			Inclusiones & N/A \\
			\hline
			Puntos de extensión & N/A \\
			\hline
			Precondiciones &  
			\begin{itemize}
				\item La aplicación ha sido cargada por un actor con rol de organizador de eventos deportivos
				\item Solo el creador/administrador principal del evento puede activar esta opcionalidad
				\item Se ha elegido un evento al cual administrar involucrados
			\end{itemize}
			\\
			\hline
			Postcondiciones & 
			\begin{itemize}
				\item El usuario añade uno o varios colaboradores a la organización del evento deportivo
				\item El usuario está en la pantalla de administración de involucrados en un evento deportivo
			\end{itemize}
			\\
			\hline
			Notas & N/A
			\\
			\hline
		\end{tabular}
		} \\
		\textbf{Fuente}: Autores
	\end{center}
\end{table}

\begin{table}[!htb]
	\caption{CU011-Añadir colaborador: Flujos de hechos}
	\label{tab:cu011_flujo}
	\begin{center}
		\resizebox{15cm}{!}{
		\begin{tabular}{|p{1.5cm}|p{6cm}|p{6.5cm}|}
			\hline
			\multicolumn{3}{|c|}{Detalle de flujo de hechos de caso de uso} \\
			\hline
			Nombre & \multicolumn{2}{|c|}{Nombre del flujo} \\
			\hline
			Paso & Acción del actor & Respuesta del sistema \\
			\hline
			 & & \\
			\hline
		\end{tabular}
		} \\
		\textbf{Fuente}: Autores
	\end{center}
\end{table}

\begin{table}[!htb]
	\caption{CU012-Retirar colaborador: Descripción}
	\label{tab:cu012_desc}
	\begin{center}
		\resizebox{15cm}{!}{
		\begin{tabular}{|p{4cm}|p{11cm}|}
			\hline
			\multicolumn{2}{|c|}{Descripción de caso de uso} \\
			\hline
			Nombre & Retirar colaborador \\
			\hline
			Identificador & CU012 \\
			\hline
			Descripción & Permite retirar un organizador del evento deportivo \\
			\hline
			Actor & Todo actor de la red social	 
			\\
			\hline
			Disparador & Se elige la opción de retirar colaborador \\
			\hline
			Inclusiones & N/A \\
			\hline
			Puntos de extensión & N/A \\
			\hline
			Precondiciones &  
			\begin{itemize}
				\item La aplicación ha sido cargada por un actor con rol de organizador de eventos deportivos
				\item Solo el creador/administrador principal del evento puede activar esta opcionalidad
				\item Se ha elegido un evento al cual administrar involucrados
			\end{itemize}
			\\
			\hline
			Postcondiciones & 
			\begin{itemize}
				\item El usuario retira uno o varios colaboradores de la organización del evento deportivo
				\item El usuario está en la pantalla de administración de involucrados en un evento deportivo
			\end{itemize}
			\\
			\hline
			Notas & N/A
			\\
			\hline
		\end{tabular}
		} \\
		\textbf{Fuente}: Autores
	\end{center}
\end{table}

\begin{table}[!htb]
	\caption{CU012-Retirar colaborador: Flujos de hechos}
	\label{tab:cu012_flujo}
	\begin{center}
		\resizebox{15cm}{!}{
		\begin{tabular}{|p{1.5cm}|p{6cm}|p{6.5cm}|}
			\hline
			\multicolumn{3}{|c|}{Detalle de flujo de hechos de caso de uso} \\
			\hline
			Nombre & \multicolumn{2}{|c|}{Nombre del flujo} \\
			\hline
			Paso & Acción del actor & Respuesta del sistema \\
			\hline
			 & & \\
			\hline
		\end{tabular}
		} \\
		\textbf{Fuente}: Autores
	\end{center}
\end{table}

\begin{table}[!htb]
	\caption{CU013-Añadir participante: Descripción}
	\label{tab:cu013_desc}
	\begin{center}
		\resizebox{15cm}{!}{
		\begin{tabular}{|p{4cm}|p{11cm}|}
			\hline
			\multicolumn{2}{|c|}{Descripción de caso de uso} \\
			\hline
			Nombre & Añadir participante \\
			\hline
			Identificador & CU013 \\
			\hline
			Descripción & Permite añadir un participante deportivo (equipo o jugador) al evento deportivo \\
			\hline
			Actor & Todo actor de la red social	 
			\\
			\hline
			Disparador & Se elige la opción de añadir un participante deportivo al evento \\
			\hline
			Inclusiones & N/A \\
			\hline
			Puntos de extensión & N/A \\
			\hline
			Precondiciones &  
			\begin{itemize}
				\item La aplicación ha sido cargada por un actor con rol de organizador de eventos deportivos
				\item Se ha elegido un evento al cual administrar involucrados
			\end{itemize}
			\\
			\hline
			Postcondiciones & 
			\begin{itemize}
				\item El usuario añade uno o varios participantes deportivos al evento deportivo
				\item El usuario está en la pantalla de administración de involucrados en un evento deportivo
			\end{itemize}
			\\
			\hline
			Notas & N/A
			\\
			\hline
		\end{tabular}
		} \\
		\textbf{Fuente}: Autores
	\end{center}
\end{table}

\begin{table}[!htb]
	\caption{CU013-Añadir participante: Flujos de hechos}
	\label{tab:cu013_flujo}
	\begin{center}
		\resizebox{15cm}{!}{
		\begin{tabular}{|p{1.5cm}|p{6cm}|p{6.5cm}|}
			\hline
			\multicolumn{3}{|c|}{Detalle de flujo de hechos de caso de uso} \\
			\hline
			Nombre & \multicolumn{2}{|c|}{Nombre del flujo} \\
			\hline
			Paso & Acción del actor & Respuesta del sistema \\
			\hline
			 & & \\
			\hline
		\end{tabular}
		} \\
		\textbf{Fuente}: Autores
	\end{center}
\end{table}

\begin{table}[!htb]
	\caption{CU014-Retirar participante: Descripción}
	\label{tab:cu014_desc}
	\begin{center}
		\resizebox{15cm}{!}{
		\begin{tabular}{|p{4cm}|p{11cm}|}
			\hline
			\multicolumn{2}{|c|}{Descripción de caso de uso} \\
			\hline
			Nombre & Retirar participante \\
			\hline
			Identificador & CU014 \\
			\hline
			Descripción & Permite retirar un participante deportivo (equipo o jugador) del evento deportivo \\
			\hline
			Actor & Todo actor de la red social	 
			\\
			\hline
			Disparador & Se elige la opción de retirar un participante deportivo al evento \\
			\hline
			Inclusiones & N/A \\
			\hline
			Puntos de extensión & N/A \\
			\hline
			Precondiciones &  
			\begin{itemize}
				\item La aplicación ha sido cargada por un actor con rol de organizador de eventos deportivos
				\item Se ha elegido un evento al cual administrar involucrados
			\end{itemize}
			\\
			\hline
			Postcondiciones & 
			\begin{itemize}
				\item El usuario retira un participante deportivo del evento deportivo
				\item El usuario está en la pantalla de administración de involucrados en un evento deportivo
			\end{itemize}
			\\
			\hline
			Notas & N/A
			\\
			\hline
		\end{tabular}
		} \\
		\textbf{Fuente}: Autores
	\end{center}
\end{table}

\begin{table}[!htb]
	\caption{CU014-Retirar participante: Flujos de hechos}
	\label{tab:cu014_flujo}
	\begin{center}
		\resizebox{15cm}{!}{
		\begin{tabular}{|p{1.5cm}|p{6cm}|p{6.5cm}|}
			\hline
			\multicolumn{3}{|c|}{Detalle de flujo de hechos de caso de uso} \\
			\hline
			Nombre & \multicolumn{2}{|c|}{Nombre del flujo} \\
			\hline
			Paso & Acción del actor & Respuesta del sistema \\
			\hline
			 & & \\
			\hline
		\end{tabular}
		} \\
		\textbf{Fuente}: Autores
	\end{center}
\end{table}

\begin{table}[!htb]
	\caption{CU015-Añadir espectador: Descripción}
	\label{tab:cu015_desc}
	\begin{center}
		\resizebox{15cm}{!}{
		\begin{tabular}{|p{4cm}|p{11cm}|}
			\hline
			\multicolumn{2}{|c|}{Descripción de caso de uso} \\
			\hline
			Nombre & Añadir espectador \\
			\hline
			Identificador & CU015 \\
			\hline
			Descripción & Permite la adición de un espectador al evento deportivo \\
			\hline
			Actor & Todo actor de la red social	 
			\\
			\hline
			Disparador & Se elige la opción de añadir un espectador deportivo al evento \\
			\hline
			Inclusiones & N/A \\
			\hline
			Puntos de extensión & N/A \\
			\hline
			Precondiciones &  
			\begin{itemize}
				\item La aplicación ha sido cargada por un actor con rol de organizador de eventos deportivos
				\item Se ha elegido un evento al cual administrar involucrados
			\end{itemize}
			\\
			\hline
			Postcondiciones & 
			\begin{itemize}
				\item El usuario añade un o varios espectadores al deportivo del evento deportivo
				\item El usuario está en la pantalla de administración de involucrados en un evento deportivo
			\end{itemize}
			\\
			\hline
			Notas & N/A
			\\
			\hline
		\end{tabular}
		} \\
		\textbf{Fuente}: Autores
	\end{center}
\end{table}

\begin{table}[!htb]
	\caption{CU015-Añadir espectador: Flujos de hechos}
	\label{tab:cu015_flujo}
	\begin{center}
		\resizebox{15cm}{!}{
		\begin{tabular}{|p{1.5cm}|p{6cm}|p{6.5cm}|}
			\hline
			\multicolumn{3}{|c|}{Detalle de flujo de hechos de caso de uso} \\
			\hline
			Nombre & \multicolumn{2}{|c|}{Nombre del flujo} \\
			\hline
			Paso & Acción del actor & Respuesta del sistema \\
			\hline
			 & & \\
			\hline
		\end{tabular}
		} \\
		\textbf{Fuente}: Autores
	\end{center}
\end{table}

\begin{table}[!htb]
	\caption{CU016-Retirar espectador: Descripción}
	\label{tab:cu016_desc}
	\begin{center}
		\resizebox{15cm}{!}{
		\begin{tabular}{|p{4cm}|p{11cm}|}
			\hline
			\multicolumn{2}{|c|}{Descripción de caso de uso} \\
			\hline
			Nombre & Retirar espectador \\
			\hline
			Identificador & CU016 \\
			\hline
			Descripción & Permite retirar un espectador del evento deportivo \\
			\hline
			Actor & Todo actor de la red social	 
			\\
			\hline
			Disparador & Se elige la opción de retirar un espectador deportivo del evento \\
			\hline
			Inclusiones & N/A \\
			\hline
			Puntos de extensión & N/A \\
			\hline
			Precondiciones &  
			\begin{itemize}
				\item La aplicación ha sido cargada por un actor con rol de organizador de eventos deportivos
				\item Se ha elegido un evento al cual administrar involucrados
			\end{itemize}
			\\
			\hline
			Postcondiciones & 
			\begin{itemize}
				\item El usuario retira uno o varios espectadores del deportivo del evento deportivo
				\item El usuario está en la pantalla de administración de involucrados en un evento deportivo
			\end{itemize}
			\\
			\hline
			Notas & N/A
			\\
			\hline
		\end{tabular}
		} \\
		\textbf{Fuente}: Autores
	\end{center}
\end{table}

\begin{table}[!htb]
	\caption{CU016-Retirar espectador: Flujos de hechos}
	\label{tab:cu016_flujo}
	\begin{center}
		\resizebox{15cm}{!}{
		\begin{tabular}{|p{1.5cm}|p{6cm}|p{6.5cm}|}
			\hline
			\multicolumn{3}{|c|}{Detalle de flujo de hechos de caso de uso} \\
			\hline
			Nombre & \multicolumn{2}{|c|}{Nombre del flujo} \\
			\hline
			Paso & Acción del actor & Respuesta del sistema \\
			\hline
			 & & \\
			\hline
		\end{tabular}
		} \\
		\textbf{Fuente}: Autores
	\end{center}
\end{table}

\subsection{Módulo de administración de torneos}

A continuación se muestran los casos de uso del módulo de administración de torneos

\begin{table}[!htb]
	\caption{CU017-Administrar torneos: Descripción}
	\label{tab:cu017_desc}
	\begin{center}
		\resizebox{15cm}{!}{
		\begin{tabular}{|p{4cm}|p{11cm}|}
			\hline
			\multicolumn{2}{|c|}{Descripción de caso de uso} \\
			\hline
			Nombre & Administrar torneos \\
			\hline
			Identificador & CU017 \\
			\hline
			Descripción & Permite la administración de torneos deportivos \\
			\hline
			Actor & Todo actor de la red social	 
			\\
			\hline
			Disparador & Se elige la opción de administrar torneos \\
			\hline
			Inclusiones & N/A \\
			\hline
			Puntos de extensión & N/A \\
			\hline
			Precondiciones &  
			\begin{itemize}
				\item La aplicación ha sido cargada por un actor con rol de organizador de eventos deportivos
			\end{itemize}
			\\
			\hline
			Postcondiciones & 
			\begin{itemize}
				\item El usuario está en la pantalla de administración de torneos
			\end{itemize}
			\\
			\hline
			Notas & 
			\begin{itemize}
				\item Generalización de:
				\begin{itemize}
					\item Crear torneo
					\item Actualizar información de torneo
					\item Gestionar formatos de torneo
					\item Agregar equipo a torneo
					\item Retirar equipo de torneo
				\end{itemize}
			\end{itemize}
			\\
			\hline
		\end{tabular}
		} \\
		\textbf{Fuente}: Autores
	\end{center}
\end{table}

\begin{table}[!htb]
	\caption{CU017-Administrar torneos: Flujos de hechos}
	\label{tab:cu017_flujo}
	\begin{center}
		\resizebox{15cm}{!}{
		\begin{tabular}{|p{1.5cm}|p{6cm}|p{6.5cm}|}
			\hline
			\multicolumn{3}{|c|}{Detalle de flujo de hechos de caso de uso} \\
			\hline
			Nombre & \multicolumn{2}{|c|}{Nombre del flujo} \\
			\hline
			Paso & Acción del actor & Respuesta del sistema \\
			\hline
			 & & \\
			\hline
		\end{tabular}
		} \\
		\textbf{Fuente}: Autores
	\end{center}
\end{table}

\begin{table}[!htb]
	\caption{CU018-Crear un torneo: Descripción}
	\label{tab:cu018_desc}
	\begin{center}
		\resizebox{15cm}{!}{
		\begin{tabular}{|p{4cm}|p{11cm}|}
			\hline
			\multicolumn{2}{|c|}{Descripción de caso de uso} \\
			\hline
			Nombre & Crear un torneo \\
			\hline
			Identificador & CU018 \\
			\hline
			Descripción & Permite la creación de un torneo \\
			\hline
			Actor & Todo actor de la red social	 
			\\
			\hline
			Disparador & Se elige la opción de crear un torneo \\
			\hline
			Inclusiones & N/A \\
			\hline
			Puntos de extensión & N/A \\
			\hline
			Precondiciones &  
			\begin{itemize}
				\item La aplicación ha sido cargada por un actor con rol de organizador de eventos deportivos
				\item Se ha elegido administrar torneo
			\end{itemize}
			\\
			\hline
			Postcondiciones & 
			\begin{itemize}
				\item El usuario ha creado o no un torneo
				\item El usuario está en la pantalla de administración de torneos
			\end{itemize}
			\\
			\hline
			Notas & N/A
			\\
			\hline
		\end{tabular}
		} \\
		\textbf{Fuente}: Autores
	\end{center}
\end{table}

\begin{table}[!htb]
	\caption{CU018-Crear un torneo: Flujos de hechos}
	\label{tab:cu018_flujo}
	\begin{center}
		\resizebox{15cm}{!}{
		\begin{tabular}{|p{1.5cm}|p{6cm}|p{6.5cm}|}
			\hline
			\multicolumn{3}{|c|}{Detalle de flujo de hechos de caso de uso} \\
			\hline
			Nombre & \multicolumn{2}{|c|}{Nombre del flujo} \\
			\hline
			Paso & Acción del actor & Respuesta del sistema \\
			\hline
			 & & \\
			\hline
		\end{tabular}
		} \\
		\textbf{Fuente}: Autores
	\end{center}
\end{table}

\begin{table}[!htb]
	\caption{CU019-Actualizar información de torneo: Descripción}
	\label{tab:cu019_desc}
	\begin{center}
		\resizebox{15cm}{!}{
		\begin{tabular}{|p{4cm}|p{11cm}|}
			\hline
			\multicolumn{2}{|c|}{Descripción de caso de uso} \\
			\hline
			Nombre & Actualizar información de torneo \\
			\hline
			Identificador & CU019 \\
			\hline
			Descripción & Actualiza la información de un torneo \\
			\hline
			Actor & Todo actor de la red social	 
			\\
			\hline
			Disparador & Se elige la opción de actualizar información de torneo \\
			\hline
			Inclusiones & N/A \\
			\hline
			Puntos de extensión & 
			\begin{itemize}
				\item Gestionar formato de torneo
			\end{itemize}	
			\\
			\hline
			Precondiciones &  
			\begin{itemize}
				\item La aplicación ha sido cargada por un actor con rol de organizador de eventos deportivos
				\item Se ha elegido administrar torneo
			\end{itemize}
			\\
			\hline
			Postcondiciones & 
			\begin{itemize}
				\item El usuario ha actualizado o no la información de un torneo
				\item El usuario está en la pantalla de administración de torneos
			\end{itemize}
			\\
			\hline
			Notas & 
			\begin{itemize}
				\item Generalización de:
				\begin{itemize}
					\item Generar calendario de encuentros
					\item Reportar resultado de encuentro
				\end{itemize}
			\end{itemize}
			\\
			\hline
		\end{tabular}
		} \\
		\textbf{Fuente}: Autores
	\end{center}
\end{table}

\begin{table}[!htb]
	\caption{CU019-Actualizar información de torneo: Flujos de hechos}
	\label{tab:cu019_flujo}
	\begin{center}
		\resizebox{15cm}{!}{
		\begin{tabular}{|p{1.5cm}|p{6cm}|p{6.5cm}|}
			\hline
			\multicolumn{3}{|c|}{Detalle de flujo de hechos de caso de uso} \\
			\hline
			Nombre & \multicolumn{2}{|c|}{Nombre del flujo} \\
			\hline
			Paso & Acción del actor & Respuesta del sistema \\
			\hline
			 & & \\
			\hline
		\end{tabular}
		} \\
		\textbf{Fuente}: Autores
	\end{center}
\end{table}

\begin{table}[!htb]
	\caption{CU020-Gestionar formato de torneo: Descripción}
	\label{tab:cu020_desc}
	\begin{center}
		\resizebox{15cm}{!}{
		\begin{tabular}{|p{4cm}|p{11cm}|}
			\hline
			\multicolumn{2}{|c|}{Descripción de caso de uso} \\
			\hline
			Nombre & Gestionar formato de torneo \\
			\hline
			Identificador & CU020 \\
			\hline
			Descripción & Permite la asignación de un formato al torneo realizado, así como también la organización de los equipos/jugadores participantes en dicho formato \\
			\hline
			Actor & Todo actor de la red social	 
			\\
			\hline
			Disparador & Se elige gestionar formato de evento \\
			\hline
			Inclusiones & N/A \\
			\hline
			Puntos de extensión & N/A
			\\
			\hline
			Precondiciones &  
			\begin{itemize}
				\item La aplicación ha sido cargada por un actor con rol de organizador de eventos deportivos
				\item Se ha elegido administrar torneo
			\end{itemize}
			\\
			\hline
			Postcondiciones & 
			\begin{itemize}
				\item El usuario ha gestionado el formato de torneo que desea
				\item El usuario está en la pantalla de administración de torneos
			\end{itemize}
			\\
			\hline
			Notas & N/A
			\\
			\hline
		\end{tabular}
		} \\
		\textbf{Fuente}: Autores
	\end{center}
\end{table}

\begin{table}[!htb]
	\caption{CU020-Gestionar formato de torneo: Flujos de hechos}
	\label{tab:cu020_flujo}
	\begin{center}
		\resizebox{15cm}{!}{
		\begin{tabular}{|p{1.5cm}|p{6cm}|p{6.5cm}|}
			\hline
			\multicolumn{3}{|c|}{Detalle de flujo de hechos de caso de uso} \\
			\hline
			Nombre & \multicolumn{2}{|c|}{Nombre del flujo} \\
			\hline
			Paso & Acción del actor & Respuesta del sistema \\
			\hline
			 & & \\
			\hline
		\end{tabular}
		} \\
		\textbf{Fuente}: Autores
	\end{center}
\end{table}

\begin{table}[!htb]
	\caption{CU021-Agregar equipo a torneo: Descripción}
	\label{tab:cu021_desc}
	\begin{center}
		\resizebox{15cm}{!}{
		\begin{tabular}{|p{4cm}|p{11cm}|}
			\hline
			\multicolumn{2}{|c|}{Descripción de caso de uso} \\
			\hline
			Nombre & Agregar equipo a torneo \\
			\hline
			Identificador & CU021 \\
			\hline
			Descripción & Agrega un equipo o jugador al torneo sin asignarlo a algún puesto en el formato elegido por el organizador del torneo \\
			\hline
			Actor & Todo actor de la red social	 
			\\
			\hline
			Disparador & Se elige agregar equipo o jugador a torneo \\
			\hline
			Inclusiones & N/A \\
			\hline
			Puntos de extensión & N/A
			\\
			\hline
			Precondiciones &  
			\begin{itemize}
				\item La aplicación ha sido cargada por un actor con rol de organizador de eventos deportivos
				\item Se ha elegido administrar torneo
				\item Se ha elegido un torneo en específico
			\end{itemize}
			\\
			\hline
			Postcondiciones & 
			\begin{itemize}
				\item El usuario ha agregado o no un equipo o jugador al torneo
				\item El usuario está en la pantalla de administración de torneos
			\end{itemize}
			\\
			\hline
			Notas & N/A
			\\
			\hline
		\end{tabular}
		} \\
		\textbf{Fuente}: Autores
	\end{center}
\end{table}

\begin{table}[!htb]
	\caption{CU021-Agregar equipo a torneo: Flujos de hechos}
	\label{tab:cu021_flujo}
	\begin{center}
		\resizebox{15cm}{!}{
		\begin{tabular}{|p{1.5cm}|p{6cm}|p{6.5cm}|}
			\hline
			\multicolumn{3}{|c|}{Detalle de flujo de hechos de caso de uso} \\
			\hline
			Nombre & \multicolumn{2}{|c|}{Nombre del flujo} \\
			\hline
			Paso & Acción del actor & Respuesta del sistema \\
			\hline
			 & & \\
			\hline
		\end{tabular}
		} \\
		\textbf{Fuente}: Autores
	\end{center}
\end{table}

\begin{table}[!htb]
	\caption{CU022-Retirar equipo de torneo: Descripción}
	\label{tab:cu022_desc}
	\begin{center}
		\resizebox{15cm}{!}{
		\begin{tabular}{|p{4cm}|p{11cm}|}
			\hline
			\multicolumn{2}{|c|}{Descripción de caso de uso} \\
			\hline
			Nombre & Retirar equipo de torneo \\
			\hline
			Identificador & CU022 \\
			\hline
			Descripción & Retira un equipo o jugador del torneo \\
			\hline
			Actor & Todo actor de la red social	 
			\\
			\hline
			Disparador & Se elige retirar equipo o jugador del torneo \\
			\hline
			Inclusiones & N/A \\
			\hline
			Puntos de extensión & N/A
			\\
			\hline
			Precondiciones &  
			\begin{itemize}
				\item La aplicación ha sido cargada por un actor con rol de organizador de eventos deportivos
				\item Se ha elegido administrar torneo
				\item Se ha elegido un torneo en específico
			\end{itemize}
			\\
			\hline
			Postcondiciones & 
			\begin{itemize}
				\item El usuario ha retirado o no un equipo o jugador del torneo
				\item El usuario está en la pantalla de administración de torneos
			\end{itemize}
			\\
			\hline
			Notas & N/A
			\\
			\hline
		\end{tabular}
		} \\
		\textbf{Fuente}: Autores
	\end{center}
\end{table}

\begin{table}[!htb]
	\caption{CU022-Retirar equipo de torneo: Flujos de hechos}
	\label{tab:cu022_flujo}
	\begin{center}
		\resizebox{15cm}{!}{
		\begin{tabular}{|p{1.5cm}|p{6cm}|p{6.5cm}|}
			\hline
			\multicolumn{3}{|c|}{Detalle de flujo de hechos de caso de uso} \\
			\hline
			Nombre & \multicolumn{2}{|c|}{Nombre del flujo} \\
			\hline
			Paso & Acción del actor & Respuesta del sistema \\
			\hline
			 & & \\
			\hline
		\end{tabular}
		} \\
		\textbf{Fuente}: Autores
	\end{center}
\end{table}

\begin{table}[!htb]
	\caption{CU023-Generar calendario de encuentros: Descripción}
	\label{tab:cu023_desc}
	\begin{center}
		\resizebox{15cm}{!}{
		\begin{tabular}{|p{4cm}|p{11cm}|}
			\hline
			\multicolumn{2}{|c|}{Descripción de caso de uso} \\
			\hline
			Nombre & Generar calendario de encuentros \\
			\hline
			Identificador & CU023 \\
			\hline
			Descripción & Genera el calendario de los encuentros a realizarse en el evento \\
			\hline
			Actor & Todo actor de la red social	 
			\\
			\hline
			Disparador & Se elige generar calendario de encuentros \\
			\hline
			Inclusiones & N/A \\
			\hline
			Puntos de extensión & N/A
			\\
			\hline
			Precondiciones &  
			\begin{itemize}
				\item La aplicación ha sido cargada por un actor con rol de organizador de eventos deportivos
				\item Se ha elegido administrar torneo
				\item Se ha elegido un torneo en específico
				\item Se ha arreglado el formato del evento correctamente
			\end{itemize}
			\\
			\hline
			Postcondiciones & 
			\begin{itemize}
				\item El usuario ha generado el calendario de encuentros
				\item El usuario está en la pantalla de administración de torneos
			\end{itemize}
			\\
			\hline
			Notas & N/A
			\\
			\hline
		\end{tabular}
		} \\
		\textbf{Fuente}: Autores
	\end{center}
\end{table}

\begin{table}[!htb]
	\caption{CU023-Generar calendario de encuentros: Flujos de hechos}
	\label{tab:cu023_flujo}
	\begin{center}
		\resizebox{15cm}{!}{
		\begin{tabular}{|p{1.5cm}|p{6cm}|p{6.5cm}|}
			\hline
			\multicolumn{3}{|c|}{Detalle de flujo de hechos de caso de uso} \\
			\hline
			Nombre & \multicolumn{2}{|c|}{Nombre del flujo} \\
			\hline
			Paso & Acción del actor & Respuesta del sistema \\
			\hline
			 & & \\
			\hline
		\end{tabular}
		} \\
		\textbf{Fuente}: Autores
	\end{center}
\end{table}

\begin{table}[!htb]
	\caption{CU024-Reportar resultado de encuentro: Descripción}
	\label{tab:cu024_desc}
	\begin{center}
		\resizebox{15cm}{!}{
		\begin{tabular}{|p{4cm}|p{11cm}|}
			\hline
			\multicolumn{2}{|c|}{Descripción de caso de uso} \\
			\hline
			Nombre & Reportar resultado de encuentro \\
			\hline
			Identificador & CU024 \\
			\hline
			Descripción & Permite reportar el resultado de un encuentro deportivo después de haber iniciado el torneo \\
			\hline
			Actor & Todo actor de la red social	 
			\\
			\hline
			Disparador & Se elige reportar resultado de un encuentro \\
			\hline
			Inclusiones & N/A \\
			\hline
			Puntos de extensión & N/A
			\\
			\hline
			Precondiciones &  
			\begin{itemize}
				\item La aplicación ha sido cargada por un actor con rol de organizador de eventos deportivos
				\item Se ha elegido administrar torneo
				\item Se ha elegido un torneo en específico
				\item Se ha elegido un encuentro específico
				\item Se ha terminado el encuentro elegido según calendario
			\end{itemize}
			\\
			\hline
			Postcondiciones & 
			\begin{itemize}
				\item El usuario ha actualizado el resultado del encuentro
				\item El usuario está en la pantalla de administración de torneos
			\end{itemize}
			\\
			\hline
			Notas & N/A
			\\
			\hline
		\end{tabular}
		} \\
		\textbf{Fuente}: Autores
	\end{center}
\end{table}

\begin{table}[!htb]
	\caption{CU024-Reportar resultado de encuentro: Flujos de hechos}
	\label{tab:cu024_flujo}
	\begin{center}
		\resizebox{15cm}{!}{
		\begin{tabular}{|p{1.5cm}|p{6cm}|p{6.5cm}|}
			\hline
			\multicolumn{3}{|c|}{Detalle de flujo de hechos de caso de uso} \\
			\hline
			Nombre & \multicolumn{2}{|c|}{Nombre del flujo} \\
			\hline
			Paso & Acción del actor & Respuesta del sistema \\
			\hline
			 & & \\
			\hline
		\end{tabular}
		} \\
		\textbf{Fuente}: Autores
	\end{center}
\end{table}

\subsection{Módulo de administración de equipos}

A continuación se muestran los casos de uso del módulo de administración de equipos

\begin{table}[!htb]
	\caption{CU025-Administración de equipos: Descripción}
	\label{tab:cu025_desc}
	\begin{center}
		\resizebox{15cm}{!}{
		\begin{tabular}{|p{4cm}|p{11cm}|}
			\hline
			\multicolumn{2}{|c|}{Descripción de caso de uso} \\
			\hline
			Nombre & Administración de equipos \\
			\hline
			Identificador & CU025 \\
			\hline
			Descripción & Permite administrar un equipo en la red social deportiva \\
			\hline
			Actor & Todo actor de la red social	 
			\\
			\hline
			Disparador & Administrar un equipo en la red social deportiva \\
			\hline
			Inclusiones & N/A \\
			\hline
			Puntos de extensión & N/A
			\\
			\hline
			Precondiciones &  
			\begin{itemize}
				\item La aplicación ha sido cargada por un actor con rol de formador de grupos deportivos
			\end{itemize}
			\\
			\hline
			Postcondiciones & 
			\begin{itemize}
				\item El usuario está en la pantalla de administración de equipos
			\end{itemize}
			\\
			\hline
			Notas & 
			\begin{itemize}
				\item Generalización de:
				\begin{itemize}
					\item Crear un equipo
					\item Actualizar información de equipo
					\item Agregar un integrante a un equipo
					\item Actualizar información de integrante de equipo
					\item Dar de baja a integrante de grupo
				\end{itemize}
			\end{itemize}
			\\
			\hline
		\end{tabular}
		} \\
		\textbf{Fuente}: Autores
	\end{center}
\end{table}

\begin{table}[!htb]
	\caption{CU025-Administración de equipos: Flujos de hechos}
	\label{tab:cu025_flujo}
	\begin{center}
		\resizebox{15cm}{!}{
		\begin{tabular}{|p{1.5cm}|p{6cm}|p{6.5cm}|}
			\hline
			\multicolumn{3}{|c|}{Detalle de flujo de hechos de caso de uso} \\
			\hline
			Nombre & \multicolumn{2}{|c|}{Nombre del flujo} \\
			\hline
			Paso & Acción del actor & Respuesta del sistema \\
			\hline
			 & & \\
			\hline
		\end{tabular}
		} \\
		\textbf{Fuente}: Autores
	\end{center}
\end{table}

\begin{table}[!htb]
	\caption{CU026-Crear un equipo: Descripción}
	\label{tab:cu026_desc}
	\begin{center}
		\resizebox{15cm}{!}{
		\begin{tabular}{|p{4cm}|p{11cm}|}
			\hline
			\multicolumn{2}{|c|}{Descripción de caso de uso} \\
			\hline
			Nombre & Crear un equipo \\
			\hline
			Identificador & CU026 \\
			\hline
			Descripción & Permite la creación de un equipo deportivo en la red social deportiva \\
			\hline
			Actor & Todo actor de la red social	 
			\\
			\hline
			Disparador & Se elige crear un equipo deportivo \\
			\hline
			Inclusiones & N/A \\
			\hline
			Puntos de extensión & N/A
			\\
			\hline
			Precondiciones &  
			\begin{itemize}
				\item La aplicación ha sido cargada por un actor con rol de formador de grupos deportivos
				\item El usuario eligió la administración de equipos
			\end{itemize}
			\\
			\hline
			Postcondiciones & 
			\begin{itemize}
				\item El usuario crea un equipo deportivo
				\item El usuario se encuentra en la pantalla de administración de equipos
				\item El usuario eligió un equipo en específico
			\end{itemize}
			\\
			\hline
			Notas & N/A
			\\
			\hline
		\end{tabular}
		} \\
		\textbf{Fuente}: Autores
	\end{center}
\end{table}

\begin{table}[!htb]
	\caption{CU026-Crear un equipo: Flujos de hechos}
	\label{tab:cu026_flujo}
	\begin{center}
		\resizebox{15cm}{!}{
		\begin{tabular}{|p{1.5cm}|p{6cm}|p{6.5cm}|}
			\hline
			\multicolumn{3}{|c|}{Detalle de flujo de hechos de caso de uso} \\
			\hline
			Nombre & \multicolumn{2}{|c|}{Nombre del flujo} \\
			\hline
			Paso & Acción del actor & Respuesta del sistema \\
			\hline
			 & & \\
			\hline
		\end{tabular}
		} \\
		\textbf{Fuente}: Autores
	\end{center}
\end{table}

\begin{table}[!htb]
	\caption{CU027-Actualizar información de equipo: Descripción}
	\label{tab:cu027_desc}
	\begin{center}
		\resizebox{15cm}{!}{
		\begin{tabular}{|p{4cm}|p{11cm}|}
			\hline
			\multicolumn{2}{|c|}{Descripción de caso de uso} \\
			\hline
			Nombre & Actualizar información de equipo \\
			\hline
			Identificador & CU027 \\
			\hline
			Descripción & Permite actualizar la información general de un equipo deportivo \\
			\hline
			Actor & Todo actor de la red social	 
			\\
			\hline
			Disparador & Se elige actualizar la información de un equipo \\
			\hline
			Inclusiones & N/A \\
			\hline
			Puntos de extensión & N/A
			\\
			\hline
			Precondiciones &  
			\begin{itemize}
				\item La aplicación ha sido cargada por un actor con rol de formador de grupos deportivos
				\item El usuario eligió la administración de equipos
				\item El usuario eligió un equipo en específico
			\end{itemize}
			\\
			\hline
			Postcondiciones & 
			\begin{itemize}
				\item El usuario actualiza la información del equipo deportivo
				\item El usuario se encuentra en la pantalla de administración de equipos
			\end{itemize}
			\\
			\hline
			Notas & N/A
			\\
			\hline
		\end{tabular}
		} \\
		\textbf{Fuente}: Autores
	\end{center}
\end{table}

\begin{table}[!htb]
	\caption{CU027-Actualizar información de equipo: Flujos de hechos}
	\label{tab:cu027_flujo}
	\begin{center}
		\resizebox{15cm}{!}{
		\begin{tabular}{|p{1.5cm}|p{6cm}|p{6.5cm}|}
			\hline
			\multicolumn{3}{|c|}{Detalle de flujo de hechos de caso de uso} \\
			\hline
			Nombre & \multicolumn{2}{|c|}{Nombre del flujo} \\
			\hline
			Paso & Acción del actor & Respuesta del sistema \\
			\hline
			 & & \\
			\hline
		\end{tabular}
		} \\
		\textbf{Fuente}: Autores
	\end{center}
\end{table}

\begin{table}[!htb]
	\caption{CU028-Agregar un integrante a un equipo: Descripción}
	\label{tab:cu028_desc}
	\begin{center}
		\resizebox{15cm}{!}{
		\begin{tabular}{|p{4cm}|p{11cm}|}
			\hline
			\multicolumn{2}{|c|}{Descripción de caso de uso} \\
			\hline
			Nombre & Agregar un integrante a un equipo \\
			\hline
			Identificador & CU028 \\
			\hline
			Descripción & Permite agregar un integrante al equipo elegido \\
			\hline
			Actor & Todo actor de la red social	 
			\\
			\hline
			Disparador & Se elige agregar un integrante a un equipo \\
			\hline
			Inclusiones & N/A \\
			\hline
			Puntos de extensión & N/A
			\\
			\hline
			Precondiciones &  
			\begin{itemize}
				\item La aplicación ha sido cargada por un actor con rol de formador de grupos deportivos
				\item El usuario eligió la administración de equipos
				\item El usuario eligió un equipo en específico
			\end{itemize}
			\\
			\hline
			Postcondiciones & 
			\begin{itemize}
				\item El usuario agrega un integrante al equipo
				\item El usuario se encuentra en la pantalla de administración de equipos
			\end{itemize}
			\\
			\hline
			Notas & N/A
			\\
			\hline
		\end{tabular}
		} \\
		\textbf{Fuente}: Autores
	\end{center}
\end{table}

\begin{table}[!htb]
	\caption{CU028-Agregar un integrante a un equipo: Flujos de hechos}
	\label{tab:cu028_flujo}
	\begin{center}
		\resizebox{15cm}{!}{
		\begin{tabular}{|p{1.5cm}|p{6cm}|p{6.5cm}|}
			\hline
			\multicolumn{3}{|c|}{Detalle de flujo de hechos de caso de uso} \\
			\hline
			Nombre & \multicolumn{2}{|c|}{Nombre del flujo} \\
			\hline
			Paso & Acción del actor & Respuesta del sistema \\
			\hline
			 & & \\
			\hline
		\end{tabular}
		} \\
		\textbf{Fuente}: Autores
	\end{center}
\end{table}

\begin{table}[!htb]
	\caption{CU029-Actualizar información de integrante de equipo: Descripción}
	\label{tab:cu029_desc}
	\begin{center}
		\resizebox{15cm}{!}{
		\begin{tabular}{|p{4cm}|p{11cm}|}
			\hline
			\multicolumn{2}{|c|}{Descripción de caso de uso} \\
			\hline
			Nombre & Actualizar información de integrante de equipo \\
			\hline
			Identificador & CU029 \\
			\hline
			Descripción & Permite la actualización la información de la información de un integrante del equipo referente al equipo mismo (ej. posición en la que éste juega) \\
			\hline
			Actor & Todo actor de la red social	 
			\\
			\hline
			Disparador & El usuario elige actualizar la información de equipo de un integrante del equipo \\
			\hline
			Inclusiones & N/A \\
			\hline
			Puntos de extensión & N/A
			\\
			\hline
			Precondiciones &  
			\begin{itemize}
				\item La aplicación ha sido cargada por un actor con rol de formador de grupos deportivos
				\item El usuario eligió la administración de equipos
				\item El usuario eligió un equipo en específico
			\end{itemize}
			\\
			\hline
			Postcondiciones & 
			\begin{itemize}
				\item El usuario cambia la información de equipo de un integrante del equipo
				\item El usuario se encuentra en la pantalla de administración de equipos
			\end{itemize}
			\\
			\hline
			Notas & N/A
			\\
			\hline
		\end{tabular}
		} \\
		\textbf{Fuente}: Autores
	\end{center}
\end{table}

\begin{table}[!htb]
	\caption{CU029-Actualizar información de integrante de equipo: Flujos de hechos}
	\label{tab:cu029_flujo}
	\begin{center}
		\resizebox{15cm}{!}{
		\begin{tabular}{|p{1.5cm}|p{6cm}|p{6.5cm}|}
			\hline
			\multicolumn{3}{|c|}{Detalle de flujo de hechos de caso de uso} \\
			\hline
			Nombre & \multicolumn{2}{|c|}{Nombre del flujo} \\
			\hline
			Paso & Acción del actor & Respuesta del sistema \\
			\hline
			 & & \\
			\hline
		\end{tabular}
		} \\
		\textbf{Fuente}: Autores
	\end{center}
\end{table}

\begin{table}[!htb]
	\caption{CU030-Dar de baja a integrante de equipo: Descripción}
	\label{tab:cu030_desc}
	\begin{center}
		\resizebox{15cm}{!}{
		\begin{tabular}{|p{4cm}|p{11cm}|}
			\hline
			\multicolumn{2}{|c|}{Descripción de caso de uso} \\
			\hline
			Nombre & Dar de baja a integrante de equipo \\
			\hline
			Identificador & CU030 \\
			\hline
			Descripción & Permite eliminar un jugador de un equipo deportivo, así como desvincularse del mismo \\
			\hline
			Actor & Todo actor de la red social	 
			\\
			\hline
			Disparador & El usuario elige dar de baja a un integrante del equipo (darse de baja también) \\
			\hline
			Inclusiones & N/A \\
			\hline
			Puntos de extensión & N/A
			\\
			\hline
			Precondiciones &  
			\begin{itemize}
				\item La aplicación ha sido cargada por un actor con rol de formador de grupos deportivos o un jugador del equipo intenta acceder a la funcionalidad
				\item El usuario eligió la administración de equipos
				\item El usuario eligió un equipo en específico
			\end{itemize}
			\\
			\hline
			Postcondiciones & 
			\begin{itemize}
				\item El usuario da de baja a un jugador del equipo (o se da de baja a si mismo)
				\item El usuario se encuentra en la pantalla de administración de equipos
			\end{itemize}
			\\
			\hline
			Notas & N/A
			\\
			\hline
		\end{tabular}
		} \\
		\textbf{Fuente}: Autores
	\end{center}
\end{table}

\begin{table}[!htb]
	\caption{CU030-Dar de baja a integrante de equipo: Flujos de hechos}
	\label{tab:cu030_flujo}
	\begin{center}
		\resizebox{15cm}{!}{
		\begin{tabular}{|p{1.5cm}|p{6cm}|p{6.5cm}|}
			\hline
			\multicolumn{3}{|c|}{Detalle de flujo de hechos de caso de uso} \\
			\hline
			Nombre & \multicolumn{2}{|c|}{Nombre del flujo} \\
			\hline
			Paso & Acción del actor & Respuesta del sistema \\
			\hline
			 & & \\
			\hline
		\end{tabular}
		} \\
		\textbf{Fuente}: Autores
	\end{center}
\end{table}

\subsection{Módulo de estadísticas}

A continuación se muestran los casos de uso del módulo de estadísticas

\begin{table}[!htb]
	\caption{CU031-Gestión de estadísticas: Descripción}
	\label{tab:cu031_desc}
	\begin{center}
		\resizebox{15cm}{!}{
		\begin{tabular}{|p{4cm}|p{11cm}|}
			\hline
			\multicolumn{2}{|c|}{Descripción de caso de uso} \\
			\hline
			Nombre & Gestión de estadísticas \\
			\hline
			Identificador & CU031 \\
			\hline
			Descripción & Permite la gestión de estadísticas a través de la red social deportiva \\
			\hline
			Actor & Todo actor de la red social	 
			\\
			\hline
			Disparador & El usuario decide calificar un objeto calificable en el SNS (evento, usuario de la red social, servicio prestado por un usuario de la red social, etc.) \\
			\hline
			Inclusiones & N/A \\
			\hline
			Puntos de extensión & N/A
			\\
			\hline
			Precondiciones &  
			\begin{itemize}
				\item La aplicación ha sido cargada por un con permisos para calificar el objeto calificable del SNS
				\item El usuario elige calificar el objeto
			\end{itemize}
			\\
			\hline
			Postcondiciones & 
			\begin{itemize}
				\item El usuario se encuentra en la pantalla de calificación
			\end{itemize}
			\\
			\hline
			Notas &
			\begin{itemize}
				\item Generalización de
				\begin{itemize}
				 \item Llevar estadística de jugador
				 \item Llevar estadística de equipo
				 \item Calificar ligas deportivas
				 \item Calificar entrenadores
				 \item Calificar organización
				 \item Llevar calificación de "mejor por categoría"
				\end{itemize}
			\end{itemize}
			\\
			\hline
		\end{tabular}
		} \\
		\textbf{Fuente}: Autores
	\end{center}
\end{table}

\begin{table}[!htb]
	\caption{CU031-Gestión de estadísticas: Flujos de hechos}
	\label{tab:cu031_flujo}
	\begin{center}
		\resizebox{15cm}{!}{
		\begin{tabular}{|p{1.5cm}|p{6cm}|p{6.5cm}|}
			\hline
			\multicolumn{3}{|c|}{Detalle de flujo de hechos de caso de uso} \\
			\hline
			Nombre & \multicolumn{2}{|c|}{Nombre del flujo} \\
			\hline
			Paso & Acción del actor & Respuesta del sistema \\
			\hline
			 & & \\
			\hline
		\end{tabular}
		} \\
		\textbf{Fuente}: Autores
	\end{center}
\end{table}

\begin{table}[!htb]
	\caption{CU032-Llevar estadística de jugador: Descripción}
	\label{tab:cu032_desc}
	\begin{center}
		\resizebox{15cm}{!}{
		\begin{tabular}{|p{4cm}|p{11cm}|}
			\hline
			\multicolumn{2}{|c|}{Descripción de caso de uso} \\
			\hline
			Nombre & Llevar estadística de jugador \\
			\hline
			Identificador & CU032 \\
			\hline
			Descripción & Permite el cálculo de estadísticas de un jugador venidas desde los datos registrados de los equipos en los que se encuentra y de datos verificables en prácticas libres para cada uno de los deportes que éste practica \\
			\hline
			Actor & 
			\begin{itemize}
				\item Jugador
				\item Equipo
				\item Entrenador
			\end{itemize}				 
			\\
			\hline
			Disparador & El usuario ingresa datos estadísticos de un jugador a la red social \\
			\hline
			Inclusiones & N/A \\
			\hline
			Puntos de extensión & 
			\begin{itemize}
				\item Gestión de niveles de juego
			\end{itemize}
			\\
			\hline
			Precondiciones &  
			\begin{itemize}
				\item La aplicación ha sido cargada como entrenador, equipo deportivo o jugador
				\item El usuario ingresa datos estadísticos del jugador
			\end{itemize}
			\\
			\hline
			Postcondiciones & 
			\begin{itemize}
				\item Se calculan datos estadísticos del jugador
			\end{itemize}
			\\
			\hline
			Notas & Los datos pueden ser accesados por cualquier usuario de la red social
			\\
			\hline
		\end{tabular}
		} \\
		\textbf{Fuente}: Autores
	\end{center}
\end{table}

\begin{table}[!htb]
	\caption{CU032-Llevar estadística de jugador: Flujos de hechos}
	\label{tab:cu032_flujo}
	\begin{center}
		\resizebox{15cm}{!}{
		\begin{tabular}{|p{1.5cm}|p{6cm}|p{6.5cm}|}
			\hline
			\multicolumn{3}{|c|}{Detalle de flujo de hechos de caso de uso} \\
			\hline
			Nombre & \multicolumn{2}{|c|}{Nombre del flujo} \\
			\hline
			Paso & Acción del actor & Respuesta del sistema \\
			\hline
			 & & \\
			\hline
		\end{tabular}
		} \\
		\textbf{Fuente}: Autores
	\end{center}
\end{table}

\begin{table}[!htb]
	\caption{CU033-Llevar estadística de equipo: Descripción}
	\label{tab:cu033_desc}
	\begin{center}
		\resizebox{15cm}{!}{
		\begin{tabular}{|p{4cm}|p{11cm}|}
			\hline
			\multicolumn{2}{|c|}{Descripción de caso de uso} \\
			\hline
			Nombre & Llevar estadística de equipo \\
			\hline
			Identificador & CU033 \\
			\hline
			Descripción & Permite el cálculo de estadísticas de un equipo venidas desde los datos registrados de éste en los eventos/torneos deportivos en los que ha participado \\
			\hline
			Actor & 
			\begin{itemize}
				\item Equipo
				\item Entrenador
			\end{itemize}				 
			\\
			\hline
			Disparador & El usuario ingresa datos estadísticos de un equipo a la red social \\
			\hline
			Inclusiones & N/A \\
			\hline
			Puntos de extensión & 
			\begin{itemize}
				\item Gestión de niveles de juego
			\end{itemize}
			\\
			\hline
			Precondiciones &  
			\begin{itemize}
				\item La aplicación ha sido cargada como entrenador, equipo deportivo o jugador
				\item El usuario se encuentra en gestión de equipo
				\item El usuario ingresa datos estadísticos del equipo
			\end{itemize}
			\\
			\hline
			Postcondiciones & 
			\begin{itemize}
				\item Se calculan datos estadísticos del equipo
			\end{itemize}
			\\
			\hline
			Notas & Los datos pueden ser accesados por cualquier usuario de la red social
			\\
			\hline
		\end{tabular}
		} \\
		\textbf{Fuente}: Autores
	\end{center}
\end{table}

\begin{table}[!htb]
	\caption{CU033-Llevar estadística de equipo: Flujos de hechos}
	\label{tab:cu033_flujo}
	\begin{center}
		\resizebox{15cm}{!}{
		\begin{tabular}{|p{1.5cm}|p{6cm}|p{6.5cm}|}
			\hline
			\multicolumn{3}{|c|}{Detalle de flujo de hechos de caso de uso} \\
			\hline
			Nombre & \multicolumn{2}{|c|}{Nombre del flujo} \\
			\hline
			Paso & Acción del actor & Respuesta del sistema \\
			\hline
			 & & \\
			\hline
		\end{tabular}
		} \\
		\textbf{Fuente}: Autores
	\end{center}
\end{table}

\begin{table}[!htb]
	\caption{CU034-Gestión de niveles de juego: Descripción}
	\label{tab:cu034_desc}
	\begin{center}
		\resizebox{15cm}{!}{
		\begin{tabular}{|p{4cm}|p{11cm}|}
			\hline
			\multicolumn{2}{|c|}{Descripción de caso de uso} \\
			\hline
			Nombre & Gestión de niveles de juego \\
			\hline
			Identificador & CU034 \\
			\hline
			Descripción & Lleva un análisis del nivel de juego de un jugador/equipo respecto de sus estadísticas y las manejadas en el resto de la red social, así como también las estadísticas de nivel de juego manejadas usualmente en un lugar donde hayan prácticas deportivas \\
			\hline
			Actor & N/A		 
			\\
			\hline
			Disparador & El usuario ingresa datos estadísticos de un equipo/jugador a la red social o se obtienen datos de un lugar deportivo, el nivel de juego manejado en dicho lugar \\
			\hline
			Inclusiones & N/A \\
			\hline
			Puntos de extensión & 
			\begin{itemize}
				\item Gestión de niveles de juego
			\end{itemize}
			\\
			\hline
			Precondiciones &  
			\begin{itemize}
				\item La aplicación ha sido cargada como entrenador, equipo deportivo o jugador
				\item El usuario ingresa datos estadísticos del equipo/jugador o la red social obtiene datos de lugares deportivos
			\end{itemize}
			\\
			\hline
			Postcondiciones & 
			\begin{itemize}
				\item Se calculan datos estadísticos del equipo
			\end{itemize}
			\\
			\hline
			Notas & Los datos pueden ser accesados por cualquier usuario de la red social
			\\
			\hline
		\end{tabular}
		} \\
		\textbf{Fuente}: Autores
	\end{center}
\end{table}

\begin{table}[!htb]
	\caption{CU034-Gestión de niveles de juego: Flujos de hechos}
	\label{tab:cu034_flujo}
	\begin{center}
		\resizebox{15cm}{!}{
		\begin{tabular}{|p{1.5cm}|p{6cm}|p{6.5cm}|}
			\hline
			\multicolumn{3}{|c|}{Detalle de flujo de hechos de caso de uso} \\
			\hline
			Nombre & \multicolumn{2}{|c|}{Nombre del flujo} \\
			\hline
			Paso & Acción del actor & Respuesta del sistema \\
			\hline
			 & & \\
			\hline
		\end{tabular}
		} \\
		\textbf{Fuente}: Autores
	\end{center}
\end{table}

\begin{table}[!htb]
	\caption{CU035-Calificar ligas deportivas: Descripción}
	\label{tab:cu035_desc}
	\begin{center}
		\resizebox{15cm}{!}{
		\begin{tabular}{|p{4cm}|p{11cm}|}
			\hline
			\multicolumn{2}{|c|}{Descripción de caso de uso} \\
			\hline
			Nombre & Calificar ligas deportivas \\
			\hline
			Identificador & CU035 \\
			\hline
			Descripción & Permite a cualquier usuario dar una calificación a una organización en la red social que se comporte como liga deportiva. A su vez, calcula la calificación global de dicha liga \\
			\hline
			Actor & Todo actor que haya tenido un vínculo con la liga deportiva				 
			\\
			\hline
			Disparador & 
			\begin{itemize}
				\item Jugador
				\item Equipo
				\item Patrocinador
				\item Organización
			\end{itemize}
			\\
			\hline
			Inclusiones & N/A \\
			\hline
			Puntos de extensión & N/A
			\\
			\hline
			Precondiciones &  
			\begin{itemize}
				\item El usuario se encuentra en zona de calificación de una organización que se comporte como liga deportiva
			\end{itemize}
			\\
			\hline
			Postcondiciones & 
			\begin{itemize}
				\item Se ha calificado la liga deportiva
				\item El usuario se encuentra en el perfíl de la liga deportiva
			\end{itemize}
			\\
			\hline
			Notas & Los datos pueden ser accesados por cualquier usuario de la red social
			\\
			\hline
		\end{tabular}
		} \\
		\textbf{Fuente}: Autores
	\end{center}
\end{table}

\begin{table}[!htb]
	\caption{CU035-Gestión de niveles de juego: Flujos de hechos}
	\label{tab:cu035_flujo}
	\begin{center}
		\resizebox{15cm}{!}{
		\begin{tabular}{|p{1.5cm}|p{6cm}|p{6.5cm}|}
			\hline
			\multicolumn{3}{|c|}{Detalle de flujo de hechos de caso de uso} \\
			\hline
			Nombre & \multicolumn{2}{|c|}{Nombre del flujo} \\
			\hline
			Paso & Acción del actor & Respuesta del sistema \\
			\hline
			 & & \\
			\hline
		\end{tabular}
		} \\
		\textbf{Fuente}: Autores
	\end{center}
\end{table}

\begin{table}[!htb]
	\caption{CU036-Calificar entrenadores: Descripción}
	\label{tab:cu036_desc}
	\begin{center}
		\resizebox{15cm}{!}{
		\begin{tabular}{|p{4cm}|p{11cm}|}
			\hline
			\multicolumn{2}{|c|}{Descripción de caso de uso} \\
			\hline
			Nombre & Calificar entrenadores \\
			\hline
			Identificador & CU036 \\
			\hline
			Descripción & Permite a cualquier usuario dar una calificación del servicio de entrenamiento dado por un entrenador deportivo \\
			\hline
			Actor & 
			\begin{itemize}
				\item Jugador
				\item Equipo
				\item Patrocinador
				\item Organización
			\end{itemize}
			\\
			\hline
			Disparador & El usuario califica los servicios ofrecidos por un entrenador \\
			\hline
			Inclusiones & N/A \\
			\hline
			Puntos de extensión & N/A
			\\
			\hline
			Precondiciones &  
			\begin{itemize}
				\item El usuario se encuentra en zona de calificación de servicios de un entrenador deportivo
			\end{itemize}
			\\
			\hline
			Postcondiciones & 
			\begin{itemize}
				\item Se ha calificado los servicios de un entrenador
				\item El usuario se encuentra en el perfíl del entrenador deportivo
			\end{itemize}
			\\
			\hline
			Notas & Los datos pueden ser accesados por cualquier usuario de la red social
			\\
			\hline
		\end{tabular}
		} \\
		\textbf{Fuente}: Autores
	\end{center}
\end{table}

\begin{table}[!htb]
	\caption{CU036-Calificar entrenadores: Flujos de hechos}
	\label{tab:cu036_flujo}
	\begin{center}
		\resizebox{15cm}{!}{
		\begin{tabular}{|p{1.5cm}|p{6cm}|p{6.5cm}|}
			\hline
			\multicolumn{3}{|c|}{Detalle de flujo de hechos de caso de uso} \\
			\hline
			Nombre & \multicolumn{2}{|c|}{Nombre del flujo} \\
			\hline
			Paso & Acción del actor & Respuesta del sistema \\
			\hline
			 & & \\
			\hline
		\end{tabular}
		} \\
		\textbf{Fuente}: Autores
	\end{center}
\end{table}

\begin{table}[!htb]
	\caption{CU037-Calificar organización: Descripción}
	\label{tab:cu037_desc}
	\begin{center}
		\resizebox{15cm}{!}{
		\begin{tabular}{|p{4cm}|p{11cm}|}
			\hline
			\multicolumn{2}{|c|}{Descripción de caso de uso} \\
			\hline
			Nombre & Calificar organización \\
			\hline
			Identificador & CU037 \\
			\hline
			Descripción & Permite a cualquier usuario dar una calificación de lose servicios ofrecidos por una organización deportiva en la red social \\
			\hline
			Actor & 
			\begin{itemize}
				\item Jugador
				\item Equipo
				\item Patrocinador
				\item Organización
			\end{itemize}
			\\
			\hline
			Disparador & El usuario califica los servicios ofrecidos por una organización deportiva \\
			\hline
			Inclusiones & N/A \\
			\hline
			Puntos de extensión & N/A
			\\
			\hline
			Precondiciones &  
			\begin{itemize}
				\item El usuario se encuentra en zona de calificación de servicios de una organización deportiva
			\end{itemize}
			\\
			\hline
			Postcondiciones & 
			\begin{itemize}
				\item Se ha calificado los servicios de una organización deportiva
				\item El usuario se encuentra en el perfíl de la organización deportiva
			\end{itemize}
			\\
			\hline
			Notas & Los datos pueden ser accesados por cualquier usuario de la red social
			\\
			\hline
		\end{tabular}
		} \\
		\textbf{Fuente}: Autores
	\end{center}
\end{table}

\begin{table}[!htb]
	\caption{CU037-Calificar organización: Flujos de hechos}
	\label{tab:cu037_flujo}
	\begin{center}
		\resizebox{15cm}{!}{
		\begin{tabular}{|p{1.5cm}|p{6cm}|p{6.5cm}|}
			\hline
			\multicolumn{3}{|c|}{Detalle de flujo de hechos de caso de uso} \\
			\hline
			Nombre & \multicolumn{2}{|c|}{Nombre del flujo} \\
			\hline
			Paso & Acción del actor & Respuesta del sistema \\
			\hline
			 & & \\
			\hline
		\end{tabular}
		} \\
		\textbf{Fuente}: Autores
	\end{center}
\end{table}

\begin{table}[!htb]
	\caption{CU038-Llevar calificación de "mejor por categoría": Descripción}
	\label{tab:cu038_desc}
	\begin{center}
		\resizebox{15cm}{!}{
		\begin{tabular}{|p{4cm}|p{11cm}|}
			\hline
			\multicolumn{2}{|c|}{Descripción de caso de uso} \\
			\hline
			Nombre & Llevar calificación de "mejor por categoría" \\
			\hline
			Identificador & CU038 \\
			\hline
			Descripción & Permite mostrar al usuario que lo desee los mejores usuarios en la busqueda que él desee realizar \\
			\hline
			Actor &
			\begin{itemize}
				\item Jugador
				\item Equipo
				\item Patrocinador
				\item Organización
			\end{itemize}
			\\
			\hline
			Disparador & El usuario hace una búsqueda por concepto de "mejor en la categoría" \\
			\hline
			Inclusiones & N/A \\
			\hline
			Puntos de extensión & N/A
			\\
			\hline
			Precondiciones &  
			\begin{itemize}
				\item El usuario se encuentra en zona de "búsqueda por mejor en la categoría"
			\end{itemize}
			\\
			\hline
			Postcondiciones & 
			\begin{itemize}
				\item La red social muestra el mejor en la categoría elegida por el usuario buscador
				\item El usuario se encuentra en zona de "búsqueda por mejor en la categoría"
			\end{itemize}
			\\
			\hline
			Notas & N/A
			\\
			\hline
		\end{tabular}
		} \\
		\textbf{Fuente}: Autores
	\end{center}
\end{table}

\begin{table}[!htb]
	\caption{CU038-Llevar calificación de "mejor por categoría": Flujos de hechos}
	\label{tab:cu038_flujo}
	\begin{center}
		\resizebox{15cm}{!}{
		\begin{tabular}{|p{1.5cm}|p{6cm}|p{6.5cm}|}
			\hline
			\multicolumn{3}{|c|}{Detalle de flujo de hechos de caso de uso} \\
			\hline
			Nombre & \multicolumn{2}{|c|}{Nombre del flujo} \\
			\hline
			Paso & Acción del actor & Respuesta del sistema \\
			\hline
			 & & \\
			\hline
		\end{tabular}
		} \\
		\textbf{Fuente}: Autores
	\end{center}
\end{table}

\subsection{Módulo de gestión de sponsors}

A continuación se muestran los casos de uso del módulo de gestión de sponsors

\begin{table}[!htb]
	\caption{CU039-Administración de sponsors: Descripción}
	\label{tab:cu039_desc}
	\begin{center}
		\resizebox{15cm}{!}{
		\begin{tabular}{|p{4cm}|p{11cm}|}
			\hline
			\multicolumn{2}{|c|}{Descripción de caso de uso} \\
			\hline
			Nombre & Administración de sponsors \\
			\hline
			Identificador & CU039 \\
			\hline
			Descripción & Permite gestionar o administrar funcionalidades ofrecidas para patrocinadores en la red social deportiva \\
			\hline
			Actor &
			\begin{itemize}
				\item Patrocinador
			\end{itemize}
			\\
			\hline
			Disparador & El patrocinador quiere entrar al módulo de gestión de funcionalidades ofrecida para éste en la red social \\
			\hline
			Inclusiones & N/A \\
			\hline
			Puntos de extensión & N/A
			\\
			\hline
			Precondiciones &  
			\begin{itemize}
				\item Se ha iniciado el SNS con un rol de patrocinador deportivo
			\end{itemize}
			\\
			\hline
			Postcondiciones & 
			\begin{itemize}
				\item El usuario se encuentra en la pantalla para gestión de patrocinios
			\end{itemize}
			\\
			\hline
			Notas & 
			\begin{itemize}
				\item Generalización de:
				\begin{itemize}
					\item Solicitar ser sponsor
					\item Dejar de ser sponsor
					\item Generación y consulta de historial de sponsorship
					\item Tracking a jugadores/equipos/eventos deportivos
				\end{itemize}
			\end{itemize}
			\\
			\hline
		\end{tabular}
		} \\
		\textbf{Fuente}: Autores
	\end{center}
\end{table}

\begin{table}[!htb]
	\caption{CU039-Administración de sponsors: Flujos de hechos}
	\label{tab:cu039_flujo}
	\begin{center}
		\resizebox{15cm}{!}{
		\begin{tabular}{|p{1.5cm}|p{6cm}|p{6.5cm}|}
			\hline
			\multicolumn{3}{|c|}{Detalle de flujo de hechos de caso de uso} \\
			\hline
			Nombre & \multicolumn{2}{|c|}{Nombre del flujo} \\
			\hline
			Paso & Acción del actor & Respuesta del sistema \\
			\hline
			 & & \\
			\hline
		\end{tabular}
		} \\
		\textbf{Fuente}: Autores
	\end{center}
\end{table}

\begin{table}[!htb]
	\caption{CU041-Dejar de ser sponsor: Descripción}
	\label{tab:cu041_desc}
	\begin{center}
		\resizebox{15cm}{!}{
		\begin{tabular}{|p{4cm}|p{11cm}|}
			\hline
			\multicolumn{2}{|c|}{Descripción de caso de uso} \\
			\hline
			Nombre & Dejar de ser sponsor \\
			\hline
			Identificador & CU041 \\
			\hline
			Descripción & Permite a un sponsor dejar de ser sponsor de un equipo/jugador/evento deportivo \\
			\hline
			Actor &
			\begin{itemize}
				\item Patrocinador
			\end{itemize}
			\\
			\hline
			Disparador & El patrocinador elige la opción "dejar de ser sponsor" \\
			\hline
			Inclusiones & N/A \\
			\hline
			Puntos de extensión & N/A
			\\
			\hline
			Precondiciones &  
			\begin{itemize}
				\item Se ha iniciado el SNS con un rol de patrocinador deportivo
				\item Se ha elegido administración de sponsor
			\end{itemize}
			\\
			\hline
			Postcondiciones & 
			\begin{itemize}
				\item El usuario ha dejado de ser patrocinador de un equipo/jugador/evento deportivo
				\item El usuario se encuentra en la pantalla para gestión de patrocinios
			\end{itemize}
			\\
			\hline
			Notas & N/A
			\\
			\hline
		\end{tabular}
		} \\
		\textbf{Fuente}: Autores
	\end{center}
\end{table}

\begin{table}[!htb]
	\caption{CU041-Dejar de ser sponsor: Flujos de hechos}
	\label{tab:cu041_flujo}
	\begin{center}
		\resizebox{15cm}{!}{
		\begin{tabular}{|p{1.5cm}|p{6cm}|p{6.5cm}|}
			\hline
			\multicolumn{3}{|c|}{Detalle de flujo de hechos de caso de uso} \\
			\hline
			Nombre & \multicolumn{2}{|c|}{Nombre del flujo} \\
			\hline
			Paso & Acción del actor & Respuesta del sistema \\
			\hline
			 & & \\
			\hline
		\end{tabular}
		} \\
		\textbf{Fuente}: Autores
	\end{center}
\end{table}

\begin{table}[!htb]
	\caption{CU043-Tracking a jugadores/equipos/eventos deportivos: Descripción}
	\label{tab:cu043_desc}
	\begin{center}
		\resizebox{15cm}{!}{
		\begin{tabular}{|p{4cm}|p{11cm}|}
			\hline
			\multicolumn{2}{|c|}{Descripción de caso de uso} \\
			\hline
			Nombre & Tracking a jugadores/equipos/eventos deportivos \\
			\hline
			Identificador & CU043 \\
			\hline
			Descripción & Permite recibir noticias de jugadores/equipos/eventos deportivos de los que se quiere ser sponsor, siempre que éste actor/evento deportivo tenga habilitada dicha opción \\
			\hline
			Actor &
			\begin{itemize}
				\item Patrocinador
			\end{itemize}
			\\
			\hline
			Disparador & El patrocinador elige hacer tracking de un jugador/equipo/evento deportivo \\
			\hline
			Inclusiones & N/A \\
			\hline
			Puntos de extensión & N/A
			\\
			\hline
			Precondiciones &  
			\begin{itemize}
				\item Se ha iniciado el SNS con un rol de patrocinador deportivo
				\item Se ha elegido administración de sponsor
			\end{itemize}
			\\
			\hline
			Postcondiciones & 
			\begin{itemize}
				\item El usuario ha hecho tracking de un jugador/equipo/evento deportivo
				\item El usuario se encuentra en la pantalla para gestión de patrocinios
			\end{itemize}
			\\
			\hline
			Notas & N/A
			\\
			\hline
		\end{tabular}
		} \\
		\textbf{Fuente}: Autores
	\end{center}
\end{table}

\begin{table}[!htb]
	\caption{CU042-Generación y consulta de historial de sponsor: Flujos de hechos}
	\label{tab:cu042_flujo}
	\begin{center}
		\resizebox{15cm}{!}{
		\begin{tabular}{|p{1.5cm}|p{6cm}|p{6.5cm}|}
			\hline
			\multicolumn{3}{|c|}{Detalle de flujo de hechos de caso de uso} \\
			\hline
			Nombre & \multicolumn{2}{|c|}{Nombre del flujo} \\
			\hline
			Paso & Acción del actor & Respuesta del sistema \\
			\hline
			 & & \\
			\hline
		\end{tabular}
		} \\
		\textbf{Fuente}: Autores
	\end{center}
\end{table}

\begin{table}[!htb]
	\caption{CU043-Tracking a jugadores/equipos/eventos deportivos: Descripción}
	\label{tab:cu043_desc}
	\begin{center}
		\resizebox{15cm}{!}{
		\begin{tabular}{|p{4cm}|p{11cm}|}
			\hline
			\multicolumn{2}{|c|}{Descripción de caso de uso} \\
			\hline
			Nombre & Tracking a jugadores/equipos/eventos deportivos \\
			\hline
			Identificador & CU043 \\
			\hline
			Descripción & Permite recibir noticias de jugadores/equipos/eventos deportivos de los que se quiere ser sponsor, siempre que éste actor/evento deportivo tenga habilitada dicha opción \\
			\hline
			Actor &
			\begin{itemize}
				\item Patrocinador
			\end{itemize}
			\\
			\hline
			Disparador & El patrocinador elige hacer tracking de un jugador/equipo/evento deportivo \\
			\hline
			Inclusiones & N/A \\
			\hline
			Puntos de extensión & N/A
			\\
			\hline
			Precondiciones &  
			\begin{itemize}
				\item Se ha iniciado el SNS con un rol de patrocinador deportivo
				\item Se ha elegido administración de sponsor
			\end{itemize}
			\\
			\hline
			Postcondiciones & 
			\begin{itemize}
				\item El usuario ha hecho tracking de un jugador/equipo/evento deportivo
				\item El usuario se encuentra en la pantalla para gestión de patrocinios
			\end{itemize}
			\\
			\hline
			Notas & N/A
			\\
			\hline
		\end{tabular}
		} \\
		\textbf{Fuente}: Autores
	\end{center}
\end{table}

\begin{table}[!htb]
	\caption{CU043-Tracking a jugadores/equipos/eventos deportivos: Flujos de hechos}
	\label{tab:cu043_flujo}
	\begin{center}
		\resizebox{15cm}{!}{
		\begin{tabular}{|p{1.5cm}|p{6cm}|p{6.5cm}|}
			\hline
			\multicolumn{3}{|c|}{Detalle de flujo de hechos de caso de uso} \\
			\hline
			Nombre & \multicolumn{2}{|c|}{Nombre del flujo} \\
			\hline
			Paso & Acción del actor & Respuesta del sistema \\
			\hline
			 & & \\
			\hline
		\end{tabular}
		} \\
		\textbf{Fuente}: Autores
	\end{center}
\end{table}