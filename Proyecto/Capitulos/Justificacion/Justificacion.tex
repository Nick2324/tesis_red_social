Los humanos, desde siempre en su evolución, han necesitado de mecanismos para comunicarse con sus congéneres. En la actualidad, uno de los mecanismos es el uso de los SNS como facebook y twitter, cada uno de ellos modificando la forma de creación de redes sociales en la actualidad. (Sección \ref{sec:red})

De acuerdo al análisis egocéntrico de las redes sociales de cada individuo, se hace conveniente la utilización de SNS para gestionar las relaciones que un individuo mantiene con otros individuos (sean personas u organizaciones) en los diferentes círculos sociales en los que se mueve. (Sección \ref{sec:egocentrico})

El círculo social o comunidad escogida para el desarrollo propuesto es la comunidad deportiva debido a que hay mucha información dispersa alrededor de internet que es ambigua y a veces inclusive errónea. A su vez, debido a que gran cantidad de deportes no han tenido una acogida grande alrededor del mundo, las comunidades que se mueven sobre uno de esos deportes son más cerradas y, por ende, pequeñas y con poca información para un público que salga de las fronteras de dichas comunidades cerradas. Lo que se quiere con este trabajo es aportar al crecimiento de las redes sociales de las personas que practiquen deporte sin importar si lo hacen a nivel profesional o aficionado por medio de un SNS orientado a los deportes en general.

Un factor de utilización masiva de las SNS es que éstas estén orientadas a un público en particular y aumenten su cobertura dependiendo de su alcance de masa crítica sobre una red social definida \cite{sna_startups}. Al construir, en principio, la red social deportiva enfocada en dos deportes en particular, la probabilidad de ganar la masa crítica es mayor y, por tanto, el SNS desarrollado puede volverse más útil con el tiempo.

La UX de los SNS (visto en la sección \ref{subsec:UX}) es otro factor, debido a que juega un papel importante pues es esta la segunda carta de presentación de un SNS. Algunas de las características que evalúan los usuarios en cuanto a la UX no son suplidas por los SNS actuales – o al menos no parcialmente - , tres de ellas (fundamentales para la acogida de un nuevo SNS) son “curiosity, learning y completeness of the social network”. Así, habiendo analizado 19 SNS orientados al deporte (Tablas \ref{tab:comparacion_redes_1} a \ref{tab:comparacion_redes_5}), se concluyó que fallaban en alguna de las tres características mencionadas.

Tener en cuenta la población a quien va dirigido el SNS a desarrollar es otro factor de éxito. Según \cite{user_behavior_online}, entre los años de adolescencia y los 40 años de edad, las personas acuden con mayor interés al uso de los SNS; al ser la comunidad del deporte comprendida en su mayoría por personas entre la adolescencia y los 40 años, aumenta aún más la probabilidad de alcanzar la masa crítica y volver útil con el tiempo el SNS.

Un último factor, que se observó, afecta la creación de redes sociales (tanto fuera de línea como en línea) (Sección \ref{sec:red}) es la distancia entre cada individual y el posible tipo de enlace que los uniría. Al ver la importancia de manejar SNS que ofrezcan servicios de geolocalización, se ha visto pertinente añadir dicho servicio a la creación del prototipo de SNS orientado a los deportes en general.

También se encontró evidencia de poca utilización de los SNS que no estaban orientadas a móviles. Dichos SNS eran utilizados mucho más por personas
que practican deportes que empiezan a tomar vuelo o deportes poco conocidos (un ejemplo de ello es el padel). El problema con dichos SNS es la
naturaleza nómada de los deportistas. Una solución a la naturaleza nómada de los deportistas y el acercamiento de los últimos a las TICs y, en este caso, a
los SNS deportivos es la aparición y utilización en masa de los smartphones.

En general, solo se encontraron dos redes sociales deportivas orientadas a cualquier deporte asociadas a aplicaciones para smartphones disponibles en el la tienda virtual de Android o en la tienda virtual de Apple (La red social de Fitivity y Huddlers) (Tablas \ref{tab:comparacion_redes_1} a \ref{tab:comparacion_redes_5}). Además, hay una ventaja real en hacer una red social orientada a dispositivos móviles y es la capacidad de movilidad que ellos brindan mientras se está utilizando el servicio \cite{spiderweb}. Dada la falta de aplicaciones móviles en el campo descrito y a su vez la importancia que toman los dispositivos móviles por sus características, se ha decidido hacer el prototipo de SNS orientado al deporte sobre tecnologías móviles.
