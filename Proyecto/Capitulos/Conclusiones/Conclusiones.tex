En cuanto al campo analizado para la obtención de un producto de software que sirva al sector deportivo, con la investigación se pudo observar la importancia que tiene el uso de tecnologías móviles debido a la facilidad de geolocalización que éstas ofrecen. A su vez, teniendo en cuenta la escogencia de una red social que se orienta a un sector de edades que no sobrepasan los 40 años, se puede decir que el factor de éxito podrá ser mayor por la demanda de redes sociales por personas por debajo de dicha edad.

En vista de lo investigado, los autores pudieron dar una filosofía al software desarrollado, orientando éste hacia el paradigma orientado a servicios, aprovechando la capacidad de éste para el desarrollo de servicios reusables, descubribles, con nivel de abstracción logrado gracias a la orquestación de sus servicios (composición de los mismos).

A lo largo del proceso investigativo, se observó la importancia que tienen las arquitecturas a la hora de dar al desarrollo una filosofía que se seguirá por el resto del proyecto. Desarrollando toda la arquitectura, con ayuda del estandar Archimate 2.0, para las funcionalidades identificadas en el desarrollo del estado del arte, se pudo ver de forma ilustrativa el proceso que conllevaba cada una de ellas y ésto, en compañía del diseño a la par con los diagramas de casos de uso, sentó bases para lo que sería la escogencia de las funcionalidades críticas del SNS para desarrollar un primer prototipo y, así, la escogencia de los servicios que serían expuestos para ésta primera fase.

Asímismo, en la etapa de modelado de interfaces de usuario, se puede notar cómo se comienza a moldear una filosofía (más allá de la impuesta por las tecnologías Android escogídas) que afectará de forma crítica tanto la evolución del prototipo hacia un producto listo para salir al mercado como el éxito de éstas por concepto de usabilidad del software.

Debido al tiempo limitado con el que los autores contaron para el desarrollo del prototipo, se dio por sentado que no era posible conseguir abarcar todos los requerimientos no funcionales que, al final, se necesitarían en el modelo que se lanzaría al mercado. Por ésta razón, los autores decidieron ceñirse a sentar bases de cada uno que, en posteriores entregas, ayudase a conseguir la completitud de los requerimientos en su totalidad.