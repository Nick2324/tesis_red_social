Luego de haber consignado los requerimientos funcionales en la sección \ref{subsec:requerimientos_funcionales} y de haber formado, con ellos, la arquitectura tratada en el capítulo \ref{chap:arquitectura}, se retrató cada funcionalidad encontrada y que sería, a final de cuentas, soportada por el SNS en los casos de uso a continuación.

Los diagramas estarán dividos por cada módulo de gestión identificado basado cada uno en la especificación de requerimientos.

En \ref{app:cu_tablas}, el lector podrá encontrar la descripción de cada caso de uso en todos los módulos tenidos en cuenta en el SNS (los que se implementarán y los restantes).

\section{Módulo de gestión de administracion de deportes}


\begin{figure}[!htb]
  \begin{center}
    \includegraphics[width=11cm]{./imagenes/casos_uso/administracion_deportes.png}
    \caption{Módulo de gestión de administracion de deportes}
    \label{fig:cu_admin_dep}
    \textbf{Fuente:} Autores
  \end{center}
\end{figure}

Este módulo ofrece funcionalidades a los actores del sistema (especialmente el administrador) para administrar los deportes que éste juega, permitiendo al jugador dar información adicional de él en cada uno de los deportes que éste juega.

\section{Módulo de gestión de usuarios}

\begin{figure}[!htb]
  \begin{center}
    \includegraphics[width=11cm]{./imagenes/casos_uso/gestion_usuarios.png}
    \caption{Módulo de gestión de self-expression}
    \label{fig:cu_self_shar}
    \textbf{Fuente:} Autores
  \end{center}
\end{figure}

Este modulo ofrece a los usuarios la funcionalidad de administrar y gestionar su información como usuario de la aplicación (datos personales, datos de contacto)
