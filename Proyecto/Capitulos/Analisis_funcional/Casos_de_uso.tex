Luego de haber consignado los requerimientos funcionales en la sección \ref{subsec:requerimientos_funcionales} y de haber formado, con ellos, la arquitectura tratada en el capítulo \ref{chap:arquitectura}, se retrató cada funcionalidad encontrada y que sería, a final de cuentas, soportada por el SNS en los casos de uso a continuación. En las figuras \ref{fig:cu1} a \ref{fig:cu70} se puede observar gráficamente la configuración de los casos de uso. La descripción de los casos de uso se hace en tres fases: La primera fase, describe aspectos no-dinámicos del caso de uso; la segunda fase comprende un flujo de hechos para el caso de uso; la tercera fase comprende la descripción de las excepciones que podría causar el caso de uso, haciendo referencia también a cual flujo es afectado con la excepción descrita.

Los diagramas estarán dividos por cada módulo de gestión identificado basado cada uno en la especificación de requerimientos.

\begin{table}[!htb]
	\caption{CU001-NOMBRE: Descripción}
	\label{tab:cu001_desc}
	\begin{center}
		\resizebox{15cm}{!}{
		\begin{tabular}{|p{4cm}|p{11cm}|}
			\hline
			\multicolumn{2}{|c|}{Descripción de caso de uso} \\
			\hline
			Nombre & \\
			\hline
			Identificador & \\
			\hline
			Descripción & \\
			\hline
			Actor & \\
			\hline
			Disparador & \\
			\hline
			Inclusiones & \\
			\hline
			Puntos de extensión & \\
			\hline
			Precondiciones & \\
			\hline
			Postcondiciones & \\
			\hline
			Notas & \\
			\hline
		\end{tabular}
		} \\
		\textbf{Fuente}: Autores
	\end{center}
\end{table}

\begin{table}[!htb]
	\caption{CU001-NOMBRE: Flujos de hechos}
	\label{tab:cu001_flujo}
	\begin{center}
		\resizebox{15cm}{!}{
		\begin{tabular}{|p{1.5cm}|p{6cm}|p{6.5cm}|}
			\hline
			\multicolumn{3}{|c|}{Detalle de flujo de hechos de caso de uso} \\
			\hline
			Nombre & \multicolumn{2}{|c|}{Nombre del flujo} \\
			\hline
			Paso & Acción del actor & Respuesta del sistema \\
			\hline
			 & & \\
			\hline
		\end{tabular}
		} \\
		\textbf{Fuente}: Autores
	\end{center}
\end{table}

\section{Módulo de administración de entrenadores}

%\input{./imagenes/casos_uso/gestion_entrenador.png}

Sobre éste módulo de entrenadores se han recopilado las funcionalidades que pretende ofrecer el SNS deportivo a los entrenadores. Dichas funcionalidades resumen la capacidad de los actores deportivos entrenadores de ofrecer su servicio deportivo a la comunidad deportiva y de hacer el seguimiento de entrenamientos con sus entrenados desde el SNS.

\clearpage

\begin{table}[!htb]
	\caption{CU008-Gestionar entrenadores: Descripción}
	\label{tab:cu008_desc}
	\begin{center}
		\resizebox{15cm}{!}{
		\begin{tabular}{|p{4cm}|p{11cm}|}
			\hline
			\multicolumn{2}{|c|}{Descripción de caso de uso} \\
			\hline
			Nombre & Gestión de entrenadores \\
			\hline
			Identificador & CU008 \\
			\hline
			Descripción & Gestiona las opciones dadas a los entrenadores deportivos en el SNS \\
			\hline
			Actor &
			\begin{itemize}
				\item Entrenador
				\item Jugador
				\item Organización
			\end{itemize}			 \\
			\hline
			Disparador & Actor con rol de entrenador deportivo elige la opción de gestión de entrenadores \\
			\hline
			Inclusiones &  \\
			\hline
			Puntos de extensión &  \\
			\hline
			Precondiciones &  
			\begin{itemize}
				\item La aplicación ha sido cargada por un actor con rol de entrenador deportivo
			\end{itemize}
			\\
			\hline
			Postcondiciones & \\
			\hline
			Notas & 
			\begin{itemize}
				\item Generalización de:
				\begin{itemize}
					\item Convertirse en entrenador de jugador/equipo
					\item Dejar de entrenar a jugador/equipo
					\item Gestionar entrenamientos de jugador/equipo
					\item Promocionar servicios de entrenamiento
				\end{itemize}
			\end{itemize}
			\\
			\hline
		\end{tabular}
		} \\
		\textbf{Fuente}: Autores
	\end{center}
\end{table}

\begin{table}[!htb]
	\caption{CU008-Gestionar entrenadores: Flujos de hechos}
	\label{tab:cu008_flujo}
	\begin{center}
		\resizebox{15cm}{!}{
		\begin{tabular}{|p{1.5cm}|p{6cm}|p{6.5cm}|}
			\hline
			\multicolumn{3}{|c|}{Detalle de flujo de hechos de caso de uso} \\
			\hline
			Nombre & \multicolumn{2}{|c|}{Nombre del flujo} \\
			\hline
			Paso & Acción del actor & Respuesta del sistema \\
			\hline
			1 & El usuario pulsa el botón de gestión de opciones de entrenador & El sistema muestra la interfaz de gestión de opciones de entrenador \\
			\hline
		\end{tabular}
		} \\
		\textbf{Fuente}: Autores
	\end{center}
\end{table}

\begin{table}[!htb]
	\caption{CU009-Convertirse en entrenador de jugador/equipo: Descripción}
	\label{tab:cu009_desc}
	\begin{center}
		\resizebox{15cm}{!}{
		\begin{tabular}{|p{4cm}|p{11cm}|}
			\hline
			\multicolumn{2}{|c|}{Descripción de caso de uso} \\
			\hline
			Nombre & Convertirse en entrenador de jugador/equipo \\
			\hline
			Identificador & CU009 \\
			\hline
			Descripción & Ayuda a pedir/aceptar la petición de ser entrenador de un jugador o equipo en el SNS \\
			\hline
			Actor &
			\begin{itemize}
				\item Entrenador
				\item Jugador
				\item Organización
			\end{itemize}			 \\
			\hline
			Disparador & Actor con rol de entrenador deportivo elige la opción de entrenar a equipo o jugador \\
			\hline
			Inclusiones &  \\
			\hline
			Puntos de extensión &  \\
			\hline
			Precondiciones &  
			\begin{itemize}
				\item La aplicación ha sido cargada por un actor con rol de entrenador deportivo
			\end{itemize}
			\\
			\hline
			Postcondiciones & 
			\begin{itemize}
				\item El usuario ha elegido ser entrenador de un jugador o equipo deportivo
			\end{itemize}
			\\
			\hline
			Notas & 
			\\
			\hline
		\end{tabular}
		} \\
		\textbf{Fuente}: Autores
	\end{center}
\end{table}

\begin{table}[!htb]
	\caption{CU009-Convertirse en entrenador de jugador/equipo: Flujos de hechos}
	\label{tab:cu009_flujo}
	\begin{center}
		\resizebox{15cm}{!}{
		\begin{tabular}{|p{1.5cm}|p{6cm}|p{6.5cm}|}
			\hline
			\multicolumn{3}{|c|}{Detalle de flujo de hechos de caso de uso} \\
			\hline
			Nombre & \multicolumn{2}{|c|}{Nombre del flujo} \\
			\hline
			Paso & Acción del actor & Respuesta del sistema \\
			\hline
			1 & El usuario eligió la gestión de opciones de entrenador & El sistema mostró la interfaz de gestión de opciones de usuario \\
			\hline
			2 & El usuario elige entrenar a un equipo o jugador en la red social & El sistema muestra la interfaz por la que el usuario puede buscar a otro en la red social para entrenarlo \\
			\hline
			3 & El usuario busca al posible entrenado en la red social & El sistema realiza la búsqueda del usuario a entrenar con el nombre proporcionado por el usuario \\
			\hline
			4 & El usuario elige el jugador o equipo a ser entrenado & El sistema envía una notificación de petición de entrenamiento al equipo o jugador elegido o, si la petición ya ha sido enviado por el jugador o equipo, agrega al usuario como entrenador de este \\
			\hline
			5 & & El sistema muestra un aviso de éxito o no de la operación \\
			\hline
			6 & El usuario continua pulsando el botón designado en el aviso para poder continuar & El sistema muestra la interfaz de convertirse en entrenador de jugador/equipo \\
			\hline
		\end{tabular}
		} \\
		\textbf{Fuente}: Autores
	\end{center}
\end{table}

\begin{table}[!htb]
	\caption{CU010-Dejar de entrenar a jugador/equipo: Descripción}
	\label{tab:cu010_desc}
	\begin{center}
		\resizebox{15cm}{!}{
		\begin{tabular}{|p{4cm}|p{11cm}|}
			\hline
			\multicolumn{2}{|c|}{Descripción de caso de uso} \\
			\hline
			Nombre & Dejar de entrenar a jugador/equipo \\
			\hline
			Identificador & CU010 \\
			\hline
			Descripción & Permite dejar de ser entrenador de un jugador o equipo en el SNS \\
			\hline
			Actor &
			\begin{itemize}
				\item Entrenador
				\item Jugador
				\item Organización
			\end{itemize}			 
			\\
			\hline
			Disparador & Actor con rol de entrenador deportivo elige la opción de dejar de ser entrenador de equipo o jugador \\
			\hline
			Inclusiones &  \\
			\hline
			Puntos de extensión &  \\
			\hline
			Precondiciones &  
			\begin{itemize}
				\item La aplicación ha sido cargada por un actor con rol de entrenador deportivo
			\end{itemize}
			\\
			\hline
			Postcondiciones & 
			\begin{itemize}
				\item El usuario ha elegido dejar de ser entrenador de un jugador o equipo deportivo
			\end{itemize}
			\\
			\hline
			Notas & 
			\\
			\hline
		\end{tabular}
		} \\
		\textbf{Fuente}: Autores
	\end{center}
\end{table}

\begin{table}[!htb]
	\caption{CU010-Dejar de entrenar a jugador/equipo: Flujos de hechos}
	\label{tab:cu010_flujo}
	\begin{center}
		\resizebox{15cm}{!}{
		\begin{tabular}{|p{1.5cm}|p{6cm}|p{6.5cm}|}
			\hline
			\multicolumn{3}{|c|}{Detalle de flujo de hechos de caso de uso} \\
			\hline
			Nombre & \multicolumn{2}{|c|}{Nombre del flujo} \\
			\hline
			Paso & Acción del actor & Respuesta del sistema \\
			\hline
			1 & El usuario eligió la gestión de opciones de entrenador & El sistema mostró la interfaz de gestión de opciones de entrenador \\
			\hline
			2 & El usuario elige la opción de abandonar el entrenamiento de un jugador o de un equipo & El sistema muestra la interfaz que permite abandonar el entrenamiento de un jugador o de un equipo \\
			\hline
			3 & El usuario ingresa una cadena de búsqueda para buscar, entre los entrenados, el entrenado que desea abandonar & El sistema muestra al usuario las posibles opciones de entrenados que encajan con la cadena de búsqueda proporcionada \\\\
			\hline
			4 & El usuario elige abandonar el entrenamiento de un equipo o jugador & El sistema muestra una alerta de la acción que está a punto de realizarse \\
			\hline
			5 & El usuario continua, pasando la alerta & El sistema posiciona al usuario en la interfaz para abandonar el entrenamiento de un jugador o un equipo \\
			\hline
		\end{tabular}
		} \\
		\textbf{Fuente}: Autores
	\end{center}
\end{table}

\begin{table}[!htb]
	\caption{CU011-Promocionar servicios de entrenamiento: Descripción}
	\label{tab:cu011_desc}
	\begin{center}
		\resizebox{15cm}{!}{
		\begin{tabular}{|p{4cm}|p{11cm}|}
			\hline
			\multicolumn{2}{|c|}{Descripción de caso de uso} \\
			\hline
			Nombre & Promocionar servicios de entrenamiento \\
			\hline
			Identificador & CU011 \\
			\hline
			Descripción & Permite la promoción de servicios de entrenamiento hacia los equipos/jugadores posiblemente interesados, así como también sobre las noticias nuevas en la red social \\
			\hline
			Actor &
			\begin{itemize}
				\item Entrenador
				\item Jugador
				\item Organización
			\end{itemize}			 
			\\
			\hline
			Disparador & Actor con rol de entrenador deportivo elige la opción de promoción de sus servicios de entrenamiento deportivo \\
			\hline
			Inclusiones &  \\
			\hline
			Puntos de extensión &  \\
			\hline
			Precondiciones &  
			\begin{itemize}
				\item La aplicación ha sido cargada por un actor con rol de entrenador deportivo
			\end{itemize}
			\\
			\hline
			Postcondiciones & 
			\begin{itemize}
				\item El usuario ha promocionado sus servicios
			\end{itemize}
			\\
			\hline
			Notas & 
			\\
			\hline
		\end{tabular}
		} \\
		\textbf{Fuente}: Autores
	\end{center}
\end{table}

\begin{table}[!htb]
	\caption{CU011-Promocionar servicios de entrenamiento: Flujos de hechos}
	\label{tab:cu011_flujo}
	\begin{center}
		\resizebox{15cm}{!}{
		\begin{tabular}{|p{1.5cm}|p{6cm}|p{6.5cm}|}
			\hline
			\multicolumn{3}{|c|}{Detalle de flujo de hechos de caso de uso} \\
			\hline
			Nombre & \multicolumn{2}{|c|}{Nombre del flujo} \\
			\hline
			Paso & Acción del actor & Respuesta del sistema \\
			\hline
			 1 & El usuario eligió la gestión de opciones de entrenador & El sistema mostró la interfaz de gestión de opciones de entrenador \\
			 \hline
			 2 & El usuario elige la opción de promoción de servicios de entrenamiento & El sistema muestra la interfaz de promoción de servicios de entrenamiento \\
			 \hline
			 3 & El usuario ingresa una cadena de búsqueda para buscar al usuario en la red social al que quiere ofrecer sus servicios & El sistema muestra todas las posibilidades que encajan con la busqueda realizada \\
			 4 & El usuario elige un usuario de la red social al cual promocionar sus servicios & El sistema realiza las acciones de promoción \\
			 \hline
			 5 & & El sistema muestra un elemento emergente que expresa el éxito o fracaso de la operación \\
			 \hline
			 6 & El usuario continua pasando el elemento emergente & El sistema situa al usuario en la interfaz de promoción de servicios de entrenamiento \\
			\hline
		\end{tabular}
		} \\
		\textbf{Fuente}: Autores
	\end{center}
\end{table}

\begin{table}[!htb]
	\caption{CU012-Gestionar planes de entrenamientos: Descripción}
	\label{tab:cu012_desc}
	\begin{center}
		\resizebox{15cm}{!}{
		\begin{tabular}{|p{4cm}|p{11cm}|}
			\hline
			\multicolumn{2}{|c|}{Descripción de caso de uso} \\
			\hline
			Nombre & Gestionar planes de entrenamientos \\
			\hline
			Identificador & CU012 \\
			\hline
			Descripción & Permite la gestión de los entrenamientos que esté realizando el entrenador a los diferentes actores de la red social a quienes les esté prestando el servicio \\
			\hline
			Actor &
			\begin{itemize}
				\item Entrenador
				\item Jugador
				\item Organización
			\end{itemize}			 
			\\
			\hline
			Disparador & Actor deportivo con rol de entrenador elige gestionar los entrenamientos de jugadores/equipos a los que él les brinda servicios \\
			\hline
			Inclusiones &  \\
			\hline
			Puntos de extensión &  \\
			\hline
			Precondiciones &  
			\begin{itemize}
				\item La aplicación ha sido cargada por un actor con rol de entrenador deportivo
			\end{itemize}
			\\
			\hline
			Postcondiciones & \\
			\hline
			Notas & 
			\begin{itemize}
				\item Generalización de:
				\begin{itemize}
					\item Llevar seguimiento de entrenamientos
					\item Administrar plan de entrenamiento
				\end{itemize}
			\end{itemize}
			\\
			\hline
		\end{tabular}
		} \\
		\textbf{Fuente}: Autores
	\end{center}
\end{table}

\begin{table}[!htb]
	\caption{CU012-Gestionar planes de entrenamientos: Flujos de hechos}
	\label{tab:cu012_flujo}
	\begin{center}
		\resizebox{15cm}{!}{
		\begin{tabular}{|p{1.5cm}|p{6cm}|p{6.5cm}|}
			\hline
			\multicolumn{3}{|c|}{Detalle de flujo de hechos de caso de uso} \\
			\hline
			Nombre & \multicolumn{2}{|c|}{Nombre del flujo} \\
			\hline
			Paso & Acción del actor & Respuesta del sistema \\
			\hline
			1 & El usuario eligió la gestión de opciones de entrenador & El sistema mostró la interfaz de gestión de opciones de entrenador \\
			\hline
			2 & El usuario elige la opción de gestión de planes deportivos & El sistema muestra la interfaz de gestión de planes deportivos \\			
			\hline
		\end{tabular}
		} \\
		\textbf{Fuente}: Autores
	\end{center}
\end{table}


\begin{table}[!htb]
	\caption{CU013-Llevar seguimiento de entrenamientos: Descripción}
	\label{tab:cu013_desc}
	\begin{center}
		\resizebox{15cm}{!}{
		\begin{tabular}{|p{4cm}|p{11cm}|}
			\hline
			\multicolumn{2}{|c|}{Descripción de caso de uso} \\
			\hline
			Nombre & Llevar seguimiento de entrenamientos/equipo \\
			\hline
			Identificador & CU013 \\
			\hline
			Descripción & Permite al entrenador la comunicación-evaluación del entrenamiento directamente con sus entrenados \\
			\hline
			Actor &
			\begin{itemize}
				\item Entrenador
				\item Jugador
				\item Organización
			\end{itemize}			 
			\\
			\hline
			Disparador & Actor deportivo con rol de entrenador elige llevar a cabo un proceso de comunicación entre él y los jugadores/equipos a los que él les brinda servicios por concepto del entrenamiento realizado \\
			\hline
			Inclusiones &  \\
			\hline
			Puntos de extensión & 
			Administrar plan deportivo \\
			\hline
			Precondiciones &  
			\begin{itemize}
				\item La aplicación ha sido cargada por un actor con rol de entrenador deportivo
				\item Se ha elegido la opción de gestionar entrenadores.
			\end{itemize}
			\\
			\hline
			Postcondiciones & 
			\begin{itemize}
				\item El usuario ha llevado a cabo el proceso de comunicación entre él y sus entrenados.
				\item El usuario se encuentra en la pantalla de seguimiento de entrenamientos.
			\end{itemize}
			\\
			\hline
			Notas & 
			\\
			\hline
		\end{tabular}
		} \\
		\textbf{Fuente}: Autores
	\end{center}
\end{table}

\begin{table}[!htb]
	\caption{CU013-Llevar seguimiento de entrenamientos: Flujos de hechos}
	\label{tab:cu013_flujo}
	\begin{center}
		\resizebox{15cm}{!}{
		\begin{tabular}{|p{1.5cm}|p{6cm}|p{6.5cm}|}
			\hline
			\multicolumn{3}{|c|}{Detalle de flujo de hechos de caso de uso} \\
			\hline
			Nombre & \multicolumn{2}{|c|}{Nombre del flujo} \\
			\hline
			Paso & Acción del actor & Respuesta del sistema \\
			\hline
			1 & El usuario ha elegido gestionar planes de entrenamiento & El sistema ha mostrado la interfaz de gestión de planes de entrenamiento, mostrando los entrenados \\
			\hline
			2 & El usuario elige uno de los entrenados para observar su rendimiento en el entrenamiento & El sistema muestra la interfaz con los datos de entrenamiento del entrenado y las opciones asociadas \\
			\hline
		\end{tabular}
		} \\
		\textbf{Fuente}: Autores
	\end{center}
\end{table}


\begin{table}[!htb]
	\caption{CU014-Administrar plan de entrenamiento: Descripción}
	\label{tab:cu014_desc}
	\begin{center}
		\resizebox{15cm}{!}{
		\begin{tabular}{|p{4cm}|p{11cm}|}
			\hline
			\multicolumn{2}{|c|}{Descripción de caso de uso} \\
			\hline
			Nombre & Administrar plan de entrenamiento \\
			\hline
			Identificador & CU014 \\
			\hline
			Descripción & Permite al entrenador la administración de los diferentes planes de entrenamiento que éste quiera crear o haya creado \\
			\hline
			Actor &
			\begin{itemize}
				\item Entrenador
				\item Jugador
				\item Organización
			\end{itemize}			 
			\\
			\hline
			Disparador & Actor deportivo con rol de entrenador elige administrar sus planes de entrenamiento \\
			\hline
			Inclusiones &  \\
			\hline
			Puntos de extensión & \\
			\hline
			Precondiciones &  
			\begin{itemize}
				\item La aplicación ha sido cargada por un actor con rol de entrenador deportivo
				\item Se ha elegido la opción de administrar plan de entrenamiento.
			\end{itemize}
			\\
			\hline
			Postcondiciones & 
			\begin{itemize}
				\item El usuario se encuentra en la pantalla de administración de planes de entrenamiento.
			\end{itemize}
			\\
			\hline
			Notas & 
			\begin{itemize}
				\item Generalización de:
				\begin{itemize}
					\item Crear plan de entrenamiento
					\item Añadir rutina a plan de entrenamiento
					\item Modificar rutina de plan de entrenamiento
					\item Eliminar rutina de plan de entrenamiento
					\item Asignar plan de entrenamiento a jugador/equipo
					\item Retirar plan de entrenamiento a jugador/equipo
					\item Eliminar plan de entrenamiento
				\end{itemize}
			\end{itemize}
			\\
			\hline
		\end{tabular}
		} \\
		\textbf{Fuente}: Autores
	\end{center}
\end{table}

\begin{table}[!htb]
	\caption{CU014-Administrar plan de entrenamiento: Flujos de hechos}
	\label{tab:cu014_flujo}
	\begin{center}
		\resizebox{15cm}{!}{
		\begin{tabular}{|p{1.5cm}|p{6cm}|p{6.5cm}|}
			\hline
			\multicolumn{3}{|c|}{Detalle de flujo de hechos de caso de uso} \\
			\hline
			Nombre & \multicolumn{2}{|c|}{Nombre del flujo} \\
			\hline
			Paso & Acción del actor & Respuesta del sistema \\
			\hline
			1 & El usuario ha elegido gestionar los planes de entrenamiento & El sistema ha mostrado la interfaz de administración de planes, mostrando los planes de entrenamiento existentes \\
			\hline
		\end{tabular}
		} \\
		\textbf{Fuente}: Autores
	\end{center}
\end{table}

\begin{table}[!htb]
	\caption{CU015-Crear plan de entrenamiento: Descripción}
	\label{tab:cu015_desc}
	\begin{center}
		\resizebox{15cm}{!}{
		\begin{tabular}{|p{4cm}|p{11cm}|}
			\hline
			\multicolumn{2}{|c|}{Descripción de caso de uso} \\
			\hline
			Nombre & Crear plan de entrenamiento \\
			\hline
			Identificador & CU015 \\
			\hline
			Descripción & Permite al entrenador crear un plan de entrenamiento \\
			\hline
			Actor &
			\begin{itemize}
				\item Entrenador
				\item Jugador
				\item Organización
			\end{itemize}			 
			\\
			\hline
			Disparador & Actor deportivo con rol de entrenador elige crear un plan de entrenamiento \\
			\hline
			Inclusiones & \\
			\hline
			Puntos de extensión & \\
			\hline
			Precondiciones &  
			\begin{itemize}
				\item La aplicación ha sido cargada por un actor con rol de entrenador deportivo
				\item Se ha elegido la opción de crear plan de entrenamiento.
			\end{itemize}
			\\
			\hline
			Postcondiciones & 
			\begin{itemize}
				\item El usuario crea un plan de entrenamiento
				\item El usuario se encuentra en la pantalla de administración de planes de entrenamiento.
			\end{itemize}
			\\
			\hline
			Notas & \\
			\hline
		\end{tabular}
		} \\
		\textbf{Fuente}: Autores
	\end{center}
\end{table}

\begin{table}[!htb]
	\caption{CU015-Crear plan de entrenamiento: Flujos de hechos}
	\label{tab:cu015_flujo}
	\begin{center}
		\resizebox{15cm}{!}{
		\begin{tabular}{|p{1.5cm}|p{6cm}|p{6.5cm}|}
			\hline
			\multicolumn{3}{|c|}{Detalle de flujo de hechos de caso de uso} \\
			\hline
			Nombre & \multicolumn{2}{|c|}{Nombre del flujo} \\
			\hline
			Paso & Acción del actor & Respuesta del sistema \\
			\hline
			1 & El usuario hubo elegido gestionar planes deportivos & El sistema ha mostrado una lista con los planes de entrenamiento \\
			\hline
			2 & El usuario pulsa el botón de creación de planes deportivos & El sistema muestra la interfaz de creación de planes deportivos \\
			\hline
			3 & El usuario pulsa el botón de creación de planes deportivos & El sistema muestra la interfaz de creación de planes deportivos \\
			\hline
			4 & El usuario ingresa todos los datos referentes a la creación correcta del plan & \\
			\hline
			5 & El usuario decide guardar los cambios & El sistema guarda los datos del plan de entrenamiento \\
			\hline
			6 & & El sistema muestra un elemento emergente informando del éxito o fracaso de la operación \\
			\hline
			7 & El actor decide continuar & El sistema muestra la interfaz de gestión de planes deportivos \\
			\hline
		\end{tabular}
		} \\
		\textbf{Fuente}: Autores
	\end{center}
\end{table}
\clearpage
\begin{table}[!htb]
	\caption{CU016-Crear rutina de plan de entrenamiento: Descripción}
	\label{tab:cu016_desc}
	\begin{center}
		\resizebox{15cm}{!}{
		\begin{tabular}{|p{4cm}|p{11cm}|}
			\hline
			\multicolumn{2}{|c|}{Descripción de caso de uso} \\
			\hline
			Nombre & Crear rutina de plan de entrenamiento \\
			\hline
			Identificador & CU016 \\
			\hline
			Descripción & Permite al entrenador crear una rutina del plan de entrenamiento mientras lo está creando/actualizando \\
			\hline
			Actor &
			\begin{itemize}
				\item Entrenador
				\item Jugador
				\item Organización
			\end{itemize}			 
			\\
			\hline
			Disparador & Actor deportivo con rol de entrenador elige crear una rutina de plan de entrenamiento \\
			\hline
			Inclusiones & \\
			\hline
			Puntos de extensión & \\
			\hline
			Precondiciones &  
			\begin{itemize}
				\item La aplicación ha sido cargada por un actor con rol de entrenador deportivo
			\end{itemize}
			\\
			\hline
			Postcondiciones & 
			\begin{itemize}
				\item El usuario modifica una rutina del plan de entrenamiento
			\end{itemize}
			\\
			\hline
			Notas & \\
			\hline
		\end{tabular}
		} \\
		\textbf{Fuente}: Autores
	\end{center}
\end{table}

\begin{table}[!htb]
	\caption{CU016-Crear rutina de plan de entrenamiento: Flujos de hechos}
	\label{tab:cu016_flujo}
	\begin{center}
		\resizebox{15cm}{!}{
		\begin{tabular}{|p{1.5cm}|p{6cm}|p{6.5cm}|}
			\hline
			\multicolumn{3}{|c|}{Detalle de flujo de hechos de caso de uso} \\
			\hline
			Nombre & \multicolumn{2}{|c|}{Nombre del flujo} \\
			\hline
			Paso & Acción del actor & Respuesta del sistema \\
			\hline
			1 & El usuario ha elegido un plan de entrenamiento en específico & El sistema ha mostrado los detalles del plan de entrenamiento deportivo \\
			\hline
			2 & El usuario pulsa el botón de creación de rutina & El sistema muestra la interfaz de creación/modificación de rutinas \\
			\hline
			3 & El usuario ingresa la información deseada a la rutina a crear & \\
			\hline
			4 & El usuario confirma la creación de la rutina & El sistema crea la rutina y la asigna inmediatamente al plan deportivo \\
			\hline
			5 & El sistema muestra un elemento emergente que informa del éxito o fracaso de la operación & \\
			\hline
			6 & El usuario continua & El sistema sitúa al usuario en los detalles del plan de entrenamiento \\
			\hline
		\end{tabular}
		} \\
		\textbf{Fuente}: Autores
	\end{center}
\end{table}

\begin{table}[!htb]
	\caption{CU017-Añadir rutina a plan de entrenamiento: Descripción}
	\label{tab:cu017_desc}
	\begin{center}
		\resizebox{15cm}{!}{
		\begin{tabular}{|p{4cm}|p{11cm}|}
			\hline
			\multicolumn{2}{|c|}{Descripción de caso de uso} \\
			\hline
			Nombre & Añadir rutina a plan de entrenamiento \\
			\hline
			Identificador & CU017 \\
			\hline
			Descripción & Permite al entrenador añadir una rutina al plan de entrenamiento. En caso de que la rutina no haya sido creada, el entrenador podrá crearla \\
			\hline
			Actor &
			\begin{itemize}
				\item Entrenador
				\item Jugador
				\item Organización
			\end{itemize}			 
			\\
			\hline
			Disparador & Actor deportivo con rol de entrenador elige añadir rutina a plan de entrenamiento \\
			\hline
			Inclusiones & \\
			\hline
			Puntos de extensión & \\
			\hline
			Precondiciones &  
			\begin{itemize}
				\item La aplicación ha sido cargada por un actor con rol de entrenador deportivo
			\end{itemize}
			\\
			\hline
			Postcondiciones & 
			\begin{itemize}
				\item El usuario añade una rutina al plan de entrenamiento
			\end{itemize}
			\\
			\hline
			Notas & \\
			\hline
		\end{tabular}
		} \\
		\textbf{Fuente}: Autores
	\end{center}
\end{table}

\begin{table}[!htb]
	\caption{CU017-Añadir rutina a plan de entrenamiento: Flujos de hechos}
	\label{tab:cu017_flujo}
	\begin{center}
		\resizebox{15cm}{!}{
		\begin{tabular}{|p{1.5cm}|p{6cm}|p{6.5cm}|}
			\hline
			\multicolumn{3}{|c|}{Detalle de flujo de hechos de caso de uso} \\
			\hline
			Nombre & \multicolumn{2}{|c|}{Nombre del flujo} \\
			\hline
			Paso & Acción del actor & Respuesta del sistema \\
			\hline
			1 & El usuario ha elegido un plan de entrenamiento en específico & El sistema ha mostrado los detalles del plan de entrenamiento escogido. El sistema carga las rutinas incluidas en el plan de entrenamiento \\
			\hline
			2 & El usuario pulsa el botón de adición de rutina & El sistema muestra la interfaz de adición de rutinas de entrenamiento a un plan deportivo \\
			\hline
			3 & El usuario busca la rutina que desea añadir & El sistema muestra el resultado de la búsqueda de la rutina \\
			\hline
			4 & El usuario elige la rutina que desea añadir al plan de entrenamiento & El sistema añade la rutina al plan \\
			\hline
			5 & & El sistema muestra un elemento emergente que informa del éxito o fracaso de la operación \\
			\hline
			6 & El usuario continua & El sistema sitúa al usuario en los detalles del plan de entrenamiento \\
			\hline
		\end{tabular}
		} \\
		\textbf{Fuente}: Autores
	\end{center}
\end{table}

\begin{table}[!htb]
	\caption{CU018-Modifcar rutina de plan de entrenamiento: Descripción}
	\label{tab:cu018_desc}
	\begin{center}
		\resizebox{15cm}{!}{
		\begin{tabular}{|p{4cm}|p{11cm}|}
			\hline
			\multicolumn{2}{|c|}{Descripción de caso de uso} \\
			\hline
			Nombre & Modifcar rutina de plan de entrenamiento \\
			\hline
			Identificador & CU018 \\
			\hline
			Descripción & Permite al entrenador modificar una rutina del plan de entrenamiento \\
			\hline
			Actor &
			\begin{itemize}
				\item Entrenador
				\item Jugador
				\item Organización
			\end{itemize}			 
			\\
			\hline
			Disparador & Actor deportivo con rol de entrenador elige modifcar rutina de plan de entrenamiento \\
			\hline
			Inclusiones & \\
			\hline
			Puntos de extensión & \\
			\hline
			Precondiciones &  
			\begin{itemize}
				\item La aplicación ha sido cargada por un actor con rol de entrenador deportivo
			\end{itemize}
			\\
			\hline
			Postcondiciones & 
			\begin{itemize}
				\item El usuario modifica una rutina del plan de entrenamiento
			\end{itemize}
			\\
			\hline
			Notas & \\
			\hline
		\end{tabular}
		} \\
		\textbf{Fuente}: Autores
	\end{center}
\end{table}

\begin{table}[!htb]
	\caption{CU018-Modifcar rutina de plan de entrenamiento: Flujos de hechos}
	\label{tab:cu018_flujo}
	\begin{center}
		\resizebox{15cm}{!}{
		\begin{tabular}{|p{1.5cm}|p{6cm}|p{6.5cm}|}
			\hline
			\multicolumn{3}{|c|}{Detalle de flujo de hechos de caso de uso} \\
			\hline
			Nombre & \multicolumn{2}{|c|}{Nombre del flujo} \\
			\hline
			Paso & Acción del actor & Respuesta del sistema \\
			\hline
			1 & El usuario ha elegido un plan de entrenamiento en específico & El sistema ha mostrado los detalles del plan deportivo elegido \\
			\hline
			2 & El usuario busca una rutina asignada al plan de entrenamiento deportivo elegido & El sistema muestra al usuario el resultado de la búsqueda de la rutina \\
			\hline
			3 & El usuario elige la rutina que desea modificar & El sistema muestra los detalles de la rutina a modificar \\
			\hline
			4 & El usuario hace las modificaciones que desea a la rutina escogida y confirma & El sistema guarda las modificaciones hechas por el usuario \\
			\hline
			5 & & El sistema muestra un elemento emergente que informa del éxito o fracaso de la operación \\
			\hline
			6 & El usuario continua & El sistema sitúa al usuario en los detalles del plan de entrenamiento \\
			\hline
		\end{tabular}
		} \\
		\textbf{Fuente}: Autores
	\end{center}
\end{table}

\begin{table}[!htb]
	\caption{CU019-Eliminar rutina de plan de entrenamiento: Descripción}
	\label{tab:cu019_desc}
	\begin{center}
		\resizebox{15cm}{!}{
		\begin{tabular}{|p{4cm}|p{11cm}|}
			\hline
			\multicolumn{2}{|c|}{Descripción de caso de uso} \\
			\hline
			Nombre & Eliminar rutina de plan de entrenamiento \\
			\hline
			Identificador & CU019 \\
			\hline
			Descripción & Permite al entrenador eliminar una rutina del plan de entrenamiento. Esta acción no elimina la rutina \\
			\hline
			Actor &
			\begin{itemize}
				\item Entrenador
				\item Jugador
				\item Organización
			\end{itemize}			 
			\\
			\hline
			Disparador & Actor deportivo con rol de entrenador elige eliminar rutina de plan de entrenamiento \\
			\hline
			Inclusiones & \\
			\hline
			Puntos de extensión & \\
			\hline
			Precondiciones &  
			\begin{itemize}
				\item La aplicación ha sido cargada por un actor con rol de entrenador deportivo
				\item El entrenador ha elegido un plan de entrenamiento en particular
				\item Se ha elegido la opción de eliminar rutina de plan de entrenamiento
			\end{itemize}
			\\
			\hline
			Postcondiciones & 
			\begin{itemize}
				\item El usuario elimina una rutina del plan de entrenamiento
				\item El usuario se encuentra en la pantalla de administración de planes de entrenamiento
			\end{itemize}
			\\
			\hline
			Notas & \\
			\hline
		\end{tabular}
		} \\
		\textbf{Fuente}: Autores
	\end{center}
\end{table}

\begin{table}[!htb]
	\caption{CU019-Eliminar rutina de plan de entrenamiento: Flujos de hechos}
	\label{tab:cu019_flujo}
	\begin{center}
		\resizebox{15cm}{!}{
		\begin{tabular}{|p{1.5cm}|p{6cm}|p{6.5cm}|}
			\hline
			\multicolumn{3}{|c|}{Detalle de flujo de hechos de caso de uso} \\
			\hline
			Nombre & \multicolumn{2}{|c|}{Nombre del flujo} \\
			\hline
			Paso & Acción del actor & Respuesta del sistema \\
			\hline
			1 & El usuario a elegido un plan de entrenamiento en específico & El sistema ha mostrado los detalles del plan de entrenamiento elegido \\
			\hline
			2 & El usuario busca la rutina que desea eliminar & El sistema devuelve los resultados de la búsqueda de la rutina \\
			\hline
			3 & El usuario pulsa el botón de eliminación de la rutina del plan de entrenamiento & El sistema elimina la rutina del plan de entrenamiento \\
			\hline
			4 & & El sistema muestra un elemento emergente que informa del éxito o fracaso de la operación \\
			\hline
			5 & El usuario continua & El sistema sitúa al usuario en los detalles del plan de entrenamiento \\
			\hline
		\end{tabular}
		} \\
		\textbf{Fuente}: Autores
	\end{center}
\end{table}

\begin{table}[!htb]
	\caption{CU020-Asignar plan de entrenamiento a jugador/equipo: Descripción}
	\label{tab:cu020_desc}
	\begin{center}
		\resizebox{15cm}{!}{
		\begin{tabular}{|p{4cm}|p{11cm}|}
			\hline
			\multicolumn{2}{|c|}{Descripción de caso de uso} \\
			\hline
			Nombre & Asignar plan de entrenamiento a jugador/equipo \\
			\hline
			Identificador & CU020 \\
			\hline
			Descripción & Permite al entrenador asignar un plan de entrenamiento a un jugador/equipo del cual él es entrenador \\
			\hline
			Actor &
			\begin{itemize}
				\item Entrenador
				\item Jugador
				\item Organización
			\end{itemize}			 
			\\
			\hline
			Disparador & Actor deportivo con rol de entrenador elige asignar plan de entrenamiento a entrenado \\
			\hline
			Inclusiones & \\
			\hline
			Puntos de extensión & \\
			\hline
			Precondiciones &  
			\begin{itemize}
				\item La aplicación ha sido cargada por un actor con rol de entrenador deportivo
				\item El entrenador ha elegido un plan de entrenamiento en particular
				\item Se ha elegido la opción de asignar plan de entrenamiento a un jugador/equipo
			\end{itemize}
			\\
			\hline
			Postcondiciones & 
			\begin{itemize}
				\item El usuario asigna plan de entrenamiento a jugador/equipo
				\item El usuario se encuentra en la pantalla de administración de planes de entrenamiento
			\end{itemize}
			\\
			\hline
			Notas & \\
			\hline
		\end{tabular}
		} \\
		\textbf{Fuente}: Autores
	\end{center}
\end{table}

\begin{table}[!htb]
	\caption{CU020-Asignar plan de entrenamiento a jugador/equipo: Flujos de hechos}
	\label{tab:cu020_flujo}
	\begin{center}
		\resizebox{15cm}{!}{
		\begin{tabular}{|p{1.5cm}|p{6cm}|p{6.5cm}|}
			\hline
			\multicolumn{3}{|c|}{Detalle de flujo de hechos de caso de uso} \\
			\hline
			Nombre & \multicolumn{2}{|c|}{Nombre del flujo} \\
			\hline
			Paso & Acción del actor & Respuesta del sistema \\
			\hline
			1 & El usuario ha elegido un plan deportivo en específico & El sistema ha mostrado los detalles del plan de entrenamiento elegido \\
			\hline
			2 & El usuario pulsa el botón para asignar el plan de entrenamiento a un usuario entrenado & El sistema muestra la interfaz de asociación de usuarios entrenados al plan de entrenamiento escogido \\
			\hline
			3 & El usuario realiza la búsqueda del usuario entrenado al que quiere asociar el plan de entrenamiento escogido & El sistema muestra el resultado de la búsqueda hecha por el usuario \\
			\hline
			4 & El usuario escoge el usuario entrenado que desea asociar y confirma su asociación & El sistema asocia al usuario entrenado escogido a el plan de entrenamiento \\
			\hline
			5 & & El sistema muestra un elemento emergente que informa del éxito o fracaso de la operación \\
			\hline
			6 & El usuario continua & El sistema sitúa al usuario en los detalles del plan de entrenamiento \\
			\hline
		\end{tabular}
		} \\
		\textbf{Fuente}: Autores
	\end{center}
\end{table}

\begin{table}[!htb]
	\caption{CU021-Retirar plan de entrenamiento a jugador/equipo: Descripción}
	\label{tab:cu021_desc}
	\begin{center}
		\resizebox{15cm}{!}{
		\begin{tabular}{|p{4cm}|p{11cm}|}
			\hline
			\multicolumn{2}{|c|}{Descripción de caso de uso} \\
			\hline
			Nombre & Retirar plan de entrenamiento a jugador/equipo \\
			\hline
			Identificador & CU021 \\
			\hline
			Descripción & Permite al entrenador retirar un plan de entrenamiento a un jugador/equipo del cual él es entrenador \\
			\hline
			Actor &
			\begin{itemize}
				\item Entrenador
				\item Jugador
				\item Organización
			\end{itemize}			 
			\\
			\hline
			Disparador & Actor deportivo con rol de entrenador elige retirar plan de entrenamiento a entrenado \\
			\hline
			Inclusiones & \\
			\hline
			Puntos de extensión & \\
			\hline
			Precondiciones &  
			\begin{itemize}
				\item La aplicación ha sido cargada por un actor con rol de entrenador deportivo
			\end{itemize}
			\\
			\hline
			Postcondiciones & 
			\begin{itemize}
				\item El usuario retira plan de entrenamiento a jugador/equipo
			\end{itemize}
			\\
			\hline
			Notas & \\
			\hline
		\end{tabular}
		} \\
		\textbf{Fuente}: Autores
	\end{center}
\end{table}

\begin{table}[!htb]
	\caption{CU021-Retirar plan de entrenamiento a jugador/equipo: Flujos de hechos}
	\label{tab:cu021_flujo}
	\begin{center}
		\resizebox{15cm}{!}{
		\begin{tabular}{|p{1.5cm}|p{6cm}|p{6.5cm}|}
			\hline
			\multicolumn{3}{|c|}{Detalle de flujo de hechos de caso de uso} \\
			\hline
			Nombre & \multicolumn{2}{|c|}{Nombre del flujo} \\
			\hline
			Paso & Acción del actor & Respuesta del sistema \\
			\hline
			1 & El usuario ha elegido un plan deportivo en específico & El sistema ha mostrado los detalles del plan de entrenamiento elegido \\
			\hline
			2 & El usuario pulsa el botón para desasignar el plan de entrenamiento a un usuario entrenado & El sistema muestra la interfaz para desasignar usuarios entrenados al plan de entrenamiento escogido \\
			\hline
			3 & El usuario realiza la búsqueda del usuario entrenado al que quiere desasignar el plan de entrenamiento escogido & El sistema muestra el resultado de la búsqueda hecha por el usuario \\
			\hline
			4 & El usuario escoge el usuario entrenado que desea desasignar y lo confirma & El sistema desasigna al usuario entrenado escogido del plan de entrenamiento \\
			\hline
			5 & & El sistema muestra un elemento emergente que informa del éxito o fracaso de la operación \\
			\hline
			6 & El usuario continua & El sistema sitúa al usuario en los detalles del plan de entrenamiento \\
			\hline
		\end{tabular}
		} \\
		\textbf{Fuente}: Autores
	\end{center}
\end{table}



\begin{table}[!htb]
	\caption{CU022-Eliminar plan de entrenamiento: Descripción}
	\label{tab:cu022_desc}
	\begin{center}
		\resizebox{15cm}{!}{
		\begin{tabular}{|p{4cm}|p{11cm}|}
			\hline
			\multicolumn{2}{|c|}{Descripción de caso de uso} \\
			\hline
			Nombre & Eliminar plan de entrenamiento \\
			\hline
			Identificador & CU022 \\
			\hline
			Descripción & Permite al entrenador eliminar un plan de entrenamiento \\
			\hline
			Actor &
			\begin{itemize}
				\item Entrenador
				\item Jugador
				\item Organización
			\end{itemize}			 
			\\
			\hline
			Disparador & Actor deportivo con rol de entrenador elige eliminar plan de entrenamiento \\
			\hline
			Inclusiones & \\
			\hline
			Puntos de extensión & \\
			\hline
			Precondiciones &  
			\begin{itemize}
				\item La aplicación ha sido cargada por un actor con rol de entrenador deportivo
				\item El entrenador ha elegido un plan de entrenamiento en particular y éste no tiene un historial con jugadores/equipos asociado
			\end{itemize}
			\\
			\hline
			Postcondiciones & 
			\begin{itemize}
				\item El usuario elimina plan de entrenamiento
			\end{itemize}
			\\
			\hline
			Notas & \\
			\hline
		\end{tabular}
		} \\
		\textbf{Fuente}: Autores
	\end{center}
\end{table}

\begin{table}[!htb]
	\caption{CU022-Eliminar plan de entrenamiento: Flujos de hechos}
	\label{tab:cu022_flujo}
	\begin{center}
		\resizebox{15cm}{!}{
		\begin{tabular}{|p{1.5cm}|p{6cm}|p{6.5cm}|}
			\hline
			\multicolumn{3}{|c|}{Detalle de flujo de hechos de caso de uso} \\
			\hline
			Nombre & \multicolumn{2}{|c|}{Nombre del flujo} \\
			\hline
			Paso & Acción del actor & Respuesta del sistema \\
			\hline
			1 & El usuario ha elegido gestionar las opciones de entrenamiento & El sistema ha mostrado la interfaz de gestión de opciones de entrenamiento \\
			\hline
			2 & El usuario busca el plan de entrenamiento que desea eliminar & El sistema retorna los resultados de la búsqueda hecha \\
			\hline
			3 & El usuario confirma la eliminación del plan de entrenamiento & El sistema pide confirmación \\
			\hline
			4 & El usuario confirma & El sistema elimina el plan de entrenamiento \\
			\hline
			5 & & El sistema muestra un elemento emergente que informa del éxito o fracaso de la operación \\
			\hline
			6 & El usuario continua & El sistema sitúa al usuario en los detalles del plan de entrenamiento \\
			\hline
		\end{tabular}
		} \\
		\textbf{Fuente}: Autores
	\end{center}
\end{table}

\clearpage

\section{Módulo de administración de eventos deportivos}

%\input{./imagenes/casos_uso/gestion_evento.png}

En cuanto a éste módulo se refiere, los autores plasmaron las funcionalidades que se le ofrecen en el SNS a los organizadores de eventos deportivos, habiendo dos grandes módulos, a saber: Administración de involucrados y la gestión de la información del evento en si.

\clearpage

\begin{table}[!htb]
	\caption{CU023-Administrar eventos deportivos: Descripción}
	\label{tab:cu023_desc}
	\begin{center}
		\resizebox{15cm}{!}{
		\begin{tabular}{|p{4cm}|p{11cm}|}
			\hline
			\multicolumn{2}{|c|}{Descripción de caso de uso} \\
			\hline
			Nombre & Administrar eventos deportivos \\
			\hline
			Identificador & CU023 \\
			\hline
			Descripción & Permite la administración de eventos deportivos realizados \\
			\hline
			Actor & Todo actor de la red social	 
			\\
			\hline
			Disparador & Se elige la opción de administrar un evento deportivo \\
			\hline
			Inclusiones &  \\
			\hline
			Puntos de extensión &  \\
			\hline
			Precondiciones &  
			\begin{itemize}
				\item La aplicación ha sido cargada por un actor con rol de organizador de eventos deportivos
			\end{itemize}
			\\
			\hline
			Postcondiciones & \\
			\hline
			Notas & 
			\begin{itemize}
				\item Generalización de:
				\begin{itemize}
					\item Crear evento deportivo
					\item Cancelar evento deportivo
					\item Actualizar información de evento deportivo
					\item Administrar involucrados
				\end{itemize}
			\end{itemize}
			\\
			\hline
		\end{tabular}
		} \\
		\textbf{Fuente}: Autores
	\end{center}
\end{table}

\begin{table}[!htb]
	\caption{CU023-Administrar eventos deportivos: Flujos de hechos}
	\label{tab:cu023_flujo}
	\begin{center}
		\resizebox{15cm}{!}{
		\begin{tabular}{|p{1.5cm}|p{6cm}|p{6.5cm}|}
			\hline
			\multicolumn{3}{|c|}{Detalle de flujo de hechos de caso de uso} \\
			\hline
			Nombre & \multicolumn{2}{|c|}{Nombre del flujo} \\
			\hline
			Paso & Acción del actor & Respuesta del sistema \\
			\hline
			1 & El usuario elige gestionar un evento deportivo & El sistema carga una lista de los eventos existentes creados en la interfaz de gestión de eventos deportivos, junto a otras herramientas \\
			\hline
		\end{tabular}
		} \\
		\textbf{Fuente}: Autores
	\end{center}
\end{table}

\begin{table}[!htb]
	\caption{CU024-Crear evento deportivo: Descripción}
	\label{tab:cu024_desc}
	\begin{center}
		\resizebox{15cm}{!}{
		\begin{tabular}{|p{4cm}|p{11cm}|}
			\hline
			\multicolumn{2}{|c|}{Descripción de caso de uso} \\
			\hline
			Nombre & Crear evento deportivo \\
			\hline
			Identificador & CU024 \\
			\hline
			Descripción & Permite la creación de un evento deportivo en la red social deportiva \\
			\hline
			Actor & Todo actor de la red social	 
			\\
			\hline
			Disparador & Se elige la opción de crear un evento deportivo \\
			\hline
			Inclusiones &  \\
			\hline
			Puntos de extensión &  \\
			\hline
			Precondiciones &  
			\begin{itemize}
				\item La aplicación ha sido cargada por un actor con rol de organizador de eventos deportivos
			\end{itemize}
			\\
			\hline
			Postcondiciones & 
			\begin{itemize}
				\item El usuario crea un evento deportivo
			\end{itemize}
			\\
			\hline
			Notas & 
			\\
			\hline
		\end{tabular}
		} \\
		\textbf{Fuente}: Autores
	\end{center}
\end{table}

\begin{table}[!htb]
	\caption{CU024-Crear evento deportivo: Flujos de hechos}
	\label{tab:cu024_flujo}
	\begin{center}
		\resizebox{15cm}{!}{
		\begin{tabular}{|p{1.5cm}|p{6cm}|p{6.5cm}|}
			\hline
			\multicolumn{3}{|c|}{Detalle de flujo de hechos de caso de uso} \\
			\hline
			Nombre & \multicolumn{2}{|c|}{Nombre del flujo} \\
			\hline
			Paso & Acción del actor & Respuesta del sistema \\
			\hline
			1 & El usuario ha escogido administrar eventos deportivos & El sistema ha mostrado la interfaz de administración de eventos deportivos \\
			\hline
			2 & El usuario pulsa el botón de creación de eventos deportivos & El sistema muestra la interfaz para elegir el tipo de evento que desea crear el usuario \\
			\hline
			3 & El usuario elige el tipo de evento y pulsa el botón de crear evento & El sistema muestra la interfaz para la adición de todos los datos del evento deportivo según su tipo (colaboradores, participantes, ubicaciones y datos generales) \\
			\hline
			4 & El usuario ingresa todos los datos que desea y salva la información & El sistema crea el evento deportivo \\
			\hline
			5 & & El sistema muestra un elemento emergente informando del éxito o fracaso de la operación \\
			\hline
			6 & El usuario continua & El sistema muestra la interfaz de administración de eventos deportivos \\
			\hline
		\end{tabular}
		} \\
		\textbf{Fuente}: Autores
	\end{center}
\end{table}

\begin{table}[!htb]
	\caption{CU025-Cancelar evento deportivo: Descripción}
	\label{tab:cu025_desc}
	\begin{center}
		\resizebox{15cm}{!}{
		\begin{tabular}{|p{4cm}|p{11cm}|}
			\hline
			\multicolumn{2}{|c|}{Descripción de caso de uso} \\
			\hline
			Nombre & Cancelar evento deportivo \\
			\hline
			Identificador & CU025 \\
			\hline
			Descripción & Permite la cancelación de un evento deportivo en la red social deportiva \\
			\hline
			Actor & Todo actor de la red social	 
			\\
			\hline
			Disparador & Se elige la opción de cancelar un evento deportivo \\
			\hline
			Inclusiones &  \\
			\hline
			Puntos de extensión &  \\
			\hline
			Precondiciones &  
			\begin{itemize}
				\item La aplicación ha sido cargada por un actor con rol de organizador de eventos deportivos
			\end{itemize}
			\\
			\hline
			Postcondiciones & 
			\begin{itemize}
				\item El usuario cancela un evento deportivo
			\end{itemize}
			\\
			\hline
			Notas & 
			\\
			\hline
		\end{tabular}
		} \\
		\textbf{Fuente}: Autores
	\end{center}
\end{table}

\begin{table}[!htb]
	\caption{CU025-Cancelar evento deportivo: Flujos de hechos}
	\label{tab:cu025_flujo}
	\begin{center}
		\resizebox{15cm}{!}{
		\begin{tabular}{|p{1.5cm}|p{6cm}|p{6.5cm}|}
			\hline
			\multicolumn{3}{|c|}{Detalle de flujo de hechos de caso de uso} \\
			\hline
			Nombre & \multicolumn{2}{|c|}{Nombre del flujo} \\
			\hline
			Paso & Acción del actor & Respuesta del sistema \\
			\hline
			1 & El usuario ha escogido administrar eventos deportivos & El sistema ha mostrado la interfaz de administración de eventos deportivos, cargando todos los eventos existentes sobre los que tiene permiso el usuario \\
			2 & El usuario elige cancelar uno de los eventos deportivos pulsando el botón de cancelar evento & El sistema cancela el evento deportivo \\
			\hline
			3 & & El sistema muestra un elemento emergente informando del éxito o fracaso de la operación \\
			\hline
			4 & El usuario continua & El sistema muestra la interfaz de administración de eventos deportivos \\
			\hline
		\end{tabular}
		} \\
		\textbf{Fuente}: Autores
	\end{center}
\end{table}

\begin{table}[!htb]
	\caption{CU026-Actualizar información de evento deportivo: Descripción}
	\label{tab:cu026_desc}
	\begin{center}
		\resizebox{15cm}{!}{
		\begin{tabular}{|p{4cm}|p{11cm}|}
			\hline
			\multicolumn{2}{|c|}{Descripción de caso de uso} \\
			\hline
			Nombre & Actualizar información de evento deportivo \\
			\hline
			Identificador & CU026 \\
			\hline
			Descripción & Permite la actualizar la información de un evento deportivo en la red social deportiva \\
			\hline
			Actor & Todo actor de la red social	 
			\\
			\hline
			Disparador & Se elige la opción de actualizar la información de un evento deportivo \\
			\hline
			Inclusiones &  \\
			\hline
			Puntos de extensión &  \\
			\hline
			Precondiciones &  
			\begin{itemize}
				\item La aplicación ha sido cargada por un actor con rol de organizador de eventos deportivos
			\end{itemize}
			\\
			\hline
			Postcondiciones & 
			\begin{itemize}
				\item El usuario cambia la información de un evento deportivo
			\end{itemize}
			\\
			\hline
			Notas & 
			\\
			\hline
		\end{tabular}
		} \\
		\textbf{Fuente}: Autores
	\end{center}
\end{table}

\begin{table}[!htb]
	\caption{CU026-Actualizar información de evento deportivo: Flujos de hechos}
	\label{tab:cu026_flujo}
	\begin{center}
		\resizebox{15cm}{!}{
		\begin{tabular}{|p{1.5cm}|p{6cm}|p{6.5cm}|}
			\hline
			\multicolumn{3}{|c|}{Detalle de flujo de hechos de caso de uso} \\
			\hline
			Nombre & \multicolumn{2}{|c|}{Nombre del flujo} \\
			\hline
			Paso & Acción del actor & Respuesta del sistema \\
			\hline
			 1 & El usuario ha escogido administrar eventos deportivos & El sistema ha mostrado la interfaz de administración de eventos deportivos, cargando todos los eventos existentes sobre los que tiene permiso el usuario \\
			\hline
			2 & El usuario busca un evento en especifico & El sistema retorna el resultado de la búsqueda \\
			3 & El usuario modifica detalles del evento elegido y salva la información & El sistema guarda la información \\
			4 & & El sistema muestra un elemento emergente informando del éxito o fracaso de la operación \\
			\hline
			5 & El usuario continua & El sistema muestra la interfaz de administración de eventos deportivos \\
			\hline
		\end{tabular}
		} \\
		\textbf{Fuente}: Autores
	\end{center}
\end{table}

\begin{table}[!htb]
	\caption{CU027-Administrar involucrados: Descripción}
	\label{tab:cu027_desc}
	\begin{center}
		\resizebox{15cm}{!}{
		\begin{tabular}{|p{4cm}|p{11cm}|}
			\hline
			\multicolumn{2}{|c|}{Descripción de caso de uso} \\
			\hline
			Nombre & Administrar involucrados \\
			\hline
			Identificador & CU027 \\
			\hline
			Descripción & Permite la administración de los stakeholder asociados a alguno de los eventos creados \\
			\hline
			Actor & Todo actor de la red social	 
			\\
			\hline
			Disparador & Se elige la opción de administración de involucrados en el evento \\
			\hline
			Inclusiones &  \\
			\hline
			Puntos de extensión &  \\
			\hline
			Precondiciones &  
			\begin{itemize}
				\item La aplicación ha sido cargada por un actor con rol de organizador de eventos deportivos
			\end{itemize}
			\\
			\hline
			Postcondiciones & \\
			\hline
			Notas & 
			\begin{itemize}
				\item Generalización de:
				\begin{itemize}
					\item Administrar colaboradores
					\item Añadir participante
					\item Modificar características de participante
					\item Consultar participantes
					\item Añadir espectador
					\item Consultar espectadores
				\end{itemize}
			\end{itemize}
			\\
			\hline
		\end{tabular}
		} \\
		\textbf{Fuente}: Autores
	\end{center}
\end{table}

\begin{table}[!htb]
	\caption{CU027-Administrar involucrados: Flujos de hechos}
	\label{tab:cu027_flujo}
	\begin{center}
		\resizebox{15cm}{!}{
		\begin{tabular}{|p{1.5cm}|p{6cm}|p{6.5cm}|}
			\hline
			\multicolumn{3}{|c|}{Detalle de flujo de hechos de caso de uso} \\
			\hline
			Nombre & \multicolumn{2}{|c|}{Nombre del flujo} \\
			\hline
			Paso & Acción del actor & Respuesta del sistema \\
			\hline
			1 & El actor elige administrar los involucrados de un evento previamente seleccionado & El sistema muestra el menú de administración de involucrados \\
			\hline
		\end{tabular}
		} \\
		\textbf{Fuente}: Autores
	\end{center}
\end{table}


\clearpage

\begin{table}[!htb]
	\caption{CU028-Administrar colaboradores: Descripción}
	\label{tab:cu028_desc}
	\begin{center}
		\resizebox{15cm}{!}{
		\begin{tabular}{|p{4cm}|p{11cm}|}
			\hline
			\multicolumn{2}{|c|}{Descripción de caso de uso} \\
			\hline
			Nombre & Administrar colaboradores\\
			\hline
			Identificador & CU028 \\
			\hline
			Descripción & Permite entrar a la administración de información de los colaboradores de un evento deportivo \\
			\hline
			Actor & Todo actor de la red social	 
			\\
			\hline
			Disparador & Se elige la opción de administración de colaboradores \\
			\hline
			Inclusiones &  \\
			\hline
			Puntos de extensión &  \\
			\hline
			Precondiciones &  
			\begin{itemize}
				\item La aplicación ha sido cargada por un actor con rol de organizador de eventos deportivos
				\item Solo el creador/administrador principal del evento puede activar esta opcionalidad
			\end{itemize}
			\\
			\hline
			Postcondiciones & \\
			\hline
			Notas &
			\begin{itemize}
				\item Generalización de:
				\begin{itemize}
					\item Añadir colaborador
					\item Retirar colaborador
				\end{itemize}
			\end{itemize}
			\\
			\hline
		\end{tabular}
		} \\
		\textbf{Fuente}: Autores
	\end{center}
\end{table}

\begin{table}[!htb]
	\caption{CU028-Administrar colaboradores: Flujos de hechos}
	\label{tab:cu028_flujo}
	\begin{center}
		\resizebox{15cm}{!}{
		\begin{tabular}{|p{1.5cm}|p{6cm}|p{6.5cm}|}
			\hline
			\multicolumn{3}{|c|}{Detalle de flujo de hechos de caso de uso} \\
			\hline
			Nombre & \multicolumn{2}{|c|}{Nombre del flujo} \\
			\hline
			Paso & Acción del actor & Respuesta del sistema \\
			\hline
			1 & El usuario a elegido un evento para administrar sus involucrados & El sistema ha mostrado el menú de administración de involucrados \\
			\hline
			2 & El usuario a elegido administrar colaboradores & El sistema realiza una búsqueda de los colaboradores y situa al usuario en la interfaz para manejarlos \\
			\hline
		\end{tabular}
		} \\
		\textbf{Fuente}: Autores
	\end{center}
\end{table}

\begin{table}[!htb]
	\caption{CU029-Añadir colaborador: Descripción}
	\label{tab:cu029_desc}
	\begin{center}
		\resizebox{15cm}{!}{
		\begin{tabular}{|p{4cm}|p{11cm}|}
			\hline
			\multicolumn{2}{|c|}{Descripción de caso de uso} \\
			\hline
			Nombre & Añadir colaborador\\
			\hline
			Identificador & CU029 \\
			\hline
			Descripción & Permite añadir un colaborador en la organización de un evento deportivo en el evento escogido \\
			\hline
			Actor & Todo actor de la red social	 
			\\
			\hline
			Disparador & Se elige la opción de añadir colaborador \\
			\hline
			Inclusiones &  \\
			\hline
			Puntos de extensión &  \\
			\hline
			Precondiciones &  
			\begin{itemize}
				\item La aplicación ha sido cargada por un actor con rol de organizador de eventos deportivos
				\item Solo el creador/administrador principal del evento puede activar esta opcionalidad
			\end{itemize}
			\\
			\hline
			Postcondiciones & 
			\begin{itemize}
				\item El usuario añade uno o varios colaboradores a la organización del evento deportivo
			\end{itemize}
			\\
			\hline
			Notas & 
			\\
			\hline
		\end{tabular}
		} \\
		\textbf{Fuente}: Autores
	\end{center}
\end{table}

\begin{table}[!htb]
	\caption{CU029-Añadir colaborador: Flujos de hechos}
	\label{tab:cu029_flujo}
	\begin{center}
		\resizebox{15cm}{!}{
		\begin{tabular}{|p{1.5cm}|p{6cm}|p{6.5cm}|}
			\hline
			\multicolumn{3}{|c|}{Detalle de flujo de hechos de caso de uso} \\
			\hline
			Nombre & \multicolumn{2}{|c|}{Nombre del flujo} \\
			\hline
			Paso & Acción del actor & Respuesta del sistema \\
			\hline
			1 & El usuario a elegido administrar colaboradores & El sistema realiza una búsqueda de los colaboradores y situa al usuario en la interfaz para manejarlos \\
			\hline
			2 & El usuario pulsa el botón de adición de colaboradores al evento & El sistema situa al usuario en la interfaz de adición de colaboradores \\
			\hline
			3 & El usuario busca otro usuario para añadir como colaborador & El sistema despliega el resultado de la búsqueda \\
			4 & El usuario pulsa el botón de añadir a colaborador & El sistema añade al nuevo colaborador \\
			\hline
			5 & & El sistema muestra un elemento emergente informando del éxito o fracaso de la operación \\
			\hline
			6 & El usuario continua & El sistema muestra la interfaz de administración de eventos deportivos \\
			\hline
		\end{tabular}
		} \\
		\textbf{Fuente}: Autores
	\end{center}
\end{table}

\begin{table}[!htb]
	\caption{CU030-Retirar colaborador: Descripción}
	\label{tab:cu030_desc}
	\begin{center}
		\resizebox{15cm}{!}{
		\begin{tabular}{|p{4cm}|p{11cm}|}
			\hline
			\multicolumn{2}{|c|}{Descripción de caso de uso} \\
			\hline
			Nombre & Retirar colaborador \\
			\hline
			Identificador & CU030 \\
			\hline
			Descripción & Permite retirar un organizador del evento deportivo \\
			\hline
			Actor & Todo actor de la red social	 
			\\
			\hline
			Disparador & Se elige la opción de retirar colaborador \\
			\hline
			Inclusiones &  \\
			\hline
			Puntos de extensión &  \\
			\hline
			Precondiciones &  
			\begin{itemize}
				\item La aplicación ha sido cargada por un actor con rol de organizador de eventos deportivos
				\item Solo el creador/administrador principal del evento puede activar esta opcionalidad
			\end{itemize}
			\\
			\hline
			Postcondiciones & 
			\begin{itemize}
				\item El usuario retira un colaborador de la organización del evento deportivo
			\end{itemize}
			\\
			\hline
			Notas & 
			\\
			\hline
		\end{tabular}
		} \\
		\textbf{Fuente}: Autores
	\end{center}
\end{table}

\begin{table}[!htb]
	\caption{CU030-Retirar colaborador: Flujos de hechos}
	\label{tab:cu030_flujo}
	\begin{center}
		\resizebox{15cm}{!}{
		\begin{tabular}{|p{1.5cm}|p{6cm}|p{6.5cm}|}
			\hline
			\multicolumn{3}{|c|}{Detalle de flujo de hechos de caso de uso} \\
			\hline
			Nombre & \multicolumn{2}{|c|}{Nombre del flujo} \\
			\hline
			Paso & Acción del actor & Respuesta del sistema \\
			\hline
			1 & El usuario a elegido administrar colaboradores & El sistema realiza una búsqueda de los colaboradores y situa al usuario en la interfaz para manejarlos \\
			\hline
			2 & El usuario busca un colaborador en específico & El sistema retorna el resultado de la búsqueda \\
			\hline
			3 & El usuario pulsa el botón para retirar colaboradores & El sistema retira al colaborador \\		
			4 & & El sistema muestra un elemento emergente informando del éxito o fracaso de la operación \\
			\hline
			5 & El usuario continua & El sistema muestra la interfaz de administración de eventos deportivos \\
			\hline
		\end{tabular}
		} \\
		\textbf{Fuente}: Autores
	\end{center}
\end{table}

\begin{table}[!htb]
	\caption{CU031-Añadir participante: Descripción}
	\label{tab:cu031_desc}
	\begin{center}
		\resizebox{15cm}{!}{
		\begin{tabular}{|p{4cm}|p{11cm}|}
			\hline
			\multicolumn{2}{|c|}{Descripción de caso de uso} \\
			\hline
			Nombre & Añadir participante \\
			\hline
			Identificador & CU031 \\
			\hline
			Descripción & Permite añadir un participante deportivo (equipo o jugador) al evento deportivo \\
			\hline
			Actor & Todo actor de la red social	 
			\\
			\hline
			Disparador & Se elige la opción de añadir un participante deportivo al evento \\
			\hline
			Inclusiones &  \\
			\hline
			Puntos de extensión &  \\
			\hline
			Precondiciones &  
			\begin{itemize}
				\item La aplicación ha sido cargada por un actor con rol de organizador de eventos deportivos
			\end{itemize}
			\\
			\hline
			Postcondiciones & 
			\begin{itemize}
				\item El usuario añade un participante deportivo al evento
			\end{itemize}
			\\
			\hline
			Notas & 
			\\
			\hline
		\end{tabular}
		} \\
		\textbf{Fuente}: Autores
	\end{center}
\end{table}

\begin{table}[!htb]
	\caption{CU031-Añadir participante: Flujos de hechos}
	\label{tab:cu031_flujo}
	\begin{center}
		\resizebox{15cm}{!}{
		\begin{tabular}{|p{1.5cm}|p{6cm}|p{6.5cm}|}
			\hline
			\multicolumn{3}{|c|}{Detalle de flujo de hechos de caso de uso} \\
			\hline
			Nombre & \multicolumn{2}{|c|}{Nombre del flujo} \\
			\hline
			Paso & Acción del actor & Respuesta del sistema \\
			\hline
			1 & El usuario a elegido un evento para administrar sus involucrados & El sistema ha mostrado el menú de administración de involucrados \\
			\hline
			2 & El usuario elige añadir un participante & El sistema muestra la interfaz de adición de participantes \\
			\hline
			3 & El usuario elige enviar una invitación o aceptar una petición de participación; en cualquier caso, el usuario busca al usuario potencial a participar & El sistema retorna el resultado de la búsqueda \\
			\hline
			4 & El usuario confirma el envío de la solicitud o de la adición del participante al evento & El sistema envía la solicitud o adiciona directamente al participante, según la función elegida \\
			5 & & El sistema muestra un elemento emergente informando del éxito o fracaso de la operación \\
			\hline
			6 & El usuario continua & El sistema muestra la interfaz de adición de participantes al evento \\
			\hline
		\end{tabular}
		} \\
		\textbf{Fuente}: Autores
	\end{center}
\end{table}

\begin{table}[!htb]
	\caption{CU032-Consultar participantes: Descripción}
	\label{tab:cu032_desc}
	\begin{center}
		\resizebox{15cm}{!}{
		\begin{tabular}{|p{4cm}|p{11cm}|}
			\hline
			\multicolumn{2}{|c|}{Descripción de caso de uso} \\
			\hline
			Nombre & Consultar participantes \\
			\hline
			Identificador & CU032 \\
			\hline
			Descripción & Permite consultar la lista de los participantes del evento deportivo \\
			\hline
			Actor & Todo actor de la red social	 
			\\
			\hline
			Disparador & Se elige la opción de consultar participantes de evento deportivo \\
			\hline
			Inclusiones &  \\
			\hline
			Puntos de extensión & 
			\begin{itemize}
				\item Retirar participante
			\end{itemize}
			\\
			\hline
			Precondiciones &  
			\begin{itemize}
				\item La aplicación ha sido cargada por un actor con rol de organizador de eventos deportivos
			\end{itemize}
			\\
			\hline
			Postcondiciones & \\
			\hline
			Notas & \\
			\hline
		\end{tabular}
		} \\
		\textbf{Fuente}: Autores
	\end{center}
\end{table}

\begin{table}[!htb]
	\caption{CU032-Consultar participantes: Flujos de hechos}
	\label{tab:cu032_flujo}
	\begin{center}
		\resizebox{15cm}{!}{
		\begin{tabular}{|p{1.5cm}|p{6cm}|p{6.5cm}|}
			\hline
			\multicolumn{3}{|c|}{Detalle de flujo de hechos de caso de uso} \\
			\hline
			Nombre & \multicolumn{2}{|c|}{Nombre del flujo} \\
			\hline
			Paso & Acción del actor & Respuesta del sistema \\
			\hline
			1 & El usuario a elegido un evento para administrar sus involucrados & El sistema ha mostrado el menú de administración de involucrados \\
			\hline
			2 & El usuario elige consultar participantes del evento deportivo & El sistema muestra una lista de los participantes del evento con herramientas para buscar sobre ella \\
			\hline
		\end{tabular}
		} \\
		\textbf{Fuente}: Autores
	\end{center}
\end{table}

\clearpage

\begin{table}[!htb]
	\caption{CU033-Retirar participante: Descripción}
	\label{tab:cu033_desc}
	\begin{center}
		\resizebox{15cm}{!}{
		\begin{tabular}{|p{4cm}|p{11cm}|}
			\hline
			\multicolumn{2}{|c|}{Descripción de caso de uso} \\
			\hline
			Nombre & Retirar participante \\
			\hline
			Identificador & CU033 \\
			\hline
			Descripción & Permite retirar un participante deportivo (equipo o jugador) del evento deportivo \\
			\hline
			Actor & Todo actor de la red social	 
			\\
			\hline
			Disparador & Se elige la opción de retirar un participante deportivo al evento \\
			\hline
			Inclusiones &  \\
			\hline
			Puntos de extensión &  \\
			\hline
			Precondiciones &  
			\begin{itemize}
				\item La aplicación ha sido cargada por un actor con rol de organizador de eventos deportivos
			\end{itemize}
			\\
			\hline
			Postcondiciones & 
			\begin{itemize}
				\item El usuario retira un participante deportivo del evento deportivo
			\end{itemize}
			\\
			\hline
			Notas & 
			\\
			\hline
		\end{tabular}
		} \\
		\textbf{Fuente}: Autores
	\end{center}
\end{table}

\begin{table}[!htb]
	\caption{CU033-Retirar participante: Flujos de hechos}
	\label{tab:cu033_flujo}
	\begin{center}
		\resizebox{15cm}{!}{
		\begin{tabular}{|p{1.5cm}|p{6cm}|p{6.5cm}|}
			\hline
			\multicolumn{3}{|c|}{Detalle de flujo de hechos de caso de uso} \\
			\hline
			Nombre & \multicolumn{2}{|c|}{Nombre del flujo} \\
			\hline
			Paso & Acción del actor & Respuesta del sistema \\
			\hline
			1 & El usuario ha consultado los participantes registrados en el evento & El sistema ha mostrado los participantes registrados en el evento \\
			\hline
			2 & El usuario ha buscado un participante & El sistema ha retornado el resultado de la búsqueda \\
			\hline
			3 & El usuario confirma el retiro del participante del evento & El sistema retira al participante del evento \\
			\hline
			4 & & El sistema muestra un elemento emergente informando del éxito o fracaso de la operación \\
			\hline
			5 & El usuario continua & El sistema muestra la interfaz de consulta de participantes \\
			\hline
		\end{tabular}
		} \\
		\textbf{Fuente}: Autores
	\end{center}
\end{table}

\begin{table}[!htb]
	\caption{CU034-Consultar espectadores: Descripción}
	\label{tab:cu034_desc}
	\begin{center}
		\resizebox{15cm}{!}{
		\begin{tabular}{|p{4cm}|p{11cm}|}
			\hline
			\multicolumn{2}{|c|}{Descripción de caso de uso} \\
			\hline
			Nombre & Consultar espectadores \\
			\hline
			Identificador & CU034 \\
			\hline
			Descripción & Permite consultar la lista de los espectadores del evento deportivo \\
			\hline
			Actor & Todo actor de la red social	 
			\\
			\hline
			Disparador & Se elige la opción de consultar espectadores de evento deportivo \\
			\hline
			Inclusiones &  \\
			\hline
			Puntos de extensión & 
			\begin{itemize}
				\item Retirar espectador
			\end{itemize}
			\\
			\hline
			Precondiciones &  
			\begin{itemize}
				\item La aplicación ha sido cargada por un actor con rol de organizador de eventos deportivos
			\end{itemize}
			\\
			\hline
			Postcondiciones & \\
			\hline
			Notas & \\
			\hline
		\end{tabular}
		} \\
		\textbf{Fuente}: Autores
	\end{center}
\end{table}

\begin{table}[!htb]
	\caption{CU034-Consultar espectadores: Flujos de hechos}
	\label{tab:cu034_flujo}
	\begin{center}
		\resizebox{15cm}{!}{
		\begin{tabular}{|p{1.5cm}|p{6cm}|p{6.5cm}|}
			\hline
			\multicolumn{3}{|c|}{Detalle de flujo de hechos de caso de uso} \\
			\hline
			Nombre & \multicolumn{2}{|c|}{Nombre del flujo} \\
			\hline
			Paso & Acción del actor & Respuesta del sistema \\
			\hline
			1 & El usuario a elegido un evento para administrar sus involucrados & El sistema ha mostrado el menú de administración de involucrados \\
			\hline
			2 & El usuario elige consultar espectadores del evento deportivo & El sistema muestra una lista de los espectadores del evento con herramientas para buscar sobre ella \\
			\hline
		\end{tabular}
		} \\
		\textbf{Fuente}: Autores
	\end{center}
\end{table}

\begin{table}[!htb]
	\caption{CU035-Retirar espectador: Descripción}
	\label{tab:cu035_desc}
	\begin{center}
		\resizebox{15cm}{!}{
		\begin{tabular}{|p{4cm}|p{11cm}|}
			\hline
			\multicolumn{2}{|c|}{Descripción de caso de uso} \\
			\hline
			Nombre & Retirar espectador \\
			\hline
			Identificador & CU035 \\
			\hline
			Descripción & Permite retirar un espectador del evento deportivo \\
			\hline
			Actor & Todo actor de la red social	 
			\\
			\hline
			Disparador & Se elige la opción de retirar un espectador deportivo del evento \\
			\hline
			Inclusiones &  \\
			\hline
			Puntos de extensión &  \\
			\hline
			Precondiciones &  
			\begin{itemize}
				\item La aplicación ha sido cargada por un actor con rol de organizador de eventos deportivos
			\end{itemize}
			\\
			\hline
			Postcondiciones & 
			\begin{itemize}
				\item El usuario retira un espectador del deportivo del evento deportivo
			\end{itemize}
			\\
			\hline
			Notas & 
			\\
			\hline
		\end{tabular}
		} \\
		\textbf{Fuente}: Autores
	\end{center}
\end{table}

\begin{table}[!htb]
	\caption{CU035-Retirar espectador: Flujos de hechos}
	\label{tab:cu035_flujo}
	\begin{center}
		\resizebox{15cm}{!}{
		\begin{tabular}{|p{1.5cm}|p{6cm}|p{6.5cm}|}
			\hline
			\multicolumn{3}{|c|}{Detalle de flujo de hechos de caso de uso} \\
			\hline
			Nombre & \multicolumn{2}{|c|}{Nombre del flujo} \\
			\hline
			Paso & Acción del actor & Respuesta del sistema \\
			\hline
			1 & El usuario ha elegido la opción de consulta de espectadores & El sistema ha mostrado una interfaz con la lista de los espectadores existentes \\
			\hline
			2 & El usuario busca un espectador en específico & El sistema muestra el resultado de la búsqueda \\
			\hline
			3 & El usuario pulsa el botón para retirar el espectador & El sistema retira al espectador \\
			\hline
			4 & & El sistema muestra un elemento emergente informando del éxito o fracaso de la operación \\
			\hline
			5 & El usuario continua & El sistema muestra la interfaz de administración de eventos deportivos \\
			\hline
		\end{tabular}
		} \\
		\textbf{Fuente}: Autores
	\end{center}
\end{table}

\begin{table}[!htb]
	\caption{CU036-Añadir espectador: Descripción}
	\label{tab:cu036_desc}
	\begin{center}
		\resizebox{15cm}{!}{
		\begin{tabular}{|p{4cm}|p{11cm}|}
			\hline
			\multicolumn{2}{|c|}{Descripción de caso de uso} \\
			\hline
			Nombre & Añadir espectador \\
			\hline
			Identificador & CU036 \\
			\hline
			Descripción & Permite la adición de un espectador al evento deportivo \\
			\hline
			Actor & Todo actor de la red social	 
			\\
			\hline
			Disparador & Se elige la opción de añadir un espectador deportivo al evento \\
			\hline
			Inclusiones &  \\
			\hline
			Puntos de extensión &  \\
			\hline
			Precondiciones &  
			\begin{itemize}
				\item La aplicación ha sido cargada por un actor con rol de organizador de eventos deportivos
			\end{itemize}
			\\
			\hline
			Postcondiciones & 
			\begin{itemize}
				\item El usuario añade un espectador al deportivo del evento deportivo
			\end{itemize}
			\\
			\hline
			Notas & 
			\begin{itemize}
				\item En este contexto, la adición hace referencia a una invitación a ser espectador, debido a que cualquier persona podrá unirse al evento cuando desee
			\end{itemize}
			\\
			\hline
		\end{tabular}
		} \\
		\textbf{Fuente}: Autores
	\end{center}
\end{table}

\begin{table}[!htb]
	\caption{CU036-Añadir espectador: Flujos de hechos}
	\label{tab:cu036_flujo}
	\begin{center}
		\resizebox{15cm}{!}{
		\begin{tabular}{|p{1.5cm}|p{6cm}|p{6.5cm}|}
			\hline
			\multicolumn{3}{|c|}{Detalle de flujo de hechos de caso de uso} \\
			\hline
			Nombre & \multicolumn{2}{|c|}{Nombre del flujo} \\
			\hline
			Paso & Acción del actor & Respuesta del sistema \\
			\hline
			1 & El usuario a elegido un evento para administrar sus involucrados & El sistema ha mostrado el menú de administración de involucrados \\
			\hline
			2 & El usuario elige añadir espectadores al evento & El sistema muestra la interfaz correspondiente a la adición de espectadores \\
			\hline
			3 & El usuario busca un usuario en la red social & El sistema muestra el resultado de la búsqueda \\
			\hline
			4 & El usuario elige uno de los usuarios que apareció luego de la búsqueda y confirma el envío de la invitación de participación como espectador & El sistema envía la invitación \\
			\hline
			5 & & El sistema muestra un elemento emergente informando del éxito o fracaso de la operación \\
			\hline
			6 & El usuario continua & El sistema muestra la interfaz de administración de eventos deportivos \\
			\hline
		\end{tabular}
		} \\
		\textbf{Fuente}: Autores
	\end{center}
\end{table}

\begin{table}[!htb]
	\caption{CU037-Modificar características de participantes: Descripción}
	\label{tab:cu037_desc}
	\begin{center}
		\resizebox{15cm}{!}{
		\begin{tabular}{|p{4cm}|p{11cm}|}
			\hline
			\multicolumn{2}{|c|}{Descripción de caso de uso} \\
			\hline
			Nombre & Modificar características de participantes \\
			\hline
			Identificador & CU037 \\
			\hline
			Descripción & Permite modificar restricciones a los posibles participantes del evento \\
			\hline
			Actor & Todo actor de la red social	 
			\\
			\hline
			Disparador & Se elige la opción de modificar restricciones a los posibles participantes del evento \\
			\hline
			Inclusiones & \\
			\hline
			Puntos de extensión & \\
			\hline
			Precondiciones &  
			\begin{itemize}
				\item La aplicación ha sido cargada por un actor con rol de organizador de eventos deportivos
			\end{itemize}
			\\
			\hline
			Postcondiciones & 
			\begin{itemize}
				\item El usuario añade un espectador al deportivo del evento deportivo
			\end{itemize}
			\\
			\hline
			Notas & 
			\\
			\hline
		\end{tabular}
		} \\
		\textbf{Fuente}: Autores
	\end{center}
\end{table}

\begin{table}[!htb]
	\caption{CU037-Modificar características de participantes: Flujos de hechos}
	\label{tab:cu037_flujo}
	\begin{center}
		\resizebox{15cm}{!}{
		\begin{tabular}{|p{1.5cm}|p{6cm}|p{6.5cm}|}
			\hline
			\multicolumn{3}{|c|}{Detalle de flujo de hechos de caso de uso} \\
			\hline
			Nombre & \multicolumn{2}{|c|}{Nombre del flujo} \\
			\hline
			Paso & Acción del actor & Respuesta del sistema \\
			\hline
			1 & El usuario a elegido un evento para administrar sus involucrados & El sistema ha mostrado el menú de administración de involucrados \\
			\hline
			2 & El usuario elige modificar las características que debe tener un participante para participar & El sistema muestra la interfaz para modificar las restricciones para los candidatos a participantes en el evento \\
			3 & El usuario hace las modificaciones que requiera y salva la información & El sistema salva la información \\
			\hline
			4 & & El sistema muestra un elemento emergente informando del éxito o fracaso de la operación \\
			\hline
			5 & El usuario continua & El sistema muestra la interfaz de administración de eventos deportivos \\
			\hline
		\end{tabular}
		} \\
		\textbf{Fuente}: Autores
	\end{center}
\end{table}

\begin{table}[!htb]
	\caption{CU038-Unir al evento: Descripción}
	\label{tab:cu038_desc}
	\begin{center}
		\resizebox{15cm}{!}{
		\begin{tabular}{|p{4cm}|p{11cm}|}
			\hline
			\multicolumn{2}{|c|}{Descripción de caso de uso} \\
			\hline
			Nombre & Unir al evento \\
			\hline
			Identificador & CU038 \\
			\hline
			Descripción & Permite gestionar la unión a un evento \\
			\hline
			Actor & Todo actor de la red social	 
			\\
			\hline
			Disparador & Se elige la opción de acciones a realizar sobre el evento \\
			\hline
			Inclusiones & \\
			\hline
			Puntos de extensión & \\
			\hline
			Precondiciones & \\
			\hline
			Postcondiciones & \\
			\hline
			Notas &
			\begin{itemize}
				\item Generalización de:
				\begin{itemize}
					\item Unir al evento como participante
					\item Unir al evento como espectador
				\end{itemize}
			\end{itemize}			 
			\\
			\hline
		\end{tabular}
		} \\
		\textbf{Fuente}: Autores
	\end{center}
\end{table}

\begin{table}[!htb]
	\caption{CU038-Unir al evento: Flujos de hechos}
	\label{tab:cu038_flujo}
	\begin{center}
		\resizebox{15cm}{!}{
		\begin{tabular}{|p{1.5cm}|p{6cm}|p{6.5cm}|}
			\hline
			\multicolumn{3}{|c|}{Detalle de flujo de hechos de caso de uso} \\
			\hline
			Nombre & \multicolumn{2}{|c|}{Nombre del flujo} \\
			\hline
			Paso & Acción del actor & Respuesta del sistema \\
			\hline
			1 & El usuario entra al timeline de un evento y elige la opción de acciones a realizar sobre el evento & El sistema muestra el menú de acciones a realizar sobre el evento \\
			\hline
		\end{tabular}
		} \\
		\textbf{Fuente}: Autores
	\end{center}
\end{table}

\begin{table}[!htb]
	\caption{CU039-Unir al evento como participante: Descripción}
	\label{tab:cu039_desc}
	\begin{center}
		\resizebox{15cm}{!}{
		\begin{tabular}{|p{4cm}|p{11cm}|}
			\hline
			\multicolumn{2}{|c|}{Descripción de caso de uso} \\
			\hline
			Nombre & Unir al evento como participante \\
			\hline
			Identificador & CU039 \\
			\hline
			Descripción & Permite gestionar la unión a un evento como participante \\
			\hline
			Actor & Todo actor de la red social	 
			\\
			\hline
			Disparador & Se elige la opción de unirse al evento como participante \\
			\hline
			Inclusiones & \\
			\hline
			Puntos de extensión & \\
			\hline
			Precondiciones & \\
			\hline
			Postcondiciones & 
			\begin{itemize}
				\item El usuario se une al evento como participante
			\end{itemize}						
			\\
			\hline
			Notas & \\
			\hline
		\end{tabular}
		} \\
		\textbf{Fuente}: Autores
	\end{center}
\end{table}

\begin{table}[!htb]
	\caption{CU039-Unir al evento como participante: Flujos de hechos}
	\label{tab:cu039_flujo}
	\begin{center}
		\resizebox{15cm}{!}{
		\begin{tabular}{|p{1.5cm}|p{6cm}|p{6.5cm}|}
			\hline
			\multicolumn{3}{|c|}{Detalle de flujo de hechos de caso de uso} \\
			\hline
			Nombre & \multicolumn{2}{|c|}{Nombre del flujo} \\
			\hline
			Paso & Acción del actor & Respuesta del sistema \\
			\hline
			1 & El usuario ha entrado al timeline de un evento y elegido la opción de acciones a realizar sobre el evento & El sistema ha mostrado el menú de acciones a realizar sobre el evento \\
			\hline
			2 & El usuario pulsa el botón de unión al evento como participante & El sistema envía la petición de entrar al evento como participante \\
			\hline
			3 & & El sistema muestra un elemento emergente informando del éxito o fracaso de la operación \\
			\hline
			4 & El usuario continua & El sistema muestra la interfaz de acciones sobre el evento \\
			\hline
		\end{tabular}
		} \\
		\textbf{Fuente}: Autores
	\end{center}
\end{table}

\begin{table}[!htb]
	\caption{CU040-Unir al evento como espectador: Descripción}
	\label{tab:cu040_desc}
	\begin{center}
		\resizebox{15cm}{!}{
		\begin{tabular}{|p{4cm}|p{11cm}|}
			\hline
			\multicolumn{2}{|c|}{Descripción de caso de uso} \\
			\hline
			Nombre & Unir al evento como espectador \\
			\hline
			Identificador & CU040 \\
			\hline
			Descripción & Permite gestionar la unión a un evento como espectador \\
			\hline
			Actor & Todo actor de la red social	 
			\\
			\hline
			Disparador & Se elige la opción de unirse al evento como espectador \\
			\hline
			Inclusiones & \\
			\hline
			Puntos de extensión & \\
			\hline
			Precondiciones & \\
			\hline
			Postcondiciones & 
			\begin{itemize}
				\item El usuario se une al evento como espectador
			\end{itemize}						
			\\
			\hline
			Notas & \\
			\hline
		\end{tabular}
		} \\
		\textbf{Fuente}: Autores
	\end{center}
\end{table}

\begin{table}[!htb]
	\caption{CU040-Unir al evento como espectador: Flujos de hechos}
	\label{tab:cu040_flujo}
	\begin{center}
		\resizebox{15cm}{!}{
		\begin{tabular}{|p{1.5cm}|p{6cm}|p{6.5cm}|}
			\hline
			\multicolumn{3}{|c|}{Detalle de flujo de hechos de caso de uso} \\
			\hline
			Nombre & \multicolumn{2}{|c|}{Nombre del flujo} \\
			\hline
			Paso & Acción del actor & Respuesta del sistema \\
			\hline
			1 & El usuario ha entrado al timeline de un evento y elegido la opción de acciones a realizar sobre el evento & El sistema ha mostrado el menú de acciones a realizar sobre el evento \\
			\hline
			2 & El usuario pulsa el botón de unión al evento como espectador & El sistema envía la petición de entrar al evento como espectador \\
			\hline
			3 & & El sistema muestra un elemento emergente informando del éxito o fracaso de la operación \\
			\hline
			4 & El usuario continua & El sistema muestra la interfaz de acciones sobre el evento \\
			\hline
		\end{tabular}
		} \\
		\textbf{Fuente}: Autores
	\end{center}
\end{table}

\clearpage

\section{Módulo de administración de torneos}

%\input{./imagenes/casos_uso/gestion_torneo.png}

El módulo de administración de torneos es una extensión al módulo de administración de eventos debido a que un torneo es también considerado un evento deportivo. Particularmente se dan funcionalidades de administración de información del torneo, así como la adición y retiro de participantes al torneo y, a su vez, la gestión del formato que se le quiera dar al torneo (que tipo de eliminatorias se van a llevar, por ejemplificar).

\begin{table}[!htb]
	\caption{CU041-Administrar torneos: Descripción}
	\label{tab:cu041_desc}
	\begin{center}
		\resizebox{15cm}{!}{
		\begin{tabular}{|p{4cm}|p{11cm}|}
			\hline
			\multicolumn{2}{|c|}{Descripción de caso de uso} \\
			\hline
			Nombre & Administrar torneos \\
			\hline
			Identificador & CU041 \\
			\hline
			Descripción & Permite la administración de torneos deportivos \\
			\hline
			Actor & Todo actor de la red social	\\
			\hline
			Disparador & Se elige la opción de administrar torneos \\
			\hline
			Inclusiones &  \\
			\hline
			Puntos de extensión &  \\
			\hline
			Precondiciones &  
			\begin{itemize}
				\item La aplicación ha sido cargada por un actor con rol de organizador de eventos deportivos
			\end{itemize}
			\\
			\hline
			Postcondiciones & 
			\\
			\hline
			Notas & 
			\begin{itemize}
				\item Generalización de:
				\begin{itemize}
					\item Crear torneo
					\item Actualizar información de torneo
					\item Gestionar formatos de torneo
					\item Agregar equipo a torneo
					\item Retirar equipo de torneo
				\end{itemize}
			\end{itemize}
			\\
			\hline
		\end{tabular}
		} \\
		\textbf{Fuente}: Autores
	\end{center}
\end{table}

\begin{table}[!htb]
	\caption{CU041-Administrar torneos: Flujos de hechos}
	\label{tab:cu041_flujo}
	\begin{center}
		\resizebox{15cm}{!}{
		\begin{tabular}{|p{1.5cm}|p{6cm}|p{6.5cm}|}
			\hline
			\multicolumn{3}{|c|}{Detalle de flujo de hechos de caso de uso} \\
			\hline
			Nombre & \multicolumn{2}{|c|}{Nombre del flujo} \\
			\hline
			Paso & Acción del actor & Respuesta del sistema \\
			\hline
			1 & El usuario elige administrar eventos y elige el tipo de evento como uno de torneo & El sistema muestra un listado de eventos tipo torneo que el usuario administra \\
			\hline
		\end{tabular}
		} \\
		\textbf{Fuente}: Autores
	\end{center}
\end{table}

\begin{table}[!htb]
	\caption{CU042-Crear un torneo: Descripción}
	\label{tab:cu042_desc}
	\begin{center}
		\resizebox{15cm}{!}{
		\begin{tabular}{|p{4cm}|p{11cm}|}
			\hline
			\multicolumn{2}{|c|}{Descripción de caso de uso} \\
			\hline
			Nombre & Crear un torneo \\
			\hline
			Identificador & CU042 \\
			\hline
			Descripción & Permite la creación de un torneo \\
			\hline
			Actor & Todo actor de la red social	 
			\\
			\hline
			Disparador & Se elige la opción de crear un torneo \\
			\hline
			Inclusiones &  \\
			\hline
			Puntos de extensión &  \\
			\hline
			Precondiciones &  
			\begin{itemize}
				\item La aplicación ha sido cargada por un actor con rol de organizador de eventos deportivos
			\end{itemize}
			\\
			\hline
			Postcondiciones & 
			\begin{itemize}
				\item El usuario ha creado un torneo
			\end{itemize}
			\\
			\hline
			Notas & 
			\\
			\hline
		\end{tabular}
		} \\
		\textbf{Fuente}: Autores
	\end{center}
\end{table}

\begin{table}[!htb]
	\caption{CU042-Crear un torneo: Flujos de hechos}
	\label{tab:cu042_flujo}
	\begin{center}
		\resizebox{15cm}{!}{
		\begin{tabular}{|p{1.5cm}|p{6cm}|p{6.5cm}|}
			\hline
			\multicolumn{3}{|c|}{Detalle de flujo de hechos de caso de uso} \\
			\hline
			Nombre & \multicolumn{2}{|c|}{Nombre del flujo} \\
			\hline
			Paso & Acción del actor & Respuesta del sistema \\
			\hline
			1 & El usuario ha elegido administrar eventos de tipo torneo & El sistema ha mostrado una lista de torneos \\
			\hline
			2 & El usuario elige el botón de creación de eventos y elige el tipo de evento torneo & El sistema muestra la interfaz de creación de eventos tipo torneo \\
			\hline
			3 & El usuario ingresa toda la información que desee del torneo y pulsa el botón para confirmar la creación & El sistema guarda los cambios \\
			\hline
			4 &  & El sistema muestra un elemento emergente informando del éxito o fracaso de la operación \\
			\hline
			5 & El usuario continua & El sistema muestra la interfaz de administración de eventos deportivos de tipo torneo \\
			\hline
		\end{tabular}
		} \\
		\textbf{Fuente}: Autores
	\end{center}
\end{table}

\begin{table}[!htb]
	\caption{CU043-Actualizar información de torneo: Descripción}
	\label{tab:cu043_desc}
	\begin{center}
		\resizebox{15cm}{!}{
		\begin{tabular}{|p{4cm}|p{11cm}|}
			\hline
			\multicolumn{2}{|c|}{Descripción de caso de uso} \\
			\hline
			Nombre & Actualizar información de torneo \\
			\hline
			Identificador & CU043 \\
			\hline
			Descripción & Actualiza la información de un torneo \\
			\hline
			Actor & Todo actor de la red social	 
			\\
			\hline
			Disparador & Se elige la opción de actualizar información de torneo \\
			\hline
			Inclusiones &  \\
			\hline
			Puntos de extensión & 
			\begin{itemize}
				\item Gestionar formato de torneo
			\end{itemize}	
			\\
			\hline
			Precondiciones &  
			\begin{itemize}
				\item La aplicación ha sido cargada por un actor con rol de organizador de eventos deportivos
			\end{itemize}
			\\
			\hline
			Postcondiciones & 
			\begin{itemize}
				\item El usuario ha actualizado la información de un torneo
			\end{itemize}
			\\
			\hline
			Notas & 
			\begin{itemize}
				\item Generalización de:
				\begin{itemize}
					\item Generar calendario de encuentros
					\item Reportar resultado de encuentro
					\item Gestionar formatos de torneo
					\item Cancelar encuentro
					\item Modificar fecha de encuentro
					\item Crear fecha de encuentro
				\end{itemize}
			\end{itemize}
			\\
			\hline
		\end{tabular}
		} \\
		\textbf{Fuente}: Autores
	\end{center}
\end{table}

\begin{table}[!htb]
	\caption{CU043-Actualizar información de torneo: Flujos de hechos}
	\label{tab:cu043_flujo}
	\begin{center}
		\resizebox{15cm}{!}{
		\begin{tabular}{|p{1.5cm}|p{6cm}|p{6.5cm}|}
			\hline
			\multicolumn{3}{|c|}{Detalle de flujo de hechos de caso de uso} \\
			\hline
			Nombre & \multicolumn{2}{|c|}{Nombre del flujo} \\
			\hline
			Paso & Acción del actor & Respuesta del sistema \\
			\hline
			1 & El usuario ha elegido administrar eventos de tipo torneo & El sistema ha mostrado una lista de torneos \\
			\hline
			2 & El usuario busca un torneo en específico & El sistema da el resultado de la búsqueda \\
			\hline
			3 & El usuario elige uno de los torneos resultados de la búsqueda & El sistema muestra la interfaz de actualización de eventos tipo torneo \\
			\hline
			4 & El usuario ingresa toda la información que desee del torneo y pulsa el botón para confirmar la actualización & El sistema guarda los cambios \\
			\hline
			5 &  & El sistema muestra un elemento emergente informando del éxito o fracaso de la operación \\
			\hline
			6 & El usuario continua & El sistema muestra la interfaz de administración de eventos deportivos de tipo torneo \\
			\hline
		\end{tabular}
		} \\
		\textbf{Fuente}: Autores
	\end{center}
\end{table}

\begin{table}[!htb]
	\caption{CU044-Gestionar formato de torneo: Descripción}
	\label{tab:cu044_desc}
	\begin{center}
		\resizebox{15cm}{!}{
		\begin{tabular}{|p{4cm}|p{11cm}|}
			\hline
			\multicolumn{2}{|c|}{Descripción de caso de uso} \\
			\hline
			Nombre & Gestionar formato de torneo \\
			\hline
			Identificador & CU044 \\
			\hline
			Descripción & Permite la asignación de un formato al torneo realizado, así como también la organización de los equipos/jugadores participantes en dicho formato \\
			\hline
			Actor & Todo actor de la red social	 
			\\
			\hline
			Disparador & Se elige gestionar formato de evento \\
			\hline
			Inclusiones &  \\
			\hline
			Puntos de extensión & 
			\\
			\hline
			Precondiciones &  
			\begin{itemize}
				\item La aplicación ha sido cargada por un actor con rol de organizador de eventos deportivos
			\end{itemize}
			\\
			\hline
			Postcondiciones & 
			\begin{itemize}
				\item El usuario ha gestionado el formato de torneo
			\end{itemize}
			\\
			\hline
			Notas & 
			\\
			\hline
		\end{tabular}
		} \\
		\textbf{Fuente}: Autores
	\end{center}
\end{table}

\begin{table}[!htb]
	\caption{CU044-Gestionar formato de torneo: Flujos de hechos}
	\label{tab:cu044_flujo}
	\begin{center}
		\resizebox{15cm}{!}{
		\begin{tabular}{|p{1.5cm}|p{6cm}|p{6.5cm}|}
			\hline
			\multicolumn{3}{|c|}{Detalle de flujo de hechos de caso de uso} \\
			\hline
			Nombre & \multicolumn{2}{|c|}{Nombre del flujo} \\
			\hline
			Paso & Acción del actor & Respuesta del sistema \\
			\hline
			1 & El usuario eligió actualizar la información del torneo & El sistema mostró la interfaz de actualización \\
			\hline
			2 & El usuario elige gestionar el formato del torneo & El sistema muestra la interfaz para la gestión del torneo \\
			\hline
			3 & El usuario asigna participantes al los partidos mostrados en el tipo de formato escogido y guarda la información & El sistema guarda los cambios \\
			\hline
			5 &  & El sistema muestra un elemento emergente informando del éxito o fracaso de la operación \\
			\hline
			6 & El usuario continua & El sistema muestra la interfaz de administración de eventos deportivos de tipo torneo \\
			\hline
		\end{tabular}
		} \\
		\textbf{Fuente}: Autores
	\end{center}
\end{table}

\begin{table}[!htb]
	\caption{CU045-Agregar participante a torneo: Descripción}
	\label{tab:cu045_desc}
	\begin{center}
		\resizebox{15cm}{!}{
		\begin{tabular}{|p{4cm}|p{11cm}|}
			\hline
			\multicolumn{2}{|c|}{Descripción de caso de uso} \\
			\hline
			Nombre & Agregar participante a torneo \\
			\hline
			Identificador & CU045 \\
			\hline
			Descripción & Agrega un equipo o jugador al torneo sin asignarlo a algún puesto en el formato elegido por el organizador del torneo \\
			\hline
			Actor & Todo actor de la red social	 
			\\
			\hline
			Disparador & Se elige agregar participante a torneo \\
			\hline
			Inclusiones &  \\
			\hline
			Puntos de extensión & 
			\\
			\hline
			Precondiciones &  
			\begin{itemize}
				\item La aplicación ha sido cargada por un actor con rol de organizador de eventos deportivos
			\end{itemize}
			\\
			\hline
			Postcondiciones & 
			\begin{itemize}
				\item El usuario ha agregado un participante al torneo
			\end{itemize}
			\\
			\hline
			Notas & 
			\\
			\hline
		\end{tabular}
		} \\
		\textbf{Fuente}: Autores
	\end{center}
\end{table}

\begin{table}[!htb]
	\caption{CU045-Agregar participante a torneo: Flujos de hechos}
	\label{tab:cu045_flujo}
	\begin{center}
		\resizebox{15cm}{!}{
		\begin{tabular}{|p{1.5cm}|p{6cm}|p{6.5cm}|}
			\hline
			\multicolumn{3}{|c|}{Detalle de flujo de hechos de caso de uso} \\
			\hline
			Nombre & \multicolumn{2}{|c|}{Nombre del flujo} \\
			\hline
			Paso & Acción del actor & Respuesta del sistema \\
			\hline
			1 & El usuario ha elegido un torneo en específico & El sistema ha mostrado los datos del torneo \\
			\hline
			2 & El usuario elige gestionar involucrados & El sistema muestra las opciones de administración de involucrados \\
			\hline
			3 & El usuario elige agregar un participante al torneo & El sistema muestra la interfaz de adición de participantes al torneo \\
			\hline
			4 & El usuario busca un posible participante en la red social para invitar/aceptar como participante en el torneo & El sistema envía la invitación o acepta al participante \\
			\hline
		\end{tabular}
		} \\
		\textbf{Fuente}: Autores
	\end{center}
\end{table}

\begin{table}[!htb]
	\caption{CU046-Retirar participante de torneo: Descripción}
	\label{tab:cu046_desc}
	\begin{center}
		\resizebox{15cm}{!}{
		\begin{tabular}{|p{4cm}|p{11cm}|}
			\hline
			\multicolumn{2}{|c|}{Descripción de caso de uso} \\
			\hline
			Nombre & Retirar equipo de torneo \\
			\hline
			Identificador & CU046 \\
			\hline
			Descripción & Retira un equipo o jugador del torneo \\
			\hline
			Actor & Todo actor de la red social	 
			\\
			\hline
			Disparador & Se elige retirar participante del torneo \\
			\hline
			Inclusiones &  \\
			\hline
			Puntos de extensión & 
			\\
			\hline
			Precondiciones &  
			\begin{itemize}
				\item La aplicación ha sido cargada por un actor con rol de organizador de eventos deportivos
			\end{itemize}
			\\
			\hline
			Postcondiciones & 
			\begin{itemize}
				\item El usuario ha retirado un participante del torneo
			\end{itemize}
			\\
			\hline
			Notas & 
			\\
			\hline
		\end{tabular}
		} \\
		\textbf{Fuente}: Autores
	\end{center}
\end{table}

\begin{table}[!htb]
	\caption{CU046-Retirar equipo de torneo: Flujos de hechos}
	\label{tab:cu046_flujo}
	\begin{center}
		\resizebox{15cm}{!}{
		\begin{tabular}{|p{1.5cm}|p{6cm}|p{6.5cm}|}
			\hline
			\multicolumn{3}{|c|}{Detalle de flujo de hechos de caso de uso} \\
			\hline
			Nombre & \multicolumn{2}{|c|}{Nombre del flujo} \\
			\hline
			Paso & Acción del actor & Respuesta del sistema \\
			\hline
			1 & El usuario ha elegido un torneo en específico & El sistema ha mostrado los datos del torneo \\
			\hline
			2 & El usuario elige gestionar involucrados & El sistema muestra las opciones de administración de involucrados \\
			\hline
			3 & El usuario consultar los participantes del torneo & El sistema muestra la lista de participantes del torneo \\
			\hline
			4 & El usuario busca un participante entre los existentes de momento & El sistema muestra los resultados de la búsqueda \\
			\hline
			5 & El usuario elige uno de los participantes y lo elimina & El sistema guarda los cambios \\
			\hline
			6 &  & El sistema muestra un elemento emergente informando del éxito o fracaso de la operación \\
			\hline
			7 & El usuario continua & El sistema muestra la interfaz de consulta de los participantes del torneo \\
			\hline
		\end{tabular}
		} \\
		\textbf{Fuente}: Autores
	\end{center}
\end{table}

\begin{table}[!htb]
	\caption{CU047-Generar calendario de encuentros: Descripción}
	\label{tab:cu047_desc}
	\begin{center}
		\resizebox{15cm}{!}{
		\begin{tabular}{|p{4cm}|p{11cm}|}
			\hline
			\multicolumn{2}{|c|}{Descripción de caso de uso} \\
			\hline
			Nombre & Generar calendario de encuentros \\
			\hline
			Identificador & CU047 \\
			\hline
			Descripción & Genera el calendario de los encuentros a realizarse en el evento \\
			\hline
			Actor & Todo actor de la red social	 
			\\
			\hline
			Disparador & Se elige generar calendario de encuentros \\
			\hline
			Inclusiones &  \\
			\hline
			Puntos de extensión & 
			\\
			\hline
			Precondiciones &  
			\begin{itemize}
				\item La aplicación ha sido cargada por un actor con rol de organizador de eventos deportivos
			\end{itemize}
			\\
			\hline
			Postcondiciones & 
			\begin{itemize}
				\item El usuario ha generado el calendario de encuentros
			\end{itemize}
			\\
			\hline
			Notas & 
			\\
			\hline
		\end{tabular}
		} \\
		\textbf{Fuente}: Autores
	\end{center}
\end{table}

\begin{table}[!htb]
	\caption{CU047-Generar calendario de encuentros: Flujos de hechos}
	\label{tab:cu047_flujo}
	\begin{center}
		\resizebox{15cm}{!}{
		\begin{tabular}{|p{1.5cm}|p{6cm}|p{6.5cm}|}
			\hline
			\multicolumn{3}{|c|}{Detalle de flujo de hechos de caso de uso} \\
			\hline
			Nombre & \multicolumn{2}{|c|}{Nombre del flujo} \\
			\hline
			Paso & Acción del actor & Respuesta del sistema \\
			\hline
			1 & El usuario ha elegido gestionar el formato del torneo  & El sistema ha mostrado la interfaz para tratar el formato del torneo \\
			\hline
			2 & El usuario elige generar el calendario de encuentros automáticamente & El sistema genera el calendario de encuentros automáticamente\\
			\hline
		\end{tabular}
		} \\
		\textbf{Fuente}: Autores
	\end{center}
\end{table}

\clearpage

\begin{table}[!htb]
	\caption{CU048-Reportar resultado de encuentro: Descripción}
	\label{tab:cu048_desc}
	\begin{center}
		\resizebox{15cm}{!}{
		\begin{tabular}{|p{4cm}|p{11cm}|}
			\hline
			\multicolumn{2}{|c|}{Descripción de caso de uso} \\
			\hline
			Nombre & Reportar resultado de encuentro \\
			\hline
			Identificador & CU048 \\
			\hline
			Descripción & Permite reportar el resultado de un encuentro deportivo después de haber iniciado el torneo \\
			\hline
			Actor & Todo actor de la red social	 
			\\
			\hline
			Disparador & Se elige reportar resultado de un encuentro \\
			\hline
			Inclusiones &  \\
			\hline
			Puntos de extensión & 
			\\
			\hline
			Precondiciones &  
			\begin{itemize}
				\item La aplicación ha sido cargada por un actor con rol de organizador de eventos deportivos
			\end{itemize}
			\\
			\hline
			Postcondiciones & 
			\begin{itemize}
				\item El sistema reporta el resultado del encuentro
			\end{itemize}
			\\
			\hline
			Notas & 
			\\
			\hline
		\end{tabular}
		} \\
		\textbf{Fuente}: Autores
	\end{center}
\end{table}

\begin{table}[!htb]
	\caption{CU048-Reportar resultado de encuentro: Flujos de hechos}
	\label{tab:cu048_flujo}
	\begin{center}
		\resizebox{15cm}{!}{
		\begin{tabular}{|p{1.5cm}|p{6cm}|p{6.5cm}|}
			\hline
			\multicolumn{3}{|c|}{Detalle de flujo de hechos de caso de uso} \\
			\hline
			Nombre & \multicolumn{2}{|c|}{Nombre del flujo} \\
			\hline
			Paso & Acción del actor & Respuesta del sistema \\
			\hline
			1 & El usuario ha elegido gestionar un encuentro & El sistema ha mostrado la interfaz de gestión del encuentro \\
			\hline
			2 & El usuario pone los resultados del encuentro y establece los cambios & El sistema envía notificaciones a los involucrados acerca de los resultados del encuentro \\
			\hline
			3 &  & El sistema muestra un elemento emergente informando del éxito o fracaso de la operación \\
			\hline
			4 & El usuario continua & El sistema muestra la interfaz de gestión del formato del torneo \\
			\hline
		\end{tabular}
		} \\
		\textbf{Fuente}: Autores
	\end{center}
\end{table}

\begin{table}[!htb]
	\caption{CU049-Crear fecha de encuentro: Descripción}
	\label{tab:cu049_desc}
	\begin{center}
		\resizebox{15cm}{!}{
		\begin{tabular}{|p{4cm}|p{11cm}|}
			\hline
			\multicolumn{2}{|c|}{Descripción de caso de uso} \\
			\hline
			Nombre & Crear fecha de encuentro \\
			\hline
			Identificador & CU049 \\
			\hline
			Descripción & Permite poner una fecha a un encuentro en el torneo \\
			\hline
			Actor & Todo actor de la red social	 
			\\
			\hline
			Disparador & Se elige poner una fecha en un encuentro del torneo \\
			\hline
			Inclusiones &  \\
			\hline
			Puntos de extensión & 
			\\
			\hline
			Precondiciones &  
			\begin{itemize}
				\item La aplicación ha sido cargada por un actor con rol de organizador de eventos deportivos
			\end{itemize}
			\\
			\hline
			Postcondiciones & 
			\begin{itemize}
				\item El usuario pone una fecha al encuentro elegido
			\end{itemize}
			\\
			\hline
			Notas & 
			\\
			\hline
		\end{tabular}
		} \\
		\textbf{Fuente}: Autores
	\end{center}
\end{table}

\begin{table}[!htb]
	\caption{CU049-Crear fecha de encuentro: Flujos de hechos}
	\label{tab:cu049_flujo}
	\begin{center}
		\resizebox{15cm}{!}{
		\begin{tabular}{|p{1.5cm}|p{6cm}|p{6.5cm}|}
			\hline
			\multicolumn{3}{|c|}{Detalle de flujo de hechos de caso de uso} \\
			\hline
			Nombre & \multicolumn{2}{|c|}{Nombre del flujo} \\
			\hline
			Paso & Acción del actor & Respuesta del sistema \\
			\hline
			1 & El usuario ha elegido gestionar formato del torneo & El sistema ha mostrado la interfaz de gestión del formato del torneo \\
			\hline
			2 & El usuario elige uno de los encuentros mostrados & El sistema muestra la interfaz de gestión de detalles de encuentro \\
			\hline
			3 & El usuario modifica las fechas/horas del encuentro & El sistema hace los cambios en hora/fecha y envía notificaciones a los involucrados \\
			\hline
		\end{tabular}
		} \\
		\textbf{Fuente}: Autores
	\end{center}
\end{table}

\begin{table}[!htb]
	\caption{CU050-Modificar fecha de encuentro: Descripción}
	\label{tab:cu050_desc}
	\begin{center}
		\resizebox{15cm}{!}{
		\begin{tabular}{|p{4cm}|p{11cm}|}
			\hline
			\multicolumn{2}{|c|}{Descripción de caso de uso} \\
			\hline
			Nombre & Modificar fecha de encuentro \\
			\hline
			Identificador & CU050 \\
			\hline
			Descripción & Permite modificar la fecha de un encuentro \\
			\hline
			Actor & Todo actor de la red social	 
			\\
			\hline
			Disparador & Se elige cambiar la fecha de un encuentro del torneo \\
			\hline
			Inclusiones &  \\
			\hline
			Puntos de extensión & 
			\\
			\hline
			Precondiciones &  
			\begin{itemize}
				\item La aplicación ha sido cargada por un actor con rol de organizador de eventos deportivos
			\end{itemize}
			\\
			\hline
			Postcondiciones & 
			\begin{itemize}
				\item El usuario ha actualizado la fecha del encuentro
			\end{itemize}
			\\
			\hline
			Notas & \\
			\hline
		\end{tabular}
		} \\
		\textbf{Fuente}: Autores
	\end{center}
\end{table}

\begin{table}[!htb]
	\caption{CU050-Modificar fecha de encuentro: Flujos de hechos}
	\label{tab:cu050_flujo}
	\begin{center}
		\resizebox{15cm}{!}{
		\begin{tabular}{|p{1.5cm}|p{6cm}|p{6.5cm}|}
			\hline
			\multicolumn{3}{|c|}{Detalle de flujo de hechos de caso de uso} \\
			\hline
			Nombre & \multicolumn{2}{|c|}{Nombre del flujo} \\
			\hline
			Paso & Acción del actor & Respuesta del sistema \\
			\hline
			1 & El usuario ha elegido gestionar formato del torneo & El sistema ha mostrado la interfaz de gestión del formato del torneo \\
			\hline
			2 & El usuario elige uno de los encuentros mostrados & El sistema muestra la interfaz de gestión de detalles de encuentro \\
			\hline
			3 & El usuario modifica las fechas/horas del encuentro & El sistema hace los cambios en hora/fecha y envía notificaciones a los involucrados \\
			\hline
		\end{tabular}
		} \\
		\textbf{Fuente}: Autores
	\end{center}
\end{table}

\begin{table}[!htb]
	\caption{CU051-Cancelar encuentro: Descripción}
	\label{tab:cu051_desc}
	\begin{center}
		\resizebox{15cm}{!}{
		\begin{tabular}{|p{4cm}|p{11cm}|}
			\hline
			\multicolumn{2}{|c|}{Descripción de caso de uso} \\
			\hline
			Nombre & Cancelar encuentro\\
			\hline
			Identificador & CU051 \\
			\hline
			Descripción & Permite cancelar un encuentro deportivo \\
			\hline
			Actor & Todo actor de la red social	 
			\\
			\hline
			Disparador & Se elige cancelar un encuentro deportivo \\
			\hline
			Inclusiones &  \\
			\hline
			Puntos de extensión & 
			\\
			\hline
			Precondiciones &  
			\begin{itemize}
				\item La aplicación ha sido cargada por un actor con rol de organizador de eventos deportivos
			\end{itemize}
			\\
			\hline
			Postcondiciones & 
			\begin{itemize}
				\item El usuario ha cancelado un encuentro
			\end{itemize}
			\\
			\hline
			Notas & \\
			\hline
		\end{tabular}
		} \\
		\textbf{Fuente}: Autores
	\end{center}
\end{table}

\begin{table}[!htb]
	\caption{CU051-Cancelar encuentro: Flujos de hechos}
	\label{tab:cu051_flujo}
	\begin{center}
		\resizebox{15cm}{!}{
		\begin{tabular}{|p{1.5cm}|p{6cm}|p{6.5cm}|}
			\hline
			\multicolumn{3}{|c|}{Detalle de flujo de hechos de caso de uso} \\
			\hline
			Nombre & \multicolumn{2}{|c|}{Nombre del flujo} \\
			\hline
			Paso & Acción del actor & Respuesta del sistema \\
			\hline
			1 & El usuario ha elegido gestionar formato del torneo & El sistema ha mostrado la interfaz de gestión del formato del torneo \\
			\hline
			2 & El usuario elige uno de los encuentros mostrados & El sistema muestra la interfaz de gestión de detalles de encuentro \\
			\hline
			3 & El usuario elige cancelar el encuentro & El sistema cancela el encuentro y envía notificaciones a los involucrados \\
			\hline
			4 &  & El sistema muestra un elemento emergente informando del éxito o fracaso de la operación \\
			\hline
			5 & El usuario continua & El sistema muestra la interfaz de gestión de formato de torneo \\
			\hline
		\end{tabular}
		} \\
		\textbf{Fuente}: Autores
	\end{center}
\end{table}

\begin{table}[!htb]
	\caption{CU052-Notificar a involucrado de un encuentro próximo: Descripción}
	\label{tab:cu052_desc}
	\begin{center}
		\resizebox{15cm}{!}{
		\begin{tabular}{|p{4cm}|p{11cm}|}
			\hline
			\multicolumn{2}{|c|}{Descripción de caso de uso} \\
			\hline
			Nombre & Notificar a involucrado de un encuentro próximo \\
			\hline
			Identificador & CU052 \\
			\hline
			Descripción & Permite dar alertas a involucrados acerca de los encuentros que están próximos a disputarse \\
			\hline
			Actor & Todo actor de la red social	 
			\\
			\hline
			Disparador & El sistema reconoce que llega al tiempo límite en el cual requiere notificar a los involucrados de los encuentros próximos \\
			\hline
			Inclusiones &  \\
			\hline
			Puntos de extensión & 
			\\
			\hline
			Precondiciones &  
			\begin{itemize}
				\item Se llega al tiempo para notificar de los encuentros próximos
			\end{itemize}
			\\
			\hline
			Postcondiciones & 
			\begin{itemize}
				\item El sistema notifica a los involucrados de los próximos encuentros a disputar
			\end{itemize}
			\\
			\hline
			Notas & 
			\\
			\hline
		\end{tabular}
		} \\
		\textbf{Fuente}: Autores
	\end{center}
\end{table}

\begin{table}[!htb]
	\caption{CU052-Notificar a involucrado de un encuentro próximo: Flujos de hechos}
	\label{tab:cu052_flujo}
	\begin{center}
		\resizebox{15cm}{!}{
		\begin{tabular}{|p{1.5cm}|p{6cm}|p{6.5cm}|}
			\hline
			\multicolumn{3}{|c|}{Detalle de flujo de hechos de caso de uso} \\
			\hline
			Nombre & \multicolumn{2}{|c|}{Nombre del flujo} \\
			\hline
			Paso & Acción del actor & Respuesta del sistema \\
			\hline
			1 & & El sistema reconoce el tiempo límite que le falta a un encuentro para empezar y envía una notificación a los involucrados \\
			\hline
		\end{tabular}
		} \\
		\textbf{Fuente}: Autores
	\end{center}
\end{table}

\begin{table}[!htb]
	\caption{CU053-Crear torneo por equipos: Descripción}
	\label{tab:cu053_desc}
	\begin{center}
		\resizebox{15cm}{!}{
		\begin{tabular}{|p{4cm}|p{11cm}|}
			\hline
			\multicolumn{2}{|c|}{Descripción de caso de uso} \\
			\hline
			Nombre & Crear torneo por equipos \\
			\hline
			Identificador & CU053 \\
			\hline
			Descripción & Permite crear un torneo con el distintivo de ser por equipos \\
			\hline
			Actor & Todo actor de la red social	 
			\\
			\hline
			Disparador & Se elige crear un torneo por equipos \\
			\hline
			Inclusiones &  \\
			\hline
			Puntos de extensión & 
			\\
			\hline
			Precondiciones &  
			\begin{itemize}
				\item La aplicación ha sido cargada por un actor con rol de organizador de eventos deportivos
			\end{itemize}
			\\
			\hline
			Postcondiciones & 
			\begin{itemize}
				\item El usuario crea un torneo por equipos
			\end{itemize}
			\\
			\hline
			Notas & 
			\\
			\hline
		\end{tabular}
		} \\
		\textbf{Fuente}: Autores
	\end{center}
\end{table}

\begin{table}[!htb]
	\caption{CU053-Crear torneo por equipos: Flujos de hechos}
	\label{tab:cu053_flujo}
	\begin{center}
		\resizebox{15cm}{!}{
		\begin{tabular}{|p{1.5cm}|p{6cm}|p{6.5cm}|}
			\hline
			\multicolumn{3}{|c|}{Detalle de flujo de hechos de caso de uso} \\
			\hline
			Nombre & \multicolumn{2}{|c|}{Nombre del flujo} \\
			\hline
			Paso & Acción del actor & Respuesta del sistema \\
			\hline
			 & & \\
			\hline
		\end{tabular}
		} \\
		\textbf{Fuente}: Autores
	\end{center}
\end{table}

\begin{table}[!htb]
	\caption{CU054-Crear torneo individual: Descripción}
	\label{tab:cu054_desc}
	\begin{center}
		\resizebox{15cm}{!}{
		\begin{tabular}{|p{4cm}|p{11cm}|}
			\hline
			\multicolumn{2}{|c|}{Descripción de caso de uso} \\
			\hline
			Nombre & Crear torneo individual \\
			\hline
			Identificador & CU054 \\
			\hline
			Descripción & Permite la creación de un torneo con un indicador de ser torneo individual \\
			\hline
			Actor & Todo actor de la red social	 
			\\
			\hline
			Disparador & Se elige crear un torneo individual \\
			\hline
			Inclusiones &  \\
			\hline
			Puntos de extensión & 
			\\
			\hline
			Precondiciones &  
			\begin{itemize}
				\item La aplicación ha sido cargada por un actor con rol de organizador de eventos deportivos
			\end{itemize}
			\\
			\hline
			Postcondiciones & 
			\begin{itemize}
				\item El usuario crea un torneo individual
			\end{itemize}
			\\
			\hline
			Notas & 
			\\
			\hline
		\end{tabular}
		} \\
		\textbf{Fuente}: Autores
	\end{center}
\end{table}

\begin{table}[!htb]
	\caption{CU054-Crear torneo individual: Flujos de hechos}
	\label{tab:cu054_flujo}
	\begin{center}
		\resizebox{15cm}{!}{
		\begin{tabular}{|p{1.5cm}|p{6cm}|p{6.5cm}|}
			\hline
			\multicolumn{3}{|c|}{Detalle de flujo de hechos de caso de uso} \\
			\hline
			Nombre & \multicolumn{2}{|c|}{Nombre del flujo} \\
			\hline
			Paso & Acción del actor & Respuesta del sistema \\
			\hline
			1 & El usuario escoge crear un torneo & El sistema muestra la interfaz de creación del torneo \\
			\hline
			2 & El usuario escoge poner el indicador de torneo individual & El sistema carga las restricciones de los torneos individuales \\
			\hline
			3 & El usuario ingresa los demás datos del torneo y salva la información & El sistema guarda la información \\
			\hline
			4 &  & El sistema muestra un elemento emergente informando del éxito o fracaso de la operación \\
			\hline
			5 & El usuario continua & El sistema muestra la interfaz de detalles del torneo \\
			\hline
		\end{tabular}
		} \\
		\textbf{Fuente}: Autores
	\end{center}
\end{table}

\clearpage

\section{Módulo de administración de equipos y grupos deportivos}

%\input{./imagenes/casos_uso/gestion_equipo_grupo.png}

A éste módulo le corresponde ofrecer funcionalidades de administración de equipos y grupos deportivos. Se decidió dejar la administración de ámbos conceptos (equipo y grupo deportivo) en el mismo módulo ya que, para efectos de las funcionalidades ofrecidas por el SNS, son bastante similares. Adicionalmente, un grupo deportivo no podrá utilizar todas las funcionalidades ofrecidas en el presente módulo, esto debido a su caracter limitado de grupo informal.

\clearpage

\begin{table}[!htb]
	\caption{CU054-Administrar equipos/grupos: Descripción}
	\label{tab:cu054_desc}
	\begin{center}
		\resizebox{15cm}{!}{
		\begin{tabular}{|p{4cm}|p{11cm}|}
			\hline
			\multicolumn{2}{|c|}{Descripción de caso de uso} \\
			\hline
			Nombre & Administrar equipos/grupos \\
			\hline
			Identificador & CU054 \\
			\hline
			Descripción & Permite administrar un equipo en la red social deportiva \\
			\hline
			Actor & Todo actor de la red social	 
			\\
			\hline
			Disparador & Administrar un equipo en la red social deportiva \\
			\hline
			Inclusiones & \\
			\hline
			Puntos de extensión &
			\\
			\hline
			Precondiciones &  
			\begin{itemize}
				\item La aplicación ha sido cargada por un actor con rol de formador de grupos deportivos
			\end{itemize}
			\\
			\hline
			Postcondiciones & \\
			\hline
			Notas & 
			\begin{itemize}
				\item Generalización de:
				\begin{itemize}
					\item Crear un equipo
					\item Actualizar información de equipo/grupo
					\item Actualizar información de integrante de equipo/grupo
					\item Crear grupo deportivo
					\item Solicitar patrocinio
				\end{itemize}
			\end{itemize}
			\\
			\hline
		\end{tabular}
		} \\
		\textbf{Fuente}: Autores
	\end{center}
\end{table}

\begin{table}[!htb]
	\caption{CU054-Administrar equipos/grupos: Flujos de hechos}
	\label{tab:cu054_flujo}
	\begin{center}
		\resizebox{15cm}{!}{
		\begin{tabular}{|p{1.5cm}|p{6cm}|p{6.5cm}|}
			\hline
			\multicolumn{3}{|c|}{Detalle de flujo de hechos de caso de uso} \\
			\hline
			Nombre & \multicolumn{2}{|c|}{Nombre del flujo} \\
			\hline
			Paso & Acción del actor & Respuesta del sistema \\
			\hline
			1 & Elige la opción de administrar grupos deportivos & El sistema muestra la interfaz con la consulta de los equipos y grupos creados por el usuario \\
			\hline
		\end{tabular}
		} \\
		\textbf{Fuente}: Autores
	\end{center}
\end{table}

\begin{table}[!htb]
	\caption{CU055-Crear un equipo: Descripción}
	\label{tab:cu055_desc}
	\begin{center}
		\resizebox{15cm}{!}{
		\begin{tabular}{|p{4cm}|p{11cm}|}
			\hline
			\multicolumn{2}{|c|}{Descripción de caso de uso} \\
			\hline
			Nombre & Crear un equipo \\
			\hline
			Identificador & CU055 \\
			\hline
			Descripción & Permite la creación de un equipo deportivo en la red social deportiva \\
			\hline
			Actor & Todo actor de la red social	 
			\\
			\hline
			Disparador & Se elige crear un equipo deportivo \\
			\hline
			Inclusiones &  \\
			\hline
			Puntos de extensión & 
			\\
			\hline
			Precondiciones &  
			\begin{itemize}
				\item La aplicación ha sido cargada por un actor con rol de formador de grupos deportivos
			\end{itemize}
			\\
			\hline
			Postcondiciones & 
			\begin{itemize}
				\item El usuario crea un equipo deportivo
			\end{itemize}
			\\
			\hline
			Notas & 
			\\
			\hline
		\end{tabular}
		} \\
		\textbf{Fuente}: Autores
	\end{center}
\end{table}

\begin{table}[!htb]
	\caption{CU055-Crear un equipo: Flujos de hechos}
	\label{tab:cu055_flujo}
	\begin{center}
		\resizebox{15cm}{!}{
		\begin{tabular}{|p{1.5cm}|p{6cm}|p{6.5cm}|}
			\hline
			\multicolumn{3}{|c|}{Detalle de flujo de hechos de caso de uso} \\
			\hline
			Nombre & \multicolumn{2}{|c|}{Nombre del flujo} \\
			\hline
			Paso & Acción del actor & Respuesta del sistema \\
			\hline
			1 & El usuario ha elegido la opción de administrar grupos deportivos & El sistema ha mostrado la interfaz con la consulta de los equipos y grupos creados por el usuario \\
			\hline
			2 & El usuario pulsa el botón para la creación de equipos deportivos & El sistema muestra la interfaz de creación con todos los datos a ser incluidos \\
			\hline
			3 & El usuario ingresa los datos del equipo y salva la información & El sistema guarda el equipo en sistema \\
			\hline
			4 & & El sistema muestra un elemento emergente informando del éxito o fracaso de la operación \\
			\hline
			5 & El usuario continua & El sistema muestra la interfaz de gestión de equipos/grupos deportivos \\
			\hline
		\end{tabular}
		} \\
		\textbf{Fuente}: Autores
	\end{center}
\end{table}

\begin{table}[!htb]
	\caption{CU056-Actualizar información de equipo/grupo: Descripción}
	\label{tab:cu056_desc}
	\begin{center}
		\resizebox{15cm}{!}{
		\begin{tabular}{|p{4cm}|p{11cm}|}
			\hline
			\multicolumn{2}{|c|}{Descripción de caso de uso} \\
			\hline
			Nombre & Actualizar información de equipo/grupo \\
			\hline
			Identificador & CU056 \\
			\hline
			Descripción & Permite actualizar la información general de un equipo/grupo deportivo \\
			\hline
			Actor & Todo actor de la red social	 
			\\
			\hline
			Disparador & Se elige actualizar la información de un equipo/grupo \\
			\hline
			Inclusiones &  \\
			\hline
			Puntos de extensión & 
			\\
			\hline
			Precondiciones &  
			\begin{itemize}
				\item La aplicación ha sido cargada por un actor con rol de formador de grupos deportivos
			\end{itemize}
			\\
			\hline
			Postcondiciones & 
			\begin{itemize}
				\item El usuario actualiza la información del equipo/grupo deportivo
			\end{itemize}
			\\
			\hline
			Notas & 
			\begin{itemize}
				\item Generalización de:
				\begin{itemize}
					\item Dar de baja a integrante de equipo/grupo
					\item Agregar un integrante a un equipo/grupo
				\end{itemize}
			\end{itemize}
			\\
			\hline
		\end{tabular}
		} \\
		\textbf{Fuente}: Autores
	\end{center}
\end{table}

\begin{table}[!htb]
	\caption{CU056-Actualizar información de equipo/grupo: Flujos de hechos}
	\label{tab:cu056_flujo}
	\begin{center}
		\resizebox{15cm}{!}{
		\begin{tabular}{|p{1.5cm}|p{6cm}|p{6.5cm}|}
			\hline
			\multicolumn{3}{|c|}{Detalle de flujo de hechos de caso de uso} \\
			\hline
			Nombre & \multicolumn{2}{|c|}{Nombre del flujo} \\
			\hline
			Paso & Acción del actor & Respuesta del sistema \\
			\hline
			1 & El usuario ha elegido la opción de administrar grupos deportivos & El sistema ha mostrado la interfaz con la consulta de los equipos y grupos creados por el usuario \\
			\hline
			2 & El usuario busca un equipo/grupo en específico & El sistema devuelve las coincidencias de la búsqueda \\
			\hline
			3 & El usuario elige un equipo/grupo deportivo de las coincidencias & El sistema muestra los detalles del equipo/grupo elegido \\
			\hline
			4 & El usuario actualiza la información del equipo/grupo deportivo y salva la información & El sistema guarda los cambios \\
			\hline
			5 & & El sistema muestra un elemento emergente informando del éxito o fracaso de la operación \\
			\hline
			6 & El usuario continua & El sistema muestra la interfaz de gestión de equipos/grupos deportivos \\
			\hline
		\end{tabular}
		} \\
		\textbf{Fuente}: Autores
	\end{center}
\end{table}

\begin{table}[!htb]
	\caption{CU057-Gestionar integrantes equipo/grupo: Descripción}
	\label{tab:cu057_desc}
	\begin{center}
		\resizebox{15cm}{!}{
		\begin{tabular}{|p{4cm}|p{11cm}|}
			\hline
			\multicolumn{2}{|c|}{Descripción de caso de uso} \\
			\hline
			Nombre & Gestionar integrantes equipo/grupo \\
			\hline
			Identificador & CU057 \\
			\hline
			Descripción & Permite gesionar los integrantes de un equipo/grupo \\
			\hline
			Actor & Todo actor de la red social	 
			\\
			\hline
			Disparador & Se elige gestionar integrantes de un equipo/grupo \\
			\hline
			Inclusiones &  \\
			\hline
			Puntos de extensión & 
			\\
			\hline
			Precondiciones &  
			\begin{itemize}
				\item La aplicación ha sido cargada por un actor con rol de formador de grupos deportivos
			\end{itemize}
			\\
			\hline
			Postcondiciones & \\
			\hline
			Notas &
			\begin{itemize}
				\item Generalización de:
				\begin{itemize}
					\item Agregar un integrante a un grupo/equipo
					\item Actualizar información de integrante de grupo/equipo
					\item Dar de baja a un integrante a un grupo/equipo
				\end{itemize}
			\end{itemize}
			\\
			\hline
		\end{tabular}
		} \\
		\textbf{Fuente}: Autores
	\end{center}
\end{table}

\begin{table}[!htb]
	\caption{CU057-Gestionar integrantes equipo/grupo: Flujos de hechos}
	\label{tab:cu057_flujo}
	\begin{center}
		\resizebox{15cm}{!}{
		\begin{tabular}{|p{1.5cm}|p{6cm}|p{6.5cm}|}
			\hline
			\multicolumn{3}{|c|}{Detalle de flujo de hechos de caso de uso} \\
			\hline
			Nombre & \multicolumn{2}{|c|}{Nombre del flujo} \\
			\hline
			Paso & Acción del actor & Respuesta del sistema \\
			\hline
			1 & El usuario ha elegido un equipo/grupo deportivo & El sistema ha mostrado los detalles del equipo/grupo elegido \\
			\hline
			2 & El usuario elige, en las acciones del equipo/grupo, gestionar integrantes & El sistema muestra la lista de integrantes del grupo/equipo \\
			\hline
		\end{tabular}
		} \\
		\textbf{Fuente}: Autores
	\end{center}
\end{table}

\begin{table}[!htb]
	\caption{CU058-Agregar un integrante a un equipo/grupo: Descripción}
	\label{tab:cu058_desc}
	\begin{center}
		\resizebox{15cm}{!}{
		\begin{tabular}{|p{4cm}|p{11cm}|}
			\hline
			\multicolumn{2}{|c|}{Descripción de caso de uso} \\
			\hline
			Nombre & Agregar/invitar un integrante a un equipo/grupo \\
			\hline
			Identificador & CU058 \\
			\hline
			Descripción & Permite agregar un integrante al equipo/grupo elegido \\
			\hline
			Actor & Todo actor de la red social	 
			\\
			\hline
			Disparador & Se elige agregar un integrante a un equipo/grupo \\
			\hline
			Inclusiones &  \\
			\hline
			Puntos de extensión & 
			\\
			\hline
			Precondiciones &  
			\begin{itemize}
				\item La aplicación ha sido cargada por un actor con rol de formador de grupos deportivos
			\end{itemize}
			\\
			\hline
			Postcondiciones & 
			\begin{itemize}
				\item El usuario agrega un integrante al equipo/grupo
			\end{itemize}
			\\
			\hline
			Notas & 
			\\
			\hline
		\end{tabular}
		} \\
		\textbf{Fuente}: Autores
	\end{center}
\end{table}

\begin{table}[!htb]
	\caption{CU058-Agregar un integrante a un equipo/grupo: Flujos de hechos}
	\label{tab:cu058_flujo}
	\begin{center}
		\resizebox{15cm}{!}{
		\begin{tabular}{|p{1.5cm}|p{6cm}|p{6.5cm}|}
			\hline
			\multicolumn{3}{|c|}{Detalle de flujo de hechos de caso de uso} \\
			\hline
			Nombre & \multicolumn{2}{|c|}{Nombre del flujo} \\
			\hline
			Paso & Acción del actor & Respuesta del sistema \\
			\hline
			1 & El usuario ha elegido un equipo/grupo deportivo & El sistema ha mostrado los detalles del equipo/grupo elegido \\
			\hline
			2 & El usuario elige, en las acciones del equipo/grupo, gestionar integrantes & El sistema muestra la lista de integrantes del grupo/equipo \\
			\hline
			3 & El usuario pulsa el botón de agregar usuario a equipo/grupo & El sistema muestra la interfaz de adición \\
			\hline
			4 & El usuario busca otro usuario entre la red social o las solicitudes de unión al grupo/equipo & El sistema muestra los resultados de la búsqueda \\
			5 & El usuario elige un usuario entre los resultados de la búsqueda y pulsa el botón de adición & El sistema agrega o invita al usuario elegido \\
			6 &  & El sistema muestra un elemento emergente informando del éxito o fracaso de la operación \\
			\hline
			7 & El usuario continua & El sistema muestra la interfaz de integrantes de equipos/grupos deportivos \\
			\hline
		\end{tabular}
		} \\
		\textbf{Fuente}: Autores
	\end{center}
\end{table}

\begin{table}[!htb]
	\caption{CU059-Actualizar información de integrante de equipo/grupo: Descripción}
	\label{tab:cu059_desc}
	\begin{center}
		\resizebox{15cm}{!}{
		\begin{tabular}{|p{4cm}|p{11cm}|}
			\hline
			\multicolumn{2}{|c|}{Descripción de caso de uso} \\
			\hline
			Nombre & Actualizar información de integrante de equipo/grupo \\
			\hline
			Identificador & CU059 \\
			\hline
			Descripción & Permite la actualización de la información de un integrante del equipo/grupo referente al equipo/grupo mismo \\
			\hline
			Actor & Todo actor de la red social	 
			\\
			\hline
			Disparador & El usuario elige actualizar la información de equipo/grupo de un integrante del equipo/grupo \\
			\hline
			Inclusiones &  \\
			\hline
			Puntos de extensión & 
			\\
			\hline
			Precondiciones &  
			\begin{itemize}
				\item La aplicación ha sido cargada por un actor con rol de formador de grupos deportivos
			\end{itemize}
			\\
			\hline
			Postcondiciones & 
			\begin{itemize}
				\item El usuario cambia la información de equipo de un integrante del equipo/grupo
			\end{itemize}
			\\
			\hline
			Notas & 
			\\
			\hline
		\end{tabular}
		} \\
		\textbf{Fuente}: Autores
	\end{center}
\end{table}

\begin{table}[!htb]
	\caption{CU059-Actualizar información de integrante de equipo/grupo: Flujos de hechos}
	\label{tab:cu059_flujo}
	\begin{center}
		\resizebox{15cm}{!}{
		\begin{tabular}{|p{1.5cm}|p{6cm}|p{6.5cm}|}
			\hline
			\multicolumn{3}{|c|}{Detalle de flujo de hechos de caso de uso} \\
			\hline
			Nombre & \multicolumn{2}{|c|}{Nombre del flujo} \\
			\hline
			Paso & Acción del actor & Respuesta del sistema \\
			\hline
			1 & El usuario ha elegido un equipo/grupo deportivo & El sistema ha mostrado los detalles del equipo/grupo elegido \\
			\hline
			2 & El usuario elige, en las acciones del equipo/grupo, gestionar integrantes & El sistema muestra la lista de integrantes del grupo/equipo \\
			\hline
			3 & El usuario busca un integrante del grupo/equipo & El sistema muestra los resultados de la búsqueda \\
			4 & El usuario elige un integrante entre los resultados de la búsqueda & El sistema muestra los detalles del integrante  \\
			5 & El usuario modifica la información del integrante que competa al grupo/equipo y la salva & El sistema guarda los cambios \\
			6 &  & El sistema muestra un elemento emergente informando del éxito o fracaso de la operación \\
			\hline
			7 & El usuario continua & El sistema muestra la interfaz de gestión de integrantes de equipos/grupos deportivos \\
			\hline
		\end{tabular}
		} \\
		\textbf{Fuente}: Autores
	\end{center}
\end{table}

\begin{table}[!htb]
	\caption{CU060-Dar de baja a integrante de equipo/grupo: Descripción}
	\label{tab:cu060_desc}
	\begin{center}
		\resizebox{15cm}{!}{
		\begin{tabular}{|p{4cm}|p{11cm}|}
			\hline
			\multicolumn{2}{|c|}{Descripción de caso de uso} \\
			\hline
			Nombre & Dar de baja a integrante de equipo/grupo \\
			\hline
			Identificador & CU060 \\
			\hline
			Descripción & Permite eliminar un jugador de un equipo deportivo, así como desvincularse del mismo \\
			\hline
			Actor & Todo actor de la red social
			\\
			\hline
			Disparador & El usuario elige dar de baja a un integrante del equipo/grupo (darse de baja también) \\
			\hline
			Inclusiones &  \\
			\hline
			Puntos de extensión & 
			\\
			\hline
			Precondiciones &  
			\begin{itemize}
				\item La aplicación ha sido cargada por un actor con rol de formador de grupos deportivos o un jugador del equipo/grupo intenta acceder a la funcionalidad
			\end{itemize}
			\\
			\hline
			Postcondiciones & 
			\begin{itemize}
				\item El usuario da de baja a un jugador del equipo/grupo (o se da de baja a si mismo)
			\end{itemize}
			\\
			\hline
			Notas & 
			\\
			\hline
		\end{tabular}
		} \\
		\textbf{Fuente}: Autores
	\end{center}
\end{table}

\begin{table}[!htb]
	\caption{CU060-Dar de baja a integrante de equipo/grupo: Flujos de hechos}
	\label{tab:cu060_flujo}
	\begin{center}
		\resizebox{15cm}{!}{
		\begin{tabular}{|p{1.5cm}|p{6cm}|p{6.5cm}|}
			\hline
			\multicolumn{3}{|c|}{Detalle de flujo de hechos de caso de uso} \\
			\hline
			Nombre & \multicolumn{2}{|c|}{Nombre del flujo} \\
			\hline
			Paso & Acción del actor & Respuesta del sistema \\
			\hline
			1 & El usuario ha elegido un equipo/grupo deportivo & El sistema ha mostrado los detalles del equipo/grupo elegido \\
			\hline
			2 & El usuario elige, en las acciones del equipo/grupo, gestionar integrantes & El sistema muestra la lista de integrantes del grupo/equipo \\
			\hline
			3 & El usuario busca un integrante del grupo/equipo & El sistema muestra los resultados de la búsqueda \\
			4 & El usuario elige un integrante entre los resultados de la búsqueda y pulsa el botón designado para retirarlo & El sistema retira al integrante del grupo/equipo  \\
			5 &  & El sistema muestra un elemento emergente informando del éxito o fracaso de la operación \\
			\hline
			6 & El usuario continua & El sistema muestra la interfaz de gestión de integrantes de equipos/grupos deportivos \\
			\hline
		\end{tabular}
		} \\
		\textbf{Fuente}: Autores
	\end{center}
\end{table}

\begin{table}[!htb]
	\caption{CU061-Crear grupo deportivo: Descripción}
	\label{tab:cu061_desc}
	\begin{center}
		\resizebox{15cm}{!}{
		\begin{tabular}{|p{4cm}|p{11cm}|}
			\hline
			\multicolumn{2}{|c|}{Descripción de caso de uso} \\
			\hline
			Nombre & Crear grupo deportivo \\
			\hline
			Identificador & CU061 \\
			\hline
			Descripción & Permite la creación de un grupo deportivo \\
			\hline
			Actor & Todo actor de la red social
			\\
			\hline
			Disparador & El usuario elige crear un grupo deportivo \\
			\hline
			Inclusiones &  \\
			\hline
			Puntos de extensión & 
			\\
			\hline
			Precondiciones &  
			\begin{itemize}
				\item La aplicación ha sido cargada por un actor con rol de formador de grupos deportivos
			\end{itemize}
			\\
			\hline
			Postcondiciones & 
			\begin{itemize}
				\item El usuario ha creado un grupo deportivo
			\end{itemize}
			\\
			\hline
			Notas & 
			\\
			\hline
		\end{tabular}
		} \\
		\textbf{Fuente}: Autores
	\end{center}
\end{table}

\begin{table}[!htb]
	\caption{CU061-Crear grupo deportivo: Flujos de hechos}
	\label{tab:cu061_flujo}
	\begin{center}
		\resizebox{15cm}{!}{
		\begin{tabular}{|p{1.5cm}|p{6cm}|p{6.5cm}|}
			\hline
			\multicolumn{3}{|c|}{Detalle de flujo de hechos de caso de uso} \\
			\hline
			Nombre & \multicolumn{2}{|c|}{Nombre del flujo} \\
			\hline
			Paso & Acción del actor & Respuesta del sistema \\
			\hline
			1 & El usuario ha elegido la opción de administrar grupos deportivos & El sistema ha mostrado la interfaz con la consulta de los equipos y grupos creados por el usuario \\
			\hline
			2 & El usuario pulsa el botón para la creación de grupos deportivos & El sistema muestra la interfaz de creación con todos los datos a ser incluidos \\
			\hline
			3 & El usuario ingresa los datos del grupo y salva la información & El sistema guarda el grupo en sistema \\
			\hline
			4 & & El sistema muestra un elemento emergente informando del éxito o fracaso de la operación \\
			\hline
			5 & El usuario continua & El sistema muestra la interfaz de gestión de equipos/grupos deportivos \\
			\hline
		\end{tabular}
		} \\
		\textbf{Fuente}: Autores
	\end{center}
\end{table}

\clearpage

\section{Módulo de estadísticas}

%\input{./imagenes/casos_uso/gestion_estadisticas.png}

Las funcionalidades ofrecidas por éste módulo son proporcionadas por el SNS para dar soporte estadístico a cáda uno de los objetos de negocio especificados en el capítulo \ref{chap:arquitectura}, en lo que se refiere al proceso estadístico. Como adición, visto desde los requerimientos funcionales, se da una funcionalidad de llevar el mejor calificado en cáda uno de los objetos de negocio, con tal de que el usuario del SNS pueda ver aquellos objetos de negocio destacados cuando el lo requiera.

\clearpage

\begin{table}[!htb]
	\caption{CU062-Gestionar estadísticas: Descripción}
	\label{tab:cu062_desc}
	\begin{center}
		\resizebox{15cm}{!}{
		\begin{tabular}{|p{4cm}|p{11cm}|}
			\hline
			\multicolumn{2}{|c|}{Descripción de caso de uso} \\
			\hline
			Nombre & Gestionar estadísticas \\
			\hline
			Identificador & CU062 \\
			\hline
			Descripción & Permite la gestión de estadísticas a través de la red social deportiva \\
			\hline
			Actor & Todo actor de la red social	 
			\\
			\hline
			Disparador & El usuario decide calificar un objeto calificable en el SNS (evento, usuario de la red social, servicio prestado por un usuario de la red social, etc.) \\
			\hline
			Inclusiones &  \\
			\hline
			Puntos de extensión & 
			\\
			\hline
			Precondiciones & \\
			\hline
			Postcondiciones & \\
			\hline
			Notas &
			\begin{itemize}
				\item Generalización de
				\begin{itemize}
				 \item Llevar estadística de jugador
				 \item Llevar estadística de equipo
				 \item Calificar ligas deportivas
				 \item Calificar entrenadores
				 \item Calificar organización
				 \item Llevar calificación de "mejor por categoría"
				 \item Llevar estadística de ubicación
				\end{itemize}
			\end{itemize}
			\\
			\hline
		\end{tabular}
		} \\
		\textbf{Fuente}: Autores
	\end{center}
\end{table}

\begin{table}[!htb]
	\caption{CU062-Gestionar estadísticas: Flujos de hechos}
	\label{tab:cu062_flujo}
	\begin{center}
		\resizebox{15cm}{!}{
		\begin{tabular}{|p{1.5cm}|p{6cm}|p{6.5cm}|}
			\hline
			\multicolumn{3}{|c|}{Detalle de flujo de hechos de caso de uso} \\
			\hline
			Nombre & \multicolumn{2}{|c|}{Nombre del flujo} \\
			\hline
			Paso & Acción del actor & Respuesta del sistema \\
			\hline
			1 & El usuario elige o hacer una calificación de un actor/evento en la red social, ingresa información estadística o hace una búsqueda de mejor calificación & El sistema muestra la interfaz correspondiente a cada funcionalidad \\
			\hline
		\end{tabular}
		} \\
		\textbf{Fuente}: Autores
	\end{center}
\end{table}

\begin{table}[!htb]
	\caption{CU063-Llevar estadística de jugador: Descripción}
	\label{tab:cu063_desc}
	\begin{center}
		\resizebox{15cm}{!}{
		\begin{tabular}{|p{4cm}|p{11cm}|}
			\hline
			\multicolumn{2}{|c|}{Descripción de caso de uso} \\
			\hline
			Nombre & Llevar estadística de jugador \\
			\hline
			Identificador & CU063 \\
			\hline
			Descripción & Permite el cálculo de estadísticas de un jugador venidas desde los datos registrados de los equipos en los que se encuentra y de datos verificables en prácticas libres para cada uno de los deportes que éste practica \\
			\hline
			Actor & 
			\begin{itemize}
				\item Jugador
				\item Equipo
				\item Entrenador
			\end{itemize}				 
			\\
			\hline
			Disparador & El usuario ingresa datos estadísticos de un jugador a la red social \\
			\hline
			Inclusiones &  \\
			\hline
			Puntos de extensión & \\
			\hline
			Precondiciones &  
			\begin{itemize}
				\item La aplicación ha sido cargada como entrenador, equipo deportivo o jugador
			\end{itemize}
			\\
			\hline
			Postcondiciones & 
			\begin{itemize}
				\item Se calculan datos estadísticos del jugador
			\end{itemize}			
			\\
			\hline
			Notas & 
			\begin{itemize}
				\item Los datos pueden ser accesados por cualquier usuario de la red social
				\item Generalización de:
				\begin{itemize}
					\item Gestión de niveles de juego
				\end{itemize}
			\end{itemize}
			\\
			\hline
		\end{tabular}
		} \\
		\textbf{Fuente}: Autores
	\end{center}
\end{table}

\begin{table}[!htb]
	\caption{CU063-Llevar estadística de jugador: Flujos de hechos}
	\label{tab:cu063_flujo}
	\begin{center}
		\resizebox{15cm}{!}{
		\begin{tabular}{|p{1.5cm}|p{6cm}|p{6.5cm}|}
			\hline
			\multicolumn{3}{|c|}{Detalle de flujo de hechos de caso de uso} \\
			\hline
			Nombre & \multicolumn{2}{|c|}{Nombre del flujo} \\
			\hline
			Paso & Acción del actor & Respuesta del sistema \\
			\hline
			1 & El usuario ingresa datos que son utilizados por el sistema para calcular estadísticos de un jugador & El sistema calcula los datos estadísticos del jugador \\
			\hline
		\end{tabular}
		} \\
		\textbf{Fuente}: Autores
	\end{center}
\end{table}

\begin{table}[!htb]
	\caption{CU064-Llevar estadística de equipo: Descripción}
	\label{tab:cu064_desc}
	\begin{center}
		\resizebox{15cm}{!}{
		\begin{tabular}{|p{4cm}|p{11cm}|}
			\hline
			\multicolumn{2}{|c|}{Descripción de caso de uso} \\
			\hline
			Nombre & Llevar estadística de equipo \\
			\hline
			Identificador & CU064 \\
			\hline
			Descripción & Permite el cálculo de estadísticas de un equipo venidas desde los datos registrados de éste en los eventos/torneos deportivos en los que ha participado \\
			\hline
			Actor & 
			\begin{itemize}
				\item Equipo
				\item Entrenador
			\end{itemize}				 
			\\
			\hline
			Disparador & El usuario ingresa datos estadísticos de un equipo a la red social \\
			\hline
			Inclusiones &  \\
			\hline
			Puntos de extensión & \\
			\hline
			Precondiciones &  
			\begin{itemize}
				\item La aplicación ha sido cargada como entrenador, equipo deportivo o jugador
			\end{itemize}
			\\
			\hline
			Postcondiciones & 
			\begin{itemize}
				\item Se calculan datos estadísticos del equipo
			\end{itemize}
			\\
			\hline
			Notas &
			\begin{itemize}
				\item Los datos pueden ser accesados por cualquier usuario de la red social
				\item Generalización de:
				\begin{itemize}
					\item Gestión de niveles de juego
				\end{itemize}
			\end{itemize}
			\\
			\hline
		\end{tabular}
		} \\
		\textbf{Fuente}: Autores
	\end{center}
\end{table}

\begin{table}[!htb]
	\caption{CU064-Llevar estadística de equipo: Flujos de hechos}
	\label{tab:cu064_flujo}
	\begin{center}
		\resizebox{15cm}{!}{
		\begin{tabular}{|p{1.5cm}|p{6cm}|p{6.5cm}|}
			\hline
			\multicolumn{3}{|c|}{Detalle de flujo de hechos de caso de uso} \\
			\hline
			Nombre & \multicolumn{2}{|c|}{Nombre del flujo} \\
			\hline
			Paso & Acción del actor & Respuesta del sistema \\
			\hline
			1 & El usuario ingresa datos que son utilizados por el sistema para calcular estadísticos de un equipo & El sistema calcula los datos estadísticos del equipo \\
			\hline
		\end{tabular}
		} \\
		\textbf{Fuente}: Autores
	\end{center}
\end{table}

\begin{table}[!htb]
	\caption{CU065-Gestión de niveles de juego: Descripción}
	\label{tab:cu065_desc}
	\begin{center}
		\resizebox{15cm}{!}{
		\begin{tabular}{|p{4cm}|p{11cm}|}
			\hline
			\multicolumn{2}{|c|}{Descripción de caso de uso} \\
			\hline
			Nombre & Gestión de niveles de juego \\
			\hline
			Identificador & CU065 \\
			\hline
			Descripción & Lleva un análisis del nivel de juego de un jugador/equipo respecto de sus estadísticas y las manejadas en el resto de la red social, así como también las estadísticas de nivel de juego manejadas usualmente en un lugar donde hayan prácticas deportivas \\
			\hline
			Actor & 		 
			\\
			\hline
			Disparador & El usuario ingresa datos estadísticos de un equipo/jugador a la red social o se obtienen datos de un lugar deportivo, el nivel de juego manejado en dicho lugar \\
			\hline
			Inclusiones &  \\
			\hline
			Puntos de extensión & 
			\begin{itemize}
				\item Gestión de niveles de juego
			\end{itemize}
			\\
			\hline
			Precondiciones &  
			\begin{itemize}
				\item La aplicación ha sido cargada como entrenador, equipo deportivo o jugador
				\item El usuario ingresa datos estadísticos del equipo/jugador o la red social obtiene datos de lugares deportivos
			\end{itemize}
			\\
			\hline
			Postcondiciones & 
			\begin{itemize}
				\item Se calculan datos estadísticos del equipo
			\end{itemize}
			\\
			\hline
			Notas & Los datos pueden ser accesados por cualquier usuario de la red social
			\\
			\hline
		\end{tabular}
		} \\
		\textbf{Fuente}: Autores
	\end{center}
\end{table}

\begin{table}[!htb]
	\caption{CU065-Gestión de niveles de juego: Flujos de hechos}
	\label{tab:cu065_flujo}
	\begin{center}
		\resizebox{15cm}{!}{
		\begin{tabular}{|p{1.5cm}|p{6cm}|p{6.5cm}|}
			\hline
			\multicolumn{3}{|c|}{Detalle de flujo de hechos de caso de uso} \\
			\hline
			Nombre & \multicolumn{2}{|c|}{Nombre del flujo} \\
			\hline
			Paso & Acción del actor & Respuesta del sistema \\
			\hline
			1 & El usuario ha ingresado datos a la red social & El sistema ha aprovechado esos datos para calcular el nivel de juego del usuario a quien apuntan los datos ingresados, según los parámetros de la red social \\
			\hline
		\end{tabular}
		} \\
		\textbf{Fuente}: Autores
	\end{center}
\end{table}

\begin{table}[!htb]
	\caption{CU066-Calificar jugadores: Descripción}
	\label{tab:cu066_desc}
	\begin{center}
		\resizebox{15cm}{!}{
		\begin{tabular}{|p{4cm}|p{11cm}|}
			\hline
			\multicolumn{2}{|c|}{Descripción de caso de uso} \\
			\hline
			Nombre & Calificar jugadores \\
			\hline
			Identificador & CU066 \\
			\hline
			Descripción & Permite a cualquier usuario dar una calificación de un jugador con el que ha interactuado \\
			\hline
			Actor & Todo actor de la red social			 
			\\
			\hline
			Disparador & Actor de la red social desea calificar un jugador \\
			\hline
			Inclusiones &  \\
			\hline
			Puntos de extensión & 
			\\
			\hline
			Precondiciones &  
			\begin{itemize}
				\item El actor tiene que tener un vínculo con el jugador
			\end{itemize}
			\\
			\hline
			Postcondiciones & 
			\begin{itemize}
				\item Se ha calificado un jugador
			\end{itemize}
			\\
			\hline
			Notas & Los datos pueden ser accesados por cualquier usuario de la red social
			\\
			\hline
		\end{tabular}
		} \\
		\textbf{Fuente}: Autores
	\end{center}
\end{table}

\begin{table}[!htb]
	\caption{CU066-Calificar jugadores: Flujos de hechos}
	\label{tab:cu066_flujo}
	\begin{center}
		\resizebox{15cm}{!}{
		\begin{tabular}{|p{1.5cm}|p{6cm}|p{6.5cm}|}
			\hline
			\multicolumn{3}{|c|}{Detalle de flujo de hechos de caso de uso} \\
			\hline
			Nombre & \multicolumn{2}{|c|}{Nombre del flujo} \\
			\hline
			Paso & Acción del actor & Respuesta del sistema \\
			\hline
			1 & El usuario entra a la zona de calificación del jugador & El sistema muestra la interfaz de calificación \\
			\hline
			2 & El usuario ingresa toda la información pedida por el sistema para realizar la calificación & \\
			\hline
			3 & El usuario califica al jugador & El sistema ingresa la calificación del usuario \\
			\hline			
			4 &  & El sistema muestra un elemento emergente informando del éxito o fracaso de la operación \\
			\hline
			5 & El usuario continua & El sistema muestra la interfaz anterior a la elección de la zona de calificación \\
			\hline
		\end{tabular}
		} \\
		\textbf{Fuente}: Autores
	\end{center}
\end{table}

\begin{table}[!htb]
	\caption{CU067-Calificar entrenadores: Descripción}
	\label{tab:cu067_desc}
	\begin{center}
		\resizebox{15cm}{!}{
		\begin{tabular}{|p{4cm}|p{11cm}|}
			\hline
			\multicolumn{2}{|c|}{Descripción de caso de uso} \\
			\hline
			Nombre & Calificar entrenadores \\
			\hline
			Identificador & CU067 \\
			\hline
			Descripción & Permite a cualquier usuario dar una calificación del servicio de entrenamiento dado por un entrenador deportivo \\
			\hline
			Actor & 
			\begin{itemize}
				\item Jugador
				\item Equipo
				\item Patrocinador
				\item Organización
			\end{itemize}
			\\
			\hline
			Disparador & El usuario califica los servicios ofrecidos por un entrenador \\
			\hline
			Inclusiones &  \\
			\hline
			Puntos de extensión & 
			\\
			\hline
			Precondiciones &  \\
			\hline
			Postcondiciones & 
			\begin{itemize}
				\item Se ha calificado los servicios de un entrenador
			\end{itemize}
			\\
			\hline
			Notas & Los datos pueden ser accesados por cualquier usuario de la red social
			\\
			\hline
		\end{tabular}
		} \\
		\textbf{Fuente}: Autores
	\end{center}
\end{table}

\begin{table}[!htb]
	\caption{CU067-Calificar entrenadores: Flujos de hechos}
	\label{tab:cu067_flujo}
	\begin{center}
		\resizebox{15cm}{!}{
		\begin{tabular}{|p{1.5cm}|p{6cm}|p{6.5cm}|}
			\hline
			\multicolumn{3}{|c|}{Detalle de flujo de hechos de caso de uso} \\
			\hline
			Nombre & \multicolumn{2}{|c|}{Nombre del flujo} \\
			\hline
			Paso & Acción del actor & Respuesta del sistema \\
			\hline
			1 & El usuario entra a la zona de calificación de los servicios del entrenador & El sistema muestra la interfaz de calificación \\
			\hline
			2 & El usuario ingresa toda la información pedida por el sistema para realizar la calificación & \\
			\hline
			3 & El usuario califica los servicios del entrenador & El sistema ingresa la calificación del los servicios del entrenador \\
			\hline			
			4 &  & El sistema muestra un elemento emergente informando del éxito o fracaso de la operación \\
			\hline
			5 & El usuario continua & El sistema muestra la interfaz anterior a la elección de la zona de calificación \\
			\hline
		\end{tabular}
		} \\
		\textbf{Fuente}: Autores
	\end{center}
\end{table}


\begin{table}[!htb]
	\caption{CU068-Calificar organización: Descripción}
	\label{tab:cu068_desc}
	\begin{center}
		\resizebox{15cm}{!}{
		\begin{tabular}{|p{4cm}|p{11cm}|}
			\hline
			\multicolumn{2}{|c|}{Descripción de caso de uso} \\
			\hline
			Nombre & Calificar organización \\
			\hline
			Identificador & CU068 \\
			\hline
			Descripción & Permite a cualquier usuario dar una calificación de los servicios ofrecidos por una organización deportiva en la red social \\
			\hline
			Actor & 
			\begin{itemize}
				\item Jugador
				\item Equipo
				\item Patrocinador
				\item Organización
			\end{itemize}
			\\
			\hline
			Disparador & El usuario califica los servicios ofrecidos por una organización deportiva \\
			\hline
			Inclusiones &  \\
			\hline
			Puntos de extensión & 
			\\
			\hline
			Precondiciones & 
			\begin{itemize}
				\item El usuario ha tomado los servicios de la organización
			\end{itemize}						
			\\
			\hline
			Postcondiciones & 
			\begin{itemize}
				\item Se ha calificado los servicios de una organización deportiva
			\end{itemize}
			\\
			\hline
			Notas & Los datos pueden ser accesados por cualquier usuario de la red social
			\\
			\hline
		\end{tabular}
		} \\
		\textbf{Fuente}: Autores
	\end{center}
\end{table}

\begin{table}[!htb]
	\caption{CU068-Calificar organización: Flujos de hechos}
	\label{tab:cu068_flujo}
	\begin{center}
		\resizebox{15cm}{!}{
		\begin{tabular}{|p{1.5cm}|p{6cm}|p{6.5cm}|}
			\hline
			\multicolumn{3}{|c|}{Detalle de flujo de hechos de caso de uso} \\
			\hline
			Nombre & \multicolumn{2}{|c|}{Nombre del flujo} \\
			\hline
			Paso & Acción del actor & Respuesta del sistema \\
			\hline
			1 & El usuario entra a la zona de calificación de los servicios ofrecidos por la organización & El sistema muestra la interfaz de calificación \\
			\hline
			2 & El usuario ingresa toda la información pedida por el sistema para realizar la calificación & \\
			\hline
			3 & El usuario califica los servicios ofrecidos por la organización & El sistema ingresa la calificación de los servicios ofrecidos por la organización \\
			\hline			
			4 &  & El sistema muestra un elemento emergente informando del éxito o fracaso de la operación \\
			\hline
			5 & El usuario continua & El sistema muestra la interfaz anterior a la elección de la zona de calificación \\
			\hline
		\end{tabular}
		} \\
		\textbf{Fuente}: Autores
	\end{center}
\end{table}

\clearpage

\begin{table}[!htb]
	\caption{CU069-Llevar estadística de ubicación: Descripción}
	\label{tab:cu069_desc}
	\begin{center}
		\resizebox{15cm}{!}{
		\begin{tabular}{|p{4cm}|p{11cm}|}
			\hline
			\multicolumn{2}{|c|}{Descripción de caso de uso} \\
			\hline
			Nombre & Llevar estadística de ubicación \\
			\hline
			Identificador & CU069 \\
			\hline
			Descripción & Permite llevar estadísticas deportivas de las ubicaciones registradas en el SNS respecto a la afluencia de gente a practicar cierto deporte, así como también del nivel de juego en dicha ubicación \\
			\hline
			Actor &
			\begin{itemize}
				\item Jugador
				\item Equipo
				\item Patrocinador
				\item Organización
			\end{itemize}
			\\
			\hline
			Disparador & Un usuario entra en la zona de la ubicación deportiva con su dispositivo móvil encendido y con el GPS en funcionamiento \\
			\hline
			Inclusiones &  \\
			\hline
			Puntos de extensión & 
			\\
			\hline
			Precondiciones &  
			\begin{itemize}
				\item El usuario tiene encendido el GPS del dispositivo móvil
				\item El usuario está ingresando a una zona deportiva registrada y se encuentra por más de 30 minutos en ella
			\end{itemize}
			\\
			\hline
			Postcondiciones & 
			\begin{itemize}
				\item La red social lleva la estadística de la ubicación
			\end{itemize}
			\\
			\hline
			Notas & 
			\\
			\hline
		\end{tabular}
		} \\
		\textbf{Fuente}: Autores
	\end{center}
\end{table}

\begin{table}[!htb]
	\caption{CU069-Llevar estadística de ubicación: Flujos de hechos}
	\label{tab:cu069_flujo}
	\begin{center}
		\resizebox{15cm}{!}{
		\begin{tabular}{|p{1.5cm}|p{6cm}|p{6.5cm}|}
			\hline
			\multicolumn{3}{|c|}{Detalle de flujo de hechos de caso de uso} \\
			\hline
			Nombre & \multicolumn{2}{|c|}{Nombre del flujo} \\
			\hline
			Paso & Acción del actor & Respuesta del sistema \\
			\hline
			1 & El usuario lleva más de 30 minutos en la misma zona registrada como deportiva & El sistema guarda la estadística de la zona deportiva \\
			\hline
		\end{tabular}
		} \\
		\textbf{Fuente}: Autores
	\end{center}
\end{table}

\begin{table}[!htb]
	\caption{CU070-Ubicar mejor por categoría: Descripción}
	\label{tab:cu070_desc}
	\begin{center}
		\resizebox{15cm}{!}{
		\begin{tabular}{|p{4cm}|p{11cm}|}
			\hline
			\multicolumn{2}{|c|}{Descripción de caso de uso} \\
			\hline
			Nombre & Ubicar mejor por categoría \\
			\hline
			Identificador & CU070 \\
			\hline
			Descripción & Permite mostrar al usuario los mejores usuarios en la busqueda que él desee realizar en las primeras posiciones de ésta \\
			\hline
			Actor &
			\begin{itemize}
				\item Jugador
				\item Equipo
				\item Patrocinador
				\item Organización
			\end{itemize}
			\\
			\hline
			Disparador & El usuario hace una búsqueda \\
			\hline
			Inclusiones &  \\
			\hline
			Puntos de extensión & 
			\\
			\hline
			Precondiciones & \\
			\hline
			Postcondiciones & \\
			\hline
			Notas &	\\
			\hline
		\end{tabular}
		} \\
		\textbf{Fuente}: Autores
	\end{center}
\end{table}

\begin{table}[!htb]
	\caption{CU070-Ubicar mejor por categoría: Flujos de hechos}
	\label{tab:cu070_flujo}
	\begin{center}
		\resizebox{15cm}{!}{
		\begin{tabular}{|p{1.5cm}|p{6cm}|p{6.5cm}|}
			\hline
			\multicolumn{3}{|c|}{Detalle de flujo de hechos de caso de uso} \\
			\hline
			Nombre & \multicolumn{2}{|c|}{Nombre del flujo} \\
			\hline
			Paso & Acción del actor & Respuesta del sistema \\
			\hline
			1 & El usuario hace una búsqueda de un usuario sobre la red social & El sistema reconoce los mejores usuarios por búsqueda y los posiciona en los primeros lugares \\
			\hline
		\end{tabular}
		} \\
		\textbf{Fuente}: Autores
	\end{center}
\end{table}

\clearpage

\section{Módulo de gestión de patrocinios}

%\input{./imagenes/casos_uso/gestion_patrocinios.png}

Este módulo ofrece funcionalidades tanto para patrocinador como patrocinado, indicando aquí que sólo el actor correspondiente (siendo patrocinador o patrocinado) puede acceder a uno de las dos grandes funcionalidades que se pueden apreciar: Por parte del patrocinador, se prestan funcionalidades de seguimiento de usuarios candidatos a patrocinar, así como la petición de patrocinio y consulta de patrocinios; Por parte del jugador, se ofrecen funcionalidades similares pero sin poder hacer seguimiento.

\clearpage

\begin{table}[!htb]
	\caption{CU071-Administrar patrocinios: Descripción}
	\label{tab:cu071_desc}
	\begin{center}
		\resizebox{15cm}{!}{
		\begin{tabular}{|p{4cm}|p{11cm}|}
			\hline
			\multicolumn{2}{|c|}{Descripción de caso de uso} \\
			\hline
			Nombre & Administrar patrocinios \\
			\hline
			Identificador & CU071 \\
			\hline
			Descripción & Permite gestionar o administrar funcionalidades ofrecidas en lo que refiere a patrocinios en la red social deportiva \\
			\hline
			Actor &
			\begin{itemize}
				\item Patrocinador
				\item Equipo
				\item Jugador
			\end{itemize}
			\\
			\hline
			Disparador & El usuario autorizado desea manejar las funcionalidades ofrecidas en el módulo de patrocinios \\
			\hline
			Inclusiones &  \\
			\hline
			Puntos de extensión & 
			\\
			\hline
			Precondiciones &  
			\begin{itemize}
				\item Se ha iniciado el SNS con un rol de patrocinador, jugador o equipo deportivo
			\end{itemize}
			\\
			\hline
			Postcondiciones & 
			\begin{itemize}
				\item El usuario se encuentra en la pantalla para gestión de patrocinios
			\end{itemize}
			\\
			\hline
			Notas & 
			\begin{itemize}
				\item Generalización de:
				\begin{itemize}
					\item Gestionar patrocinios de patrocinador
					\item Gestionar patrocinios de equipo/jugador
				\end{itemize}
			\end{itemize}
			\\
			\hline
		\end{tabular}
		} \\
		\textbf{Fuente}: Autores
	\end{center}
\end{table}

\begin{table}[!htb]
	\caption{CU071-Administrar patrocinios: Flujos de hechos}
	\label{tab:cu071_flujo}
	\begin{center}
		\resizebox{15cm}{!}{
		\begin{tabular}{|p{1.5cm}|p{6cm}|p{6.5cm}|}
			\hline
			\multicolumn{3}{|c|}{Detalle de flujo de hechos de caso de uso} \\
			\hline
			Nombre & \multicolumn{2}{|c|}{Nombre del flujo} \\
			\hline
			Paso & Acción del actor & Respuesta del sistema \\
			\hline
			1 & El usuario elige administrar patrocinios deportivos & El sistema muestra la interfaz de patrocinio según el rol cargado: Si es un rol de patrocinador, se cargarán las funcionalidades de patrocinadores; si es un rol no-patrocinador, entonces se cargarán las funcionalidades restantes \\
			\hline
		\end{tabular}
		} \\
		\textbf{Fuente}: Autores
	\end{center}
\end{table}

\begin{table}[!htb]
	\caption{CU072-Gestionar patrocinios de patrocinador: Descripción}
	\label{tab:cu072_desc}
	\begin{center}
		\resizebox{15cm}{!}{
		\begin{tabular}{|p{4cm}|p{11cm}|}
			\hline
			\multicolumn{2}{|c|}{Descripción de caso de uso} \\
			\hline
			Nombre & Gestionar patrocinios de patrocinador \\
			\hline
			Identificador & CU072 \\
			\hline
			Descripción & Permite gestionar o administrar funcionalidades ofrecidas para patrocinadores en la red social deportiva \\
			\hline
			Actor &
			\begin{itemize}
				\item Patrocinador
				\item Equipo
				\item Jugador
			\end{itemize}
			\\
			\hline
			Disparador & El patrocinador quiere entrar al módulo de gestión de funcionalidades ofrecida para éste en la red social \\
			\hline
			Inclusiones & N/A \\
			\hline
			Puntos de extensión & N/A
			\\
			\hline
			Precondiciones &  
			\begin{itemize}
				\item Se ha iniciado el SNS con un rol de patrocinador deportivo
			\end{itemize}
			\\
			\hline
			Postcondiciones & 
			\begin{itemize}
				\item El usuario se encuentra en la pantalla para gestión de patrocinios de patrocinador
			\end{itemize}
			\\
			\hline
			Notas & 
			\begin{itemize}
				\item Generalización de:
				\begin{itemize}
					\item Aceptar ser patrocinador
					\item Solicitar ser sponsor
					\item Dejar de ser sponsor
					\item Consultar historial de patrocinios
					\item Gestionar tracking
				\end{itemize}
			\end{itemize}
			\\
			\hline
		\end{tabular}
		} \\
		\textbf{Fuente}: Autores
	\end{center}
\end{table}

\begin{table}[!htb]
	\caption{CU072-Gestionar patrocinios de patrocinador: Flujos de hechos}
	\label{tab:cu072_flujo}
	\begin{center}
		\resizebox{15cm}{!}{
		\begin{tabular}{|p{1.5cm}|p{6cm}|p{6.5cm}|}
			\hline
			\multicolumn{3}{|c|}{Detalle de flujo de hechos de caso de uso} \\
			\hline
			Nombre & \multicolumn{2}{|c|}{Nombre del flujo} \\
			\hline
			Paso & Acción del actor & Respuesta del sistema \\
			\hline
			1 & El usuario ha elegido gestionar patrocinios & El sistema ha mostrado la interfaz de patrocinios correspondiente a un patrocinador con un listado de patrocinados y funcionalidades de eliminación del vinculo de patrocinio y de consultar el detalle del patrocinado \\
			\hline
		\end{tabular}
		} \\
		\textbf{Fuente}: Autores
	\end{center}
\end{table}

\input{./Capitulos/Analisis_funcional/Gestion_patrocinios/CU073-Gestionar elementos de patrocinio.tex}

\begin{table}[!htb]
	\caption{CU074-Gestionar material de patrocinio: Descripción}
	\label{tab:cu074_desc}
	\begin{center}
		\resizebox{15cm}{!}{
		\begin{tabular}{|p{4cm}|p{11cm}|}
			\hline
			\multicolumn{2}{|c|}{Descripción de caso de uso} \\
			\hline
			Nombre & Gestionar material de patrocinio \\
			\hline
			Identificador & CU074 \\
			\hline
			Descripción & Permite a un patrocinador llevar una lista de los recursos materiales con los que cuenta y con los que patrocinará a los posibles patrocinados \\
			\hline
			Actor &
			\begin{itemize}
				\item Patrocinador
			\end{itemize}
			\\
			\hline
			Disparador & El patrocinador elige la opción de gestión de materiales de patrocinio \\
			\hline
			Inclusiones &  \\
			\hline
			Puntos de extensión & 
			\\
			\hline
			Precondiciones &  
			\begin{itemize}
				\item Se ha iniciado el SNS con un rol de patrocinador deportivo
			\end{itemize}
			\\
			\hline
			Postcondiciones & 
			\begin{itemize}
				\item El usuario ha hecho cambios en los materiales de patrocinio que maneja
			\end{itemize}
			\\
			\hline
			Notas & 
			\\
			\hline
		\end{tabular}
		} \\
		\textbf{Fuente}: Autores
	\end{center}
\end{table}

\begin{table}[!htb]
	\caption{CU074-Gestionar material de patrocinio: Flujos de hechos}
	\label{tab:cu074_flujo}
	\begin{center}
		\resizebox{15cm}{!}{
		\begin{tabular}{|p{1.5cm}|p{6cm}|p{6.5cm}|}
			\hline
			\multicolumn{3}{|c|}{Detalle de flujo de hechos de caso de uso} \\
			\hline
			Nombre & \multicolumn{2}{|c|}{Nombre del flujo} \\
			\hline
			Paso & Acción del actor & Respuesta del sistema \\
			\hline
			1 & El usuario ha elegido gestionar patrocinios de patrocinado & El sistema ha mostrado la interfaz de modificación de los parámetros del patrocinio \\
			\hline
			2 & El usuario elige gestionar materiales deportivos adicionando o borrando materiales deportivos con los que patrocina y guarda acciones & El sistema realiza las acciones pedidas \\
			\hline
			3 & & El sistema muestra un elemento emergente informando del éxito o fracaso de la operación \\
			\hline
			4 & El usuario continua & El sistema muestra la interfaz de gestión de detalles del patrocinio \\
			\hline
		\end{tabular}
		} \\
		\textbf{Fuente}: Autores
	\end{center}
\end{table}

\begin{table}[!htb]
	\caption{CU075-Gestionar servicios de patrocinio: Descripción}
	\label{tab:cu075_desc}
	\begin{center}
		\resizebox{15cm}{!}{
		\begin{tabular}{|p{4cm}|p{11cm}|}
			\hline
			\multicolumn{2}{|c|}{Descripción de caso de uso} \\
			\hline
			Nombre & Gestionar servicios de patrocinio \\
			\hline
			Identificador & CU075 \\
			\hline
			Descripción & Permite a un patrocinador llevar una lista de los servicios con los que cuenta y con los que patrocinará a los posibles patrocinados \\
			\hline
			Actor &
			\begin{itemize}
				\item Patrocinador
			\end{itemize}
			\\
			\hline
			Disparador & El patrocinador elige la opción de gestión de servicios de patrocinio \\
			\hline
			Inclusiones & N/A \\
			\hline
			Puntos de extensión & N/A
			\\
			\hline
			Precondiciones &  
			\begin{itemize}
				\item Se ha iniciado el SNS con un rol de patrocinador deportivo
			\end{itemize}
			\\
			\hline
			Postcondiciones & 
			\begin{itemize}
				\item El usuario ha hecho cambios en los servicios de patrocinio que maneja
			\end{itemize}
			\\
			\hline
			Notas & N/A
			\\
			\hline
		\end{tabular}
		} \\
		\textbf{Fuente}: Autores
	\end{center}
\end{table}

\begin{table}[!htb]
	\caption{CU075-Gestionar servicios de patrocinio: Flujos de hechos}
	\label{tab:cu075_flujo}
	\begin{center}
		\resizebox{15cm}{!}{
		\begin{tabular}{|p{1.5cm}|p{6cm}|p{6.5cm}|}
			\hline
			\multicolumn{3}{|c|}{Detalle de flujo de hechos de caso de uso} \\
			\hline
			Nombre & \multicolumn{2}{|c|}{Nombre del flujo} \\
			\hline
			Paso & Acción del actor & Respuesta del sistema \\
			\hline
			1 & El usuario ha elegido gestionar patrocinios de patrocinado & El sistema ha mostrado la interfaz de modificación de los parámetros del patrocinio \\
			\hline
			2 & El usuario elige gestionar servicios deportivos adicionando o borrando servicios deportivos con los que patrocina y guarda acciones & El sistema realiza las acciones pedidas \\
			\hline
			3 & & El sistema muestra un elemento emergente informando del éxito o fracaso de la operación \\
			\hline
			4 & El usuario continua & El sistema muestra la interfaz de gestión de detalles del patrocinio \\
			\hline
		\end{tabular}
		} \\
		\textbf{Fuente}: Autores
	\end{center}
\end{table}

\begin{table}[!htb]
	\caption{CU076-Gestionar capacidad de financiamiento: Descripción}
	\label{tab:cu076_desc}
	\begin{center}
		\resizebox{15cm}{!}{
		\begin{tabular}{|p{4cm}|p{11cm}|}
			\hline
			\multicolumn{2}{|c|}{Descripción de caso de uso} \\
			\hline
			Nombre & Gestionar capacidad de financiamiento \\
			\hline
			Identificador & CU076 \\
			\hline
			Descripción & Permite a un patrocinador llevar la capacidad de financiamiento con la que cuenta y con la que patrocinará a los posibles patrocinados \\
			\hline
			Actor &
			\begin{itemize}
				\item Patrocinador
			\end{itemize}
			\\
			\hline
			Disparador & El patrocinador elige la opción de gestión de capacidad de financiamiento \\
			\hline
			Inclusiones &  \\
			\hline
			Puntos de extensión & 
			\\
			\hline
			Precondiciones &  
			\begin{itemize}
				\item Se ha iniciado el SNS con un rol de patrocinador deportivo
			\end{itemize}
			\\
			\hline
			Postcondiciones & 
			\begin{itemize}
				\item El usuario ha hecho cambios en la capacidad de financiamiento que maneja
			\end{itemize}
			\\
			\hline
			Notas & 
			\\
			\hline
		\end{tabular}
		} \\
		\textbf{Fuente}: Autores
	\end{center}
\end{table}

\begin{table}[!htb]
	\caption{CU076-Gestionar capacidad de financiamiento: Flujos de hechos}
	\label{tab:cu076_flujo}
	\begin{center}
		\resizebox{15cm}{!}{
		\begin{tabular}{|p{1.5cm}|p{6cm}|p{6.5cm}|}
			\hline
			\multicolumn{3}{|c|}{Detalle de flujo de hechos de caso de uso} \\
			\hline
			Nombre & \multicolumn{2}{|c|}{Nombre del flujo} \\
			\hline
			Paso & Acción del actor & Respuesta del sistema \\
			\hline
			1 & El usuario ha elegido gestionar patrocinios de patrocinado & El sistema ha mostrado la interfaz de modificación de los parámetros del patrocinio \\
			\hline
			2 & El usuario elige gestionar capacidad de financiamiento adicionando o borrando capacidad de financiamiento con la que patrocina y guarda acciones & El sistema realiza las acciones pedidas \\
			\hline
			3 & & El sistema muestra un elemento emergente informando del éxito o fracaso de la operación \\
			\hline
			4 & El usuario continua & El sistema muestra la interfaz de gestión de detalles del patrocinio \\
			\hline
		\end{tabular}
		} \\
		\textbf{Fuente}: Autores
	\end{center}
\end{table}

\begin{table}[!htb]
	\caption{CU077-Aceptar ser patrocinador: Descripción}
	\label{tab:cu077_desc}
	\begin{center}
		\resizebox{15cm}{!}{
		\begin{tabular}{|p{4cm}|p{11cm}|}
			\hline
			\multicolumn{2}{|c|}{Descripción de caso de uso} \\
			\hline
			Nombre & Aceptar ser patrocinador \\
			\hline
			Identificador & CU077 \\
			\hline
			Descripción & Permite a un patrocinador aceptar una solicitud de convertirse en patrocinador de un actor/evento deportivo \\
			\hline
			Actor &
			\begin{itemize}
				\item Patrocinador
			\end{itemize}
			\\
			\hline
			Disparador & El patrocinador elige la opción aceptar ser patrocinador \\
			\hline
			Inclusiones & N/A \\
			\hline
			Puntos de extensión & N/A
			\\
			\hline
			Precondiciones &  
			\begin{itemize}
				\item Se ha iniciado el SNS con un rol de patrocinador deportivo
				\item Se ha elegido gestionar patrocinios de patrocinador
			\end{itemize}
			\\
			\hline
			Postcondiciones & 
			\begin{itemize}
				\item El usuario ha aceptado ser patrocinador de un actor/evento deportivo
				\item El usuario se encuentra en la pantalla para gestión de patrocinios
			\end{itemize}
			\\
			\hline
			Notas & N/A
			\\
			\hline
		\end{tabular}
		} \\
		\textbf{Fuente}: Autores
	\end{center}
\end{table}

\begin{table}[!htb]
	\caption{CU077-Aceptar ser patrocinador: Flujos de hechos}
	\label{tab:cu077_flujo}
	\begin{center}
		\resizebox{15cm}{!}{
		\begin{tabular}{|p{1.5cm}|p{6cm}|p{6.5cm}|}
			\hline
			\multicolumn{3}{|c|}{Detalle de flujo de hechos de caso de uso} \\
			\hline
			Nombre & \multicolumn{2}{|c|}{Nombre del flujo} \\
			\hline
			Paso & Acción del actor & Respuesta del sistema \\
			\hline
			1 & El usuario elige gestionar solicitudes de patrocinio & El sistema muestra una lista de solicitudes de patrocinados potenciales que han pedido patrocinio \\
			\hline
			2 & El usuario busca una solicitud en específico & El sistema muestra los resultados de la búsqueda \\
			\hline
			3 & El usuario elige una solicitud en específico & El sistema muestra una ventana de confirmación \\
			\hline
			4 & El usuario confirma & El sistema acepta la solicitud \\
			\hline
			5 & El usuario continua & El sistema muestra la interfaz de gestión de solicitudes de patrocinio \\
			\hline
		\end{tabular}
		} \\
		\textbf{Fuente}: Autores
	\end{center}
\end{table}



\begin{table}[!htb]
	\caption{CU041-Solicitar ser patrocinador: Descripción}
	\label{tab:cu041_desc}
	\begin{center}
		\resizebox{15cm}{!}{
		\begin{tabular}{|p{4cm}|p{11cm}|}
			\hline
			\multicolumn{2}{|c|}{Descripción de caso de uso} \\
			\hline
			Nombre & Solicitar ser patrocinador \\
			\hline
			Identificador & CU041 \\
			\hline
			Descripción & Permite a un patrocinador solicitar serlo de un actor/evento deportivo \\
			\hline
			Actor &
			\begin{itemize}
				\item Patrocinador
			\end{itemize}
			\\
			\hline
			Disparador & El patrocinador elige la opción solicitar ser patrocinador \\
			\hline
			Inclusiones & N/A \\
			\hline
			Puntos de extensión & N/A
			\\
			\hline
			Precondiciones &  
			\begin{itemize}
				\item Se ha iniciado el SNS con un rol de patrocinador deportivo
				\item Se ha elegido gestionar patrocinios de patrocinador
			\end{itemize}
			\\
			\hline
			Postcondiciones & 
			\begin{itemize}
				\item El usuario ha solicitado de ser patrocinador de un actor/evento deportivo
				\item El usuario se encuentra en la pantalla para gestión de patrocinios
			\end{itemize}
			\\
			\hline
			Notas & N/A
			\\
			\hline
		\end{tabular}
		} \\
		\textbf{Fuente}: Autores
	\end{center}
\end{table}

\begin{table}[!htb]
	\caption{CU041-Solicitar ser patrocinador: Flujos de hechos}
	\label{tab:cu041_flujo}
	\begin{center}
		\resizebox{15cm}{!}{
		\begin{tabular}{|p{1.5cm}|p{6cm}|p{6.5cm}|}
			\hline
			\multicolumn{3}{|c|}{Detalle de flujo de hechos de caso de uso} \\
			\hline
			Nombre & \multicolumn{2}{|c|}{Nombre del flujo} \\
			\hline
			Paso & Acción del actor & Respuesta del sistema \\
			\hline
			 & & \\
			\hline
		\end{tabular}
		} \\
		\textbf{Fuente}: Autores
	\end{center}
\end{table}

\begin{table}[!htb]
	\caption{CU041-Dejar de ser patrocinador: Descripción}
	\label{tab:cu041_desc}
	\begin{center}
		\resizebox{15cm}{!}{
		\begin{tabular}{|p{4cm}|p{11cm}|}
			\hline
			\multicolumn{2}{|c|}{Descripción de caso de uso} \\
			\hline
			Nombre & Dejar de ser patrocinador \\
			\hline
			Identificador & CU041 \\
			\hline
			Descripción & Permite a un patrocinador dejar de serlo de un actor/evento deportivo \\
			\hline
			Actor &
			\begin{itemize}
				\item Patrocinador
			\end{itemize}
			\\
			\hline
			Disparador & El patrocinador elige la opción "dejar de ser patrocinador" \\
			\hline
			Inclusiones & N/A \\
			\hline
			Puntos de extensión & N/A
			\\
			\hline
			Precondiciones &  
			\begin{itemize}
				\item Se ha iniciado el SNS con un rol de patrocinador deportivo
				\item Se ha elegido gestionar patrocinios de patrocinador
			\end{itemize}
			\\
			\hline
			Postcondiciones & 
			\begin{itemize}
				\item El usuario ha dejado de ser patrocinador de un actor/evento deportivo
				\item El usuario se encuentra en la pantalla para gestión de patrocinios
			\end{itemize}
			\\
			\hline
			Notas & N/A
			\\
			\hline
		\end{tabular}
		} \\
		\textbf{Fuente}: Autores
	\end{center}
\end{table}

\clearpage

\begin{table}[!htb]
	\caption{CU041-Dejar de ser patrocinador: Flujos de hechos}
	\label{tab:cu041_flujo}
	\begin{center}
		\resizebox{15cm}{!}{
		\begin{tabular}{|p{1.5cm}|p{6cm}|p{6.5cm}|}
			\hline
			\multicolumn{3}{|c|}{Detalle de flujo de hechos de caso de uso} \\
			\hline
			Nombre & \multicolumn{2}{|c|}{Nombre del flujo} \\
			\hline
			Paso & Acción del actor & Respuesta del sistema \\
			\hline
			 & & \\
			\hline
		\end{tabular}
		} \\
		\textbf{Fuente}: Autores
	\end{center}
\end{table}

\begin{table}[!htb]
	\caption{CU041-Consultar historial de patrocinios: Descripción}
	\label{tab:cu041_desc}
	\begin{center}
		\resizebox{15cm}{!}{
		\begin{tabular}{|p{4cm}|p{11cm}|}
			\hline
			\multicolumn{2}{|c|}{Descripción de caso de uso} \\
			\hline
			Nombre & Consultar historial de patrocinios \\
			\hline
			Identificador & CU041 \\
			\hline
			Descripción & Permite a un patrocinador consultar el historial de sus patrocinios \\
			\hline
			Actor &
			\begin{itemize}
				\item Patrocinador
			\end{itemize}
			\\
			\hline
			Disparador & El patrocinador elige la opción de consulta de historial de patrocinios \\
			\hline
			Inclusiones & N/A \\
			\hline
			Puntos de extensión & N/A
			\\
			\hline
			Precondiciones &  
			\begin{itemize}
				\item Se ha iniciado el SNS con un rol de patrocinador deportivo
				\item Se ha elegido gestionar patrocinios de patrocinador
			\end{itemize}
			\\
			\hline
			Postcondiciones & 
			\begin{itemize}
				\item El usuario ha consultado el historial de patrocinio
				\item El usuario se encuentra en la pantalla para gestión de patrocinios
			\end{itemize}
			\\
			\hline
			Notas & N/A
			\\
			\hline
		\end{tabular}
		} \\
		\textbf{Fuente}: Autores
	\end{center}
\end{table}

\begin{table}[!htb]
	\caption{CU041-Consultar historial de patrocinios: Flujos de hechos}
	\label{tab:cu041_flujo}
	\begin{center}
		\resizebox{15cm}{!}{
		\begin{tabular}{|p{1.5cm}|p{6cm}|p{6.5cm}|}
			\hline
			\multicolumn{3}{|c|}{Detalle de flujo de hechos de caso de uso} \\
			\hline
			Nombre & \multicolumn{2}{|c|}{Nombre del flujo} \\
			\hline
			Paso & Acción del actor & Respuesta del sistema \\
			\hline
			 & & \\
			\hline
		\end{tabular}
		} \\
		\textbf{Fuente}: Autores
	\end{center}
\end{table}

\begin{table}[!htb]
	\caption{CU041-Actualizar detalles de patrocinio: Descripción}
	\label{tab:cu041_desc}
	\begin{center}
		\resizebox{15cm}{!}{
		\begin{tabular}{|p{4cm}|p{11cm}|}
			\hline
			\multicolumn{2}{|c|}{Descripción de caso de uso} \\
			\hline
			Nombre & Actualizar detalles de patrocinio \\
			\hline
			Identificador & CU041 \\
			\hline
			Descripción & Actualiza los detalles del patrocinio producido entre el actor deportivo y el patrocinador (el apoyo material, en servicios o financiero realizado por el patrocinador deportivo) \\
			\hline
			Actor &
			\begin{itemize}
				\item Patrocinador
			\end{itemize}
			\\
			\hline
			Disparador & El patrocinador elige la opción de actualización de detalles de patrocinio \\
			\hline
			Inclusiones & N/A \\
			\hline
			Puntos de extensión & N/A
			\\
			\hline
			Precondiciones &  
			\begin{itemize}
				\item Se ha iniciado el SNS con un rol de patrocinador deportivo
			\end{itemize}
			\\
			\hline
			Postcondiciones & 
			\begin{itemize}
				\item El usuario ha actualizado los detalles del patrocinio deportivo
			\end{itemize}
			\\
			\hline
			Notas & N/A
			\\
			\hline
		\end{tabular}
		} \\
		\textbf{Fuente}: Autores
	\end{center}
\end{table}

\begin{table}[!htb]
	\caption{CU041-Actualizar detalles de patrocinio: Flujos de hechos}
	\label{tab:cu041_flujo}
	\begin{center}
		\resizebox{15cm}{!}{
		\begin{tabular}{|p{1.5cm}|p{6cm}|p{6.5cm}|}
			\hline
			\multicolumn{3}{|c|}{Detalle de flujo de hechos de caso de uso} \\
			\hline
			Nombre & \multicolumn{2}{|c|}{Nombre del flujo} \\
			\hline
			Paso & Acción del actor & Respuesta del sistema \\
			\hline
			 & & \\
			\hline
		\end{tabular}
		} \\
		\textbf{Fuente}: Autores
	\end{center}
\end{table}

\begin{table}[!htb]
	\caption{CU043-Gestionar tracking: Descripción}
	\label{tab:cu043_desc}
	\begin{center}
		\resizebox{15cm}{!}{
		\begin{tabular}{|p{4cm}|p{11cm}|}
			\hline
			\multicolumn{2}{|c|}{Descripción de caso de uso} \\
			\hline
			Nombre & Gestionar tracking \\
			\hline
			Identificador & CU043 \\
			\hline
			Descripción & Permite recibir noticias de jugadores/equipos/eventos deportivos de los que se quiere ser sponsor, siempre que éste actor/evento deportivo tenga habilitada dicha opción \\
			\hline
			Actor &
			\begin{itemize}
				\item Patrocinador
			\end{itemize}
			\\
			\hline
			Disparador & El patrocinador elige gestionar tracking \\
			\hline
			Inclusiones & N/A \\
			\hline
			Puntos de extensión & N/A
			\\
			\hline
			Precondiciones &  
			\begin{itemize}
				\item Se ha iniciado el SNS con un rol de patrocinador deportivo
				\item Se ha elegido gestionar patrocinios de patrocinador
			\end{itemize}
			\\
			\hline
			Postcondiciones & 
			\begin{itemize}
				\item El usuario se encuentra en la pantalla para gestión de tracking
			\end{itemize}
			\\
			\hline
			Notas & N/A
			\\
			\hline
		\end{tabular}
		} \\
		\textbf{Fuente}: Autores
	\end{center}
\end{table}

\begin{table}[!htb]
	\caption{CU042-Gestionar tracking: Flujos de hechos}
	\label{tab:cu042_flujo}
	\begin{center}
		\resizebox{15cm}{!}{
		\begin{tabular}{|p{1.5cm}|p{6cm}|p{6.5cm}|}
			\hline
			\multicolumn{3}{|c|}{Detalle de flujo de hechos de caso de uso} \\
			\hline
			Nombre & \multicolumn{2}{|c|}{Nombre del flujo} \\
			\hline
			Paso & Acción del actor & Respuesta del sistema \\
			\hline
			 & & \\
			\hline
		\end{tabular}
		} \\
		\textbf{Fuente}: Autores
	\end{center}
\end{table}

\begin{table}[!htb]
	\caption{CU043-Seguir jugador: Descripción}
	\label{tab:cu043_desc}
	\begin{center}
		\resizebox{15cm}{!}{
		\begin{tabular}{|p{4cm}|p{11cm}|}
			\hline
			\multicolumn{2}{|c|}{Descripción de caso de uso} \\
			\hline
			Nombre & Seguir jugador \\
			\hline
			Identificador & CU043 \\
			\hline
			Descripción & Permite seguir la actividad de un jugador \\
			\hline
			Actor &
			\begin{itemize}
				\item Patrocinador
			\end{itemize}
			\\
			\hline
			Disparador & El patrocinador elige gestionar tracking \\
			\hline
			Inclusiones & N/A \\
			\hline
			Puntos de extensión & N/A
			\\
			\hline
			Precondiciones &  
			\begin{itemize}
				\item Se ha iniciado el SNS con un rol de patrocinador deportivo
				\item Se ha elegido gestionar patrocinios de patrocinador
			\end{itemize}
			\\
			\hline
			Postcondiciones & 
			\begin{itemize}
				\item El usuario sigue la actividad de un jugador
			\end{itemize}
			\\
			\hline
			Notas & N/A
			\\
			\hline
		\end{tabular}
		} \\
		\textbf{Fuente}: Autores
	\end{center}
\end{table}

\begin{table}[!htb]
	\caption{CU042-Seguir jugador: Flujos de hechos}
	\label{tab:cu042_flujo}
	\begin{center}
		\resizebox{15cm}{!}{
		\begin{tabular}{|p{1.5cm}|p{6cm}|p{6.5cm}|}
			\hline
			\multicolumn{3}{|c|}{Detalle de flujo de hechos de caso de uso} \\
			\hline
			Nombre & \multicolumn{2}{|c|}{Nombre del flujo} \\
			\hline
			Paso & Acción del actor & Respuesta del sistema \\
			\hline
			 & & \\
			\hline
		\end{tabular}
		} \\
		\textbf{Fuente}: Autores
	\end{center}
\end{table}

\begin{table}[!htb]
	\caption{CU043-Dejar de seguir jugador: Descripción}
	\label{tab:cu043_desc}
	\begin{center}
		\resizebox{15cm}{!}{
		\begin{tabular}{|p{4cm}|p{11cm}|}
			\hline
			\multicolumn{2}{|c|}{Descripción de caso de uso} \\
			\hline
			Nombre & Dejar de seguir jugador \\
			\hline
			Identificador & CU043 \\
			\hline
			Descripción & Permite dejar de seguir la actividad de un jugador \\
			\hline
			Actor &
			\begin{itemize}
				\item Patrocinador
			\end{itemize}
			\\
			\hline
			Disparador & El patrocinador elige gestionar tracking \\
			\hline
			Inclusiones & N/A \\
			\hline
			Puntos de extensión & N/A
			\\
			\hline
			Precondiciones &  
			\begin{itemize}
				\item Se ha iniciado el SNS con un rol de patrocinador deportivo
				\item Se ha elegido gestionar patrocinios de patrocinador
			\end{itemize}
			\\
			\hline
			Postcondiciones & 
			\begin{itemize}
				\item El usuario deja de seguir la actividad de un jugador
			\end{itemize}
			\\
			\hline
			Notas & N/A
			\\
			\hline
		\end{tabular}
		} \\
		\textbf{Fuente}: Autores
	\end{center}
\end{table}

\clearpage

\begin{table}[!htb]
	\caption{CU042-Dejar de seguir jugador: Flujos de hechos}
	\label{tab:cu042_flujo}
	\begin{center}
		\resizebox{15cm}{!}{
		\begin{tabular}{|p{1.5cm}|p{6cm}|p{6.5cm}|}
			\hline
			\multicolumn{3}{|c|}{Detalle de flujo de hechos de caso de uso} \\
			\hline
			Nombre & \multicolumn{2}{|c|}{Nombre del flujo} \\
			\hline
			Paso & Acción del actor & Respuesta del sistema \\
			\hline
			 & & \\
			\hline
		\end{tabular}
		} \\
		\textbf{Fuente}: Autores
	\end{center}
\end{table}

\begin{table}[!htb]
	\caption{CU043-Seguir equipo: Descripción}
	\label{tab:cu043_desc}
	\begin{center}
		\resizebox{15cm}{!}{
		\begin{tabular}{|p{4cm}|p{11cm}|}
			\hline
			\multicolumn{2}{|c|}{Descripción de caso de uso} \\
			\hline
			Nombre & Seguir equipo \\
			\hline
			Identificador & CU043 \\
			\hline
			Descripción & Permite seguir la actividad de un equipo \\
			\hline
			Actor &
			\begin{itemize}
				\item Patrocinador
			\end{itemize}
			\\
			\hline
			Disparador & El patrocinador elige gestionar tracking \\
			\hline
			Inclusiones & N/A \\
			\hline
			Puntos de extensión & N/A
			\\
			\hline
			Precondiciones &  
			\begin{itemize}
				\item Se ha iniciado el SNS con un rol de patrocinador deportivo
				\item Se ha elegido gestionar patrocinios de patrocinador
			\end{itemize}
			\\
			\hline
			Postcondiciones & 
			\begin{itemize}
				\item El usuario sigue la actividad de un equipo
			\end{itemize}
			\\
			\hline
			Notas & N/A
			\\
			\hline
		\end{tabular}
		} \\
		\textbf{Fuente}: Autores
	\end{center}
\end{table}

\begin{table}[!htb]
	\caption{CU042-Seguir equipo: Flujos de hechos}
	\label{tab:cu042_flujo}
	\begin{center}
		\resizebox{15cm}{!}{
		\begin{tabular}{|p{1.5cm}|p{6cm}|p{6.5cm}|}
			\hline
			\multicolumn{3}{|c|}{Detalle de flujo de hechos de caso de uso} \\
			\hline
			Nombre & \multicolumn{2}{|c|}{Nombre del flujo} \\
			\hline
			Paso & Acción del actor & Respuesta del sistema \\
			\hline
			 & & \\
			\hline
		\end{tabular}
		} \\
		\textbf{Fuente}: Autores
	\end{center}
\end{table}

\begin{table}[!htb]
	\caption{CU043-Dejar de seguir equipo: Descripción}
	\label{tab:cu043_desc}
	\begin{center}
		\resizebox{15cm}{!}{
		\begin{tabular}{|p{4cm}|p{11cm}|}
			\hline
			\multicolumn{2}{|c|}{Descripción de caso de uso} \\
			\hline
			Nombre & Dejar de seguir equipo \\
			\hline
			Identificador & CU043 \\
			\hline
			Descripción & Permite dejar de seguir la actividad de un equipo \\
			\hline
			Actor &
			\begin{itemize}
				\item Patrocinador
			\end{itemize}
			\\
			\hline
			Disparador & El patrocinador elige gestionar tracking \\
			\hline
			Inclusiones & N/A \\
			\hline
			Puntos de extensión & N/A
			\\
			\hline
			Precondiciones &  
			\begin{itemize}
				\item Se ha iniciado el SNS con un rol de patrocinador deportivo
				\item Se ha elegido gestionar patrocinios de patrocinador
			\end{itemize}
			\\
			\hline
			Postcondiciones & 
			\begin{itemize}
				\item El usuario deja de seguir la actividad de un equipo
			\end{itemize}
			\\
			\hline
			Notas & N/A
			\\
			\hline
		\end{tabular}
		} \\
		\textbf{Fuente}: Autores
	\end{center}
\end{table}

\begin{table}[!htb]
	\caption{CU042-Dejar de seguir equipo: Flujos de hechos}
	\label{tab:cu042_flujo}
	\begin{center}
		\resizebox{15cm}{!}{
		\begin{tabular}{|p{1.5cm}|p{6cm}|p{6.5cm}|}
			\hline
			\multicolumn{3}{|c|}{Detalle de flujo de hechos de caso de uso} \\
			\hline
			Nombre & \multicolumn{2}{|c|}{Nombre del flujo} \\
			\hline
			Paso & Acción del actor & Respuesta del sistema \\
			\hline
			 & & \\
			\hline
		\end{tabular}
		} \\
		\textbf{Fuente}: Autores
	\end{center}
\end{table}

\begin{table}[!htb]
	\caption{CU043-Seguir organización: Descripción}
	\label{tab:cu043_desc}
	\begin{center}
		\resizebox{15cm}{!}{
		\begin{tabular}{|p{4cm}|p{11cm}|}
			\hline
			\multicolumn{2}{|c|}{Descripción de caso de uso} \\
			\hline
			Nombre & Seguir organización \\
			\hline
			Identificador & CU043 \\
			\hline
			Descripción & Permite seguir la actividad de una organización \\
			\hline
			Actor &
			\begin{itemize}
				\item Patrocinador
			\end{itemize}
			\\
			\hline
			Disparador & El patrocinador elige gestionar tracking \\
			\hline
			Inclusiones & N/A \\
			\hline
			Puntos de extensión & N/A
			\\
			\hline
			Precondiciones &  
			\begin{itemize}
				\item Se ha iniciado el SNS con un rol de patrocinador deportivo
				\item Se ha elegido gestionar patrocinios de patrocinador
			\end{itemize}
			\\
			\hline
			Postcondiciones & 
			\begin{itemize}
				\item El usuario sigue la actividad de una organización
			\end{itemize}
			\\
			\hline
			Notas & N/A
			\\
			\hline
		\end{tabular}
		} \\
		\textbf{Fuente}: Autores
	\end{center}
\end{table}

\begin{table}[!htb]
	\caption{CU042-Seguir organización: Flujos de hechos}
	\label{tab:cu042_flujo}
	\begin{center}
		\resizebox{15cm}{!}{
		\begin{tabular}{|p{1.5cm}|p{6cm}|p{6.5cm}|}
			\hline
			\multicolumn{3}{|c|}{Detalle de flujo de hechos de caso de uso} \\
			\hline
			Nombre & \multicolumn{2}{|c|}{Nombre del flujo} \\
			\hline
			Paso & Acción del actor & Respuesta del sistema \\
			\hline
			 & & \\
			\hline
		\end{tabular}
		} \\
		\textbf{Fuente}: Autores
	\end{center}
\end{table}

\begin{table}[!htb]
	\caption{CU043-Dejar de seguir organización: Descripción}
	\label{tab:cu043_desc}
	\begin{center}
		\resizebox{15cm}{!}{
		\begin{tabular}{|p{4cm}|p{11cm}|}
			\hline
			\multicolumn{2}{|c|}{Descripción de caso de uso} \\
			\hline
			Nombre & Dejar de seguir organización \\
			\hline
			Identificador & CU043 \\
			\hline
			Descripción & Permite dejar de seguir la actividad de una organización \\
			\hline
			Actor &
			\begin{itemize}
				\item Patrocinador
			\end{itemize}
			\\
			\hline
			Disparador & El patrocinador elige gestionar tracking \\
			\hline
			Inclusiones & N/A \\
			\hline
			Puntos de extensión & N/A
			\\
			\hline
			Precondiciones &  
			\begin{itemize}
				\item Se ha iniciado el SNS con un rol de patrocinador deportivo
				\item Se ha elegido gestionar patrocinios de patrocinador
			\end{itemize}
			\\
			\hline
			Postcondiciones & 
			\begin{itemize}
				\item El usuario deja de seguir la actividad de una organización
			\end{itemize}
			\\
			\hline
			Notas & N/A
			\\
			\hline
		\end{tabular}
		} \\
		\textbf{Fuente}: Autores
	\end{center}
\end{table}

\begin{table}[!htb]
	\caption{CU042-Dejar de seguir organización: Flujos de hechos}
	\label{tab:cu042_flujo}
	\begin{center}
		\resizebox{15cm}{!}{
		\begin{tabular}{|p{1.5cm}|p{6cm}|p{6.5cm}|}
			\hline
			\multicolumn{3}{|c|}{Detalle de flujo de hechos de caso de uso} \\
			\hline
			Nombre & \multicolumn{2}{|c|}{Nombre del flujo} \\
			\hline
			Paso & Acción del actor & Respuesta del sistema \\
			\hline
			 & & \\
			\hline
		\end{tabular}
		} \\
		\textbf{Fuente}: Autores
	\end{center}
\end{table}

\begin{table}[!htb]
	\caption{CU043-Consultar información de seguido: Descripción}
	\label{tab:cu043_desc}
	\begin{center}
		\resizebox{15cm}{!}{
		\begin{tabular}{|p{4cm}|p{11cm}|}
			\hline
			\multicolumn{2}{|c|}{Descripción de caso de uso} \\
			\hline
			Nombre & Consultar información de seguido \\
			\hline
			Identificador & CU043 \\
			\hline
			Descripción & Permite consultar noticias de jugadores/equipos/eventos deportivos de los que se quiere ser patrocinador, siempre que éste actor/evento deportivo tenga habilitada dicha opción \\
			\hline
			Actor &
			\begin{itemize}
				\item Patrocinador
			\end{itemize}
			\\
			\hline
			Disparador & El patrocinador elige hacer tracking de un jugador/equipo/evento deportivo \\
			\hline
			Inclusiones & N/A \\
			\hline
			Puntos de extensión & N/A
			\\
			\hline
			Precondiciones &  
			\begin{itemize}
				\item Se ha iniciado el SNS con un rol de patrocinador deportivo
				\item Se ha elegido administración de sponsor
			\end{itemize}
			\\
			\hline
			Postcondiciones & 
			\begin{itemize}
				\item El usuario ha hecho tracking de un jugador/equipo/evento deportivo
				\item El usuario se encuentra en la pantalla para gestión de patrocinios
			\end{itemize}
			\\
			\hline
			Notas & N/A
			\\
			\hline
		\end{tabular}
		} \\
		\textbf{Fuente}: Autores
	\end{center}
\end{table}

\begin{table}[!htb]
	\caption{CU043-Consultar información de seguido: Flujos de hechos}
	\label{tab:cu043_flujo}
	\begin{center}
		\resizebox{15cm}{!}{
		\begin{tabular}{|p{1.5cm}|p{6cm}|p{6.5cm}|}
			\hline
			\multicolumn{3}{|c|}{Detalle de flujo de hechos de caso de uso} \\
			\hline
			Nombre & \multicolumn{2}{|c|}{Nombre del flujo} \\
			\hline
			Paso & Acción del actor & Respuesta del sistema \\
			\hline
			 & & \\
			\hline
		\end{tabular}
		} \\
		\textbf{Fuente}: Autores
	\end{center}
\end{table}

\begin{table}[!htb]
	\caption{CU042-Gestionar patrocinios de equipo/jugador: Descripción}
	\label{tab:cu042_desc}
	\begin{center}
		\resizebox{15cm}{!}{
		\begin{tabular}{|p{4cm}|p{11cm}|}
			\hline
			\multicolumn{2}{|c|}{Descripción de caso de uso} \\
			\hline
			Nombre & Gestionar patrocinios de equipo/jugador \\
			\hline
			Identificador & CU042 \\
			\hline
			Descripción & Permite gestionar los patrocinios deportivos del equipo/jugador \\
			\hline
			Actor & Todo actor de la red social
			\\
			\hline
			Disparador & El usuario elige gestionar patrocinios de equipo/jugador \\
			\hline
			Inclusiones & N/A \\
			\hline
			Puntos de extensión & N/A
			\\
			\hline
			Precondiciones &  
			\begin{itemize}
				\item La aplicación ha sido cargada por un jugador o actor con rol de formador de grupos deportivos
				\item El usuario eligió la gestión de patrocinios deportivos al equipo/jugador
			\end{itemize}
			\\
			\hline
			Postcondiciones & 
			\begin{itemize}
				\item El usuario se encuentra en la pantalla de gestión de patrocinios deportivos de equipo/jugador
			\end{itemize}
			\\
			\hline
			Notas & N/A
			\\
			\hline
		\end{tabular}
		} \\
		\textbf{Fuente}: Autores
	\end{center}
\end{table}

\clearpage

\begin{table}[!htb]
	\caption{CU041-Gestionar patrocinios de equipo/jugador: Flujos de hechos}
	\label{tab:cu041_flujo}
	\begin{center}
		\resizebox{15cm}{!}{
		\begin{tabular}{|p{1.5cm}|p{6cm}|p{6.5cm}|}
			\hline
			\multicolumn{3}{|c|}{Detalle de flujo de hechos de caso de uso} \\
			\hline
			Nombre & \multicolumn{2}{|c|}{Nombre del flujo} \\
			\hline
			Paso & Acción del actor & Respuesta del sistema \\
			\hline
			 & & \\
			\hline
		\end{tabular}
		} \\
		\textbf{Fuente}: Autores
	\end{center}
\end{table}

\begin{table}[!htb]
	\caption{CU042-Solicitar patrocinio: Descripción}
	\label{tab:cu042_desc}
	\begin{center}
		\resizebox{15cm}{!}{
		\begin{tabular}{|p{4cm}|p{11cm}|}
			\hline
			\multicolumn{2}{|c|}{Descripción de caso de uso} \\
			\hline
			Nombre & Solicitar patrocinio \\
			\hline
			Identificador & CU042 \\
			\hline
			Descripción & Permite solicitar patrocinio deportivo a un ente que pueda proporcionarlo \\
			\hline
			Actor & Todo actor de la red social
			\\
			\hline
			Disparador & El usuario elige solicitar un patrocinio \\
			\hline
			Inclusiones & N/A \\
			\hline
			Puntos de extensión & N/A
			\\
			\hline
			Precondiciones &  
			\begin{itemize}
				\item La aplicación ha sido cargada por un jugador o actor con rol de formador de grupos deportivos
				\item El usuario eligió la gestión de patrocinios deportivos al equipo/jugador
				\item El usuario eligió solicitar un patrocinio
			\end{itemize}
			\\
			\hline
			Postcondiciones & 
			\begin{itemize}
				\item El usuario ha enviado una petición de patrocinio
				\item El usuario se encuentra en la pantalla de gestión de patrocinios deportivos al equipo/jugador
			\end{itemize}
			\\
			\hline
			Notas & N/A
			\\
			\hline
		\end{tabular}
		} \\
		\textbf{Fuente}: Autores
	\end{center}
\end{table}

\begin{table}[!htb]
	\caption{CU041-Solicitar patrocinio: Flujos de hechos}
	\label{tab:cu041_flujo}
	\begin{center}
		\resizebox{15cm}{!}{
		\begin{tabular}{|p{1.5cm}|p{6cm}|p{6.5cm}|}
			\hline
			\multicolumn{3}{|c|}{Detalle de flujo de hechos de caso de uso} \\
			\hline
			Nombre & \multicolumn{2}{|c|}{Nombre del flujo} \\
			\hline
			Paso & Acción del actor & Respuesta del sistema \\
			\hline
			 & & \\
			\hline
		\end{tabular}
		} \\
		\textbf{Fuente}: Autores
	\end{center}
\end{table}

\begin{table}[!htb]
	\caption{CU042-Dejar patrocinio: Descripción}
	\label{tab:cu042_desc}
	\begin{center}
		\resizebox{15cm}{!}{
		\begin{tabular}{|p{4cm}|p{11cm}|}
			\hline
			\multicolumn{2}{|c|}{Descripción de caso de uso} \\
			\hline
			Nombre & Dejar patrocinio \\
			\hline
			Identificador & CU042 \\
			\hline
			Descripción & Permite dejar un patrocinio deportivo \\
			\hline
			Actor & Todo actor de la red social
			\\
			\hline
			Disparador & El usuario elige dejar un patrocinio \\
			\hline
			Inclusiones & N/A \\
			\hline
			Puntos de extensión & N/A
			\\
			\hline
			Precondiciones &  
			\begin{itemize}
				\item La aplicación ha sido cargada por un jugador o actor con rol de formador de grupos deportivos
				\item El usuario eligió la gestión de patrocinios deportivos al equipo/jugador
				\item El usuario eligió dejar un patrocinio
			\end{itemize}
			\\
			\hline
			Postcondiciones & 
			\begin{itemize}
				\item El usuario ha dejado un patrocinio deportivo
				\item El usuario se encuentra en la pantalla de gestión de patrocinios deportivos al equipo/jugador
			\end{itemize}
			\\
			\hline
			Notas & N/A
			\\
			\hline
		\end{tabular}
		} \\
		\textbf{Fuente}: Autores
	\end{center}
\end{table}

\begin{table}[!htb]
	\caption{CU041-Dejar patrocinio: Flujos de hechos}
	\label{tab:cu041_flujo}
	\begin{center}
		\resizebox{15cm}{!}{
		\begin{tabular}{|p{1.5cm}|p{6cm}|p{6.5cm}|}
			\hline
			\multicolumn{3}{|c|}{Detalle de flujo de hechos de caso de uso} \\
			\hline
			Nombre & \multicolumn{2}{|c|}{Nombre del flujo} \\
			\hline
			Paso & Acción del actor & Respuesta del sistema \\
			\hline
			 & & \\
			\hline
		\end{tabular}
		} \\
		\textbf{Fuente}: Autores
	\end{center}
\end{table}

\begin{table}[!htb]
	\caption{CU042-Consultar patrocinios: Descripción}
	\label{tab:cu042_desc}
	\begin{center}
		\resizebox{15cm}{!}{
		\begin{tabular}{|p{4cm}|p{11cm}|}
			\hline
			\multicolumn{2}{|c|}{Descripción de caso de uso} \\
			\hline
			Nombre & Consultar patrocinios \\
			\hline
			Identificador & CU042 \\
			\hline
			Descripción & Permite consultar los patrocinios deportivos a los que ha accedido el jugador/equipo \\
			\hline
			Actor & Todo actor de la red social
			\\
			\hline
			Disparador & El usuario elige consultar patrocinios \\
			\hline
			Inclusiones & N/A \\
			\hline
			Puntos de extensión & N/A
			\\
			\hline
			Precondiciones &  
			\begin{itemize}
				\item La aplicación ha sido cargada por un jugador o actor con rol de formador de grupos deportivos
				\item El usuario eligió la gestión de patrocinios deportivos al equipo/jugador
				\item El usuario eligió consultar patrocinios
			\end{itemize}
			\\
			\hline
			Postcondiciones & 
			\begin{itemize}
				\item El usuario se encuentra en la pantalla de consulta de patrocinios
			\end{itemize}
			\\
			\hline
			Notas & N/A
			\\
			\hline
		\end{tabular}
		} \\
		\textbf{Fuente}: Autores
	\end{center}
\end{table}

\begin{table}[!htb]
	\caption{CU041-Consultar patrocinios: Flujos de hechos}
	\label{tab:cu041_flujo}
	\begin{center}
		\resizebox{15cm}{!}{
		\begin{tabular}{|p{1.5cm}|p{6cm}|p{6.5cm}|}
			\hline
			\multicolumn{3}{|c|}{Detalle de flujo de hechos de caso de uso} \\
			\hline
			Nombre & \multicolumn{2}{|c|}{Nombre del flujo} \\
			\hline
			Paso & Acción del actor & Respuesta del sistema \\
			\hline
			 & & \\
			\hline
		\end{tabular}
		} \\
		\textbf{Fuente}: Autores
	\end{center}
\end{table}

\begin{table}[!htb]
	\caption{CU042-Aceptar solicitud de patrocinio: Descripción}
	\label{tab:cu042_desc}
	\begin{center}
		\resizebox{15cm}{!}{
		\begin{tabular}{|p{4cm}|p{11cm}|}
			\hline
			\multicolumn{2}{|c|}{Descripción de caso de uso} \\
			\hline
			Nombre & Aceptar solicitud de patrocinio \\
			\hline
			Identificador & CU042 \\
			\hline
			Descripción & Permite al jugador/equipo aceptar la propuesta de patrocinio de un patrocinador \\
			\hline
			Actor & Todo actor de la red social
			\\
			\hline
			Disparador & El usuario elige aceptar solicitud de patrocinio \\
			\hline
			Inclusiones & N/A \\
			\hline
			Puntos de extensión & N/A
			\\
			\hline
			Precondiciones &  
			\begin{itemize}
				\item La aplicación ha sido cargada por un jugador o actor con rol de formador de grupos deportivos
				\item El usuario eligió la gestión de patrocinios deportivos al equipo/jugador
				\item El usuario eligió aceptar solicitud de patrocinio
			\end{itemize}
			\\
			\hline
			Postcondiciones & 
			\begin{itemize}
				\item El usuario se encuentra en la pantalla de gestión de patrocinios de equipo/jugador
			\end{itemize}
			\\
			\hline
			Notas & N/A
			\\
			\hline
		\end{tabular}
		} \\
		\textbf{Fuente}: Autores
	\end{center}
\end{table}

\begin{table}[!htb]
	\caption{CU041-Aceptar solicitud de patrocinio: Flujos de hechos}
	\label{tab:cu041_flujo}
	\begin{center}
		\resizebox{15cm}{!}{
		\begin{tabular}{|p{1.5cm}|p{6cm}|p{6.5cm}|}
			\hline
			\multicolumn{3}{|c|}{Detalle de flujo de hechos de caso de uso} \\
			\hline
			Nombre & \multicolumn{2}{|c|}{Nombre del flujo} \\
			\hline
			Paso & Acción del actor & Respuesta del sistema \\
			\hline
			 & & \\
			\hline
		\end{tabular}
		} \\
		\textbf{Fuente}: Autores
	\end{center}
\end{table}

\section{Módulo de gestión de administracion de jugadores}

%\input{./imagenes/casos_uso/administracion_jugadores.png}

Este módulo ofrece funcionalidades a un jugador para administrar los deportes que éste juega, permitiendo al jugador dar informcación adicional de él en cada uno de los deportes que éste juega.

\clearpage
\begin{table}[!htb]
	\caption{CU001-Administrar jugador: Descripción}
	\label{tab:cu001_desc}
	\begin{center}
		\resizebox{15cm}{!}{
		\begin{tabular}{|p{4cm}|p{11cm}|}
			\hline
			\multicolumn{2}{|c|}{Descripción de caso de uso} \\
			\hline
			Nombre & Administrar jugador \\
			\hline
			Identificador & CU \\
			\hline
			Descripción & Permite la administración de características especialmente de jugador \\
			\hline
			Actor & 
			\begin{itemize}
				\item Jugador
			\end{itemize} \\
			\hline
			Disparador & El jugador quiere administrar la información personal suya, refiriendose en éste a las del jugador especificamente \\
			\hline
			Inclusiones & \\
			\hline
			Puntos de extensión & \\
			\hline
			Precondiciones & 
			\begin{itemize}
				\item El SNS se inicia con rol de jugador
			\end{itemize} \\
			\hline
			Postcondiciones & \\
			\hline
			Notas & 
			\begin{itemize}
				\item Generalización de:
				\begin{itemize}
					\item Actualizar información de jugador
				\end{itemize}
			\end{itemize} \\
			\hline 
		\end{tabular}
		} \\
		\textbf{Fuente}: Autores
	\end{center}
\end{table}

\begin{table}[!htb]
	\caption{CU001-Administrar jugador: Flujos de hechos }
	\label{tab:cu001_flujo}
	\begin{center}
		\resizebox{15cm}{!}{
		\begin{tabular}{|p{1.5cm}|p{6cm}|p{6.5cm}|}
			\hline
			\multicolumn{3}{|c|}{Detalle de flujo de hechos de caso de uso} \\
			\hline
			Nombre & \multicolumn{2}{|c|}{Nombre del flujo} \\
			\hline
			Paso & Acción del actor & Respuesta del sistema \\
			\hline
			 & & \\
			\hline
		\end{tabular}
		} \\
		\textbf{Fuente}: Autores
	\end{center}
\end{table}

\begin{table}[!htb]
	\caption{CU001-Actualizar información de jugador: Descripción}
	\label{tab:cu001_desc}
	\begin{center}
		\resizebox{15cm}{!}{
		\begin{tabular}{|p{4cm}|p{11cm}|}
			\hline
			\multicolumn{2}{|c|}{Descripción de caso de uso} \\
			\hline
			Nombre & Actualizar información de jugador \\
			\hline
			Identificador & CU\\
			\hline
			Descripción & Permite actualizar la información de jugador de un actor jugador en la red social \\
			\hline
			Actor & 
			\begin{itemize}
				\item Jugador
			\end{itemize} \\
			\hline
			Disparador & El jugador quiere actualizar información personal suya, refiriendose en éste a las del jugador especificamente \\
			\hline
			Inclusiones & \\
			\hline
			Puntos de extensión & \\
			\hline
			Precondiciones & 
			\begin{itemize}
				\item El SNS se inicia con rol de jugador
			\end{itemize} \\
			\hline
			Postcondiciones & \\
			\hline
			Notas & 
			\begin{itemize}
				\item Generalización de:
				\begin{itemize}
			 		\item Gestionar deportes practicados
				\end{itemize}
			\end{itemize} \\
			\hline
		\end{tabular}
		} \\
		\textbf{Fuente}: Autores
	\end{center}
\end{table}

\begin{table}[!htb]
	\caption{CU001-Actualizar información de jugador: Flujos de hechos }
	\label{tab:cu001_flujo}
	\begin{center}
		\resizebox{15cm}{!}{
		\begin{tabular}{|p{1.5cm}|p{6cm}|p{6.5cm}|}
			\hline
			\multicolumn{3}{|c|}{Detalle de flujo de hechos de caso de uso} \\
			\hline
			Nombre & \multicolumn{2}{|c|}{Nombre del flujo} \\
			\hline
			Paso & Acción del actor & Respuesta del sistema \\
			\hline
			 & & \\
			\hline
		\end{tabular}
		} \\
		\textbf{Fuente}: Autores
	\end{center}
\end{table}

\begin{table}[!htb]
	\caption{CU001-Gestionar deportes practicados: Descripción}
	\label{tab:cu001_desc}
	\begin{center}
		\resizebox{15cm}{!}{
		\begin{tabular}{|p{4cm}|p{11cm}|}
			\hline
			\multicolumn{2}{|c|}{Descripción de caso de uso} \\
			\hline
			Nombre & Gestionar deportes practicados \\
			\hline
			Identificador & CU\\
			\hline
			Descripción & Permite la gestión de la información de jugador correspondiente a los deportes practicados \\
			\hline
			Actor & 
			\begin{itemize}
				\item Jugador
			\end{itemize} \\
			\hline
			Disparador & El jugador quiere gestionar la información correspondiente a los deportes que él practica \\
			\hline
			Inclusiones & \\
			\hline
			Puntos de extensión & \\
			\hline
			Precondiciones & 
			\begin{itemize}
				\item El SNS se inicia con rol de jugador
			\end{itemize} \\
			\hline
			Postcondiciones & \\
			\hline
			Notas & 
			\begin{itemize}
				\item Generalización de:
				\begin{itemize}
			 		\item Agregar deporte practicado
			 		\item Eliminar deporte practicado
			 		\item Actualizar información de deporte
				\end{itemize}
			\end{itemize} \\
			\hline
		\end{tabular}
		} \\
		\textbf{Fuente}: Autores
	\end{center}
\end{table}

\begin{table}[!htb]
	\caption{CU001-Gestionar deportes practicados: Flujos de hechos }
	\label{tab:cu001_flujo}
	\begin{center}
		\resizebox{15cm}{!}{
		\begin{tabular}{|p{1.5cm}|p{6cm}|p{6.5cm}|}
			\hline
			\multicolumn{3}{|c|}{Detalle de flujo de hechos de caso de uso} \\
			\hline
			Nombre & \multicolumn{2}{|c|}{Nombre del flujo} \\
			\hline
			Paso & Acción del actor & Respuesta del sistema \\
			\hline
			 & & \\
			\hline
		\end{tabular}
		} \\
		\textbf{Fuente}: Autores
	\end{center}
\end{table}

\begin{table}[!htb]
	\caption{CU001-Agregar deporte practicado: Descripción}
	\label{tab:cu001_desc}
	\begin{center}
		\resizebox{15cm}{!}{
		\begin{tabular}{|p{4cm}|p{11cm}|}
			\hline
			\multicolumn{2}{|c|}{Descripción de caso de uso} \\
			\hline
			Nombre & Agregar deporte practicado \\
			\hline
			Identificador & CU\\
			\hline
			Descripción & Permite agregar un deporte practicado por el jugador \\
			\hline
			Actor & 
			\begin{itemize}
				\item Jugador
			\end{itemize} \\
			\hline
			Disparador & El jugador quiere agregar un deporte practicado por él \\
			\hline
			Inclusiones & \\
			\hline
			Puntos de extensión & \\
			\hline
			Precondiciones & 
			\begin{itemize}
				\item El SNS se inicia con rol de jugador
			\end{itemize} \\
			\hline
			Postcondiciones & 
			\begin{itemize}
				\item El jugador agrega un deporte practicado por él
			\end{itemize} \\
			\hline
			Notas & \\
			\hline
		\end{tabular}
		} \\
		\textbf{Fuente}: Autores
	\end{center}
\end{table}

\begin{table}[!htb]
	\caption{CU001-Agregar deporte practicado: Flujos de hechos }
	\label{tab:cu001_flujo}
	\begin{center}
		\resizebox{15cm}{!}{
		\begin{tabular}{|p{1.5cm}|p{6cm}|p{6.5cm}|}
			\hline
			\multicolumn{3}{|c|}{Detalle de flujo de hechos de caso de uso} \\
			\hline
			Nombre & \multicolumn{2}{|c|}{Nombre del flujo} \\
			\hline
			Paso & Acción del actor & Respuesta del sistema \\
			\hline
			 & & \\
			\hline
		\end{tabular}
		} \\
		\textbf{Fuente}: Autores
	\end{center}
\end{table}

\begin{table}[!htb]
	\caption{CU001-Eliminar deporte practicado: Descripción}
	\label{tab:cu001_desc}
	\begin{center}
		\resizebox{15cm}{!}{
		\begin{tabular}{|p{4cm}|p{11cm}|}
			\hline
			\multicolumn{2}{|c|}{Descripción de caso de uso} \\
			\hline
			Nombre & Eliminar deporte practicado \\
			\hline
			Identificador & CU\\
			\hline
			Descripción & Permite eliminar un deporte practicado por el jugador \\
			\hline
			Actor & 
			\begin{itemize}
				\item Jugador
			\end{itemize} \\
			\hline
			Disparador & El jugador quiere eliminar un deporte practicado por él \\
			\hline
			Inclusiones & \\
			\hline
			Puntos de extensión & \\
			\hline
			Precondiciones & 
			\begin{itemize}
				\item El SNS se inicia con rol de jugador
			\end{itemize} \\
			\hline
			Postcondiciones & 
			\begin{itemize}
				\item El jugador elimina un deporte practicado por él
			\end{itemize} \\
			\hline
			Notas & \\
			\hline
		\end{tabular}
		} \\
		\textbf{Fuente}: Autores
	\end{center}
\end{table}

\begin{table}[!htb]
	\caption{CU001-Eliminar deporte practicado: Flujos de hechos }
	\label{tab:cu001_flujo}
	\begin{center}
		\resizebox{15cm}{!}{
		\begin{tabular}{|p{1.5cm}|p{6cm}|p{6.5cm}|}
			\hline
			\multicolumn{3}{|c|}{Detalle de flujo de hechos de caso de uso} \\
			\hline
			Nombre & \multicolumn{2}{|c|}{Nombre del flujo} \\
			\hline
			Paso & Acción del actor & Respuesta del sistema \\
			\hline
			 & & \\
			\hline
		\end{tabular}
		} \\
		\textbf{Fuente}: Autores
	\end{center}
\end{table}

\begin{table}[!htb]
	\caption{CU001-Actualizar deporte practicado: Descripción}
	\label{tab:cu001_desc}
	\begin{center}
		\resizebox{15cm}{!}{
		\begin{tabular}{|p{4cm}|p{11cm}|}
			\hline
			\multicolumn{2}{|c|}{Descripción de caso de uso} \\
			\hline
			Nombre & Actualizar deporte practicado \\
			\hline
			Identificador & CU\\
			\hline
			Descripción & Permite actualizar la información de un deporte practicado por el jugador \\
			\hline
			Actor & 
			\begin{itemize}
				\item Jugador
			\end{itemize} \\
			\hline
			Disparador & El jugador quiere actualizar la información un deporte practicado por él \\
			\hline
			Inclusiones & \\
			\hline
			Puntos de extensión & \\
			\hline
			Precondiciones & 
			\begin{itemize}
				\item El SNS se inicia con rol de jugador
			\end{itemize} \\
			\hline
			Postcondiciones & 
			\begin{itemize}
				\item El jugador actualiza la información un deporte practicado por él
			\end{itemize} \\
			\hline
			Notas & \\
			\hline
		\end{tabular}
		} \\
		\textbf{Fuente}: Autores
	\end{center}
\end{table}

\begin{table}[!htb]
	\caption{CU001-Actualizar deporte practicado: Flujos de hechos }
	\label{tab:cu001_flujo}
	\begin{center}
		\resizebox{15cm}{!}{
		\begin{tabular}{|p{1.5cm}|p{6cm}|p{6.5cm}|}
			\hline
			\multicolumn{3}{|c|}{Detalle de flujo de hechos de caso de uso} \\
			\hline
			Nombre & \multicolumn{2}{|c|}{Nombre del flujo} \\
			\hline
			Paso & Acción del actor & Respuesta del sistema \\
			\hline
			 & & \\
			\hline
		\end{tabular}
		} \\
		\textbf{Fuente}: Autores
	\end{center}
\end{table}

\section{Módulo de gestión de self-sharing}

%\input{./imagenes/casos_uso/gestion_self_sharing.png}

Este módulo ofrece funcionalidades propias de redes social similares a Facebook. Por medio de éste el usuario podrá manejar la comunicación de él con sus amigos a traves de un timeline, del servicio de mensajería instantánea y de la posibilidad de compartir elementos multimedia como lo son videos y fotos, así como también mostrar y actualizar su información personal.

\clearpage

\begin{table}[!htb]
	\caption{CU094-Gestionar self-sharing: Descripción}
	\label{tab:cu094_desc}
	\begin{center}
		\resizebox{15cm}{!}{
		\begin{tabular}{|p{4cm}|p{11cm}|}
			\hline
			\multicolumn{2}{|c|}{Descripción de caso de uso} \\
			\hline
			Nombre & Gestionar self-sharing \\
			\hline
			Identificador & CU094 \\
			\hline
			Descripción & Permite la gestión de todas las características self-sharing proporcionadas por el SNS \\
			\hline
			Actor & 
			\begin{itemize}
				\item Todos
			\end{itemize} \\
			\hline
			Disparador & El actor desea gestionar las opciones proporcionadas por el módulo de self-sharing para él, una organización, un equipo/grupo deportivo o un evento deportivo \\
			\hline
			Inclusiones & \\
			\hline
			Puntos de extensión & \\
			\hline
			Precondiciones & \\
			\hline
			Postcondiciones & \\
			\hline
			Notas & 
			\begin{itemize}
				\item Soporta el uso tanto en los actores deportivos como en los eventos. Esta característica es desencadenada por todos los casos de uso que se desprenden de éste
				\item Generalización de:
				\begin{itemize}
					\item Gestionar timelines
					\item Gestionar material multimedia
					\item Gestionar mensajería instantánea
				\end{itemize}
			\end{itemize} \\
			\hline 
		\end{tabular}
		} \\
		\textbf{Fuente}: Autores
	\end{center}
\end{table}

\begin{table}[!htb]
	\caption{CU094-Gestionar self-sharing: Flujos de hechos }
	\label{tab:cu094_flujo}
	\begin{center}
		\resizebox{15cm}{!}{
		\begin{tabular}{|p{1.5cm}|p{6cm}|p{6.5cm}|}
			\hline
			\multicolumn{3}{|c|}{Detalle de flujo de hechos de caso de uso} \\
			\hline
			Nombre & \multicolumn{2}{|c|}{Nombre del flujo} \\
			\hline
			Paso & Acción del actor & Respuesta del sistema \\
			\hline
			1 & El usuario elige gestionar self-sharing & El sistema muestra las opciones proporcionadas por el módulo de self-sharing \\
			\hline
		\end{tabular}
		} \\
		\textbf{Fuente}: Autores
	\end{center}
\end{table}

\begin{table}[!htb]
	\caption{CU095-Gestionar timeline: Descripción}
	\label{tab:cu095_desc}
	\begin{center}
		\resizebox{15cm}{!}{
		\begin{tabular}{|p{4cm}|p{11cm}|}
			\hline
			\multicolumn{2}{|c|}{Descripción de caso de uso} \\
			\hline
			Nombre & Gestionar timeline \\
			\hline
			Identificador & CU095 \\
			\hline
			Descripción & Permite la gestión de un timeline para cada actor/evento en la red social \\
			\hline
			Actor & 
			\begin{itemize}
				\item Todos
			\end{itemize} \\
			\hline
			Disparador & El actor desea gestionar el timeline de él, una organización, un equipo/grupo deportivo o un evento deportivo \\
			\hline
			Inclusiones & \\
			\hline
			Puntos de extensión & \\
			\hline
			Precondiciones & \\
			\hline
			Postcondiciones & \\
			\hline
			Notas & 
			\begin{itemize}
				\item Soporta el uso tanto en los actores deportivos como en los eventos. Esta característica es desencadenada por todos los casos de uso que se desprenden de éste
				\item Generalización de:
				\begin{itemize}
					\item Crear post
					\item Comentar un post
					\item Eliminar un post
				\end{itemize}
				\item Si se desea ver el timeline de un actor/evento deportivo, se ha de buscar primero
			\end{itemize} \\
			\hline 
		\end{tabular}
		} \\
		\textbf{Fuente}: Autores
	\end{center}
\end{table}

\begin{table}[!htb]
	\caption{CU095-Gestionar timeline: Flujos de hechos }
	\label{tab:cu095_flujo}
	\begin{center}
		\resizebox{15cm}{!}{
		\begin{tabular}{|p{1.5cm}|p{6cm}|p{6.5cm}|}
			\hline
			\multicolumn{3}{|c|}{Detalle de flujo de hechos de caso de uso} \\
			\hline
			Nombre & \multicolumn{2}{|c|}{Nombre del flujo} \\
			\hline
			Paso & Acción del actor & Respuesta del sistema \\
			\hline
			1 & El usuario ha elegido gestionar self-sharing & El sistema ha mostrado las opciones de self-sharing que tiene el usuario a disposición \\
			\hline
			2 & El usuario elige gestionar su timeline, el timeline general o el timeline de un actor/evento deportivo & El sistema muestra el timeline elegido \\
			\hline
			3 & & \\
			\hline
		\end{tabular}
		} \\
		\textbf{Fuente}: Autores
	\end{center}
\end{table}

\begin{table}[!htb]
	\caption{CU096-Crear post: Descripción}
	\label{tab:cu096_desc}
	\begin{center}
		\resizebox{15cm}{!}{
		\begin{tabular}{|p{4cm}|p{11cm}|}
			\hline
			\multicolumn{2}{|c|}{Descripción de caso de uso} \\
			\hline
			Nombre & Crear post \\
			\hline
			Identificador & CU096 \\
			\hline
			Descripción & Permite la creación de un post sobre un timeline \\
			\hline
			Actor & 
			\begin{itemize}
				\item Todos
			\end{itemize} \\
			\hline
			Disparador & El actor desea crear un post sobre un timeline propio o ajeno, según sus privilegios se lo permitan \\
			\hline
			Inclusiones & \\
			\hline
			Puntos de extensión & \\
			\hline
			Precondiciones & 
			\begin{itemize}
				\item El actor tiene privilegios para la creación del post
			\end{itemize} \\
			\hline
			Postcondiciones &
			\begin{itemize}
				\item El actor crea un post sobre el timeline elegido
			\end{itemize} \\
			\hline
			Notas & 
			\begin{itemize}
				\item Soporta el uso tanto en los actores deportivos como en los eventos. Esta característica es desencadenada por todos los casos de uso que se desprenden de éste
			\end{itemize} \\
			\hline 
		\end{tabular}
		} \\
		\textbf{Fuente}: Autores
	\end{center}
\end{table}

\begin{table}[!htb]
	\caption{CU096-Crear post: Flujos de hechos}
	\label{tab:cu096_flujo}
	\begin{center}
		\resizebox{15cm}{!}{
		\begin{tabular}{|p{1.5cm}|p{6cm}|p{6.5cm}|}
			\hline
			\multicolumn{3}{|c|}{Detalle de flujo de hechos de caso de uso} \\
			\hline
			Nombre & \multicolumn{2}{|c|}{Nombre del flujo} \\
			\hline
			Paso & Acción del actor & Respuesta del sistema \\
			\hline
			1 & El usuario ha elegido gestionar timeline & El sistema ha mostrado el timeline propio o de otro actor/evento en la red social \\
			\hline
			2 & El usuario escribe en el cuadro de texto dirigido a la escritura de posts y elige publicarlo & El sistema guarda el post y lo publica en el timeline \\
			\hline
			3 & & De haber un error, el sistema mostrará un mensaje indicándolo \\
			\hline
			4 & & El sistema ubica al usuario en el timeline \\
			\hline
		\end{tabular}
		} \\
		\textbf{Fuente}: Autores
	\end{center}
\end{table}

\begin{table}[!htb]
	\caption{CU097-Comentar un post: Descripción}
	\label{tab:cu097_desc}
	\begin{center}
		\resizebox{15cm}{!}{
		\begin{tabular}{|p{4cm}|p{11cm}|}
			\hline
			\multicolumn{2}{|c|}{Descripción de caso de uso} \\
			\hline
			Nombre & Comentar un post \\
			\hline
			Identificador & CU097 \\
			\hline
			Descripción & Permite comentar un post creado sobre un timeline \\
			\hline
			Actor & 
			\begin{itemize}
				\item Todos
			\end{itemize} \\
			\hline
			Disparador & El actor desea comentar un post creado sobre un timeline propio o ageno, según sus privilegios se lo permitan \\
			\hline
			Inclusiones & \\
			\hline
			Puntos de extensión & \\
			\hline
			Precondiciones & 
			\begin{itemize}
				\item El actor tiene privilegios para comentar el post creado
			\end{itemize} \\
			\hline
			Postcondiciones &
			\begin{itemize}
				\item El actor comenta un post creado sobre el timeline elegido
			\end{itemize} \\
			\hline
			Notas & 
			\begin{itemize}
				\item Soporta el uso tanto en los actores deportivos como en los eventos. Esta característica es desencadenada por todos los casos de uso que se desprenden de éste
			\end{itemize} \\
			\hline 
		\end{tabular}
		} \\
		\textbf{Fuente}: Autores
	\end{center}
\end{table}

\begin{table}[!htb]
	\caption{CU097-Comentar un post: Flujos de hechos }
	\label{tab:cu097_flujo}
	\begin{center}
		\resizebox{15cm}{!}{
		\begin{tabular}{|p{1.5cm}|p{6cm}|p{6.5cm}|}
			\hline
			\multicolumn{3}{|c|}{Detalle de flujo de hechos de caso de uso} \\
			\hline
			Nombre & \multicolumn{2}{|c|}{Nombre del flujo} \\
			\hline
			Paso & Acción del actor & Respuesta del sistema \\
			\hline
			1 & El usuario ha elegido gestionar timeline & El sistema ha mostrado el timeline propio o de otro actor/evento en la red social \\
			\hline
			2 & El usuario se ubica en un post y elige comentar acerca de él. Guarda los cambios & El sistema guarda el comentario en el sistema y lo publica en el post \\
			\hline
			3 & & De haber un error, el sistema mostrará un mensaje indicándolo \\
			\hline
			4 & & El sistema ubica al usuario en el post elegido \\
			\hline
		\end{tabular}
		} \\
		\textbf{Fuente}: Autores
	\end{center}
\end{table}

\begin{table}[!htb]
	\caption{CU098-Eliminar un post: Descripción}
	\label{tab:cu098_desc}
	\begin{center}
		\resizebox{15cm}{!}{
		\begin{tabular}{|p{4cm}|p{11cm}|}
			\hline
			\multicolumn{2}{|c|}{Descripción de caso de uso} \\
			\hline
			Nombre & Eliminar un post \\
			\hline
			Identificador & CU098 \\
			\hline
			Descripción & Permite eliminar un post creado sobre un timeline administrado por el actor o creado por este \\
			\hline
			Actor & 
			\begin{itemize}
				\item Todos
			\end{itemize} \\
			\hline
			Disparador & El actor desea eliminar un post creado sobre un timeline propio o uno creado por él \\
			\hline
			Inclusiones & \\
			\hline
			Puntos de extensión & \\
			\hline
			Precondiciones & 
			\begin{itemize}
				\item El actor tiene privilegios para eliminar el post creado
			\end{itemize} \\
			\hline
			Postcondiciones &
			\begin{itemize}
				\item El actor elimina un post creado sobre el timeline elegido
			\end{itemize} \\
			\hline
			Notas & 
			\begin{itemize}
				\item Soporta el uso tanto en los actores deportivos como en los eventos. Esta característica es desencadenada por todos los casos de uso que se desprenden de éste
			\end{itemize} \\
			\hline 
		\end{tabular}
		} \\
		\textbf{Fuente}: Autores
	\end{center}
\end{table}

\begin{table}[!htb]
	\caption{CU098-Eliminar un post: Flujos de hechos }
	\label{tab:cu098_flujo}
	\begin{center}
		\resizebox{15cm}{!}{
		\begin{tabular}{|p{1.5cm}|p{6cm}|p{6.5cm}|}
			\hline
			\multicolumn{3}{|c|}{Detalle de flujo de hechos de caso de uso} \\
			\hline
			Nombre & \multicolumn{2}{|c|}{Nombre del flujo} \\
			\hline
			Paso & Acción del actor & Respuesta del sistema \\
			\hline
			1 & El usuario ha elegido gestionar timeline & El sistema ha mostrado el timeline propio o de otro actor/evento en la red social \\
			\hline
			2 & El usuario se posiciona en un post y elige borrarlo & El sistema guarda los cambios \\
			\hline
			3 & & De haber un error, el sistema mostrará un mensaje indicándolo \\
			\hline
			4 & & El sistema ubica al usuario en el timeline elegido \\
			\hline
		\end{tabular}
		} \\
		\textbf{Fuente}: Autores
	\end{center}
\end{table}

\begin{table}[!htb]
	\caption{CU099-Gestionar material multimedia: Descripción}
	\label{tab:cu099_desc}
	\begin{center}
		\resizebox{15cm}{!}{
		\begin{tabular}{|p{4cm}|p{11cm}|}
			\hline
			\multicolumn{2}{|c|}{Descripción de caso de uso} \\
			\hline
			Nombre & Gestionar material multimedia \\
			\hline
			Identificador & CU099 \\
			\hline
			Descripción & Permite la gestión del material multimedia del actor o evento deportivo desde el que se ha activado la funcionalidad \\
			\hline
			Actor & 
			\begin{itemize}
				\item Todos
			\end{itemize} \\
			\hline
			Disparador & El actor desea gestionar material multimedia del actor o evento deportivo desde el que se ha activado la funcionalidad \\
			\hline
			Inclusiones & \\
			\hline
			Puntos de extensión & \\
			\hline
			Precondiciones & \\
			\hline
			Postcondiciones & \\
			\hline
			Notas & 
			\begin{itemize}
				\item Soporta el uso tanto en los actores deportivos como en los eventos. Esta característica es desencadenada por todos los casos de uso que se desprenden de éste
				\item Generalización de:
				\begin{itemize}
					\item Gestionar albumes de fotos
					\item Gestionar videos
				\end{itemize}
			\end{itemize} \\
			\hline 
		\end{tabular}
		} \\
		\textbf{Fuente}: Autores
	\end{center}
\end{table}

\begin{table}[!htb]
	\caption{CU099-Gestionar material multimedia: Flujos de hechos }
	\label{tab:cu099_flujo}
	\begin{center}
		\resizebox{15cm}{!}{
		\begin{tabular}{|p{1.5cm}|p{6cm}|p{6.5cm}|}
			\hline
			\multicolumn{3}{|c|}{Detalle de flujo de hechos de caso de uso} \\
			\hline
			Nombre & \multicolumn{2}{|c|}{Nombre del flujo} \\
			\hline
			Paso & Acción del actor & Respuesta del sistema \\
			\hline
			1 & El usuario elige gestionar material multimedia (Albumes de fotos o videos) & El sistema muestra la interfaz del material multimedia elegido \\
			\hline
		\end{tabular}
		} \\
		\textbf{Fuente}: Autores
	\end{center}
\end{table}

\begin{table}[!htb]
	\caption{CU100-Gestionar albumes de fotos: Descripción}
	\label{tab:cu100_desc}
	\begin{center}
		\resizebox{15cm}{!}{
		\begin{tabular}{|p{4cm}|p{11cm}|}
			\hline
			\multicolumn{2}{|c|}{Descripción de caso de uso} \\
			\hline
			Nombre & Gestionar albumes de fotos \\
			\hline
			Identificador & CU100 \\
			\hline
			Descripción & Permite la gestión de albumes de fotos del actor o evento deportivo desde el que se ha activado la funcionalidad \\
			\hline
			Actor & 
			\begin{itemize}
				\item Todos
			\end{itemize} \\
			\hline
			Disparador & El actor desea gestionar albumes de fotos del actor o evento deportivo desde el que se ha activado la funcionalidad \\
			\hline
			Inclusiones & \\
			\hline
			Puntos de extensión & \\
			\hline
			Precondiciones & \\
			\hline
			Postcondiciones & \\
			\hline
			Notas & 
			\begin{itemize}
				\item Soporta el uso tanto en los actores deportivos como en los eventos. Esta característica es desencadenada por todos los casos de uso que se desprenden de éste
				\item Generalización de:
				\begin{itemize}
					\item Crear album de fotos
					\item Subir fotos a album
					\item Eliminar album
					\item Ver album
					\item Gestionar fotos de album
					\item Actualizar album de fotos
				\end{itemize}
			\end{itemize} \\
			\hline 
		\end{tabular}
		} \\
		\textbf{Fuente}: Autores
	\end{center}
\end{table}

\begin{table}[!htb]
	\caption{CU100-Gestionar albumes de fotos: Flujos de hechos }
	\label{tab:cu100_flujo}
	\begin{center}
		\resizebox{15cm}{!}{
		\begin{tabular}{|p{1.5cm}|p{6cm}|p{6.5cm}|}
			\hline
			\multicolumn{3}{|c|}{Detalle de flujo de hechos de caso de uso} \\
			\hline
			Nombre & \multicolumn{2}{|c|}{Nombre del flujo} \\
			\hline
			Paso & Acción del actor & Respuesta del sistema \\
			\hline
			1 & El usuario ha elegido gestionar self-sharing & El sistema ha mostrado las opciones de gestión self-sharing \\
			\hline
			2 & El usuario elige gestionar fotos & El sistema muestra la interfaz de gestión de fotos \\
			\hline
		\end{tabular}
		} \\
		\textbf{Fuente}: Autores
	\end{center}
\end{table}

\clearpage

\begin{table}[!htb]
	\caption{CU101-Crear album de fotos: Descripción}
	\label{tab:cu001_desc}
	\begin{center}
		\resizebox{15cm}{!}{
		\begin{tabular}{|p{4cm}|p{11cm}|}
			\hline
			\multicolumn{2}{|c|}{Descripción de caso de uso} \\
			\hline
			Nombre & Crear album de fotos \\
			\hline
			Identificador & CU101 \\
			\hline
			Descripción & Permite la creación de un album de fotos del actor o evento deportivo desde el que se ha activado la funcionalidad \\
			\hline
			Actor & 
			\begin{itemize}
				\item Todos
			\end{itemize} \\
			\hline
			Disparador & El actor desea crear album de fotos del actor o evento deportivo desde el que se ha activado la funcionalidad \\
			\hline
			Inclusiones & \\
			\hline
			Puntos de extensión & 
			\begin{itemize}
				\item Gestionar fotos de album
			\end{itemize}						
			\\
			\hline
			Precondiciones &  \\
			\hline
			Postcondiciones & 
			\begin{itemize}
				\item El actor crea un album de fotos
			\end{itemize} \\
			\hline
			Notas & 
			\begin{itemize}
				\item Soporta el uso tanto en los actores deportivos como en los eventos. Esta característica es desencadenada por todos los casos de uso que se desprenden de éste
			\end{itemize} \\
			\hline 
		\end{tabular}
		} \\
		\textbf{Fuente}: Autores
	\end{center}
\end{table}

\begin{table}[!htb]
	\caption{CU101-Crear album de fotos: Flujos de hechos }
	\label{tab:cu101_flujo}
	\begin{center}
		\resizebox{15cm}{!}{
		\begin{tabular}{|p{1.5cm}|p{6cm}|p{6.5cm}|}
			\hline
			\multicolumn{3}{|c|}{Detalle de flujo de hechos de caso de uso} \\
			\hline
			Nombre & \multicolumn{2}{|c|}{Nombre del flujo} \\
			\hline
			Paso & Acción del actor & Respuesta del sistema \\
			\hline
			1 & El usuario elige gestionar fotos & El sistema muestra la interfaz de gestión de fotos \\
			\hline
			2 & El usuario elige crear un album de fotos & El sistema muestra la interfaz de creación del album de fotos  \\
			\hline
			3 & El usuario ingresa la información requerida y elige, también, gestionar las fotos del nuevo album & El sistema muestra la interfaz de gestión de fotos cuando el usuario lo requiera en orden de crear el album \\
			\hline
			4 & El usuario elige crear el album recién modificado & El sistema guarda cambios \\
			\hline
			5 & & El sistema muestra un elemento emergente informando del éxito o fracaso de la operación \\
			\hline
			6 & El usuario continua & El sistema muestra la interfaz de gestión de albumes de fotos \\
			\hline
		\end{tabular}
		} \\
		\textbf{Fuente}: Autores
	\end{center}
\end{table}

\begin{table}[!htb]
	\caption{CU102-Eliminar album: Descripción}
	\label{tab:cu102_desc}
	\begin{center}
		\resizebox{15cm}{!}{
		\begin{tabular}{|p{4cm}|p{11cm}|}
			\hline
			\multicolumn{2}{|c|}{Descripción de caso de uso} \\
			\hline
			Nombre & Eliminar album \\
			\hline
			Identificador & CU102 \\
			\hline
			Descripción & Permite la eliminación de un album de fotos del actor o evento deportivo desde el que se ha activado la funcionalidad \\
			\hline
			Actor & 
			\begin{itemize}
				\item Todos
			\end{itemize} \\
			\hline
			Disparador & El actor desea eliminar album de fotos del actor o evento deportivo desde el que se ha activado la funcionalidad \\
			\hline
			Inclusiones & \\
			\hline
			Puntos de extensión & \\
			\hline
			Precondiciones &  \\
			\hline
			Postcondiciones & 
			\begin{itemize}
				\item El actor elimina un album de fotos
			\end{itemize} \\
			\hline
			Notas & 
			\begin{itemize}
				\item Soporta el uso tanto en los actores deportivos como en los eventos. Esta característica es desencadenada por todos los casos de uso que se desprenden de éste
			\end{itemize} \\
			\hline 
		\end{tabular}
		} \\
		\textbf{Fuente}: Autores
	\end{center}
\end{table}

\begin{table}[!htb]
	\caption{CU102-Eliminar album: Flujos de hechos }
	\label{tab:cu102_flujo}
	\begin{center}
		\resizebox{15cm}{!}{
		\begin{tabular}{|p{1.5cm}|p{6cm}|p{6.5cm}|}
			\hline
			\multicolumn{3}{|c|}{Detalle de flujo de hechos de caso de uso} \\
			\hline
			Nombre & \multicolumn{2}{|c|}{Nombre del flujo} \\
			\hline
			Paso & Acción del actor & Respuesta del sistema \\
			\hline
			 & & \\
			\hline
		\end{tabular}
		} \\
		\textbf{Fuente}: Autores
	\end{center}
\end{table}

\begin{table}[!htb]
	\caption{CU103-Actualizar album de fotos: Descripción}
	\label{tab:cu103_desc}
	\begin{center}
		\resizebox{15cm}{!}{
		\begin{tabular}{|p{4cm}|p{11cm}|}
			\hline
			\multicolumn{2}{|c|}{Descripción de caso de uso} \\
			\hline
			Nombre & Actualizar album de fotos \\
			\hline
			Identificador & CU103 \\
			\hline
			Descripción & Permite la actualización de la información del album de fotos elegido \\
			\hline
			Actor & 
			\begin{itemize}
				\item Todos
			\end{itemize} \\
			\hline
			Disparador & El actor desea actualizar la información del actor o evento deportivo desde el que se ha activado la funcionalidad \\
			\hline
			Inclusiones & \\
			\hline
			Puntos de extensión & 
			\begin{itemize}
				\item Gestionar fotos de album
			\end{itemize}						
			\\
			\hline
			Precondiciones & \\
			\hline
			Postcondiciones & 
			\begin{itemize}
				\item El actor actualiza la información del album de fotos
			\end{itemize} \\
			\hline
			Notas & 
			\begin{itemize}
				\item Soporta el uso tanto en los actores deportivos como en los eventos. Esta característica es desencadenada por todos los casos de uso que se desprenden de éste
			\end{itemize} \\
			\hline 
		\end{tabular}
		} \\
		\textbf{Fuente}: Autores
	\end{center}
\end{table}

\begin{table}[!htb]
	\caption{CU103-Actualizar album de fotos: Flujos de hechos }
	\label{tab:cu103_flujo}
	\begin{center}
		\resizebox{15cm}{!}{
		\begin{tabular}{|p{1.5cm}|p{6cm}|p{6.5cm}|}
			\hline
			\multicolumn{3}{|c|}{Detalle de flujo de hechos de caso de uso} \\
			\hline
			Nombre & \multicolumn{2}{|c|}{Nombre del flujo} \\
			\hline
			Paso & Acción del actor & Respuesta del sistema \\
			\hline
			1 & El usuario ha elegido gestionar sus albumes de fotos & El sistema ha mostrado la interfaz de gestión de albumes \\
			\hline
			2 & El usuario busca un album en específico & El sistema arroja los resultados de la búsqueda \\
			\hline
			3 & El usuario elige un album de los resultados arrojados & El sistema muestra la interfaz de actualización de album \\
			\hline
			4 & El usuario hace los cambios que necesite en el album y guarda la información & El sistema guarda los cambios \\
			\hline
			5 & & El sistema muestra un elemento emergente informando del éxito o fracaso de la operación \\
			\hline
			6 & El usuario continua & El sistema muestra la interfaz de gestión de albumes de fotos \\
			\hline
		\end{tabular}
		} \\
		\textbf{Fuente}: Autores
	\end{center}
\end{table}

\begin{table}[!htb]
	\caption{CU104-Gestionar fotos de album: Descripción}
	\label{tab:cu104_desc}
	\begin{center}
		\resizebox{15cm}{!}{
		\begin{tabular}{|p{4cm}|p{11cm}|}
			\hline
			\multicolumn{2}{|c|}{Descripción de caso de uso} \\
			\hline
			Nombre & Gestionar fotos de album\\
			\hline
			Identificador & CU104 \\
			\hline
			Descripción & Permite la gestión de las fotos del album que ha sido elegido \\
			\hline
			Actor & 
			\begin{itemize}
				\item Todos
			\end{itemize} \\
			\hline
			Disparador & El actor desea gestionar las fotos del album elegido \\
			\hline
			Inclusiones & \\
			\hline
			Puntos de extensión & \\
			\hline
			Precondiciones & \\
			\hline
			Postcondiciones & \\
			\hline
			Notas & 
			\begin{itemize}
				\item Soporta el uso tanto en los actores deportivos como en los eventos. Esta característica es desencadenada por todos los casos de uso que se desprenden de éste
				\item Generalización de:
				\begin{itemize}
					\item Subir fotos a album
					\item Eliminar foto de album
					\item Ver foto
					\item Actualizar foto
				\end{itemize}
			\end{itemize} \\
			\hline 
		\end{tabular}
		} \\
		\textbf{Fuente}: Autores
	\end{center}
\end{table}

\begin{table}[!htb]
	\caption{CU104-Gestionar fotos de album: Flujos de hechos }
	\label{tab:cu104_flujo}
	\begin{center}
		\resizebox{15cm}{!}{
		\begin{tabular}{|p{1.5cm}|p{6cm}|p{6.5cm}|}
			\hline
			\multicolumn{3}{|c|}{Detalle de flujo de hechos de caso de uso} \\
			\hline
			Nombre & \multicolumn{2}{|c|}{Nombre del flujo} \\
			\hline
			Paso & Acción del actor & Respuesta del sistema \\
			\hline
			1 & El usuario ha elegido un album en específico & El sistema ha mostrado los detalles del album y, a su vez, una lista de las fotos \\
			\hline
		\end{tabular}
		} \\
		\textbf{Fuente}: Autores
	\end{center}
\end{table}

\begin{table}[!htb]
	\caption{CU105-Subir fotos a album: Descripción}
	\label{tab:cu105_desc}
	\begin{center}
		\resizebox{15cm}{!}{
		\begin{tabular}{|p{4cm}|p{11cm}|}
			\hline
			\multicolumn{2}{|c|}{Descripción de caso de uso} \\
			\hline
			Nombre & Subir fotos a album \\
			\hline
			Identificador & CU105 \\
			\hline
			Descripción & Permite subir fotos a algún album en creación o ya creado \\
			\hline
			Actor & 
			\begin{itemize}
				\item Todos
			\end{itemize} \\
			\hline
			Disparador & El actor desea subir fotos a algún album en creación o ya creado \\
			\hline
			Inclusiones & \\
			\hline
			Puntos de extensión & \\
			\hline
			Precondiciones & \\
			\hline
			Postcondiciones &
			\begin{itemize}
				\item El actor sube la cantidad de fotos que él requiera al album elegido
			\end{itemize}
			\\
			\hline
			Notas & 
			\begin{itemize}
				\item Soporta el uso tanto en los actores deportivos como en los eventos. Esta característica es desencadenada por todos los casos de uso que se desprenden de éste
			\end{itemize} \\
			\hline 
		\end{tabular}
		} \\
		\textbf{Fuente}: Autores
	\end{center}
\end{table}

\begin{table}[!htb]
	\caption{CU105-Subir fotos a album: Flujos de hechos }
	\label{tab:cu105_flujo}
	\begin{center}
		\resizebox{15cm}{!}{
		\begin{tabular}{|p{1.5cm}|p{6cm}|p{6.5cm}|}
			\hline
			\multicolumn{3}{|c|}{Detalle de flujo de hechos de caso de uso} \\
			\hline
			Nombre & \multicolumn{2}{|c|}{Nombre del flujo} \\
			\hline
			Paso & Acción del actor & Respuesta del sistema \\
			\hline
			1 & El usuario ha elegido gestionar las fotos del album & El sistema ha mostrado las fotos en el album \\
			\hline
			2 & El usuario ha elegido subir una foto & El sistema ha mostrado la interfaz para la subida de fotos \\
			\hline
			3 & El usuario busca, sube una foto desde su dispositivo móvil y guarda cambios & El sistema sube la foto y la adiciona a la vista actual de fotos \\
			\hline
			4 & & El sistema muestra un elemento emergente informando del éxito o fracaso de la operación \\
			\hline
			5 & El usuario continua & El sistema muestra la interfaz de gestión de fotos de albumes \\
			\hline
		\end{tabular}
		} \\
		\textbf{Fuente}: Autores
	\end{center}
\end{table}

\begin{table}[!htb]
	\caption{CU106-Eliminar foto de album: Descripción}
	\label{tab:cu106_desc}
	\begin{center}
		\resizebox{15cm}{!}{
		\begin{tabular}{|p{4cm}|p{11cm}|}
			\hline
			\multicolumn{2}{|c|}{Descripción de caso de uso} \\
			\hline
			Nombre & Eliminar foto de album \\
			\hline
			Identificador & CU106 \\
			\hline
			Descripción & Permite eliminar una foto de algún album en creación o ya creado \\
			\hline
			Actor & 
			\begin{itemize}
				\item Todos
			\end{itemize} \\
			\hline
			Disparador & El actor desea eliminar una foto de algún album en creación o ya creado \\
			\hline
			Inclusiones & \\
			\hline
			Puntos de extensión & \\
			\hline
			Precondiciones & \\
			\hline
			Postcondiciones &
			\begin{itemize}
				\item El actor elimina la foto que él requiere del album elegido
			\end{itemize}
			\\
			\hline
			Notas & 
			\begin{itemize}
				\item Soporta el uso tanto en los actores deportivos como en los eventos. Esta característica es desencadenada por todos los casos de uso que se desprenden de éste
			\end{itemize} \\
			\hline 
		\end{tabular}
		} \\
		\textbf{Fuente}: Autores
	\end{center}
\end{table}

\begin{table}[!htb]
	\caption{CU106-Eliminar foto de album: Flujos de hechos }
	\label{tab:cu106_flujo}
	\begin{center}
		\resizebox{15cm}{!}{
		\begin{tabular}{|p{1.5cm}|p{6cm}|p{6.5cm}|}
			\hline
			\multicolumn{3}{|c|}{Detalle de flujo de hechos de caso de uso} \\
			\hline
			Nombre & \multicolumn{2}{|c|}{Nombre del flujo} \\
			\hline
			Paso & Acción del actor & Respuesta del sistema \\
			\hline
			1 & El usuario ha elegido gestionar las fotos de una album & El sistema ha mostrado las fotos presentes en el album \\
			\hline
			2 & El usuario elige borrar una foto del album & El sistema borra la foto del album \\
			\hline
			3 & & El sistema muestra un elemento emergente informando del éxito o fracaso de la operación \\
			\hline
			4 & El usuario continua & El sistema muestra la interfaz de gestión de fotos de albumes \\
			\hline
		\end{tabular}
		} \\
		\textbf{Fuente}: Autores
	\end{center}
\end{table}

\begin{table}[!htb]
	\caption{CU107-Ver foto: Descripción}
	\label{tab:cu107_desc}
	\begin{center}
		\resizebox{15cm}{!}{
		\begin{tabular}{|p{4cm}|p{11cm}|}
			\hline
			\multicolumn{2}{|c|}{Descripción de caso de uso} \\
			\hline
			Nombre & Ver foto \\
			\hline
			Identificador & CU107 \\
			\hline
			Descripción & Permite ver una foto de algún album elegido \\
			\hline
			Actor & 
			\begin{itemize}
				\item Todos
			\end{itemize} \\
			\hline
			Disparador & El actor desea ver una foto de algún album elegido \\
			\hline
			Inclusiones & \\
			\hline
			Puntos de extensión & \\
			\hline
			Precondiciones & \\
			\hline
			Postcondiciones & \\
			\hline
			Notas & 
			\begin{itemize}
				\item Soporta el uso tanto en los actores deportivos como en los eventos. Esta característica es desencadenada por todos los casos de uso que se desprenden de éste
				\item Generalización de:
				\begin{itemize}
					\item Actualizar foto
				\end{itemize}
			\end{itemize} \\
			\hline 
		\end{tabular}
		} \\
		\textbf{Fuente}: Autores
	\end{center}
\end{table}
\clearpage
\begin{table}[!htb]
	\caption{CU107-Ver foto: Flujos de hechos }
	\label{tab:cu107_flujo}
	\begin{center}
		\resizebox{15cm}{!}{
		\begin{tabular}{|p{1.5cm}|p{6cm}|p{6.5cm}|}
			\hline
			\multicolumn{3}{|c|}{Detalle de flujo de hechos de caso de uso} \\
			\hline
			Nombre & \multicolumn{2}{|c|}{Nombre del flujo} \\
			\hline
			Paso & Acción del actor & Respuesta del sistema \\
			\hline
			1 & El usuario ha elegido un album de fotos en específico & El sistema ha mostrado una lista de fotos del album elegido \\
			\hline
			2 & El usuario elige una de las fotos de la lista & El sistema muestra la imagen en pantalla completa \\
			\hline
		\end{tabular}
		} \\
		\textbf{Fuente}: Autores
	\end{center}
\end{table}

\begin{table}[!htb]
	\caption{CU108-Actualizar foto: Descripción}
	\label{tab:cu001_desc}
	\begin{center}
		\resizebox{15cm}{!}{
		\begin{tabular}{|p{4cm}|p{11cm}|}
			\hline
			\multicolumn{2}{|c|}{Descripción de caso de uso} \\
			\hline
			Nombre & Actualizar foto \\
			\hline
			Identificador & CU108 \\
			\hline
			Descripción & Permite actualizar la información de una foto de un album \\
			\hline
			Actor & 
			\begin{itemize}
				\item Todos
			\end{itemize} \\
			\hline
			Disparador & El actor desea actualizar la información de una foto de un album \\
			\hline
			Inclusiones & \\
			\hline
			Puntos de extensión & \\
			\hline
			Precondiciones & \\
			\hline
			Postcondiciones &
			\begin{itemize}
				\item El actor actualiza la información de una foto de un album
			\end{itemize}
			\\
			\hline
			Notas & 
			\begin{itemize}
				\item Soporta el uso tanto en los actores deportivos como en los eventos. Esta característica es desencadenada por todos los casos de uso que se desprenden de éste
			\end{itemize} \\
			\hline 
		\end{tabular}
		} \\
		\textbf{Fuente}: Autores
	\end{center}
\end{table}

\begin{table}[!htb]
	\caption{CU108-Actualizar foto: Flujos de hechos }
	\label{tab:cu108_flujo}
	\begin{center}
		\resizebox{15cm}{!}{
		\begin{tabular}{|p{1.5cm}|p{6cm}|p{6.5cm}|}
			\hline
			\multicolumn{3}{|c|}{Detalle de flujo de hechos de caso de uso} \\
			\hline
			Nombre & \multicolumn{2}{|c|}{Nombre del flujo} \\
			\hline
			Paso & Acción del actor & Respuesta del sistema \\
			\hline
			1 & El usuario ha elegido una foto de entre la lista de fotos de un album previamente elegido & El sistema ha mostrado la foto en pantalla completa \\
			\hline
			2 & El usuario cambia detalles de la foto y actualiza & El sistema guarda los cambios \\
			\hline
			3 & & El sistema muestra un elemento emergente informando del éxito o fracaso de la operación \\
			\hline
			4 & El usuario continua & El sistema muestra la interfaz de la foto en pantalla completa \\
			\hline
		\end{tabular}
		} \\
		\textbf{Fuente}: Autores
	\end{center}
\end{table}

\begin{table}[!htb]
	\caption{CU109-Gestionar videos: Descripción}
	\label{tab:cu109_desc}
	\begin{center}
		\resizebox{15cm}{!}{
		\begin{tabular}{|p{4cm}|p{11cm}|}
			\hline
			\multicolumn{2}{|c|}{Descripción de caso de uso} \\
			\hline
			Nombre & Gestionar videos \\
			\hline
			Identificador & CU109 \\
			\hline
			Descripción & Permite al usuario gestionar videos suyos o de cualquier organización/grupo/equipo o evento que él administre o sobre los que tenga dicho privilegio \\
			\hline
			Actor & 
			\begin{itemize}
				\item Todos
			\end{itemize} \\
			\hline
			Disparador & El actor desea gestionar los videos suyos o de cualquier organización/grupo/equipo o evento que él administre o sobre los que tenga dicho privilegio \\
			\hline
			Inclusiones & \\
			\hline
			Puntos de extensión & \\
			\hline
			Precondiciones & \\
			\hline
			Postcondiciones & \\
			\hline
			Notas & 
			\begin{itemize}
				\item Soporta el uso tanto en los actores deportivos como en los eventos. Esta característica es desencadenada por todos los casos de uso que se desprenden de éste
				\item Generalizacion de:
				\begin{itemize}
					\item Reproducir un video
					\item Modificar detalles de video
					\item Subir un video
					\item Eliminar un video
				\end{itemize}
			\end{itemize} \\
			\hline 
		\end{tabular}
		} \\
		\textbf{Fuente}: Autores
	\end{center}
\end{table}

\begin{table}[!htb]
	\caption{CU109-Gestionar videos: Flujos de hechos }
	\label{tab:cu109_flujo}
	\begin{center}
		\resizebox{15cm}{!}{
		\begin{tabular}{|p{1.5cm}|p{6cm}|p{6.5cm}|}
			\hline
			\multicolumn{3}{|c|}{Detalle de flujo de hechos de caso de uso} \\
			\hline
			Nombre & \multicolumn{2}{|c|}{Nombre del flujo} \\
			\hline
			Paso & Acción del actor & Respuesta del sistema \\
			\hline
			1 & El usuario elige gestionar sus videos  & El sistema muestra la interfaz de gestión de videos \\
			\hline
		\end{tabular}
		} \\
		\textbf{Fuente}: Autores
	\end{center}
\end{table}

\begin{table}[!htb]
	\caption{CU110-Reproducir un video: Descripción}
	\label{tab:cu110_desc}
	\begin{center}
		\resizebox{15cm}{!}{
		\begin{tabular}{|p{4cm}|p{11cm}|}
			\hline
			\multicolumn{2}{|c|}{Descripción de caso de uso} \\
			\hline
			Nombre & Reproducir un video \\
			\hline
			Identificador & CU110 \\
			\hline
			Descripción & Permite la reproducción de un video subido al SNS \\
			\hline
			Actor & 
			\begin{itemize}
				\item Todos
			\end{itemize} \\
			\hline
			Disparador & El actor desea reproducir un video subido al SNS \\
			\hline
			Inclusiones & \\
			\hline
			Puntos de extensión & \\
			\hline
			Precondiciones & 
			\begin{itemize}
				\item El actor tiene privilegios para reproducir el video
			\end{itemize}
			\\
			\hline
			Postcondiciones & 						
			\\
			\hline
			Notas & 
			\begin{itemize}
				\item Soporta el uso tanto en los actores deportivos como en los eventos. Esta característica es desencadenada por todos los casos de uso que se desprenden de éste
			\end{itemize} \\
			\hline 
		\end{tabular}
		} \\
		\textbf{Fuente}: Autores
	\end{center}
\end{table}

\begin{table}[!htb]
	\caption{CU110-Reproducir un video: Flujos de hechos }
	\label{tab:cu110_flujo}
	\begin{center}
		\resizebox{15cm}{!}{
		\begin{tabular}{|p{1.5cm}|p{6cm}|p{6.5cm}|}
			\hline
			\multicolumn{3}{|c|}{Detalle de flujo de hechos de caso de uso} \\
			\hline
			Nombre & \multicolumn{2}{|c|}{Nombre del flujo} \\
			\hline
			Paso & Acción del actor & Respuesta del sistema \\
			\hline
			1 & El usuario ha elegido la gestión de sus videos & El sistema ha mostrado la interfaz de gestión de videos \\
			\hline
			2 & El usuario busca un video en específico & El sistema muestra el resultado de la búsqueda \\
			\hline
			3 & El usuario elige un video entre los resultados de la búsqueda & El sistema muestra el video \\
			\hline
			4 & El usuario elige reproducir el video & El sistema reproduce el video \\
			\hline
		\end{tabular}
		} \\
		\textbf{Fuente}: Autores
	\end{center}
\end{table}

\begin{table}[!htb]
	\caption{CU111-Subir un video: Descripción}
	\label{tab:cu111_desc}
	\begin{center}
		\resizebox{15cm}{!}{
		\begin{tabular}{|p{4cm}|p{11cm}|}
			\hline
			\multicolumn{2}{|c|}{Descripción de caso de uso} \\
			\hline
			Nombre & Subir un video \\
			\hline
			Identificador & CU111 \\
			\hline
			Descripción & Permite la subida de un video al SNS \\
			\hline
			Actor & 
			\begin{itemize}
				\item Todos
			\end{itemize} \\
			\hline
			Disparador & El actor desea subir un video al SNS \\
			\hline
			Inclusiones & \\
			\hline
			Puntos de extensión & \\
			\hline
			Precondiciones & 
			\begin{itemize}
				\item El actor tiene privilegios para subir el video
			\end{itemize}
			\\
			\hline
			Postcondiciones &
			\begin{itemize}
				\item El actor sube un video al SNS
			\end{itemize}			 						
			\\
			\hline
			Notas & 
			\begin{itemize}
				\item Soporta el uso tanto en los actores deportivos como en los eventos. Esta característica es desencadenada por todos los casos de uso que se desprenden de éste
			\end{itemize} \\
			\hline 
		\end{tabular}
		} \\
		\textbf{Fuente}: Autores
	\end{center}
\end{table}

\begin{table}[!htb]
	\caption{CU111-Subir un video: Flujos de hechos }
	\label{tab:cu111_flujo}
	\begin{center}
		\resizebox{15cm}{!}{
		\begin{tabular}{|p{1.5cm}|p{6cm}|p{6.5cm}|}
			\hline
			\multicolumn{3}{|c|}{Detalle de flujo de hechos de caso de uso} \\
			\hline
			Nombre & \multicolumn{2}{|c|}{Nombre del flujo} \\
			\hline
			Paso & Acción del actor & Respuesta del sistema \\
			\hline
			1 & El usuario ha elegido la gestión de sus videos & El sistema ha mostrado la interfaz de gestión de videos \\
			\hline
			2 & El usuario elige subir un video & El sistema muestra la interfaz de subida de videos \\
			\hline
			3 & El usuario ingresa los datos del video, sube el video y guarda los cambios & El sistema guarda los cambios \\
			\hline
			4 & & El sistema muestra un elemento emergente informando del éxito o fracaso de la operación \\
			\hline
			5 & El usuario continua & El sistema muestra la interfaz de gestión de videos \\
			\hline
		\end{tabular}
		} \\
		\textbf{Fuente}: Autores
	\end{center}
\end{table}

\begin{table}[!htb]
	\caption{CU112-Eliminar un video: Descripción}
	\label{tab:cu112_desc}
	\begin{center}
		\resizebox{15cm}{!}{
		\begin{tabular}{|p{4cm}|p{11cm}|}
			\hline
			\multicolumn{2}{|c|}{Descripción de caso de uso} \\
			\hline
			Nombre & Eliminar un video \\
			\hline
			Identificador & CU \\
			\hline
			Descripción & Permite la eliminación de un video subido al SNS \\
			\hline
			Actor & 
			\begin{itemize}
				\item Todos
			\end{itemize} \\
			\hline
			Disparador & El actor desea eliminar un video subido al SNS \\
			\hline
			Inclusiones & \\
			\hline
			Puntos de extensión & \\
			\hline
			Precondiciones & 
			\begin{itemize}
				\item El actor tiene privilegios para eliminar el video
			\end{itemize}
			\\
			\hline
			Postcondiciones &
			\begin{itemize}
				\item El actor elimina un video del SNS
			\end{itemize}			 						
			\\
			\hline
			Notas & 
			\begin{itemize}
				\item Soporta el uso tanto en los actores deportivos como en los eventos. Esta característica es desencadenada por todos los casos de uso que se desprenden de éste
			\end{itemize} \\
			\hline 
		\end{tabular}
		} \\
		\textbf{Fuente}: Autores
	\end{center}
\end{table}

\begin{table}[!htb]
	\caption{CU112-Eliminar un video: Flujos de hechos }
	\label{tab:cu112_flujo}
	\begin{center}
		\resizebox{15cm}{!}{
		\begin{tabular}{|p{1.5cm}|p{6cm}|p{6.5cm}|}
			\hline
			\multicolumn{3}{|c|}{Detalle de flujo de hechos de caso de uso} \\
			\hline
			Nombre & \multicolumn{2}{|c|}{Nombre del flujo} \\
			\hline
			Paso & Acción del actor & Respuesta del sistema \\
			\hline
			1 & El usuario ha elegido la gestión de sus videos & El sistema ha mostrado la interfaz de gestión de videos \\
			2 & El usuario busca un video en específico & El sistema arroja los resultados de la búsqueda \\
			\hline
			3 & El usuario elige eliminar un video & El sistema guarda los cambios\\
			\hline
			4 & & El sistema muestra un elemento emergente informando del éxito o fracaso de la operación \\
			\hline
			5 & El usuario continua & El sistema muestra la interfaz de gestión de videos \\
			\hline
		\end{tabular}
		} \\
		\textbf{Fuente}: Autores
	\end{center}
\end{table}


\clearpage

\begin{table}[!htb]
	\caption{CU113-Modificar detalles de un video: Descripción}
	\label{tab:cu113_desc}
	\begin{center}
		\resizebox{15cm}{!}{
		\begin{tabular}{|p{4cm}|p{11cm}|}
			\hline
			\multicolumn{2}{|c|}{Descripción de caso de uso} \\
			\hline
			Nombre & Modificar detalles de un video \\
			\hline
			Identificador & CU113 \\
			\hline
			Descripción & Permite la modificación de los detalles de un video subido al SNS \\
			\hline
			Actor & 
			\begin{itemize}
				\item Todos
			\end{itemize} \\
			\hline
			Disparador & El actor desea modificar los detalles de un video subido al SNS \\
			\hline
			Inclusiones & \\
			\hline
			Puntos de extensión & \\
			\hline
			Precondiciones & \\
			\hline
			Postcondiciones &
			\begin{itemize}
				\item El actor modifica los detalles de un video del subido por él al SNS
			\end{itemize}			 						
			\\
			\hline
			Notas & 
			\begin{itemize}
				\item Soporta el uso tanto en los actores deportivos como en los eventos. Esta característica es desencadenada por todos los casos de uso que se desprenden de éste
			\end{itemize} \\
			\hline 
		\end{tabular}
		} \\
		\textbf{Fuente}: Autores
	\end{center}
\end{table}

\begin{table}[!htb]
	\caption{CU113-Modificar detalles de un video: Flujos de hechos }
	\label{tab:cu001_flujo}
	\begin{center}
		\resizebox{15cm}{!}{
		\begin{tabular}{|p{1.5cm}|p{6cm}|p{6.5cm}|}
			\hline
			\multicolumn{3}{|c|}{Detalle de flujo de hechos de caso de uso} \\
			\hline
			Nombre & \multicolumn{2}{|c|}{Nombre del flujo} \\
			\hline
			Paso & Acción del actor & Respuesta del sistema \\
			\hline
			1 & El usuario ha elegido la gestión de sus videos & El sistema ha mostrado la interfaz de gestión de videos \\
			2 & El usuario busca un video en específico & El sistema arroja los resultados de la búsqueda \\
			\hline
			3 & El usuario pulsa en el video & El sistema muestra los detalles del video \\
			\hline
			4 & El usuario modifica los detalles del video y guarda & El sistema guarda cambios \\
			\hline
			5 & & El sistema muestra un elemento emergente informando del éxito o fracaso de la operación \\
			\hline
			6 & El usuario continua & El sistema muestra la interfaz de gestión de videos \\
			\hline
		\end{tabular}
		} \\
		\textbf{Fuente}: Autores
	\end{center}
\end{table}

\begin{table}[!htb]
	\caption{CU114-Gestionar mensajería instantánea: Descripción}
	\label{tab:cu102_desc}
	\begin{center}
		\resizebox{15cm}{!}{
		\begin{tabular}{|p{4cm}|p{11cm}|}
			\hline
			\multicolumn{2}{|c|}{Descripción de caso de uso} \\
			\hline
			Nombre & Gestionar mensajería instantánea \\
			\hline
			Identificador & CU114 \\
			\hline
			Descripción & La gestión de un canal de mensajería instantánea para cada actor/evento de la red social \\
			\hline
			Actor & 
			\begin{itemize}
				\item Todos
			\end{itemize} \\
			\hline
			Disparador & El actor desea gestionar mensajería instantánea \\
			\hline
			Inclusiones & \\
			\hline
			Puntos de extensión & \\
			\hline
			Precondiciones &  \\
			\hline
			Postcondiciones & \\
			\hline
			Notas & 
			\begin{itemize}
				\item Soporta el uso tanto en los actores deportivos como en los eventos. Esta característica es desencadenada por todos los casos de uso que se desprenden de éste
				\item Generalización de:
				\begin{itemize}
					\item Crear conversación
					\item Responder conversación
				\end{itemize}
			\end{itemize} \\
			\hline 
		\end{tabular}
		} \\
		\textbf{Fuente}: Autores
	\end{center}
\end{table}

\begin{table}[!htb]
	\caption{CU114-Gestionar mensajería instantánea: Flujos de hechos }
	\label{tab:cu102_flujo}
	\begin{center}
		\resizebox{15cm}{!}{
		\begin{tabular}{|p{1.5cm}|p{6cm}|p{6.5cm}|}
			\hline
			\multicolumn{3}{|c|}{Detalle de flujo de hechos de caso de uso} \\
			\hline
			Nombre & \multicolumn{2}{|c|}{Nombre del flujo} \\
			\hline
			Paso & Acción del actor & Respuesta del sistema \\
			\hline
			1 & El usuario elige gestionar su mensajería instantánea & El sistema muestra la interfaz de búsqueda de actores/eventos con los que puede entablar mensaje el usuario \\
			\hline
		\end{tabular}
		} \\
		\textbf{Fuente}: Autores
	\end{center}
\end{table}

\begin{table}[!htb]
	\caption{CU115-Crear conversación: Descripción}
	\label{tab:cu102_desc}
	\begin{center}
		\resizebox{15cm}{!}{
		\begin{tabular}{|p{4cm}|p{11cm}|}
			\hline
			\multicolumn{2}{|c|}{Descripción de caso de uso} \\
			\hline
			Nombre & Crear conversación \\
			\hline
			Identificador & CU115 \\
			\hline
			Descripción & Permite la creación de un nuevo chat con un actor/evento \\
			\hline
			Actor & 
			\begin{itemize}
				\item Todos
			\end{itemize} \\
			\hline
			Disparador & El actor desea crear un nuevo chat con un actor/evento \\
			\hline
			Inclusiones & \\
			\hline
			Puntos de extensión & \\
			\hline
			Precondiciones &  \\
			\hline
			Postcondiciones & 
			\begin{itemize}
				\item El actor crea un chat con un actor/evento
			\end{itemize} \\
			\hline
			Notas & 
			\begin{itemize}
				\item Soporta el uso tanto en los actores deportivos como en los eventos. Esta característica es desencadenada por todos los casos de uso que se desprenden de éste
			\end{itemize} \\
			\hline 
		\end{tabular}
		} \\
		\textbf{Fuente}: Autores
	\end{center}
\end{table}

\begin{table}[!htb]
	\caption{CU115-Crear conversación: Flujos de hechos }
	\label{tab:cu102_flujo}
	\begin{center}
		\resizebox{15cm}{!}{
		\begin{tabular}{|p{1.5cm}|p{6cm}|p{6.5cm}|}
			\hline
			\multicolumn{3}{|c|}{Detalle de flujo de hechos de caso de uso} \\
			\hline
			Nombre & \multicolumn{2}{|c|}{Nombre del flujo} \\
			\hline
			Paso & Acción del actor & Respuesta del sistema \\
			\hline
			1 & El usuario ha elegido gestionar su mensajería instantánea & El sistema ha mostrado la interfaz de búsqueda de actores/eventos con los que puede entablar mensaje el usuario \\
			\hline
			2 & El usuario elige crear una conversación & El sistema muestra la interfaz de búsqueda de usuario sobre la red social para iniciar chat \\
			\hline
			3 & El usuario busca otro o un evento en la red social & El sistema arroja los resultados  \\
			\hline
			4 & El usuario elige uno de los resultados de la búsqueda & El sistema crea el chat \\
			\hline
			4 & & El sistema muestra un elemento emergente informando del éxito o fracaso de la operación \\
			\hline
			5 & El usuario continua & El sistema muestra la interfaz de chat \\
			\hline
		\end{tabular}
		} \\
		\textbf{Fuente}: Autores
	\end{center}
\end{table}

\begin{table}[!htb]
	\caption{CU116-Responder conversación: Descripción}
	\label{tab:cu102_desc}
	\begin{center}
		\resizebox{15cm}{!}{
		\begin{tabular}{|p{4cm}|p{11cm}|}
			\hline
			\multicolumn{2}{|c|}{Descripción de caso de uso} \\
			\hline
			Nombre & Eliminar album \\
			\hline
			Identificador & CU102 \\
			\hline
			Descripción & Permite responder/escribir las conversaciones en chats \\
			\hline
			Actor & 
			\begin{itemize}
				\item Todos
			\end{itemize} \\
			\hline
			Disparador & El actor desea contestar/escribir un mensaje \\
			\hline
			Inclusiones & \\
			\hline
			Puntos de extensión & \\
			\hline
			Precondiciones &  \\
			\hline
			Postcondiciones & 
			\begin{itemize}
				\item El actor contesta/escribe un mensaje
			\end{itemize} \\
			\hline
			Notas & 
			\begin{itemize}
				\item Soporta el uso tanto en los actores deportivos como en los eventos. Esta característica es desencadenada por todos los casos de uso que se desprenden de éste
			\end{itemize} \\
			\hline 
		\end{tabular}
		} \\
		\textbf{Fuente}: Autores
	\end{center}
\end{table}

\begin{table}[!htb]
	\caption{CU102-Eliminar album: Flujos de hechos }
	\label{tab:cu102_flujo}
	\begin{center}
		\resizebox{15cm}{!}{
		\begin{tabular}{|p{1.5cm}|p{6cm}|p{6.5cm}|}
			\hline
			\multicolumn{3}{|c|}{Detalle de flujo de hechos de caso de uso} \\
			\hline
			Nombre & \multicolumn{2}{|c|}{Nombre del flujo} \\
			\hline
			Paso & Acción del actor & Respuesta del sistema \\
			\hline
			1 & El usuario ha elegido gestionar su mensajería instantánea & El sistema ha mostrado la interfaz de búsqueda de actores/eventos con los que puede entablar mensaje el usuario \\
			\hline
			2 & El usuario busca otro o un evento en las conversaciones & El sistema arroja los resultados  \\
			\hline
			3 & El usuario elige uno de los resultados de la búsqueda & El sistema muestra el chat con los mensajes que le han llegado al usuario, así como los que él ha enviado \\
			\hline
			4 & El usuario responde/escribe al otro usuario/evento en la conversación & El sistema envía el mensaje \\
			\hline
		\end{tabular}
		} \\
		\textbf{Fuente}: Autores
	\end{center}
\end{table}

\clearpage