\begin{table}[!htb]
	\caption{CU097-Comentar un post: Descripción}
	\label{tab:cu097_desc}
	\begin{center}
		\resizebox{15cm}{!}{
		\begin{tabular}{|p{4cm}|p{11cm}|}
			\hline
			\multicolumn{2}{|c|}{Descripción de caso de uso} \\
			\hline
			Nombre & Comentar un post \\
			\hline
			Identificador & CU097 \\
			\hline
			Descripción & Permite comentar un post creado sobre un timeline \\
			\hline
			Actor & 
			\begin{itemize}
				\item Todos
			\end{itemize} \\
			\hline
			Disparador & El actor desea comentar un post creado sobre un timeline propio o ageno, según sus privilegios se lo permitan \\
			\hline
			Inclusiones & \\
			\hline
			Puntos de extensión & \\
			\hline
			Precondiciones & 
			\begin{itemize}
				\item El actor tiene privilegios para comentar el post creado
			\end{itemize} \\
			\hline
			Postcondiciones &
			\begin{itemize}
				\item El actor comenta un post creado sobre el timeline elegido
			\end{itemize} \\
			\hline
			Notas & 
			\begin{itemize}
				\item Soporta el uso tanto en los actores deportivos como en los eventos. Esta característica es desencadenada por todos los casos de uso que se desprenden de éste
			\end{itemize} \\
			\hline 
		\end{tabular}
		} \\
		\textbf{Fuente}: Autores
	\end{center}
\end{table}

\begin{table}[!htb]
	\caption{CU097-Comentar un post: Flujos de hechos }
	\label{tab:cu097_flujo}
	\begin{center}
		\resizebox{15cm}{!}{
		\begin{tabular}{|p{1.5cm}|p{6cm}|p{6.5cm}|}
			\hline
			\multicolumn{3}{|c|}{Detalle de flujo de hechos de caso de uso} \\
			\hline
			Nombre & \multicolumn{2}{|c|}{Nombre del flujo} \\
			\hline
			Paso & Acción del actor & Respuesta del sistema \\
			\hline
			1 & El usuario ha elegido gestionar timeline & El sistema ha mostrado el timeline propio o de otro actor/evento en la red social \\
			\hline
			2 & El usuario se ubica en un post y elige comentar acerca de él. Guarda los cambios & El sistema guarda el comentario en el sistema y lo publica en el post \\
			\hline
			3 & & De haber un error, el sistema mostrará un mensaje indicándolo \\
			\hline
			4 & & El sistema ubica al usuario en el post elegido \\
			\hline
		\end{tabular}
		} \\
		\textbf{Fuente}: Autores
	\end{center}
\end{table}