\begin{table}[!htb]
	\caption{CU095-Gestionar timeline: Descripción}
	\label{tab:cu095_desc}
	\begin{center}
		\resizebox{15cm}{!}{
		\begin{tabular}{|p{4cm}|p{11cm}|}
			\hline
			\multicolumn{2}{|c|}{Descripción de caso de uso} \\
			\hline
			Nombre & Gestionar timeline \\
			\hline
			Identificador & CU095 \\
			\hline
			Descripción & Permite la gestión de un timeline para cada actor/evento en la red social \\
			\hline
			Actor & 
			\begin{itemize}
				\item Todos
			\end{itemize} \\
			\hline
			Disparador & El actor desea gestionar el timeline de él, una organización, un equipo/grupo deportivo o un evento deportivo \\
			\hline
			Inclusiones & \\
			\hline
			Puntos de extensión & \\
			\hline
			Precondiciones & \\
			\hline
			Postcondiciones & \\
			\hline
			Notas & 
			\begin{itemize}
				\item Soporta el uso tanto en los actores deportivos como en los eventos. Esta característica es desencadenada por todos los casos de uso que se desprenden de éste
				\item Generalización de:
				\begin{itemize}
					\item Crear post
					\item Comentar un post
					\item Eliminar un post
				\end{itemize}
				\item Si se desea ver el timeline de un actor/evento deportivo, se ha de buscar primero
			\end{itemize} \\
			\hline 
		\end{tabular}
		} \\
		\textbf{Fuente}: Autores
	\end{center}
\end{table}

\begin{table}[!htb]
	\caption{CU095-Gestionar timeline: Flujos de hechos }
	\label{tab:cu095_flujo}
	\begin{center}
		\resizebox{15cm}{!}{
		\begin{tabular}{|p{1.5cm}|p{6cm}|p{6.5cm}|}
			\hline
			\multicolumn{3}{|c|}{Detalle de flujo de hechos de caso de uso} \\
			\hline
			Nombre & \multicolumn{2}{|c|}{Nombre del flujo} \\
			\hline
			Paso & Acción del actor & Respuesta del sistema \\
			\hline
			1 & El usuario ha elegido gestionar self-sharing & El sistema ha mostrado las opciones de self-sharing que tiene el usuario a disposición \\
			\hline
			2 & El usuario elige gestionar su timeline, el timeline general o el timeline de un actor/evento deportivo & El sistema muestra el timeline elegido \\
			\hline
			3 & & \\
			\hline
		\end{tabular}
		} \\
		\textbf{Fuente}: Autores
	\end{center}
\end{table}