\begin{table}[!htb]
	\caption{CU114-Gestionar mensajería instantánea: Descripción}
	\label{tab:cu102_desc}
	\begin{center}
		\resizebox{15cm}{!}{
		\begin{tabular}{|p{4cm}|p{11cm}|}
			\hline
			\multicolumn{2}{|c|}{Descripción de caso de uso} \\
			\hline
			Nombre & Gestionar mensajería instantánea \\
			\hline
			Identificador & CU114 \\
			\hline
			Descripción & La gestión de un canal de mensajería instantánea para cada actor/evento de la red social \\
			\hline
			Actor & 
			\begin{itemize}
				\item Todos
			\end{itemize} \\
			\hline
			Disparador & El actor desea gestionar mensajería instantánea \\
			\hline
			Inclusiones & \\
			\hline
			Puntos de extensión & \\
			\hline
			Precondiciones &  \\
			\hline
			Postcondiciones & \\
			\hline
			Notas & 
			\begin{itemize}
				\item Soporta el uso tanto en los actores deportivos como en los eventos. Esta característica es desencadenada por todos los casos de uso que se desprenden de éste
				\item Generalización de:
				\begin{itemize}
					\item Crear conversación
					\item Responder conversación
				\end{itemize}
			\end{itemize} \\
			\hline 
		\end{tabular}
		} \\
		\textbf{Fuente}: Autores
	\end{center}
\end{table}

\begin{table}[!htb]
	\caption{CU114-Gestionar mensajería instantánea: Flujos de hechos }
	\label{tab:cu102_flujo}
	\begin{center}
		\resizebox{15cm}{!}{
		\begin{tabular}{|p{1.5cm}|p{6cm}|p{6.5cm}|}
			\hline
			\multicolumn{3}{|c|}{Detalle de flujo de hechos de caso de uso} \\
			\hline
			Nombre & \multicolumn{2}{|c|}{Nombre del flujo} \\
			\hline
			Paso & Acción del actor & Respuesta del sistema \\
			\hline
			1 & El usuario elige gestionar su mensajería instantánea & El sistema muestra la interfaz de búsqueda de actores/eventos con los que puede entablar mensaje el usuario \\
			\hline
		\end{tabular}
		} \\
		\textbf{Fuente}: Autores
	\end{center}
\end{table}