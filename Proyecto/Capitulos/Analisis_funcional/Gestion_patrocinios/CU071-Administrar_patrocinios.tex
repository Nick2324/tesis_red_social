\begin{table}[!htb]
	\caption{CU071-Administrar patrocinios: Descripción}
	\label{tab:cu071_desc}
	\begin{center}
		\resizebox{15cm}{!}{
		\begin{tabular}{|p{4cm}|p{11cm}|}
			\hline
			\multicolumn{2}{|c|}{Descripción de caso de uso} \\
			\hline
			Nombre & Administrar patrocinios \\
			\hline
			Identificador & CU071 \\
			\hline
			Descripción & Permite gestionar o administrar funcionalidades ofrecidas en lo que refiere a patrocinios en la red social deportiva \\
			\hline
			Actor &
			\begin{itemize}
				\item Patrocinador
				\item Equipo
				\item Jugador
			\end{itemize}
			\\
			\hline
			Disparador & El usuario autorizado desea manejar las funcionalidades ofrecidas en el módulo de patrocinios \\
			\hline
			Inclusiones &  \\
			\hline
			Puntos de extensión & 
			\\
			\hline
			Precondiciones &  
			\begin{itemize}
				\item Se ha iniciado el SNS con un rol de patrocinador, jugador o equipo deportivo
			\end{itemize}
			\\
			\hline
			Postcondiciones & 
			\begin{itemize}
				\item El usuario se encuentra en la pantalla para gestión de patrocinios
			\end{itemize}
			\\
			\hline
			Notas & 
			\begin{itemize}
				\item Generalización de:
				\begin{itemize}
					\item Gestionar patrocinios de patrocinador
					\item Gestionar patrocinios de equipo/jugador
				\end{itemize}
			\end{itemize}
			\\
			\hline
		\end{tabular}
		} \\
		\textbf{Fuente}: Autores
	\end{center}
\end{table}

\begin{table}[!htb]
	\caption{CU071-Administrar patrocinios: Flujos de hechos}
	\label{tab:cu071_flujo}
	\begin{center}
		\resizebox{15cm}{!}{
		\begin{tabular}{|p{1.5cm}|p{6cm}|p{6.5cm}|}
			\hline
			\multicolumn{3}{|c|}{Detalle de flujo de hechos de caso de uso} \\
			\hline
			Nombre & \multicolumn{2}{|c|}{Nombre del flujo} \\
			\hline
			Paso & Acción del actor & Respuesta del sistema \\
			\hline
			1 & El usuario elige administrar patrocinios deportivos & El sistema muestra la interfaz de patrocinio según el rol cargado: Si es un rol de patrocinador, se cargarán las funcionalidades de patrocinadores; si es un rol no-patrocinador, entonces se cargarán las funcionalidades restantes \\
			\hline
		\end{tabular}
		} \\
		\textbf{Fuente}: Autores
	\end{center}
\end{table}