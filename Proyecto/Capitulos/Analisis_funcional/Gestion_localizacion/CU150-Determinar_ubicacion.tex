\begin{table}[!htb]
	\caption{CU150-Determinar ubicación: Descripción}
	\label{tab:cu150_desc}
	\begin{center}
		\resizebox{15cm}{!}{
		\begin{tabular}{|p{4cm}|p{11cm}|}
			\hline
			\multicolumn{2}{|c|}{Descripción de caso de uso} \\
			\hline
			Nombre & Determinar ubicación\\
			\hline
			Identificador & CU150\\
			\hline
			Descripcion & Funcionalidad que le permite al usuario obtener su ubicación actual\\
			\hline
			Actor & 
			\begin{itemize}
				\item Actor deportivo.
			\end{itemize} \\
			\hline
			Disparador & El usuario necesita obtener su ubicación actual.\\
			\hline
			Inclusiones & \\
			\hline
			Extensiones & \\
			\hline
			Precondiciones & \\
			\hline
			Postcondiciones & \\
			\hline
			Notas & \\
			\hline
		\end{tabular}
		} \\
		\textbf{Fuente}: Autores
	\end{center}
\end{table}
\begin{table}[!htb]
	\caption{CU150-Determinar ubicación:Flujos de hechos}
	\label{tab:cu150_flujo}
	\begin{center}
		\resizebox{15cm}{!}{
		\begin{tabular}{|p{1.5cm}|p{6cm}|p{6.5cm}|}
			\hline
			\multicolumn{3}{|c|}{Descripción de caso de uso} \\
			\hline
			Nombre & \multicolumn{2}{|c|}{Nombre del flujo} \\
			\hline
			Paso & Acción del actor & Respuesta del sistema\\
			\hline
				1 & El usuario realiza una cción en la aplicación en la que sea posible utilizar su ubicación actual (Sube una foto, registra un nuevo evento, registra un logar de entrenamiento, etc.) & La aplicación obtiene los datos de su ubicación actual y los muestra en pantalla.\\
				2 & El usuarioestá de acuerdo con los datos mostrados y completa la acción que está realizando. & El sistema realiza la acción utilizando los datos obtenidos de la ubicación.\\
			\hline
		\end{tabular}
		} \\
		\textbf{Fuente}: Autores
	\end{center}
\end{table}
