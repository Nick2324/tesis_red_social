A continuación se muestran los casos de uso del módulo de administración de torneos

\begin{table}[!htb]
	\caption{CU027-Administrar torneos: Descripción}
	\label{tab:cu027_desc}
	\begin{center}
		\resizebox{15cm}{!}{
		\begin{tabular}{|p{4cm}|p{11cm}|}
			\hline
			\multicolumn{2}{|c|}{Descripción de caso de uso} \\
			\hline
			Nombre & Administrar torneos \\
			\hline
			Identificador & CU027 \\
			\hline
			Descripción & Permite la administración de torneos deportivos \\
			\hline
			Actor & Todo actor de la red social	 
			\\
			\hline
			Disparador & Se elige la opción de administrar torneos \\
			\hline
			Inclusiones & N/A \\
			\hline
			Puntos de extensión & N/A \\
			\hline
			Precondiciones &  
			\begin{itemize}
				\item La aplicación ha sido cargada por un actor con rol de organizador de eventos deportivos
			\end{itemize}
			\\
			\hline
			Postcondiciones & 
			\begin{itemize}
				\item El usuario está en la pantalla de administración de torneos
			\end{itemize}
			\\
			\hline
			Notas & 
			\begin{itemize}
				\item Generalización de:
				\begin{itemize}
					\item Crear torneo
					\item Actualizar información de torneo
					\item Gestionar formatos de torneo
					\item Agregar equipo a torneo
					\item Retirar equipo de torneo
				\end{itemize}
			\end{itemize}
			\\
			\hline
		\end{tabular}
		} \\
		\textbf{Fuente}: Autores
	\end{center}
\end{table}

\begin{table}[!htb]
	\caption{CU027-Administrar torneos: Flujos de hechos}
	\label{tab:cu027_flujo}
	\begin{center}
		\resizebox{15cm}{!}{
		\begin{tabular}{|p{1.5cm}|p{6cm}|p{6.5cm}|}
			\hline
			\multicolumn{3}{|c|}{Detalle de flujo de hechos de caso de uso} \\
			\hline
			Nombre & \multicolumn{2}{|c|}{Nombre del flujo} \\
			\hline
			Paso & Acción del actor & Respuesta del sistema \\
			\hline
			 & & \\
			\hline
		\end{tabular}
		} \\
		\textbf{Fuente}: Autores
	\end{center}
\end{table}

\begin{table}[!htb]
	\caption{CU028-Crear un torneo: Descripción}
	\label{tab:cu028_desc}
	\begin{center}
		\resizebox{15cm}{!}{
		\begin{tabular}{|p{4cm}|p{11cm}|}
			\hline
			\multicolumn{2}{|c|}{Descripción de caso de uso} \\
			\hline
			Nombre & Crear un torneo \\
			\hline
			Identificador & CU018 \\
			\hline
			Descripción & Permite la creación de un torneo \\
			\hline
			Actor & Todo actor de la red social	 
			\\
			\hline
			Disparador & Se elige la opción de crear un torneo \\
			\hline
			Inclusiones & N/A \\
			\hline
			Puntos de extensión & N/A \\
			\hline
			Precondiciones &  
			\begin{itemize}
				\item La aplicación ha sido cargada por un actor con rol de organizador de eventos deportivos
				\item Se ha elegido administrar torneo
			\end{itemize}
			\\
			\hline
			Postcondiciones & 
			\begin{itemize}
				\item El usuario ha creado o no un torneo
				\item El usuario está en la pantalla de administración de torneos
			\end{itemize}
			\\
			\hline
			Notas & N/A
			\\
			\hline
		\end{tabular}
		} \\
		\textbf{Fuente}: Autores
	\end{center}
\end{table}

\begin{table}[!htb]
	\caption{CU028-Crear un torneo: Flujos de hechos}
	\label{tab:cu028_flujo}
	\begin{center}
		\resizebox{15cm}{!}{
		\begin{tabular}{|p{1.5cm}|p{6cm}|p{6.5cm}|}
			\hline
			\multicolumn{3}{|c|}{Detalle de flujo de hechos de caso de uso} \\
			\hline
			Nombre & \multicolumn{2}{|c|}{Nombre del flujo} \\
			\hline
			Paso & Acción del actor & Respuesta del sistema \\
			\hline
			 & & \\
			\hline
		\end{tabular}
		} \\
		\textbf{Fuente}: Autores
	\end{center}
\end{table}

\begin{table}[!htb]
	\caption{CU029-Actualizar información de torneo: Descripción}
	\label{tab:cu029_desc}
	\begin{center}
		\resizebox{15cm}{!}{
		\begin{tabular}{|p{4cm}|p{11cm}|}
			\hline
			\multicolumn{2}{|c|}{Descripción de caso de uso} \\
			\hline
			Nombre & Actualizar información de torneo \\
			\hline
			Identificador & CU029 \\
			\hline
			Descripción & Actualiza la información de un torneo \\
			\hline
			Actor & Todo actor de la red social	 
			\\
			\hline
			Disparador & Se elige la opción de actualizar información de torneo \\
			\hline
			Inclusiones & N/A \\
			\hline
			Puntos de extensión & 
			\begin{itemize}
				\item Gestionar formato de torneo
			\end{itemize}	
			\\
			\hline
			Precondiciones &  
			\begin{itemize}
				\item La aplicación ha sido cargada por un actor con rol de organizador de eventos deportivos
				\item Se ha elegido administrar torneo
			\end{itemize}
			\\
			\hline
			Postcondiciones & 
			\begin{itemize}
				\item El usuario ha actualizado o no la información de un torneo
				\item El usuario está en la pantalla de administración de torneos
			\end{itemize}
			\\
			\hline
			Notas & 
			\begin{itemize}
				\item Generalización de:
				\begin{itemize}
					\item Generar calendario de encuentros
					\item Reportar resultado de encuentro
				\end{itemize}
			\end{itemize}
			\\
			\hline
		\end{tabular}
		} \\
		\textbf{Fuente}: Autores
	\end{center}
\end{table}

\begin{table}[!htb]
	\caption{CU029-Actualizar información de torneo: Flujos de hechos}
	\label{tab:cu029_flujo}
	\begin{center}
		\resizebox{15cm}{!}{
		\begin{tabular}{|p{1.5cm}|p{6cm}|p{6.5cm}|}
			\hline
			\multicolumn{3}{|c|}{Detalle de flujo de hechos de caso de uso} \\
			\hline
			Nombre & \multicolumn{2}{|c|}{Nombre del flujo} \\
			\hline
			Paso & Acción del actor & Respuesta del sistema \\
			\hline
			 & & \\
			\hline
		\end{tabular}
		} \\
		\textbf{Fuente}: Autores
	\end{center}
\end{table}

\begin{table}[!htb]
	\caption{CU030-Gestionar formato de torneo: Descripción}
	\label{tab:cu030_desc}
	\begin{center}
		\resizebox{15cm}{!}{
		\begin{tabular}{|p{4cm}|p{11cm}|}
			\hline
			\multicolumn{2}{|c|}{Descripción de caso de uso} \\
			\hline
			Nombre & Gestionar formato de torneo \\
			\hline
			Identificador & CU030 \\
			\hline
			Descripción & Permite la asignación de un formato al torneo realizado, así como también la organización de los equipos/jugadores participantes en dicho formato \\
			\hline
			Actor & Todo actor de la red social	 
			\\
			\hline
			Disparador & Se elige gestionar formato de evento \\
			\hline
			Inclusiones & N/A \\
			\hline
			Puntos de extensión & N/A
			\\
			\hline
			Precondiciones &  
			\begin{itemize}
				\item La aplicación ha sido cargada por un actor con rol de organizador de eventos deportivos
				\item Se ha elegido administrar torneo
			\end{itemize}
			\\
			\hline
			Postcondiciones & 
			\begin{itemize}
				\item El usuario ha gestionado el formato de torneo que desea
				\item El usuario está en la pantalla de administración de torneos
			\end{itemize}
			\\
			\hline
			Notas & N/A
			\\
			\hline
		\end{tabular}
		} \\
		\textbf{Fuente}: Autores
	\end{center}
\end{table}

\begin{table}[!htb]
	\caption{CU030-Gestionar formato de torneo: Flujos de hechos}
	\label{tab:cu030_flujo}
	\begin{center}
		\resizebox{15cm}{!}{
		\begin{tabular}{|p{1.5cm}|p{6cm}|p{6.5cm}|}
			\hline
			\multicolumn{3}{|c|}{Detalle de flujo de hechos de caso de uso} \\
			\hline
			Nombre & \multicolumn{2}{|c|}{Nombre del flujo} \\
			\hline
			Paso & Acción del actor & Respuesta del sistema \\
			\hline
			 & & \\
			\hline
		\end{tabular}
		} \\
		\textbf{Fuente}: Autores
	\end{center}
\end{table}

\begin{table}[!htb]
	\caption{CU031-Agregar participante a torneo: Descripción}
	\label{tab:cu031_desc}
	\begin{center}
		\resizebox{15cm}{!}{
		\begin{tabular}{|p{4cm}|p{11cm}|}
			\hline
			\multicolumn{2}{|c|}{Descripción de caso de uso} \\
			\hline
			Nombre & Agregar participante a torneo \\
			\hline
			Identificador & CU031 \\
			\hline
			Descripción & Agrega un equipo o jugador al torneo sin asignarlo a algún puesto en el formato elegido por el organizador del torneo \\
			\hline
			Actor & Todo actor de la red social	 
			\\
			\hline
			Disparador & Se elige agregar participante a torneo \\
			\hline
			Inclusiones & N/A \\
			\hline
			Puntos de extensión & N/A
			\\
			\hline
			Precondiciones &  
			\begin{itemize}
				\item La aplicación ha sido cargada por un actor con rol de organizador de eventos deportivos
				\item Se ha elegido administrar torneo
				\item Se ha elegido un torneo en específico
			\end{itemize}
			\\
			\hline
			Postcondiciones & 
			\begin{itemize}
				\item El usuario ha agregado o no un participante al torneo
				\item El usuario está en la pantalla de administración de torneos
			\end{itemize}
			\\
			\hline
			Notas & N/A
			\\
			\hline
		\end{tabular}
		} \\
		\textbf{Fuente}: Autores
	\end{center}
\end{table}

\begin{table}[!htb]
	\caption{CU031-Agregar participante a torneo: Flujos de hechos}
	\label{tab:cu031_flujo}
	\begin{center}
		\resizebox{15cm}{!}{
		\begin{tabular}{|p{1.5cm}|p{6cm}|p{6.5cm}|}
			\hline
			\multicolumn{3}{|c|}{Detalle de flujo de hechos de caso de uso} \\
			\hline
			Nombre & \multicolumn{2}{|c|}{Nombre del flujo} \\
			\hline
			Paso & Acción del actor & Respuesta del sistema \\
			\hline
			 & & \\
			\hline
		\end{tabular}
		} \\
		\textbf{Fuente}: Autores
	\end{center}
\end{table}

\begin{table}[!htb]
	\caption{CU032-Retirar participante de torneo: Descripción}
	\label{tab:cu032_desc}
	\begin{center}
		\resizebox{15cm}{!}{
		\begin{tabular}{|p{4cm}|p{11cm}|}
			\hline
			\multicolumn{2}{|c|}{Descripción de caso de uso} \\
			\hline
			Nombre & Retirar equipo de torneo \\
			\hline
			Identificador & CU032 \\
			\hline
			Descripción & Retira un equipo o jugador del torneo \\
			\hline
			Actor & Todo actor de la red social	 
			\\
			\hline
			Disparador & Se elige retirar participante del torneo \\
			\hline
			Inclusiones & N/A \\
			\hline
			Puntos de extensión & N/A
			\\
			\hline
			Precondiciones &  
			\begin{itemize}
				\item La aplicación ha sido cargada por un actor con rol de organizador de eventos deportivos
				\item Se ha elegido administrar torneo
				\item Se ha elegido un torneo en específico
			\end{itemize}
			\\
			\hline
			Postcondiciones & 
			\begin{itemize}
				\item El usuario ha retirado o no un participante del torneo
				\item El usuario está en la pantalla de administración de torneos
			\end{itemize}
			\\
			\hline
			Notas & N/A
			\\
			\hline
		\end{tabular}
		} \\
		\textbf{Fuente}: Autores
	\end{center}
\end{table}

\begin{table}[!htb]
	\caption{CU032-Retirar equipo de torneo: Flujos de hechos}
	\label{tab:cu032_flujo}
	\begin{center}
		\resizebox{15cm}{!}{
		\begin{tabular}{|p{1.5cm}|p{6cm}|p{6.5cm}|}
			\hline
			\multicolumn{3}{|c|}{Detalle de flujo de hechos de caso de uso} \\
			\hline
			Nombre & \multicolumn{2}{|c|}{Nombre del flujo} \\
			\hline
			Paso & Acción del actor & Respuesta del sistema \\
			\hline
			 & & \\
			\hline
		\end{tabular}
		} \\
		\textbf{Fuente}: Autores
	\end{center}
\end{table}

\begin{table}[!htb]
	\caption{CU033-Generar calendario de encuentros: Descripción}
	\label{tab:cu033_desc}
	\begin{center}
		\resizebox{15cm}{!}{
		\begin{tabular}{|p{4cm}|p{11cm}|}
			\hline
			\multicolumn{2}{|c|}{Descripción de caso de uso} \\
			\hline
			Nombre & Generar calendario de encuentros \\
			\hline
			Identificador & CU033 \\
			\hline
			Descripción & Genera el calendario de los encuentros a realizarse en el evento \\
			\hline
			Actor & Todo actor de la red social	 
			\\
			\hline
			Disparador & Se elige generar calendario de encuentros \\
			\hline
			Inclusiones & N/A \\
			\hline
			Puntos de extensión & N/A
			\\
			\hline
			Precondiciones &  
			\begin{itemize}
				\item La aplicación ha sido cargada por un actor con rol de organizador de eventos deportivos
				\item Se ha elegido administrar torneo
				\item Se ha elegido un torneo en específico
				\item Se ha arreglado el formato del evento correctamente
			\end{itemize}
			\\
			\hline
			Postcondiciones & 
			\begin{itemize}
				\item El usuario ha generado el calendario de encuentros
				\item El usuario está en la pantalla de administración de torneos
			\end{itemize}
			\\
			\hline
			Notas & N/A
			\\
			\hline
		\end{tabular}
		} \\
		\textbf{Fuente}: Autores
	\end{center}
\end{table}

\begin{table}[!htb]
	\caption{CU033-Generar calendario de encuentros: Flujos de hechos}
	\label{tab:cu033_flujo}
	\begin{center}
		\resizebox{15cm}{!}{
		\begin{tabular}{|p{1.5cm}|p{6cm}|p{6.5cm}|}
			\hline
			\multicolumn{3}{|c|}{Detalle de flujo de hechos de caso de uso} \\
			\hline
			Nombre & \multicolumn{2}{|c|}{Nombre del flujo} \\
			\hline
			Paso & Acción del actor & Respuesta del sistema \\
			\hline
			 & & \\
			\hline
		\end{tabular}
		} \\
		\textbf{Fuente}: Autores
	\end{center}
\end{table}

\begin{table}[!htb]
	\caption{CU034-Reportar resultado de encuentro: Descripción}
	\label{tab:cu034_desc}
	\begin{center}
		\resizebox{15cm}{!}{
		\begin{tabular}{|p{4cm}|p{11cm}|}
			\hline
			\multicolumn{2}{|c|}{Descripción de caso de uso} \\
			\hline
			Nombre & Reportar resultado de encuentro \\
			\hline
			Identificador & CU034 \\
			\hline
			Descripción & Permite reportar el resultado de un encuentro deportivo después de haber iniciado el torneo \\
			\hline
			Actor & Todo actor de la red social	 
			\\
			\hline
			Disparador & Se elige reportar resultado de un encuentro \\
			\hline
			Inclusiones & N/A \\
			\hline
			Puntos de extensión & N/A
			\\
			\hline
			Precondiciones &  
			\begin{itemize}
				\item La aplicación ha sido cargada por un actor con rol de organizador de eventos deportivos
				\item Se ha elegido administrar torneo
				\item Se ha elegido un torneo en específico
				\item Se ha elegido un encuentro específico
				\item Se ha terminado el encuentro elegido según calendario
			\end{itemize}
			\\
			\hline
			Postcondiciones & 
			\begin{itemize}
				\item El usuario ha actualizado el resultado del encuentro
				\item El usuario está en la pantalla de administración de torneos
			\end{itemize}
			\\
			\hline
			Notas & N/A
			\\
			\hline
		\end{tabular}
		} \\
		\textbf{Fuente}: Autores
	\end{center}
\end{table}

\begin{table}[!htb]
	\caption{CU034-Reportar resultado de encuentro: Flujos de hechos}
	\label{tab:cu034_flujo}
	\begin{center}
		\resizebox{15cm}{!}{
		\begin{tabular}{|p{1.5cm}|p{6cm}|p{6.5cm}|}
			\hline
			\multicolumn{3}{|c|}{Detalle de flujo de hechos de caso de uso} \\
			\hline
			Nombre & \multicolumn{2}{|c|}{Nombre del flujo} \\
			\hline
			Paso & Acción del actor & Respuesta del sistema \\
			\hline
			 & & \\
			\hline
		\end{tabular}
		} \\
		\textbf{Fuente}: Autores
	\end{center}
\end{table}