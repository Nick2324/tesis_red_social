\clearpage
\begin{table}[!htb]
	\caption{CU001-Administrar deportes: Descripción}
	\label{tab:cu001_desc}
	\begin{center}
		\resizebox{15cm}{!}{
		\begin{tabular}{|p{4cm}|p{11cm}|}
			\hline
			\multicolumn{2}{|c|}{Descripción de caso de uso} \\
			\hline
			Nombre & Administrar deportes \\
			\hline
			Identificador & CU \\
			\hline
			Descripción & Permite la administración de la información de de deportes registrados en la red social \\
			\hline
			Actor & 
			\begin{itemize}
				\item Todos
			\end{itemize} \\
			\hline
			Disparador & El jugador quiere administrar la información de deportes \\
			\hline
			Inclusiones & \\
			\hline
			Puntos de extensión & \\
			\hline
			Precondiciones & \\
			\hline
			Postcondiciones & \\
			\hline
			Notas & 
			\begin{itemize}
				\item Generalización de:
				\begin{itemize}
					\item Agregar un deporte
					\item Actualizar un deporte
					\item Dar de baja un deporte
				\end{itemize}
			\end{itemize} \\
			\hline 
		\end{tabular}
		} \\
		\textbf{Fuente}: Autores
	\end{center}
\end{table}

\begin{table}[!htb]
	\caption{CU001-Administrar deportes: Flujos de hechos }
	\label{tab:cu001_flujo}
	\begin{center}
		\resizebox{15cm}{!}{
		\begin{tabular}{|p{1.5cm}|p{6cm}|p{6.5cm}|}
			\hline
			\multicolumn{3}{|c|}{Detalle de flujo de hechos de caso de uso} \\
			\hline
			Nombre & \multicolumn{2}{|c|}{Nombre del flujo} \\
			\hline
			Paso & Acción del actor & Respuesta del sistema \\
			\hline
			 & & \\
			\hline
		\end{tabular}
		} \\
		\textbf{Fuente}: Autores
	\end{center}
\end{table}

\begin{table}[!htb]
	\caption{CU001-Agregar un deporte: Descripción}
	\label{tab:cu001_desc}
	\begin{center}
		\resizebox{15cm}{!}{
		\begin{tabular}{|p{4cm}|p{11cm}|}
			\hline
			\multicolumn{2}{|c|}{Descripción de caso de uso} \\
			\hline
			Nombre & Agregar un deporte \\
			\hline
			Identificador & CU\\
			\hline
			Descripción & Permite agregar a la red social un nuevo deporte \\
			\hline
			Actor & 
			\begin{itemize}
				\item Administrador
			\end{itemize} \\
			\hline
			Disparador & El administrador desea crear un nuevo deporte en la red social y activa dicha opción \\
			\hline
			Inclusiones & \\
			\hline
			Puntos de extensión & \\
			\hline
			Precondiciones & 
			\begin{itemize}
				\item El SNS se inicia con rol de administrador
			\end{itemize} \\
			\hline
			Postcondiciones & 
			\begin{itemize}
				\item El administrador crea un nuevo deporte sobre la red social
			\end{itemize}						
			\\
			\hline
			Notas & \\
			\hline
		\end{tabular}
		} \\
		\textbf{Fuente}: Autores
	\end{center}
\end{table}

\begin{table}[!htb]
	\caption{CU001-Agregar un deporte: Flujos de hechos }
	\label{tab:cu001_flujo}
	\begin{center}
		\resizebox{15cm}{!}{
		\begin{tabular}{|p{1.5cm}|p{6cm}|p{6.5cm}|}
			\hline
			\multicolumn{3}{|c|}{Detalle de flujo de hechos de caso de uso} \\
			\hline
			Nombre & \multicolumn{2}{|c|}{Nombre del flujo} \\
			\hline
			Paso & Acción del actor & Respuesta del sistema \\
			\hline
			 & & \\
			\hline
		\end{tabular}
		} \\
		\textbf{Fuente}: Autores
	\end{center}
\end{table}

\begin{table}[!htb]
	\caption{CU001-Actualizar un deporte: Descripción}
	\label{tab:cu001_desc}
	\begin{center}
		\resizebox{15cm}{!}{
		\begin{tabular}{|p{4cm}|p{11cm}|}
			\hline
			\multicolumn{2}{|c|}{Descripción de caso de uso} \\
			\hline
			Nombre & Actualizar un deporte \\
			\hline
			Identificador & CU\\
			\hline
			Descripción & Permite al usuario actualizar la información de un deporte ya creado sobre la red social \\
			\hline
			Actor & 
			\begin{itemize}
				\item Administrador
			\end{itemize} \\
			\hline
			Disparador & El administrador desea actualizar la información de un deporte ya creado y elige la opción correspondiente \\
			\hline
			Inclusiones & \\
			\hline
			Puntos de extensión & \\
			\hline
			Precondiciones & 
			\begin{itemize}
				\item El SNS se inicia con rol de administrador
			\end{itemize} \\
			\hline
			Postcondiciones & 
			\begin{itemize}
				\item El administrador actualiza la información de un deporte con éxito
			\end{itemize}						
			\\
			\hline
			Notas &
			\begin{itemize}
				\item Gestionar implementos
				\item Gestionar recursos externos
				\item Gestionar movimientos técnicos
				\item Gestionar movimientos tácticos
			\end{itemize}
			\\
			\hline
		\end{tabular}
		} \\
		\textbf{Fuente}: Autores
	\end{center}
\end{table}

\begin{table}[!htb]
	\caption{CU001-Actualizar un deporte: Flujos de hechos }
	\label{tab:cu001_flujo}
	\begin{center}
		\resizebox{15cm}{!}{
		\begin{tabular}{|p{1.5cm}|p{6cm}|p{6.5cm}|}
			\hline
			\multicolumn{3}{|c|}{Detalle de flujo de hechos de caso de uso} \\
			\hline
			Nombre & \multicolumn{2}{|c|}{Nombre del flujo} \\
			\hline
			Paso & Acción del actor & Respuesta del sistema \\
			\hline
			 & & \\
			\hline
		\end{tabular}
		} \\
		\textbf{Fuente}: Autores
	\end{center}
\end{table}

\begin{table}[!htb]
	\caption{CU001-Gestionar implementos: Descripción}
	\label{tab:cu001_desc}
	\begin{center}
		\resizebox{15cm}{!}{
		\begin{tabular}{|p{4cm}|p{11cm}|}
			\hline
			\multicolumn{2}{|c|}{Descripción de caso de uso} \\
			\hline
			Nombre & Gestionar implementos \\
			\hline
			Identificador & CU \\
			\hline
			Descripción & Permite al usuario gestionar información de implementos usados en el deporte \\
			\hline
			Actor & 
			\begin{itemize}
				\item Administrador
			\end{itemize} \\
			\hline
			Disparador & El administrador desea gestionar información de implementos usados en el deporte y elige la opción correspondiente \\
			\hline
			Inclusiones & \\
			\hline
			Puntos de extensión & \\
			\hline
			Precondiciones & 
			\begin{itemize}
				\item El SNS se inicia con rol de administrador
			\end{itemize} \\
			\hline
			Postcondiciones & \\
			\hline
			Notas & \\
			\hline
		\end{tabular}
		} \\
		\textbf{Fuente}: Autores
	\end{center}
\end{table}

\begin{table}[!htb]
	\caption{CU001-Gestionar implementos: Flujos de hechos }
	\label{tab:cu001_flujo}
	\begin{center}
		\resizebox{15cm}{!}{
		\begin{tabular}{|p{1.5cm}|p{6cm}|p{6.5cm}|}
			\hline
			\multicolumn{3}{|c|}{Detalle de flujo de hechos de caso de uso} \\
			\hline
			Nombre & \multicolumn{2}{|c|}{Nombre del flujo} \\
			\hline
			Paso & Acción del actor & Respuesta del sistema \\
			\hline
			 & & \\
			\hline
		\end{tabular}
		} \\
		\textbf{Fuente}: Autores
	\end{center}
\end{table}

\begin{table}[!htb]
	\caption{CU001-Gestionar movimientos técnicos: Descripción}
	\label{tab:cu001_desc}
	\begin{center}
		\resizebox{15cm}{!}{
		\begin{tabular}{|p{4cm}|p{11cm}|}
			\hline
			\multicolumn{2}{|c|}{Descripción de caso de uso} \\
			\hline
			Nombre & Gestionar movimientos técnicos \\
			\hline
			Identificador & CU \\
			\hline
			Descripción & Permite al usuario gestionar información de movimientos técnicos usados en el deporte \\
			\hline
			Actor & 
			\begin{itemize}
				\item Administrador
			\end{itemize} \\
			\hline
			Disparador & El administrador desea gestionar información de movimientos técnicos usados en el deporte y elige la opción correspondiente \\
			\hline
			Inclusiones & \\
			\hline
			Puntos de extensión & \\
			\hline
			Precondiciones & 
			\begin{itemize}
				\item El SNS se inicia con rol de administrador
			\end{itemize} \\
			\hline
			Postcondiciones & \\
			\hline
			Notas & \\
			\hline
		\end{tabular}
		} \\
		\textbf{Fuente}: Autores
	\end{center}
\end{table}

\begin{table}[!htb]
	\caption{CU001-Gestionar movimientos técnicos: Flujos de hechos }
	\label{tab:cu001_flujo}
	\begin{center}
		\resizebox{15cm}{!}{
		\begin{tabular}{|p{1.5cm}|p{6cm}|p{6.5cm}|}
			\hline
			\multicolumn{3}{|c|}{Detalle de flujo de hechos de caso de uso} \\
			\hline
			Nombre & \multicolumn{2}{|c|}{Nombre del flujo} \\
			\hline
			Paso & Acción del actor & Respuesta del sistema \\
			\hline
			 & & \\
			\hline
		\end{tabular}
		} \\
		\textbf{Fuente}: Autores
	\end{center}
\end{table}

\begin{table}[!htb]
	\caption{CU001-Gestionar movimientos tácticos: Descripción}
	\label{tab:cu001_desc}
	\begin{center}
		\resizebox{15cm}{!}{
		\begin{tabular}{|p{4cm}|p{11cm}|}
			\hline
			\multicolumn{2}{|c|}{Descripción de caso de uso} \\
			\hline
			Nombre & Gestionar movimientos tácticos \\
			\hline
			Identificador & CU \\
			\hline
			Descripción & Permite al usuario gestionar información de movimientos tácticos usados en el deporte \\
			\hline
			Actor & 
			\begin{itemize}
				\item Administrador
			\end{itemize} \\
			\hline
			Disparador & El administrador desea gestionar información de movimientos tácticos usados en el deporte y elige la opción correspondiente \\
			\hline
			Inclusiones & \\
			\hline
			Puntos de extensión & \\
			\hline
			Precondiciones & 
			\begin{itemize}
				\item El SNS se inicia con rol de administrador
			\end{itemize} \\
			\hline
			Postcondiciones & \\
			\hline
			Notas & \\
			\hline
		\end{tabular}
		} \\
		\textbf{Fuente}: Autores
	\end{center}
\end{table}

\begin{table}[!htb]
	\caption{CU001-Gestionar movimientos tácticos: Flujos de hechos }
	\label{tab:cu001_flujo}
	\begin{center}
		\resizebox{15cm}{!}{
		\begin{tabular}{|p{1.5cm}|p{6cm}|p{6.5cm}|}
			\hline
			\multicolumn{3}{|c|}{Detalle de flujo de hechos de caso de uso} \\
			\hline
			Nombre & \multicolumn{2}{|c|}{Nombre del flujo} \\
			\hline
			Paso & Acción del actor & Respuesta del sistema \\
			\hline
			 & & \\
			\hline
		\end{tabular}
		} \\
		\textbf{Fuente}: Autores
	\end{center}
\end{table}

\begin{table}[!htb]
	\caption{CU001-Gestionar recursos externos: Descripción}
	\label{tab:cu001_desc}
	\begin{center}
		\resizebox{15cm}{!}{
		\begin{tabular}{|p{4cm}|p{11cm}|}
			\hline
			\multicolumn{2}{|c|}{Descripción de caso de uso} \\
			\hline
			Nombre & Gestionar recursos externos \\
			\hline
			Identificador & CU \\
			\hline
			Descripción & Permite al usuario gestionar información de recursos externos que contienen información del deporte (ej. reglas del deporte) \\
			\hline
			Actor & 
			\begin{itemize}
				\item Administrador
			\end{itemize} \\
			\hline
			Disparador & El administrador desea gestionar información de recursos externos que contienen información del deporte y elige la opción correspondiente \\
			\hline
			Inclusiones & \\
			\hline
			Puntos de extensión & \\
			\hline
			Precondiciones & 
			\begin{itemize}
				\item El SNS se inicia con rol de administrador
			\end{itemize} \\
			\hline
			Postcondiciones & \\
			\hline
			Notas & \\
			\hline
		\end{tabular}
		} \\
		\textbf{Fuente}: Autores
	\end{center}
\end{table}

\begin{table}[!htb]
	\caption{CU001-Gestionar recursos externos: Flujos de hechos }
	\label{tab:cu001_flujo}
	\begin{center}
		\resizebox{15cm}{!}{
		\begin{tabular}{|p{1.5cm}|p{6cm}|p{6.5cm}|}
			\hline
			\multicolumn{3}{|c|}{Detalle de flujo de hechos de caso de uso} \\
			\hline
			Nombre & \multicolumn{2}{|c|}{Nombre del flujo} \\
			\hline
			Paso & Acción del actor & Respuesta del sistema \\
			\hline
			 & & \\
			\hline
		\end{tabular}
		} \\
		\textbf{Fuente}: Autores
	\end{center}
\end{table}

\begin{table}[!htb]
	\caption{CU001-Dar de baja un deporte: Descripción}
	\label{tab:cu001_desc}
	\begin{center}
		\resizebox{15cm}{!}{
		\begin{tabular}{|p{4cm}|p{11cm}|}
			\hline
			\multicolumn{2}{|c|}{Descripción de caso de uso} \\
			\hline
			Nombre & Agregar deporte practicado \\
			\hline
			Identificador & CU\\
			\hline
			Descripción & Permite dar de baja un deporte creado sobre la red social \\
			\hline
			Actor & 
			\begin{itemize}
				\item Administrador
			\end{itemize} \\
			\hline
			Disparador & El administrador desea dar de baja un deporte creado sobre la red social y elige la opción correspondiente\\
			\hline
			Inclusiones & \\
			\hline
			Puntos de extensión & \\
			\hline
			Precondiciones & 
			\begin{itemize}
				\item El SNS se inicia con rol de administrador
			\end{itemize} \\
			\hline
			Postcondiciones & 
			\begin{itemize}
				\item El administrador da de baja un deporte
			\end{itemize} \\
			\hline
			Notas & \\
			\hline
		\end{tabular}
		} \\
		\textbf{Fuente}: Autores
	\end{center}
\end{table}

\begin{table}[!htb]
	\caption{CU001-Dar de baja un deporte: Flujos de hechos }
	\label{tab:cu001_flujo}
	\begin{center}
		\resizebox{15cm}{!}{
		\begin{tabular}{|p{1.5cm}|p{6cm}|p{6.5cm}|}
			\hline
			\multicolumn{3}{|c|}{Detalle de flujo de hechos de caso de uso} \\
			\hline
			Nombre & \multicolumn{2}{|c|}{Nombre del flujo} \\
			\hline
			Paso & Acción del actor & Respuesta del sistema \\
			\hline
			 & & \\
			\hline
		\end{tabular}
		} \\
		\textbf{Fuente}: Autores
	\end{center}
\end{table}