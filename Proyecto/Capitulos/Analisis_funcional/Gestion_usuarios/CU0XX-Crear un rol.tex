\begin{table}[!htb]
	\caption{CU0XX-Crear un rol: Descripci�n}
	\label{tab:cu0xx_desc}
	\begin{center}
		\resizebox{15cm}{!}{
		\begin{tabular}{|p{4cm}|p{11cm}|}
			\hline
			\multicolumn{2}{|c|}{Descripci�n de caso de uso} \\
			\hline
			Nombre & Crear un rol\\
			\hline
			Identificador & CU0XX\\
			\hline
			Descripcion & Funcionalidad que le permite al usuario administrador crear un nuevo rol en la aplicaci�n.\\
			\hline
			Actor & 
			\begin{itemize}
				\item Administrador
			\end{itemize}
			\hline
			Disparador & El usuario administrador necesita crear un nuevo rol en la aplicaci�n.\\
			\hline
			Inclusiones & N/A\\
			\hline
			Extensiones & N/A\\
			\hline
			Precondiciones & 
			\begin{itemize}
				\item 
			\end{itemize}
			\hline
			Postcondiciones & 
			\begin{itemize}
				\item 
			\end{itemize}\\
			\hline
			Notas & N/A\\
			\hline
		\end{tabular}
		} \\
		\textbf{Fuente}: Autores
	\end{center}
\end{table}
\begin{table}[!htb]
	\caption{CU0XX-Crear un rol:Flujos de hechos}
	\label{tab:cu0xx_flujo}
	\begin{center}
		\resizebox{15cm}{!}{
		\begin{tabular}{|p{1.5cm}|p{6cm}|p{6.5cm}|}
			\hline
			\multicolumn{3}{|c|}{Descripci�n de caso de uso} \\
			\hline
			Nombre & \multicolumn{2}{|c|}{Nombre del flujo} \\
			\hline
			Paso & Acci�n del actor & Respuesta del sistema\\
			\hline
				1 & El usuario selecciona la opci�n Administrar. & Se despliega el men� de administraci�n de la aplicaci�n.\\
				2 & El usuario selecciona la opci�n Adminisrar Roles. & Se despliega el men� de administraci�n de roles.\\
				3 & El usuario selecciona la opci�n Crear un rol. & Se despliega el formulario de creaci�n de roles.\\
				4 & Luego de llenar el formulario, el usuario selecciona la opcion Crear. & Se solicita confirmaci�n al usuario de la creaci�n del nuevo rol.\\
				5 & El usuario confirma la creaci�n del nuevo rol. & Se muestra mensaje informando el exito de la operaci�n. El sistema regresa al men� de administraci�n.\\
			\hline
		\end{tabular}
		} \\
		\textbf{Fuente}: Autores
	\end{center}
\end{table}
