\begin{table}[!htb]
	\caption{CU144-Actualizar un rol: Descripción}
	\label{tab:cu144_desc}
	\begin{center}
		\resizebox{15cm}{!}{
		\begin{tabular}{|p{4cm}|p{11cm}|}
			\hline
			\multicolumn{2}{|c|}{Descripción de caso de uso} \\
			\hline
			Nombre & Actualizar un rol\\
			\hline
			Identificador & CU144\\
			\hline
			Descripcion & Funcionalidad que le permite al usuario administrador actualizar la información asociada a un rol en la aplicación.\\
			\hline
			Actor & 
			\begin{itemize}
				\item Administrador
			\end{itemize}
			\hline
			Disparador & El usuario administrador necesita actualizar un rol en la aplicación.\\
			\hline
			Inclusiones & N/A\\
			\hline
			Extensiones & N/A\\
			\hline
			Precondiciones & 
			\begin{itemize}
				\item 
			\end{itemize}
			\hline
			Postcondiciones & 
			\begin{itemize}
				\item 
			\end{itemize}\\
			\hline
			Notas & N/A\\
			\hline
		\end{tabular}
		} \\
		\textbf{Fuente}: Autores
	\end{center}
\end{table}
\begin{table}[!htb]
	\caption{CU144-Actualizar un rol:Flujos de hechos}
	\label{tab:cu144_flujo}
	\begin{center}
		\resizebox{15cm}{!}{
		\begin{tabular}{|p{1.5cm}|p{6cm}|p{6.5cm}|}
			\hline
			\multicolumn{3}{|c|}{Descripción de caso de uso} \\
			\hline
			Nombre & \multicolumn{2}{|c|}{Nombre del flujo} \\
			\hline
			Paso & Acción del actor & Respuesta del sistema\\
			\hline
				1 & El usuario selecciona la opción Administrar. & Se despliega el menú de administración de la aplicación.\\
				2 & El usuario selecciona la opción Adminisrar Roles. & Se despliega el menú de administración de roles.\\
				3 & El usuario selecciona la opción Roles. & Se despliega el listado de los roles actuales.\\
				4 & El usuario selecciona la opcion actualizar en el rol que desea actualizar. & Se despliega el formulario de actualización de un rol\\
				5 & Luego de llenar los datos del formulario, el usuario selecciona la opción Actualizar & Se solicita confirmación de la actualización del rol\\
				6 & El usuario confirma la actualización del nuevo rol. & Se muestra mensaje informando el exito de la operación. El sistema regresa al menú de administración.\\
			\hline
		\end{tabular}
		} \\
		\textbf{Fuente}: Autores
	\end{center}
\end{table}
