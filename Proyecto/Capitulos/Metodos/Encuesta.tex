\section{Encuesta inicial}

Con el fin de definir el grupo focal de la aplicación a desarrollar, se practicó una encuesta inicial por medio de internet. Gracias a esta encuesta, se pudo qué deportes son los mas populares, los menos populares, y de qué forma los jóvenes interactuan para practicar estos deportes. Teniendo en cuenta que el objetivo final de la aplicación es ayudar a que la práctica deportiva se masifique, se decidio que inicialmente no se van a tener en cuenta los deportes más populares (Futbol, Baloncesto, Ciclismo) e implementar la aplicación para que soporte los deportes menos populares. En este orden de ideas, los deportes seleccionados como pioneros en la aplicacion son el Rugby y el Tenis.

\subsection{Análisis de resultados}

La encuesta se practicó a 155 personas, a través de internet, y valiéndonos de grupos y sitios web que frecuentan los jóvenes de Bogotá, en su mayoría estudiantes universitarios.\\

\begin{itemize}
  \item Edad \\
  Los rangos de edad de los encuestados varían de 13 hasta 60 años, concentrándose en el rango de 20 a 27 años. Aún cuando esta pregunta no refleja ningún comportamiento de análisis, refleja que la encuesta fue practicada, en su gran mayoría, a los jóvenes Bogotanos.
  \item Ocupación \\
  Se puede apreciar como el 66\% de los encuestados son estudiantes. Los estudiantes suelen tener grupos de amigos/conocidos en su lugar de estudio con quienes pasan tiempo por fuera de sus lugares de estudio, son jóvenes que, en su mayoría, están disfrutando de su etapa de estudiantes universitarios, concentrándose mayoritariamente en sus estudios. Por otra parte, un 25\% de los encuestados dicen ser Empleados, y teniendo en cuenta como y a quien se le realiza la encuesta, se puede suponer que son estudiantes que, a parte de estudiar, también tienen que trabajar.
  \item Elementos electrónicos \\
  El elemento que mas  dicen tener los encuestados es el computador portátil (37\%) , lo cual tiene sentido teniendo en cuenta las necesidades de un estudiante universitario, seguido muy de cerca del computador de escritorio (26\%) y el smartphone (26\%). En la mayoría de los casos, aseguran tener tanto computador portátil como smarthpone. De allí se puede deducir que a los jóvenes les gusta estar en constante conexión con el mundo digital y la internet.
  \item Lugar de acceso a internet \\
  El lugar desde el que se accede a internet con mayor frecuencia es el hogar con un 43\%, seguido de el lugar de estudio (23\%) y del internet móvil(13\%). Adicionalmente, los encuestados aseguran que los lugares en los que duran mas tiempo conectados son el hogar (72\%) y el internet móvil (15\%). Esto muestra que hay preferencia en conectarse desde lugares y dispositivos en los que se sienten mas en privado (o en control) de quienes tienen acceso a la información contenida por estos dispositivos.
  \item ¿Practica deporte? \\
  El 63\% de los encuestados asegura practicar algún deporte, mientras el 37\% no. Las razones por las que este importante porcentaje de la población no practican algún deporte sale del alcance de esta primera encuesta.
  \item Deportes practicados \\
  En los deportes practicados, resaltan el Fútbol (27\%), Baloncesto (13\%) y Ciclismo (13\%), mientras que los deportes en los que se requieren implementos o lugares especializados no son tan comunes (Tenis 7\%, Escalada deportiva 3\%, Patinaje 1\% y no se practican deportes como Rugby, Fútbol Americano o Golf)
  \item Métodos de búsqueda \\
  Para analizar los métodos de búsqueda, se realizaron preguntas enfocadas a la búsqueda de nuevos deportes, implementos, lugares y grupos o equipos para practicar estos deportes. El común denominador para cada una de ellas fue consultar con los amigos, en donde siempre fue de las opciones mas populares, solo superada por la consulta de tiendas deportivas (en el caso de la búsqueda de implementos) y el CouchSurfing (en el caso de la búsqueda de un nuevo deporte), demostrando que las opiniones de los amigos/conocidos tienen mayor importancia que cualquier otra forma de búsqueda.
\end{itemize}


