\section{Recursos de Software}
En la tabla \ref{tab:rec_software} se muestran los costos estimados en los que se incurrirá para el desarrollo del proyecto con respecto a recursos de software.
  \begin{table}[!htb]
    \caption{Recursos de software}
    \label{tab:rec_software}
    \begin{center}
    \resizebox{10cm}{!}{
          \begin{tabular}{|p{5cm}|p{7cm}|}
          \hline
          Recurso & Descripción\\
          \hline \hline
          Debian Versión 7.4 (wheezy),32-bit & Sistema operativo en el que se realizará el desarrollo \\
          \hline
          Eclipse IDE & Entorno de desarrollo de código abierto \\
          \hline
          Android Studio & IDE proporcionado por Gooogle que brinda un entorno de desarrollo para construir aplicaciones Android \\
          \hline
          Android 4.0 Ice Cream & API para el desarrollo de aplicaciones móviles para Android 4.0 y superiores \\
          \hline
          Enterprise Architect 11 & Herramienta para modelado de software (en nuestro caso, UML)  \\
		  \hline
          Archi 3.0 & Herramienta libre y gratuita para crear modelos en el estándar Archimate \\
		  \hline
		  Evolus Pencil 2.0.5 & Herramienta libre para el diseño de interfaces gráficas de usuario \\
		  \hline
		  Neo4j 2.1.7 & Motor de base de datos orientado a grafos \\
		  \hline          
          WildFly 8.2.0 & Servidor de aplicaciones libre y gratuito que se utiliza como backend de la aplicación \\
          \hline
        \end{tabular}
    } \\
      \footnotesize \textbf{Nota:} Los costos incurridos en instalación, configuración o capacitaciones están cubiertos en el salario del developement team.
    \end{center}
  \end{table} 
