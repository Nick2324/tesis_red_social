\section{Scrum}

A continuación, se explican los diferentes roles y artefactos que existen en el marco de trabajo de Scrum.

\subsection{Equipo Scrum (Scrum Team)}

\subsubsection{Product Owner (dueño del producto)}

Es el responsable de gestionar el product backlog y el trabajo del equipo de desarrollo. Entre sus funciones se encuentran:
\begin{itemize}
		  \item Expresar los elementos del product backlog.
		  \item Ordenar de la mejor manera posible los elementos del product backlog para lograr el objetivo final.
		  \item Asegurarse de que el equipo de desarrollo entiende los items del product backlog.
		\end{itemize}
		
\subsubsection{Development Team (Equipo de desarrollo)}

Son los encargados de llevar a cabo el incremento al producto en cada iteración o sprint. El equipo de desarrollo es el encargado de organizar y gestionar su propio trabajo.

El equipo de desarrollo se caracteriza por:
	
\begin{itemize}
		  \item Es autoorganizado: Se le dice al equipo que debe hacer, pero el es libre de decidir como lo hace.
		  \item Son multifuncionales: Se tienen integrantes que manejan diferentes áreas de experticia para ayudar a realizar el incremento necesario.
		  \item No se reconocen los sub-equipos que se puedan formar, la responsabilidad de lo que se haga recae en e equipo de desarrollo como un todo.
		\end{itemize}

\subsubsection{Scrum Master}

Es el encargado de que se cumpla la teoría de scrum a lo largo de todo el proyecto. El Scrum Master ayuda a las personas externas al Equipo Scrum a entender qué interacciones con el Equipo Scrum pueden ser de ayuda y cuáles no \cite{scrum_guide}.

\textbf{Servicios que ofrece al Product Owner}
	
\begin{itemize}
		  \item Encontrar técnicas para gestionar la Lista de Producto de manera efectiva
		  \item Ayudar al Equipo Scrum a entender la necesidad de contar con elementos de Lista de Producto claros y concisos
		  \item Entender la planificación del producto en un entorno empírico
		  \item Asegurar que el Dueño de Producto conozca cómo ordenar la Lista de Producto para maximizar el valor
		  \item Entender y practicar la agilidad; y,
		  \item Facilitar los eventos de Scrum según se requiera o necesite.
\end{itemize}
		\cite{scrum_guide}

\textbf{Servicios que ofrece al Developement Team}
	
\begin{itemize}
		  \item Guiar al Equipo de Desarrollo en ser autoorganizado y multifuncional;
		  \item Ayudar al Equipo de Desarrollo a crear productos de alto valor;
		  \item Eliminar impedimentos para el progreso del Equipo de Desarrollo;
		  \item Facilitar los eventos de Scrum según se requiera o necesite; y,
		  \item Guiar al Equipo de Desarrollo en el entorno de organizaciones en las que Scrum aún no ha sido adoptado y entendido por completo.
\end{itemize}
		\cite{scrum_guide}

\textbf{Servicios que ofrece a la organización}
	
\begin{itemize}
		  \item Liderar y guiar a la organización en la adopción de Scrum;
		  \item Planificar las implementaciones de Scrum en la organización; 
		  \item Ayudar a los empleados e interesados a entender y llevar a cabo Scrum y el desarrollo empírico de producto;
		  \item Motivar cambios que incrementen la productividad del Equipo Scrum; y,
		  \item Trabajar con otros Scrum Masters para incrementar la efectividad de la aplicación de Scrum en la organización.
\end{itemize}
		\cite{scrum_guide}
		
\subsection{Eventos}

\subsubsection{Sprint}

El Sprint representa un espacio de tiempo, no mayor a un mes, en el que se trabaja para crear un incremento en el desarrollo del proyecto. Es conveniente que los sprint tengan una duración consistente a lo largo del proyecto, y un nuevo sprint inicia tan pronto el actual termina.

Cada sprint debe tener un objetivo definido (Sprint Goal), un plan flexible y concepto de ``terminado'' claro. 

\subsubsection{Sprint Planning Meeting (Reunión de Planificación de Sprint)}

Esta reunion se lleva a cabo al inicio de cada Sprint y no tiene una duración mayor a 8 horas. En esta reunión, que se lleva a cabo en presencia de todo el equipo scrum, se crea un plan para el sprint que inicia. En este plan se responden dos preguntas fundamentales, ¿Qué puede entregarse en el Incremento resultante del Sprint que comienza? y ¿Cómo se conseguirá hacer el trabajo necesario para entregar el Incremento?

\subsubsection{Daily Scrum (Scrum Diario)}

Esta reunion, que no debe durar mas de 15 minutos, se realiza a diario entre los miembros del development team. El objetivo de esta reunion es socializar lo que se hizo en las ultimas 24 horas y planear que hacer en las proximas 24 horas. Se evalua si se esta haciendo lo necesario para cumplir el sprint goal y, si es necesario, se puede adaptar o redefinir el trabajo del resto del sprint.

\subsubsection{Sprint Review (Revisión de Sprint)}

Al final de cada sprint se realiza esta reunion cuyo objetivo es el de socializar lo que se hizo en el presente sprint. En esta reunion se descuten cosas como qué fue bien durante el Sprint, qué problemas aparecieron y cómo fueron resueltos esos problemas. Al final de la revisión se debe generar un product backlog actualizado con los elementos que se proponen para el siguiente sprint. Esta reunion tiene una duracion no mayor a 4 horas.

\subsubsection{Sprint Retrospective (Retrospectiva de Sprint)}

Esta reunión es similar al sprint review, pero en lugar de tratar el Qué se hizo, se trata el Cómo se hizo. Al final de esta reunión se genera un plan para mejorar el desempeño del equipo de scrum para que los sprint posteriores sean de mayor provecho para el proyecto.

\subsection{Artefactos}

\subsubsection{Sprint Goal (Objetivo del Sprint)}

El sprint goal es una meta que se plantea al inicio de cada sprint que puede ser alcanzada mediante el incremento en el proyecto. Este sprint goal ``Proporciona una guía al Equipo de Desarrollo acerca de por qué está construyendo el incremento'' \cite{scrum_guide}. Es importante y necesario que el objetivo sea claro, coherente y sea entendido por todos los integrantes del equipo de scrum.

\subsubsection{Product Backlog (Lista de Producto)}

Esta lista representa todos los requisitos que tenga el proyecto o producto que son conocidos y entendidos en un momento determinado del desarrollo. Debido a la naturaleza cambiante y dinámica del entorno, la lista nunca está vacía. A medida que el producto evoluciona, la retroalimentación que se obtiene ayuda a completar la lista y a refinar el producto final.

\subsubsection{Sprint Backlog (Lista de Pendientes del Sprint)}

Esta lista esta compuesta por los diferentes items seleccionados del broduct backlog que van a ser tratados en cada sprint. Adicionalmente, se incluye un plan que ayude a conseguir el objetivo del sprint.
	
La Lista de Pendientes del Sprint es una predicción hecha por el Equipo de Desarrollo acerca de qué funcionalidad formará parte del próximo Incremento y del trabajo necesario para entregar esa funcionalidad en un Incremento \cite{scrum_guide}.
