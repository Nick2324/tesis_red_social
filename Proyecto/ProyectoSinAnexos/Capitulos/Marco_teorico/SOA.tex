\section{Service-Oriented Architecture (SOA)}
\subsection{Proceso de Análisis y diseño orientados a servicios}

En la etapa de análisis y diseño, el arquitecto de software se reúne con un analista de negocio, el cual diseñará unos servicios candidatos que después se convertirán en los servicios incluidos en los blueprints. Luego de tener los planos, el arquitecto escoge un subconjunto de dichos candidatos a servicios para ser implementados físicamente, dotándolos de algún método para realizar la composición de servicios. En la figura \ref{fig:seis} se muestra el proceso de análisis y diseño orientado a servicios.

\begin{figure}[!htb]
  \begin{center}
    \includegraphics[width=11cm]{./imagenes/6.png}
    \caption{proceso de análisis y diseño orientado a servicios}
    \label{fig:seis}
    \textbf{Fuente:}  \cite{soa_principles}
  \end{center}
\end{figure}

\subsection{Metas y beneficios de la computación orientada a servicios}

Las metas y los beneficios que trae la implementación de la computación orientada a servicios en una organización son:

\begin{itemize}
  \item \textbf{Incremento de la interoperabilidad intrínseca}: Es una meta de la computación orientada a servicios ya que la composición de servicios requiere que, sin necesidad de integrar un servicio con el otro, estos puedan intercambiar información para desarrollar la funcionalidad a la que han sido inscritos. Aplicando principios de diseño orientados a servicios, así como también estándares de diseño orientados a servicios, se logra el incremento de la interoperabilidad intrínseca.
  \item \textbf{Federación incrementada}: Un entorno de TI federado es aquel en el cual los recursos y aplicaciones se manejan y gobiernan con autonomía y por sí mismos. En el ámbito de la orientación a servicios, cada servicio puede tener su propia implementación independiente y aun así comunicarse. Esto se logra por medio de una especial atención a los estándares de diseño.
  \item \textbf{Incremento de opciones en la escogencia de proveedores de servicios}: Como la federación en la computación orientada a servicios es una meta, la escogencia de proveedores de servicios diferentes es un beneficio ya que la composición de servicios puede ser lograda sin importar que proveedor provea de un servicio específico.
  \item \textbf{Incremento de la alineación entre el negocio y las tecnologías}: Ya que en el proceso de diseño actúan tanto el analista de negocio como el arquitecto de software, la alineación entre negocio y tecnologías es incrementada. La reconfiguración en la composición de servicios, además, provee alineación extra al poder cambiar el proceso de negocio y la composición de los servicios al mismo tiempo.
  \item \textbf{ROI}: Al hacerse inventarios de servicios y, además, ser los servicios reutilizables a lo largo del tiempo y compuestos de diferente forma gracias a la composición de servicios, se da una relación costo/beneficio más baja que con otros paradigmas usados.
  \item \textbf{Agilidad organizacional incrementada}: Debido a la orientación a servicios, se hace una composición rápida de los servicios que se tengan y se crean los que se necesiten para agilizar el proceso del departamento de TI y así agilizar los procesos subyacentes.
  \item \textbf{Reducción de cargas al departamento TI}: Debido a la agilidad organizacional cuando es aplicada la orientación a servicios, son reducidos costos operacionales (tiempo u otros recursos) y el departamento de TI adquiere un papel activo en el sector estratégico.
\end{itemize}

\subsection{Principios de diseño SOA}

Según \cite{soa_principles}, hay ocho principios de diseño sobre la computación orientada a servicios. A continuación se dará una definición de cada uno de ellos.

\begin{itemize}
  \item \textbf{Standardized Service Contract (Contrato de servicio estandarizado)}: Exposición de capacidades de los servicios por medio de los contratos de servicio.

 \item \textbf{Service Loose Coupling (Bajo acoplamiento de servicios)}: Este principio busca el desacoplamiento de la implementación de los servicios y los consumidores de dichos servicios (sus capacidades), así como también del contrato de servicio. Este principio garantiza la interoperabilidad servidor – contrato – consumidor, sin necesidad de tener un alto acoplamiento entre los tres.

 \item \textbf{Service Abstraction (Abstracción de servicio)}: Este principio enfatiza en mostrar las capacidades de servicio que sean necesarias, escondiendo aquellas que no deben publicarse en los contratos de servicio. El manejo de metadata es imprescindible en este principo debido a que, por la aplicación de este, los servicios tenderán a ser mayormente agnósticos y podrán ser utilizados en diversas composiciones de servicios, así como también en la publicación de los mismos (para que sean descubribles).

 \item \textbf{Service reusability (Reusabilidad de servicio)}: Enfatiza en la necesidad de la creación de servicios agnósticos para que estos puedan ser utilizados en gran variedad de escenarios y tengan una vida útil más larga.

 \item \textbf{Service Autonomy (Autonomia de servicio)}: Se enfoca en la necesidad de un servicio de controlar su ambiente (entorno y recursos propios) en orden de volver este un servicio confiable y predecible.

 \item \textbf{Service Statelessness (Servicios sin estado)}: Se enfoca en la construcción de servicios que, en lo posible, presciendan de utilizar estados (guardar información dentro de ellos mismos para utilizarla en un futuro). Si la cantidad de estados es excesiva, el servicio perderá la capacidad de ser escalable y disponible.

 \item \textbf{Service Discoverability (Capacidad de descubrimiento de servicios)}: Se centra en el diseño de servicios entendibles y facilmente identificados, haciendo posible el aumento del ROI (Return On Invesment). 

 \item \textbf{Service Composability (Capacidad de composición de servicios)}: El más complejo de los principios, que se enfoca en la composición de servicios, en la variedad de configuraciones con que estos pueden hacerse una vez la creación de los servicios sea lograda con la utilización de los principios anteriormente enunciados.
\end{itemize}

