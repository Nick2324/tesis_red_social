\begin{table}[!htb]
	\caption{CU065-Gestión de niveles de juego: Descripción}
	\label{tab:cu065_desc}
	\begin{center}
		\resizebox{15cm}{!}{
		\begin{tabular}{|p{4cm}|p{11cm}|}
			\hline
			\multicolumn{2}{|c|}{Descripción de caso de uso} \\
			\hline
			Nombre & Gestión de niveles de juego \\
			\hline
			Identificador & CU065 \\
			\hline
			Descripción & Lleva un análisis del nivel de juego de un jugador/equipo respecto de sus estadísticas y las manejadas en el resto de la red social, así como también las estadísticas de nivel de juego manejadas usualmente en un lugar donde hayan prácticas deportivas \\
			\hline
			Actor & 		 
			\\
			\hline
			Disparador & El usuario ingresa datos estadísticos de un equipo/jugador a la red social o se obtienen datos de un lugar deportivo, el nivel de juego manejado en dicho lugar \\
			\hline
			Inclusiones &  \\
			\hline
			Puntos de extensión & 
			\begin{itemize}
				\item Gestión de niveles de juego
			\end{itemize}
			\\
			\hline
			Precondiciones &  
			\begin{itemize}
				\item La aplicación ha sido cargada como entrenador, equipo deportivo o jugador
				\item El usuario ingresa datos estadísticos del equipo/jugador o la red social obtiene datos de lugares deportivos
			\end{itemize}
			\\
			\hline
			Postcondiciones & 
			\begin{itemize}
				\item Se calculan datos estadísticos del equipo
			\end{itemize}
			\\
			\hline
			Notas & Los datos pueden ser accesados por cualquier usuario de la red social
			\\
			\hline
		\end{tabular}
		} \\
		\textbf{Fuente}: Autores
	\end{center}
\end{table}

\begin{table}[!htb]
	\caption{CU065-Gestión de niveles de juego: Flujos de hechos}
	\label{tab:cu065_flujo}
	\begin{center}
		\resizebox{15cm}{!}{
		\begin{tabular}{|p{1.5cm}|p{6cm}|p{6.5cm}|}
			\hline
			\multicolumn{3}{|c|}{Detalle de flujo de hechos de caso de uso} \\
			\hline
			Nombre & \multicolumn{2}{|c|}{Nombre del flujo} \\
			\hline
			Paso & Acción del actor & Respuesta del sistema \\
			\hline
			1 & El usuario ha ingresado datos a la red social & El sistema ha aprovechado esos datos para calcular el nivel de juego del usuario a quien apuntan los datos ingresados, según los parámetros de la red social \\
			\hline
		\end{tabular}
		} \\
		\textbf{Fuente}: Autores
	\end{center}
\end{table}