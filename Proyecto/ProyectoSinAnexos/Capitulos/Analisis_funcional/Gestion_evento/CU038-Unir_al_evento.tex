\begin{table}[!htb]
	\caption{CU038-Unir al evento: Descripción}
	\label{tab:cu038_desc}
	\begin{center}
		\resizebox{15cm}{!}{
		\begin{tabular}{|p{4cm}|p{11cm}|}
			\hline
			\multicolumn{2}{|c|}{Descripción de caso de uso} \\
			\hline
			Nombre & Unir al evento \\
			\hline
			Identificador & CU038 \\
			\hline
			Descripción & Permite gestionar la unión a un evento \\
			\hline
			Actor & Todo actor de la red social	 
			\\
			\hline
			Disparador & Se elige la opción de acciones a realizar sobre el evento \\
			\hline
			Inclusiones & \\
			\hline
			Puntos de extensión & \\
			\hline
			Precondiciones & \\
			\hline
			Postcondiciones & \\
			\hline
			Notas &
			\begin{itemize}
				\item Generalización de:
				\begin{itemize}
					\item Unir al evento como participante
					\item Unir al evento como espectador
				\end{itemize}
			\end{itemize}			 
			\\
			\hline
		\end{tabular}
		} \\
		\textbf{Fuente}: Autores
	\end{center}
\end{table}

\begin{table}[!htb]
	\caption{CU038-Unir al evento: Flujos de hechos}
	\label{tab:cu038_flujo}
	\begin{center}
		\resizebox{15cm}{!}{
		\begin{tabular}{|p{1.5cm}|p{6cm}|p{6.5cm}|}
			\hline
			\multicolumn{3}{|c|}{Detalle de flujo de hechos de caso de uso} \\
			\hline
			Nombre & \multicolumn{2}{|c|}{Nombre del flujo} \\
			\hline
			Paso & Acción del actor & Respuesta del sistema \\
			\hline
			1 & El usuario entra al timeline de un evento y elige la opción de acciones a realizar sobre el evento & El sistema muestra el menú de acciones a realizar sobre el evento \\
			\hline
		\end{tabular}
		} \\
		\textbf{Fuente}: Autores
	\end{center}
\end{table}