\begin{table}[!htb]
	\caption{CU120-Consultar noticias de foros: Descripción}
	\label{tab:cu120_desc}
	\begin{center}
		\resizebox{15cm}{!}{
		\begin{tabular}{|p{4cm}|p{11cm}|}
			\hline
			\multicolumn{2}{|c|}{Descripción de caso de uso} \\
			\hline
			Nombre & Consultar noticias de foros\\
			\hline
			Identificador & CU120\\
			\hline
			Descripcion & Funcionalidad que le permite al usuario consultar noticias que fueron publicadas en la aplicación.\\
			\hline
			Actor & 
			\begin{itemize}
				\item Jugador
				\item Entrenador
				\item Organización
			\end{itemize} \\
			\hline
			Disparador & El usuario quiere leer noticias originadas en la aplicación\\
			\hline
			Inclusiones & \\
			\hline
			Extensiones & \\
			\hline
			Precondiciones & \\
			\hline
			Postcondiciones & \\
			\hline
			Notas & \\
			\hline
		\end{tabular}
		} \\
		\textbf{Fuente}: Autores
	\end{center}
\end{table}
\begin{table}[!htb]
	\caption{CU120-Consultar noticias de foros:Flujos de hechos}
	\label{tab:cu120_flujo}
	\begin{center}
		\resizebox{15cm}{!}{
		\begin{tabular}{|p{1.5cm}|p{6cm}|p{6.5cm}|}
			\hline
			\multicolumn{3}{|c|}{Descripción de caso de uso} \\
			\hline
			Nombre & \multicolumn{2}{|c|}{Nombre del flujo} \\
			\hline
			Paso & Acción del actor & Respuesta del sistema\\
			\hline
				1 & El usuario selecciona la opción Noticias & Se despliega el listado de noticias mas recientes.\\
				2 & El usuario selecciona la opción Fuentes & Se despliega el menú de fuentes.\\
				3 & El usuario marca como fuente "Foros" y regresa al listado de noticias & Se actualiza el listado de noticias para mostrar las noticias extraidas de los foros.\\
			\hline
		\end{tabular}
		} \\
		\textbf{Fuente}: Autores
	\end{center}
\end{table}
