\begin{table}[!htb]
	\caption{CU152-Generar mapa de ubicaciones cercanas: Descripción}
	\label{tab:cu152_desc}
	\begin{center}
		\resizebox{15cm}{!}{
		\begin{tabular}{|p{4cm}|p{11cm}|}
			\hline
			\multicolumn{2}{|c|}{Descripción de caso de uso} \\
			\hline
			Nombre & Generar mapa de ubicaciones cercanas\\
			\hline
			Identificador & CU149\\
			\hline
			Descripcion & Funcionalidad que le permite al usuario ver que ubicaciones están cerca de él.\\
			\hline
			Actor & 
			\begin{itemize}
				\item Actor deportivo
			\end{itemize} \\
			\hline
			Disparador & El usuario quiere saber que ubicaciones están cerca de él.\\
			\hline
			Inclusiones & CU150-Determinar ubicación\\
			\hline
			Extensiones & \\
			\hline
			Precondiciones & \\
			\hline
			Postcondiciones & \\
			\hline
			Notas & \\
			\hline
		\end{tabular}
		} \\
		\textbf{Fuente}: Autores
	\end{center}
\end{table}
\begin{table}[!htb]
	\caption{CU152-Generar mapa de ubicaciones cercanas:Flujos de hechos}
	\label{tab:cu152_flujo}
	\begin{center}
		\resizebox{15cm}{!}{
		\begin{tabular}{|p{1.5cm}|p{6cm}|p{6.5cm}|}
			\hline
			\multicolumn{3}{|c|}{Descripción de caso de uso} \\
			\hline
			Nombre & \multicolumn{2}{|c|}{Nombre del flujo} \\
			\hline
			Paso & Acción del actor & Respuesta del sistema\\
			\hline
				1 & El usuario selecciona la opción ¿Qué hay cerca de mi?. & Se solicita confirmación del usuario para obtener su ubicación actual.\\
				2 & El usuario confirma la obtencion de su ubicación. & Se despliega un mapa con las ubicaciones cercanas a la ubicación del usuario.\\
			\hline
		\end{tabular}
		} \\
		\textbf{Fuente}: Autores
	\end{center}
\end{table}
