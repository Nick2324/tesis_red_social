\begin{table}[!htb]
	\caption{CU151-Registrar nueva ubicación: Descripción}
	\label{tab:cu151_desc}
	\begin{center}
		\resizebox{15cm}{!}{
		\begin{tabular}{|p{4cm}|p{11cm}|}
			\hline
			\multicolumn{2}{|c|}{Descripción de caso de uso} \\
			\hline
			Nombre & Registrar nueva ubicación\\
			\hline
			Identificador & CU151\\
			\hline
			Descripcion & Funcionalidad que le permite al usuario registrar una nueva ubicación en la aplicación.\\
			\hline
			Actor & 
			\begin{itemize}
				\item Actor deportivo
			\end{itemize} \\
			\hline
			Disparador & El usuario quiere registrar una nueva ubicación.\\
			\hline
			Inclusiones & CU151-Determinar ubicación\\
			\hline
			Extensiones & \\
			\hline
			Precondiciones & \\
			\hline
			Postcondiciones & \\
			\hline
			Notas & \\
			\hline
		\end{tabular}
		} \\
		\textbf{Fuente}: Autores
	\end{center}
\end{table}
\begin{table}[!htb]
	\caption{CU151-Registrar nueva ubicación:Flujos de hechos}
	\label{tab:cu151_flujo}
	\begin{center}
		\resizebox{15cm}{!}{
		\begin{tabular}{|p{1.5cm}|p{6cm}|p{6.5cm}|}
			\hline
			\multicolumn{3}{|c|}{Descripción de caso de uso} \\
			\hline
			Nombre & \multicolumn{2}{|c|}{Nombre del flujo} \\
			\hline
			Paso & Acción del actor & Respuesta del sistema\\
			\hline
				1 & El usuario selecciona la opción registrar ubicación & Se despliega el formulario de registro de nueva ubicación.\\
				2 & El usuario selecciona la opción Obtener mi ubicación. & Se obtiene la ubicación del usuario y se completan los campos asociados a la misma en el formularo de registro.\\
				3 & Luego de llenar los campos faltantes, el usuario selecciona la opcion Registrar ubicación. & Se solicita confirmación de la acción\\
				4 & El usuario confirma la acción & Se muestra mensaje informando el exito de la operación. El sistema regresa al punto desde el que se invocó el registro.\\
			\hline
		\end{tabular}
		} \\
		\textbf{Fuente}: Autores
	\end{center}
\end{table}
