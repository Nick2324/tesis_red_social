\begin{table}[!htb]
	\caption{CU002-Agregar un deporte: Descripción}
	\label{tab:cu002_desc}
	\begin{center}
		\resizebox{15cm}{!}{
		\begin{tabular}{|p{4cm}|p{11cm}|}
			\hline
			\multicolumn{2}{|c|}{Descripción de caso de uso} \\
			\hline
			Nombre & Agregar un deporte \\
			\hline
			Identificador & CU002\\
			\hline
			Descripción & Permite agregar a la red social un nuevo deporte \\
			\hline
			Actor & 
			\begin{itemize}
				\item Administrador
			\end{itemize} \\
			\hline
			Disparador & El administrador desea crear un nuevo deporte en la red social y activa dicha opción \\
			\hline
			Inclusiones & \\
			\hline
			Puntos de extensión & \\
			\hline
			Precondiciones & 
			\begin{itemize}
				\item El SNS se inicia con rol de administrador
			\end{itemize} \\
			\hline
			Postcondiciones & 
			\begin{itemize}
				\item El administrador crea un nuevo deporte sobre la red social
			\end{itemize}						
			\\
			\hline
			Notas & \\
			\hline
		\end{tabular}
		} \\
		\textbf{Fuente}: Autores
	\end{center}
\end{table}

\begin{table}[!htb]
	\caption{CU002-Agregar un deporte: Flujos de hechos }
	\label{tab:cu002_flujo}
	\begin{center}
		\resizebox{15cm}{!}{
		\begin{tabular}{|p{1.5cm}|p{6cm}|p{6.5cm}|}
			\hline
			\multicolumn{3}{|c|}{Detalle de flujo de hechos de caso de uso} \\
			\hline
			Nombre & \multicolumn{2}{|c|}{Nombre del flujo} \\
			\hline
			Paso & Acción del actor & Respuesta del sistema \\
			\hline
			1 & El usuario pulsa el botón para administrar deportes & El sistema se dirige a la muestra de los deportes existentes en la red social \\
			\hline
			2 & El usuario pulsa el botón para crear un nuevo deporte & El sistema muestra los campos a llenar para el registro del nuevo deporte en la red social \\
			\hline
			3 & El usuario llena los datos que la red social requiere para crear el nuevo deporte sobre ella & \\
			\hline
			4 & El usuario pulsa el botón para crear el deporte & El sistema guarda el deporte si éste no existe ya \\
			\hline
			5 & & El sistema muestra el estado del sistema (creación exitosa o fallida) \\
			\hline
			6 & El usuario recibe el mensaje y decide realizar alguna otra opción & \\
			\hline
		\end{tabular}
		} \\
		\textbf{Fuente}: Autores
	\end{center}
\end{table}