\begin{table}[!htb]
	\caption{CU096-Crear post: Descripción}
	\label{tab:cu096_desc}
	\begin{center}
		\resizebox{15cm}{!}{
		\begin{tabular}{|p{4cm}|p{11cm}|}
			\hline
			\multicolumn{2}{|c|}{Descripción de caso de uso} \\
			\hline
			Nombre & Crear post \\
			\hline
			Identificador & CU096 \\
			\hline
			Descripción & Permite la creación de un post sobre un timeline \\
			\hline
			Actor & 
			\begin{itemize}
				\item Todos
			\end{itemize} \\
			\hline
			Disparador & El actor desea crear un post sobre un timeline propio o ajeno, según sus privilegios se lo permitan \\
			\hline
			Inclusiones & \\
			\hline
			Puntos de extensión & \\
			\hline
			Precondiciones & 
			\begin{itemize}
				\item El actor tiene privilegios para la creación del post
			\end{itemize} \\
			\hline
			Postcondiciones &
			\begin{itemize}
				\item El actor crea un post sobre el timeline elegido
			\end{itemize} \\
			\hline
			Notas & 
			\begin{itemize}
				\item Soporta el uso tanto en los actores deportivos como en los eventos. Esta característica es desencadenada por todos los casos de uso que se desprenden de éste
			\end{itemize} \\
			\hline 
		\end{tabular}
		} \\
		\textbf{Fuente}: Autores
	\end{center}
\end{table}

\begin{table}[!htb]
	\caption{CU096-Crear post: Flujos de hechos}
	\label{tab:cu096_flujo}
	\begin{center}
		\resizebox{15cm}{!}{
		\begin{tabular}{|p{1.5cm}|p{6cm}|p{6.5cm}|}
			\hline
			\multicolumn{3}{|c|}{Detalle de flujo de hechos de caso de uso} \\
			\hline
			Nombre & \multicolumn{2}{|c|}{Nombre del flujo} \\
			\hline
			Paso & Acción del actor & Respuesta del sistema \\
			\hline
			1 & El usuario ha elegido gestionar timeline & El sistema ha mostrado el timeline propio o de otro actor/evento en la red social \\
			\hline
			2 & El usuario escribe en el cuadro de texto dirigido a la escritura de posts y elige publicarlo & El sistema guarda el post y lo publica en el timeline \\
			\hline
			3 & & De haber un error, el sistema mostrará un mensaje indicándolo \\
			\hline
			4 & & El sistema ubica al usuario en el timeline \\
			\hline
		\end{tabular}
		} \\
		\textbf{Fuente}: Autores
	\end{center}
\end{table}