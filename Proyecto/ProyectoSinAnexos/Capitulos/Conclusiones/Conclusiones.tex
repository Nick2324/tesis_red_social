En cuanto al campo analizado para la obtención de un producto de software que sirva al sector deportivo, con la investigación se pudo observar la importancia que tiene el uso de tecnologías móviles debido a la facilidad de geolocalización que éstas ofrecen. A su vez, teniendo en cuenta la escogencia de una red social que se orienta a un sector de edades que no sobrepasan los 40 años, se puede decir que el factor de éxito podrá ser mayor por la demanda de redes sociales por personas por debajo de dicha edad. Dada la encuesta sobre el ámbito universitario y encontrando que más de la mitad aseveraban practicar al menos un deporte y estar por debajo del umbral de los 40 años, se pudo asegurar una población objetivo adecuada. En ésta encuesta, el aspecto más importante fue encontrar que los jovenes no sólo utilizaban los metodos usuales para la consulta de lugares para practicar deportes (amigos), sino que también utilizaban redes sociales como CouchSurfing, lo cual abre la puerta a que ellos prefieran una red social enfocada a deportes y encontrada por medios electrónicos (en nuestro caso, un móvil).

Cuando los autores recolectaron las funcionalidades ofrecidas en éste momento por las SNS deportivas escogidas para estudio, se dieron cuenta de cómo se podía articular todo sobre el ámbito móvil, debido a que la mitad de ellas hacían énfasis en encuentros deportivos, búsqueda de lugares deportivos y búsqueda de personas para practicar un deporte. Más allá de éstas funcionalidades, las demás tienen la capacidad de ser adecuadas al entorno móvil con tal de que el usuario pueda disfrutar tanto de las funcionalidades para él claves como de otras con las que lo dotamos.

Debido al tiempo limitado con el que los autores contaron para el desarrollo del prototipo, se dio por sentado que no era posible conseguir abarcar todos los requerimientos no funcionales que, al final, se necesitarían en el modelo que se lanzaría al mercado. Por ésta razón, los autores decidieron ceñirse a sentar bases de cada uno definiendo la arquitectura, haciendo un análisis funcional y un diseño de interfaz de usuario de las funcionalidades en su totalidad, sin embargo, mostrando en el prototipo generado en la fase de construcción las características que más captaban la atención de los usuarios, esto es, aquellas que tenían que ver con encuentros deportivos y geolocalización de los mismos.

En vista de lo investigado, los autores pudieron dar una filosofía al software desarrollado, orientando éste hacia el paradigma orientado a servicios, aprovechando la capacidad de éste para el desarrollo de servicios reusables, descubribles, con nivel de abstracción logrado gracias a la orquestación de sus servicios (composición de los mismos).

Asímismo, en la etapa de modelado de interfaces de usuario, se puede notar cómo se comienza a moldear una filosofía (más allá de la impuesta por las tecnologías Android escogídas) que afectará de forma crítica tanto la evolución del prototipo hacia un producto listo para salir al mercado como el éxito de éstas por concepto de usabilidad del software.

Una vez terminada la fase de diseño, los autores hicieron un análisis a la arquitectura hecha, escogiendo en base a ella las funcionalidades primarias que irían en el prototipo en la fase de construcción, basandose en los métodos utilizados y en el estado del arte. Debido a la conexión existente entre las funcionalidades y la arquitectura en la que fueron ilustradas, una vez elegidas las funcionalidades fueron elegidos aquellos servicios que se podían trazar desde ellas y, utilizando la reusabilidad táctica, fueron desarrollados los módulos que dan soporte a cada una de las capacidades primordiales de estos, asegurando la disponibilidad de funcionalidades de geolocalizacion, de manejo de eventos, usuarios y deportes para ayudar a los usuarios de la SNS, en el prototipo, a llevar a cabo tareas sobre su red en dichos aspectos.