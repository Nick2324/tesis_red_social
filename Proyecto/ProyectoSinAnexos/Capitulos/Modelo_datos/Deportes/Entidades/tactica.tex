\begin{table}[!htb]
	\caption{Entidad E\_Tactica}
	\label{tab:entidad_010}
	\begin{center}
		\resizebox{16cm}{!}{
		\begin{tabular}{|p{4cm}|p{4cm}|p{4cm}|p{4cm}|}
			\hline
			\multicolumn{4}{|c|}{Entidad} \\
			\hline
			\multicolumn{2}{|c|}{Nombre} &
			\multicolumn{2}{c|}{E\_Tactica}  \\
			\hline
			\multicolumn{2}{|c|}{Descripción} &
			\multicolumn{2}{c|}{Tacticas deportivas usadas en partidos que son ingeniadas por entrenadores. Puede ser tanto una jugada como una formación (para deportes en equipo)}  \\
			\hline
			\multicolumn{4}{|c|}{Atributos} \\
			\hline
			Nombre & Tipo & Longitud & Descripción \\
			\hline
			id [P] &
			Numerico &
			&
			Identificación única de la táctica en la base de datos \\
			\hline
			nombre &
			String &
			50 &
			Nombre de la táctica \\
			\hline
			descripcion &
			String &
			1000 &
			Descripción de la táctica \\
			\hline
			\multicolumn{4}{|c|}{Relaciones a entidades} \\
			\hline
			\multicolumn{2}{|c|}{Relación} & \multicolumn{2}{c|}{Entidad} \\
			\hline
			\multicolumn{2}{|c|}{R\_RelacionImagen} & 
			\multicolumn{2}{c|}{E\_Imagen} \\
			\hline
			\multicolumn{2}{|c|}{R\_RelacionVideo} & 
			\multicolumn{2}{c|}{E\_Video} \\
			\hline
		\end{tabular}
		} \\
		\textbf{Fuente}: Autores
	\end{center}
\end{table}