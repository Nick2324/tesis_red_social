% Please add the following required packages to your document preamble:
% \usepackage[table,xcdraw]{xcolor}
% If you use beamer only pass "xcolor=table" option, i.e. \documentclass[xcolor=table]{beamer}
\begin{table}[!htb]
	\caption{Relacion R\_FotoAlbum}
	\label{tab:relacion_fotoalbum}
	\begin{center}
		\resizebox{16cm}{!}{
\begin{tabular}{|l|l|l|l|}
\hline
\multicolumn{4}{|c|}{\textbf{Relacion}} \\ \hline
\multicolumn{2}{|l|}{\textbf{Nombre}} & \multicolumn{2}{l|}{R\_FotoAlbum} \\ \hline
\multicolumn{2}{|l|}{\textbf{Descripcion}} & \multicolumn{2}{l|}{Se crea cuando una relación por cada foto que haga parte del album.} \\ \hline
\multicolumn{4}{|c|}{\textbf{Atributos}} \\ \hline
Nombre & Tipo & Longitud & Descripción \\
			\hline
desde & String & 10 & Fecha desde que la foto hace parte del album. Formato DD/MM/YYYY \\ \hline
Tipo de relación & \multicolumn{3}{c|}{Asimetrica} \\ \hline
\multicolumn{4}{|l|}{\textbf{Relaciones a entidades}} \\ \hline
\multicolumn{2}{|l|}{} & \multicolumn{2}{l|}{} \\ \hline
\end{tabular}
		} \\
		\textbf{Fuente}: Autores
		\end{center}
	\end{table}