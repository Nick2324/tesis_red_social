% Please add the following required packages to your document preamble:
% \usepackage[table,xcdraw]{xcolor}
% If you use beamer only pass "xcolor=table" option, i.e. \documentclass[xcolor=table]{beamer}
\begin{table}[!htb]
	\caption{Entidad E\_Timeline}
	\label{tab:entidad_timeline}
	\begin{center}
		\resizebox{16cm}{!}{
\begin{tabular}{|l|l|l|l|l|}
\hline
\multicolumn{5}{|c|}{\textbf{Entidad}} \\ \hline
\textbf{Nombre} & \multicolumn{4}{l|}{E\_Timeline} \\ \hline
\textbf{Descripcion} & \multicolumn{4}{l|}{Hace referencia al muro donde un usuario puede publicar un post.} \\ \hline
\multicolumn{5}{|c|}{\textbf{Atributos}} \\ \hline
Nombre & Tipo & Longitud & Descripción \\
			\hline
\multicolumn{2}{|l|}{idTimeline[P]} & String & 30 &  Representa el identificador del timeline, utilizado para detectar en donde se publica un post. \\ \hline
\multicolumn{5}{|c|}{\textbf{Relaciones a entidades}} \\ \hline
\multicolumn{2}{|l|}{\textbf{Relacion}} & \multicolumn{3}{l|}{\textbf{Entidad}} \\ \hline
\multicolumn{2}{|l|}{TimelineUsuario} & \multicolumn{3}{l|}{Usuario} \\ \hline
\multicolumn{2}{|l|}{TimelineEvento} & \multicolumn{3}{l|}{Evento} \\ \hline
\multicolumn{2}{|l|}{TimelineGrupo} & \multicolumn{3}{l|}{Grupo} \\ \hline
\end{tabular}
		} \\
		\textbf{Fuente}: Autores
		\end{center}
	\end{table}