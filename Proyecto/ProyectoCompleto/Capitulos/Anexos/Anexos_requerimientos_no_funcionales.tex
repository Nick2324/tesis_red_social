\chapter{Escenarios de calidad de requerimientos no funcionales en detalle}
\label{app:req_no_funcionales}

A continuación se muestran todos los escenarios de calidad tenidos en cuenta para cada uno de los requerimientos no funcionales que se buscan tener en cuenta al momento de realizar el prototipo.

\section{QiU}

En esta sección se exponen, de la tabla \ref{tab:ec1_qiu} a \ref{tab:ec8_qiu}, escenarios de calidad correspondientes a las áreas de usabilidad y UX.

\begin{table}[!htb]
	\caption{QiU - Escenario de calidad 1}
	\label{tab:ec1_qiu}
	\begin{center}
		\resizebox{11cm}{!}{
		\begin{tabular}{|p{4cm}|p{7cm}|}
			\hline
			\multicolumn{2}{|c|}{Escenario de calidad} \\
			\hline
			Estímulo & El usuario requiere realizar todas las tareas que él necesite en el SNS\\
			\hline
			Fuente & Usuario final\\
			\hline
			Ambiente & Producción, pruebas\\
			\hline
			Elementos & Capa de presentación, capa de negocio\\
			\hline
			Respuesta & El usuario logra hacer las tareas que requiere\\
			\hline
			Medidas de respuesta & El usuario logra hacer el 80\% de las tareas que requiere en el SNS deportivo (medida hechas sobre las funcionalidades que este ofrece)\\
			\hline
		\end{tabular}
		} \\
		\textbf{Fuente}: Autores
	\end{center}
\end{table}

\begin{table}[!htb]
	\caption{QiU - Escenario de calidad 2}
	\label{tab:ec2_qiu}
	\begin{center}
		\resizebox{11cm}{!}{
		\begin{tabular}{|p{4cm}|p{7cm}|}
			\hline
			\multicolumn{2}{|c|}{Escenario de calidad} \\
			\hline
			Estímulo & Se requiere que las funcionalidades sean ejecutadas eficientemente, es decir, en un tiempo corto \\
			\hline
			Fuente & Usuario final \\
			\hline
			Ambiente & Producción, pruebas \\
			\hline
			Elementos & Capa de presentación, capa de negocio \\
			\hline
			Respuesta & La capacidad que tiene el usuario de realizar una tarea de forma rápida en la aplicación es aceptable\\
			\hline
			Medidas de respuesta & Al usuario le toma una media de 30 segundos realizar las tareas que requiere en la red social\\
			\hline
		\end{tabular}
		} \\
		\textbf{Fuente}: Autores
	\end{center}
\end{table}

\begin{table}[!htb]
	\caption{QiU - Escenario de calidad 3}
	\label{tab:ec3_qiu}
	\begin{center}
		\resizebox{11cm}{!}{
		\begin{tabular}{|p{4cm}|p{7cm}|}
			\hline
			\multicolumn{2}{|c|}{Escenario de calidad} \\
			\hline
			Estímulo & Se requiere que los usuarios estén satisfechos con la interfaz: Una interfáz práctica que los haga, a su vez, sentir bien (UX) \\
			\hline
			Fuente & Usuario final \\
			\hline
			Ambiente & Producción \\
			\hline
			Elementos & Capa de presentación, capa de negocio \\
			\hline
			Respuesta & El usuario se siente a gusto con la aplicación y además logra hacer las tareas que él requiere facilmente \\
			\hline
			Medidas de respuesta & La medida se satisfacción del usuario está en el 65\% \\
			\hline
		\end{tabular}
		} \\
		\textbf{Fuente}: Autores
	\end{center}
\end{table}

\begin{table}[!htb]
	\caption{QiU - Escenario de calidad 4}
	\label{tab:ec4_qiu}
	\begin{center}
		\resizebox{11cm}{!}{
		\begin{tabular}{|p{4cm}|p{7cm}|}
			\hline
			\multicolumn{2}{|c|}{Escenario de calidad} \\
			\hline
			Estímulo & Se requiere que el usuario aprenda a utilizar las funcionalidades gruesas del software (las principales) en poco tiempo \\
			\hline
			Fuente & Usuario final \\
			\hline
			Ambiente & Producción, pruebas \\
			\hline
			Elementos & Capa de presentación, capa de negocio \\
			\hline
			Respuesta & Las curvas de aprendizaje referentes a las funcionalidades gruesas del software según el tipo de usuario son empinadas \\
			\hline
			Medidas de respuesta & El usuario aprende a utilizar las funciones de su rol en menos de 1 hora \\
			\hline
		\end{tabular}
		} \\
		\textbf{Fuente}: Autores
	\end{center}
\end{table}

\begin{table}[!htb]
	\caption{QiU - Escenario de calidad 5}
	\label{tab:ec5_qiu}
	\begin{center}
		\resizebox{11cm}{!}{
		\begin{tabular}{|p{4cm}|p{7cm}|}
			\hline
			\multicolumn{2}{|c|}{Escenario de calidad} \\
			\hline
			Estímulo & Se requiere dar confianza al usuario, que este pueda entender cada mensaje de posible error que se pueda presentar en el SNS \\
			\hline
			Fuente & Usuario final \\
			\hline
			Ambiente & Producción \\
			\hline
			Elementos & Capa de presentación, capa de negocio \\
			\hline
			Respuesta & El usuario confía en el software puesto que éste muestra de manera clara los errores presentados por el software \\
			\hline
			Medidas de respuesta & El usuario expresa que el 95\% o más de los errores que presenta el software son presentados en un lenguaje natural, entendible para él \\
			\hline
		\end{tabular}
		} \\
		\textbf{Fuente}: Autores
	\end{center}
\end{table}

\begin{table}[!htb]
	\caption{QiU - Escenario de calidad 6}
	\label{tab:ec6_qiu}
	\begin{center}
		\resizebox{11cm}{!}{
		\begin{tabular}{|p{4cm}|p{7cm}|}
			\hline
			\multicolumn{2}{|c|}{Escenario de calidad} \\
			\hline
			Estímulo & Se requiere que el usuario sepa cuales son los diferentes roles que actúan en el SNS \\
			\hline
			Fuente & Usuario final \\
			\hline
			Ambiente & Análisis \\
			\hline
			Elementos & Capa de presentación, capa de negocio \\
			\hline
			Respuesta & El software muestra roles claros a cada tipo de usuario en su interfaz \\
			\hline
			Medidas de respuesta & Los usuarios identifican con rapidéz cuales son los roles existentes en la aplicación, teniendo una efectividad del 90\% \\
			\hline
		\end{tabular}
		} \\
		\textbf{Fuente}: Autores
	\end{center}
\end{table}

\begin{table}[!htb]
	\caption{QiU - Escenario de calidad 7}
	\label{tab:ec7_qiu}
	\begin{center}
		\resizebox{11cm}{!}{
		\begin{tabular}{|p{4cm}|p{7cm}|}
			\hline
			\multicolumn{2}{|c|}{Escenario de calidad} \\
			\hline
			Estímulo & Se requiere que el usuario sepa cuales son las funcionalidades ofrecidas por el SNS \\
			\hline
			Fuente & Usuario final \\
			\hline
			Ambiente & Producción, pruebas \\
			\hline
			Elementos & Documentación, capa de presentación \\
			\hline
			Respuesta & El usuario puede reconocer las fuciones que le ofrece el software y utilizarlas para sus propositos \\
			\hline
			Medidas de respuesta & El usuario logra reconocer el 80\% de las funcionalidades que le ofrece el software según su rol o roles \\
			\hline
		\end{tabular}
		} \\
		\textbf{Fuente}: Autores
	\end{center}
\end{table}

\begin{table}[!htb]
	\caption{QiU - Escenario de calidad 8}
	\label{tab:ec8_qiu}
	\begin{center}
		\resizebox{11cm}{!}{
		\begin{tabular}{|p{4cm}|p{7cm}|}
			\hline
			\multicolumn{2}{|c|}{Escenario de calidad} \\
			\hline
			Estímulo & Se requiere dar al usuario confianza en el software, que él sepa siempre que hace el software \\
			\hline
			Fuente & Usuario final \\
			\hline
			Ambiente & Producción, pruebas \\
			\hline
			Elementos & Capa de presentación \\
			\hline
			Respuesta & El usuario sabe que funcionalidad está ejecutandose en el software en la mayoría del tiempo \\
			\hline
			Medidas de respuesta & El usuario logra saber con una efectividad del 90\% que funcionalidad está ejecutando el software en determinado momento \\
			\hline
		\end{tabular}
		} \\
		\textbf{Fuente}: Autores
	\end{center}
\end{table}

\section{Reusabilidad}

De la tabla \ref{tab:ec1_reusabilidad} a \ref{tab:ec3_reusabilidad} se exponen los escenarios de calidad basados en el principio de diseño del paradigma orientado a servicios reusabilidad.

\begin{table}[!htb]
	\caption{Reusabilidad - Escenario de calidad 1}
	\label{tab:ec1_reusabilidad}
	\begin{center}
		\resizebox{11cm}{!}{
		\begin{tabular}{|p{4cm}|p{7cm}|}
			\hline
			\multicolumn{2}{|c|}{Escenario de calidad} \\
			\hline
			Estímulo & Se requiere que el software cumpla una reusabilidad táctica \\
			\hline
			Fuente & Equipo de desarrollo, propietario del software \\
			\hline
			Ambiente & Diseño \\
			\hline
			Elementos & Capa de negocio, capa de persistencia \\
			\hline
			Respuesta & Las capacidades de los servicios deben tener una perfecta correspondencia con los requerimientos sin adicionar capacidades sin uso \\
			\hline
			Medidas de respuesta & Todos las capacidades puestas a los servicios son  utilizadas en su composición con otros servicios y responden a los requerimientos funcionales iniciales \\
			\hline
		\end{tabular}
		} \\
		\textbf{Fuente}: Autores
	\end{center}
\end{table}

\begin{table}[!htb]
	\caption{Reusabilidad - Escenario de calidad 2}
	\label{tab:ec2_reusabilidad}
	\begin{center}
		\resizebox{11cm}{!}{
		\begin{tabular}{|p{4cm}|p{7cm}|}
			\hline
			\multicolumn{2}{|c|}{Escenario de calidad} \\
			\hline
			Estímulo & El software debe tener una cantidad mayor de servicios agnósticos que de servicios del tipo ''task'' \\
			\hline
			Fuente & Equipo de desarrollo \\
			\hline
			Ambiente & Diseño \\
			\hline
			Elementos & Capa de negocio, capa de persistencia \\
			\hline
			Respuesta & Los blueprint de servicios son agnósticos puesto que los servicios pertenecientes a este son en su mayoría así  y, por tanto, tienen un mayor nivel de reuso \\
			\hline
			Medidas de respuesta & El 70\% de los servicios son agnósticos en su uso, esto es, son servicios del tipo ''entity'' y ''utility'' \\
			\hline
		\end{tabular}
		} \\
		\textbf{Fuente}: Autores
	\end{center}
\end{table}

\begin{table}[!htb]
	\caption{Reusabilidad - Escenario de calidad 3}
	\label{tab:ec3_reusabilidad}
	\begin{center}
		\resizebox{11cm}{!}{
		\begin{tabular}{|p{4cm}|p{7cm}|}
			\hline
			\multicolumn{2}{|c|}{Escenario de calidad} \\
			\hline
			Estímulo & Se requiere la creación de estandares de definición de contratos de software para así poder interconectar servicios sin necesidad de conversiones, volviendolos más reusables \\
			\hline
			Fuente & Equipo de desarrollo, propietario del software \\
			\hline
			Ambiente & Diseño, desarrollo \\
			\hline
			Elementos & Todas las capas \\
			\hline
			Respuesta & Hay estandares concisos para cada blueprint de servicios \\
			\hline
			Medidas de respuesta & La cantidad de elementos difusos en su definición es 0 \\
			\hline
		\end{tabular}
		} \\
		\textbf{Fuente}: Autores
	\end{center}
\end{table}

\section{Mantenibilidad}

A continuación, de las tablas \ref{tab:ec1_mantenibilidad} a \ref{tab:ec3_mantenibilidad} se exponen escenarios de calidad relacionados a la mantenibilidad.

\begin{table}[!htb]
	\caption{Mantenibilidad - Escenario de calidad 1}
	\label{tab:ec1_mantenibilidad}
	\begin{center}
		\resizebox{11cm}{!}{
		\begin{tabular}{|p{4cm}|p{7cm}|}
			\hline
			\multicolumn{2}{|c|}{Escenario de calidad} \\
			\hline
			Estímulo & Se quiere que el software tenga una fácil adaptación a capacidades nuevas que pudiese llegar a adquirir un servicio\\
			\hline
			Fuente & Propietario del software, equipo de desarrollo \\
			\hline
			Ambiente & Desarrollo, mantenimiento \\
			\hline
			Elementos & Todas las capas \\
			\hline
			Respuesta & El software es adaptable con un nivel de complejidad aceptable \\
			\hline
			Medidas de respuesta & El nivel de complejidad por unidad debe tener un nivel (+) y el nivel de duplicación debe estar en (++) \\
			\hline
		\end{tabular}
		} \\
		\textbf{Fuente}: Autores
	\end{center}
\end{table}

\begin{table}[!htb]
	\caption{Mantenibilidad - Escenario de calidad 2}
	\label{tab:ec2_mantenibilidad}
	\begin{center}
		\resizebox{11cm}{!}{
		\begin{tabular}{|p{4cm}|p{7cm}|}
			\hline
			\multicolumn{2}{|c|}{Escenario de calidad} \\
			\hline
			Estímulo & Se requiere que el software sea facilmente analizable debido a que los servicios que sean hechos pueden ser analizados con mayor facilidad \\
			\hline
			Fuente & Equipo de desarrollo \\
			\hline
			Ambiente & Desarrollo, mantenimiento \\
			\hline
			Elementos & Todas las capas \\
			\hline
			Respuesta & Los servicios tienen un tamaño manejable \\
			\hline
			Medidas de respuesta & El tamaño por unidad (esto es, por servicio) debe estar en (0) \\
			\hline
		\end{tabular}
		} \\
		\textbf{Fuente}: Autores
	\end{center}
\end{table}

\begin{table}[!htb]
	\caption{Mantenibilidad - Escenario de calidad 3}
	\label{tab:ec3_mantenibilidad}
	\begin{center}
		\resizebox{11cm}{!}{
		\begin{tabular}{|p{4cm}|p{7cm}|}
			\hline
			\multicolumn{2}{|c|}{Escenario de calidad} \\
			\hline
			Estímulo & Se requiere que los servicios tengan una complejidad manejable \\
			\hline
			Fuente & Equipo de desarrollo \\
			\hline
			Ambiente & Diseño, desarrollo, mantenimiento \\
			\hline
			Elementos & Todas las capas \\
			\hline
			Respuesta & Los servicios implementados tienen una complejidad manejable \\
			\hline
			Medidas de respuesta & La medida CBM (Complexity-based Message) debe estar por debajo de 10 y el CBO (Complexity-based Operations) debe estar por debajo de 20 \\
			\hline
		\end{tabular}
		} \\
		\textbf{Fuente}: Autores
	\end{center}
\end{table}

\section{Interoperabilidad}

En el cuadro \ref{tab:ec1_interoperabilidad} se expone la forma en la que será evaluada la interoperabilidad (ya intrinseca) del software desarrollado.

\begin{table}[!htb]
	\caption{Interoperabilidad - Escenario de calidad 1}
	\label{tab:ec1_interoperabilidad}
	\begin{center}
		\resizebox{11cm}{!}{
		\begin{tabular}{|p{4cm}|p{7cm}|}
			\hline
			\multicolumn{2}{|c|}{Escenario de calidad} \\
			\hline
			Estímulo & Se requiere adoptar una estricta política de estandarización de contratos de servicio con el fin de utilizar la mayor cantidad de servicios web ofrecidos que suplan los requerimientos funcionales \\
			\hline
			Fuente & Equipo de desarrollo, propietario del software \\
			\hline
			Ambiente & Analisis, Diseño, Desarrollo \\
			\hline
			Elementos & Todas las capas \\
			\hline
			Respuesta & Definición de estándares de forma clara en el documento de especificación \\
			\hline
			Medidas de respuesta & Alineamiento de estándares con cada artefacto trabajado en el SNS \\
			\hline
		\end{tabular}
		} \\
		\textbf{Fuente}: Autores
	\end{center}
\end{table}

\section{Seguridad}

En las tablas \ref{tab:ec1_seguridad} a \ref{tab:ec4_seguridad} se especifican los escenarios de seguridad a ser valorados.

\begin{table}[!htb]
	\caption{Seguridad - Escenario de calidad 1}
	\label{tab:ec1_seguridad}
	\begin{center}
		\resizebox{11cm}{!}{
		\begin{tabular}{|p{4cm}|p{7cm}|}
			\hline
			\multicolumn{2}{|c|}{Escenario de calidad} \\
			\hline
			Estímulo & Se requiere aplicar el principio de ''Fail Securely'', esto es, nunca mostrar información que pueda ser tratada para convertirla en una vulnerabilidad atacable \\
			\hline
			Fuente & Propietario del software \\
			\hline
			Ambiente & Desarrollo, producción, mantenimiento \\
			\hline
			Elementos & Capa de negocio, capa de persistencia \\
			\hline
			Respuesta & Ningún servicio muestra información sensible para este cuando se presenta un fallo \\
			\hline
			Medidas de respuesta & La cuenta de errores mostrados al usuario con detalles para los ingenieros de software es 0 \\
			\hline
		\end{tabular}
		} \\
		\textbf{Fuente}: Autores
	\end{center}
\end{table}

\begin{table}[!htb]
	\caption{Seguridad - Escenario de calidad 2}
	\label{tab:ec2_seguridad}
	\begin{center}
		\resizebox{11cm}{!}{
		\begin{tabular}{|p{4cm}|p{7cm}|}
			\hline
			\multicolumn{2}{|c|}{Escenario de calidad} \\
			\hline
			Estímulo & Se requiere deshabilitar todas las funcionalidades que no sean utilizadas en el momento en que el software esté en producción \\
			\hline
			Fuente & Usuario final, propietario del software \\
			\hline
			Ambiente & Producción \\
			\hline
			Elementos & Capa de presentación, capa de negocio (ofrecimiento de servicios que no son ''releseables'')\\
			\hline
			Respuesta & No hay ninguna funcionalidad incompleta mostrada a los clientes del (los) servicios \\
			\hline
			Medidas de respuesta & El número de funcionalidades ofrecidas y que están incompletas es 0, tanto al usuario final del SNS como a los posibles consumidores de servicios \\
			\hline
		\end{tabular}
		} \\
		\textbf{Fuente}: Autores
	\end{center}
\end{table}

\clearpage

\begin{table}[!htb]
	\caption{Seguridad - Escenario de calidad 3}
	\label{tab:ec3_seguridad}
	\begin{center}
		\resizebox{11cm}{!}{
		\begin{tabular}{|p{4cm}|p{7cm}|}
			\hline
			\multicolumn{2}{|c|}{Escenario de calidad} \\
			\hline
			Estímulo & Se requiere manejar un sistema de autenticación-autorización para cada rol posible en la red social \\
			\hline
			Fuente & Propietario del software, usuario final \\
			\hline
			Ambiente & Desarrollo, producción, mantenimiento \\
			\hline
			Elementos & Todas las capas \\
			\hline
			Respuesta & Todas las operaciones que pudiese tener un servicio están soportadas sobre un esquema de seguridad autenticación-autorización \\
			\hline
			Medidas de respuesta & La cantidad de servicios que no utiliza el sistema de autenticación-autorización es 0\\
			\hline
		\end{tabular}
		} \\
		\textbf{Fuente}: Autores
	\end{center}
\end{table}

\begin{table}[!htb]
	\caption{Seguridad - Escenario de calidad 4}
	\label{tab:ec4_seguridad}
	\begin{center}
		\resizebox{11cm}{!}{
		\begin{tabular}{|p{4cm}|p{7cm}|}
			\hline
			\multicolumn{2}{|c|}{Escenario de calidad} \\
			\hline
			Estímulo & Se requiere que los mensajes pasados entre servicios estén encriptados para tener un mayor control de la información confidencial recibida entre servicios \\
			\hline
			Fuente & Propietario del software, usuario final \\
			\hline
			Ambiente & Desarrollo, producción, mantenimiento \\
			\hline
			Elementos & Todas las capas \\
			\hline
			Respuesta & Se utiliza un sistema de encripción para cada mensaje entre servicios \\
			\hline
			Medidas de respuesta & La cantidad de mensajes sin encripción que son pasados de un servicio
a otro es 0 \\
			\hline
		\end{tabular}
		} \\
		\textbf{Fuente}: Autores
	\end{center}
\end{table}

\section{Entendimiento}

En las tablas \ref{tab:ec1_entendimiento} a \ref{tab:ec2_entendimiento} se especifican los escenarios de entendimiento a ser valorados.

\begin{table}[!htb]
	\caption{Entendimiento - Escenario de calidad 1}
	\label{tab:ec1_entendimiento}
	\begin{center}
		\resizebox{11cm}{!}{
			\begin{tabular}{|p{4cm}|p{7cm}|}
			\hline
			\multicolumn{2}{|c|}{Escenario de calidad} \\
			\hline
			\textbf{Estimulo} & Se trata de extender o mantener el software, agregando mas funcionalidades u optimizando las existentes. \\
			\hline
			\textbf{Fuente} & Desarrollador de software. Mercado. \\
			\hline
			\textbf{Ambiente} & Aplicación en mantenimiento. \\
			\hline
			\textbf{Elementos} & Todo el sistema. \\
			\hline
			\textbf{Respuesta} & La modificacion o extension de funcionalidades se hace de manera satisfactoria \\
			\hline
			\textbf{Medidas de respuesta} & La modificacion o extension de funcionalidades no se demora mas de 96 horas de trabajo \\
			\hline
			\end{tabular}
			}\\
		\textbf{Fuente}: Autores
	\end{center}
\end{table}

\begin{table}[!htb]
	\caption{Entendimiento - Escenario de calidad 2}
	\label{tab:ec2_entendimiento}
	\begin{center}
		\resizebox{11cm}{!}{
			\begin{tabular}{|p{4cm}|p{7cm}|}
			\hline
			\multicolumn{2}{|c|}{Escenario de calidad} \\
			\hline
			\textbf{Estimulo} & Un usuario sin experiencia en el manejo de la aplicación intenta usarla por primera vez. \\
			\hline
			\textbf{Fuente} & Usuario final. \\
			\hline
			\textbf{Ambiente} & Funcionamiento normal. \\
			\hline
			\textbf{Elementos} & Capa de presentación, Manuales de usuario. \\
			\hline
			\textbf{Respuesta} & El usuario crea, actualiza y navega satisfactoriamente en la aplicación \\
			\hline
			\textbf{Medidas de respuesta} & El tiempo requerido para aprender a utilizar, y a familiarizarse con la aplicacion no supera los 15 minutos en la mestra seleccionada. \\
			\hline
			\end{tabular}
			}\\
		\textbf{Fuente}: Autores
	\end{center}
\end{table}

\section{Robustez y fiabilidad}

En las tablas \ref{tab:ec1_rob_fia} a \ref{tab:ec2_rob_fia} se especifican los escenarios de robustez y fiabilidad a ser valorados.

\begin{table}[!htb]
	\caption{Robustez y fiabilidad - Escenario de calidad 1}
	\label{tab:ec1_rob_fia}
	\begin{center}
		\resizebox{11cm}{!}{
			\begin{tabular}{|p{4cm}|p{7cm}|}
			\hline
			\multicolumn{2}{|c|}{Escenario de calidad} \\
			\hline
			\textbf{Estimulo} & Los diferentes servidores utilizados funcionan por un periodo de 1 mes continuo. \\
			\hline
			\textbf{Fuente} & Duracion de funcionamietno de la aplicación. \\
			\hline
			\textbf{Ambiente} & Funcionamiento normal. \\
			\hline
			\textbf{Elementos} & Capa de negocio, Capa de persistencia. \\
			\hline
			\textbf{Respuesta} & La aplicacion funciona con normalidad en el periodo establecido. \\
			\hline
			\textbf{Medidas de respuesta} & La aplicacion debe estar disponible por lo menos un 99\% del tiempo establecido. \\
			\hline
			\end{tabular}
			}\\
		\textbf{Fuente}: Autores
	\end{center}
\end{table}

\begin{table}[!htb]
	\caption{Robustez y fiabilidad - Escenario de calidad 2}
	\label{tab:ec2_rob_fia}
	\begin{center}
		\resizebox{11cm}{!}{
			\begin{tabular}{|p{4cm}|p{7cm}|}
			\hline
			\multicolumn{2}{|c|}{Escenario de calidad} \\
			\hline
			\textbf{Estimulo} & La cantidad de servidores disponibles se ve disminuida de manera inesperada. \\
			\hline
			\textbf{Fuente} & Causa externa. \\
			\hline
			\textbf{Ambiente} & Funcionamiento normal. \\
			\hline
			\textbf{Elementos} & Capa de negocio, Capa de persistencia. \\
			\hline
			\textbf{Respuesta} & El sistema sigue funcionando en modo degenerado. \\
			\hline
			\textbf{Medidas de respuesta} & El sistema es capaz de atender, como mínimo, un 85\% de los usuarios que normalmente atiende. \\
			\hline
			\end{tabular}
			}\\
		\textbf{Fuente}: Autores
	\end{center}
\end{table}

\section{Rendimiento}

A continuación, en el cuadro \ref{tab:rendimiento_esc}, se presenta el escenario de calidad para el requerimiento de rendimiento.

\begin{table}[!htb]
	\caption{Rendimiento - Escenario de calidad 1}
	\label{tab:rendimiento_esc}
	\begin{center}
		\resizebox{11cm}{!}{
			\begin{tabular}{|p{4cm}|p{7cm}|}
			\hline
			\multicolumn{2}{|c|}{Escenario de calidad} \\
			\hline
			\textbf{Estimulo} & El usuario requiere respuesta a las peticiones que requiera en un tiempo prudencial. \\
			\hline
			\textbf{Fuente} & Usuario final. \\
			\hline
			\textbf{Ambiente} & Producción. \\
			\hline
			\textbf{Elementos} & Todas las capas. \\
			\hline
			\textbf{Respuesta} & El sistema da respuesta a peticiones del usuario. \\
			\hline
			\textbf{Medidas de respuesta} & El sistema tardará 5 segundos o menos dando una respuesta a una petición de un usuario. \\
			\hline
			\end{tabular}
			}\\
		\textbf{Fuente}: Autores
	\end{center}
\end{table}