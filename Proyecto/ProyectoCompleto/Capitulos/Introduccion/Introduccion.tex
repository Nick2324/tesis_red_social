El uso de los medios informáticos para la formación de comunidades deportivas en las que los deportistas puedan formar y gestionar sus redes sociales es restringido debido al modo de vida del deportista. Es usual que por medio de facebook y twitter los deportistas creen sus redes sociales. Sin embargo, facebook y twitter añaden información basura para los deportistas y no ofrecen servicios que han de ser propios de una red social deportiva.

En este documento se expone una propuesta para el desarrollo de una red social deportiva por medio de la teoría de redes sociales y SOA, así como la investigación del estado del arte de las redes sociales deportivas. Lo que se pretende con el documento es exponer el problema existente que hay entre la utilización de las TIC y la comunicación (a modo de red social) entre las comunidades deportivas y, además, se pretende exponer el desarrollo de un prototipo de SNS que resuelva dicho problema.

Primero, el lector encontrará un acercamiento al problema que se resolverá en la definición del problema, la justificación y los objetivos. Más adelante, el lector podrá echar un vistazo a la teoría que está detrás del problema a resolver y que es necesaria para su solución. Se presenta también al lector el marco a utilizar para la solución del problema en las secciones de metodología, cronograma y un estudio de presupuestos. Por último, el documento contiene los análisis funcionales, la creación de la arquitectura y del modelo de datos que llevará el prototipo de SNS construido.
