\section{Comunicación}

Los científicos han estudiado el porqué de las relaciones complejas entre los humanos en comparación a la complejidad presentada en las relaciones entre otros animales. Una de las hipótesis, Social Brain Hypothesis (SBH) postula que el crecimiento cognitivo humano y sus intrincadas relaciones sociales se deben a “la necesidad de nuestros ancestros de mantener e incrementar el número de relaciones sociales con diferentes grupos para sobrevivir en las extremadamente desafiantes condiciones ambientales originadas durante la última era glacial”.\cite{dynamics}

El hombre, en su continua evolución, ha utilizado el lenguaje como una herramienta creadora de conocimiento transferible a sus congéneres o cualquier otro ser que interactuase con él. Con esto, “los humanos han desarrollado el lenguaje como un instrumento ligero y conveniente para mantener sus relaciones” \cite{dynamics}. 

En la comunicación entre congéneres, el lenguaje puede ser dividido en dos funciones: función de transmisión de información (gossip) y función de entendimiento del estado interno (estado mental) del congénere (mentalisation) \cite{dynamics}. Estas funciones de transmisión y entendimiento del otro han permitido que dos o varios humanos puedan asociarse entre sí formando redes sociales.
