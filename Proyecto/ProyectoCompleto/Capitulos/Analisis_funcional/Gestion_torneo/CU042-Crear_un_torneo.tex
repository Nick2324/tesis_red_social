\begin{table}[!htb]
	\caption{CU042-Crear un torneo: Descripción}
	\label{tab:cu042_desc}
	\begin{center}
		\resizebox{15cm}{!}{
		\begin{tabular}{|p{4cm}|p{11cm}|}
			\hline
			\multicolumn{2}{|c|}{Descripción de caso de uso} \\
			\hline
			Nombre & Crear un torneo \\
			\hline
			Identificador & CU042 \\
			\hline
			Descripción & Permite la creación de un torneo \\
			\hline
			Actor & Todo actor de la red social	 
			\\
			\hline
			Disparador & Se elige la opción de crear un torneo \\
			\hline
			Inclusiones &  \\
			\hline
			Puntos de extensión &  \\
			\hline
			Precondiciones &  
			\begin{itemize}
				\item La aplicación ha sido cargada por un actor con rol de organizador de eventos deportivos
			\end{itemize}
			\\
			\hline
			Postcondiciones & 
			\begin{itemize}
				\item El usuario ha creado un torneo
			\end{itemize}
			\\
			\hline
			Notas & 
			\\
			\hline
		\end{tabular}
		} \\
		\textbf{Fuente}: Autores
	\end{center}
\end{table}

\begin{table}[!htb]
	\caption{CU042-Crear un torneo: Flujos de hechos}
	\label{tab:cu042_flujo}
	\begin{center}
		\resizebox{15cm}{!}{
		\begin{tabular}{|p{1.5cm}|p{6cm}|p{6.5cm}|}
			\hline
			\multicolumn{3}{|c|}{Detalle de flujo de hechos de caso de uso} \\
			\hline
			Nombre & \multicolumn{2}{|c|}{Nombre del flujo} \\
			\hline
			Paso & Acción del actor & Respuesta del sistema \\
			\hline
			1 & El usuario ha elegido administrar eventos de tipo torneo & El sistema ha mostrado una lista de torneos \\
			\hline
			2 & El usuario elige el botón de creación de eventos y elige el tipo de evento torneo & El sistema muestra la interfaz de creación de eventos tipo torneo \\
			\hline
			3 & El usuario ingresa toda la información que desee del torneo y pulsa el botón para confirmar la creación & El sistema guarda los cambios \\
			\hline
			4 &  & El sistema muestra un elemento emergente informando del éxito o fracaso de la operación \\
			\hline
			5 & El usuario continua & El sistema muestra la interfaz de administración de eventos deportivos de tipo torneo \\
			\hline
		\end{tabular}
		} \\
		\textbf{Fuente}: Autores
	\end{center}
\end{table}