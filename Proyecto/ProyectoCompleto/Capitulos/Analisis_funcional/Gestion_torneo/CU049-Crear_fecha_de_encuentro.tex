\begin{table}[!htb]
	\caption{CU049-Crear fecha de encuentro: Descripción}
	\label{tab:cu049_desc}
	\begin{center}
		\resizebox{15cm}{!}{
		\begin{tabular}{|p{4cm}|p{11cm}|}
			\hline
			\multicolumn{2}{|c|}{Descripción de caso de uso} \\
			\hline
			Nombre & Crear fecha de encuentro \\
			\hline
			Identificador & CU049 \\
			\hline
			Descripción & Permite poner una fecha a un encuentro en el torneo \\
			\hline
			Actor & Todo actor de la red social	 
			\\
			\hline
			Disparador & Se elige poner una fecha en un encuentro del torneo \\
			\hline
			Inclusiones &  \\
			\hline
			Puntos de extensión & 
			\\
			\hline
			Precondiciones &  
			\begin{itemize}
				\item La aplicación ha sido cargada por un actor con rol de organizador de eventos deportivos
			\end{itemize}
			\\
			\hline
			Postcondiciones & 
			\begin{itemize}
				\item El usuario pone una fecha al encuentro elegido
			\end{itemize}
			\\
			\hline
			Notas & 
			\\
			\hline
		\end{tabular}
		} \\
		\textbf{Fuente}: Autores
	\end{center}
\end{table}

\begin{table}[!htb]
	\caption{CU049-Crear fecha de encuentro: Flujos de hechos}
	\label{tab:cu049_flujo}
	\begin{center}
		\resizebox{15cm}{!}{
		\begin{tabular}{|p{1.5cm}|p{6cm}|p{6.5cm}|}
			\hline
			\multicolumn{3}{|c|}{Detalle de flujo de hechos de caso de uso} \\
			\hline
			Nombre & \multicolumn{2}{|c|}{Nombre del flujo} \\
			\hline
			Paso & Acción del actor & Respuesta del sistema \\
			\hline
			1 & El usuario ha elegido gestionar formato del torneo & El sistema ha mostrado la interfaz de gestión del formato del torneo \\
			\hline
			2 & El usuario elige uno de los encuentros mostrados & El sistema muestra la interfaz de gestión de detalles de encuentro \\
			\hline
			3 & El usuario modifica las fechas/horas del encuentro & El sistema hace los cambios en hora/fecha y envía notificaciones a los involucrados \\
			\hline
		\end{tabular}
		} \\
		\textbf{Fuente}: Autores
	\end{center}
\end{table}