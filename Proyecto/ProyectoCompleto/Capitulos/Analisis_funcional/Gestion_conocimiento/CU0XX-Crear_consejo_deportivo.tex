\begin{table}[!htb]
	\caption{CU135-Crear consejo deportivo: Descripción}
	\label{tab:cu135_desc}
	\begin{center}
		\resizebox{15cm}{!}{
		\begin{tabular}{|p{4cm}|p{11cm}|}
			\hline
			\multicolumn{2}{|c|}{Descripción de caso de uso} \\
			\hline
			Nombre & Crear consejo deportivo \\
			\hline
			Identificador & CU135 \\
			\hline
			Descripción & Funcionalidad que permite crear un nuevo consejo deportivo en la aplicación\\
			\hline
			Actor & 
			\begin{itemize}
				\item \textit{Consejero deportivo.}
			\end{itemize} \\
			\hline
			Disparador &  Un usuario decide crear un nuevo consejo deportivo.\\
			\hline
			Inclusiones &  \\
			\hline
			Puntos de extensión &  \\
			\hline
			Precondiciones & \\
			\hline
			Postcondiciones & \\
			\hline
			Notas &  \\
			\hline
		\end{tabular}
		} \\
		\textbf{Fuente}: Autores
	\end{center}
\end{table}

\begin{table}[!htb]
	\caption{CU135-Crear consejo deportivo: Flujos de hechos}
	\label{tab:cu135_flujo}
	\begin{center}
		\resizebox{15cm}{!}{
		\begin{tabular}{|p{1.5cm}|p{6cm}|p{6.5cm}|}
			\hline
			\multicolumn{3}{|c|}{Detalle de flujo de hechos de caso de uso} \\
			\hline
			Nombre & \multicolumn{2}{|c|}{Nombre del flujo} \\
			\hline
			Paso & Acción del actor & Respuesta del sistema \\
			\hline
			1 & El usuario ingresa al menu de gestión del conocimiento. & Se despliega el menu de gestión del conocimiento. \\
			2 & El usuario selecciona la opcion Contenidos deportivos. & Se despliegan las opciones de menú a las cuales el rol del usuario tiene acceso. \\
			3 & El usuario selecciona la opcion Crear un nuevo consejo deportivo. & Se despliega el formulario de creación de un nuevo consejo deportivo. \\
			4 & Luego de llenar los campos obligatorios del formulario, el usuario selecciona la opción Crear consejo. & El sistema solicita confirmación de la creación del nuevo consejo deportivo. \\
			5 & El usuario confirma la creación del nuevo consejo. & Se crea el nuevo consejo deportivo en la base de conocimientos de la aplicación. El sistema regresa al menu de gestión del conocimiento. \\
			\hline
		\end{tabular}
		} \\
		\textbf{Fuente}: Autores
	\end{center}
\end{table}