\begin{table}[!htb]
	\caption{CU060-Dar de baja a integrante de equipo/grupo: Descripción}
	\label{tab:cu060_desc}
	\begin{center}
		\resizebox{15cm}{!}{
		\begin{tabular}{|p{4cm}|p{11cm}|}
			\hline
			\multicolumn{2}{|c|}{Descripción de caso de uso} \\
			\hline
			Nombre & Dar de baja a integrante de equipo/grupo \\
			\hline
			Identificador & CU060 \\
			\hline
			Descripción & Permite eliminar un jugador de un equipo deportivo, así como desvincularse del mismo \\
			\hline
			Actor & Todo actor de la red social
			\\
			\hline
			Disparador & El usuario elige dar de baja a un integrante del equipo/grupo (darse de baja también) \\
			\hline
			Inclusiones &  \\
			\hline
			Puntos de extensión & 
			\\
			\hline
			Precondiciones &  
			\begin{itemize}
				\item La aplicación ha sido cargada por un actor con rol de formador de grupos deportivos o un jugador del equipo/grupo intenta acceder a la funcionalidad
			\end{itemize}
			\\
			\hline
			Postcondiciones & 
			\begin{itemize}
				\item El usuario da de baja a un jugador del equipo/grupo (o se da de baja a si mismo)
			\end{itemize}
			\\
			\hline
			Notas & 
			\\
			\hline
		\end{tabular}
		} \\
		\textbf{Fuente}: Autores
	\end{center}
\end{table}

\begin{table}[!htb]
	\caption{CU060-Dar de baja a integrante de equipo/grupo: Flujos de hechos}
	\label{tab:cu060_flujo}
	\begin{center}
		\resizebox{15cm}{!}{
		\begin{tabular}{|p{1.5cm}|p{6cm}|p{6.5cm}|}
			\hline
			\multicolumn{3}{|c|}{Detalle de flujo de hechos de caso de uso} \\
			\hline
			Nombre & \multicolumn{2}{|c|}{Nombre del flujo} \\
			\hline
			Paso & Acción del actor & Respuesta del sistema \\
			\hline
			1 & El usuario ha elegido un equipo/grupo deportivo & El sistema ha mostrado los detalles del equipo/grupo elegido \\
			\hline
			2 & El usuario elige, en las acciones del equipo/grupo, gestionar integrantes & El sistema muestra la lista de integrantes del grupo/equipo \\
			\hline
			3 & El usuario busca un integrante del grupo/equipo & El sistema muestra los resultados de la búsqueda \\
			4 & El usuario elige un integrante entre los resultados de la búsqueda y pulsa el botón designado para retirarlo & El sistema retira al integrante del grupo/equipo  \\
			5 &  & El sistema muestra un elemento emergente informando del éxito o fracaso de la operación \\
			\hline
			6 & El usuario continua & El sistema muestra la interfaz de gestión de integrantes de equipos/grupos deportivos \\
			\hline
		\end{tabular}
		} \\
		\textbf{Fuente}: Autores
	\end{center}
\end{table}