\begin{table}[!htb]
	\caption{CU035-Retirar espectador: Descripción}
	\label{tab:cu035_desc}
	\begin{center}
		\resizebox{15cm}{!}{
		\begin{tabular}{|p{4cm}|p{11cm}|}
			\hline
			\multicolumn{2}{|c|}{Descripción de caso de uso} \\
			\hline
			Nombre & Retirar espectador \\
			\hline
			Identificador & CU035 \\
			\hline
			Descripción & Permite retirar un espectador del evento deportivo \\
			\hline
			Actor & Todo actor de la red social	 
			\\
			\hline
			Disparador & Se elige la opción de retirar un espectador deportivo del evento \\
			\hline
			Inclusiones &  \\
			\hline
			Puntos de extensión &  \\
			\hline
			Precondiciones &  
			\begin{itemize}
				\item La aplicación ha sido cargada por un actor con rol de organizador de eventos deportivos
			\end{itemize}
			\\
			\hline
			Postcondiciones & 
			\begin{itemize}
				\item El usuario retira un espectador del deportivo del evento deportivo
			\end{itemize}
			\\
			\hline
			Notas & 
			\\
			\hline
		\end{tabular}
		} \\
		\textbf{Fuente}: Autores
	\end{center}
\end{table}

\begin{table}[!htb]
	\caption{CU035-Retirar espectador: Flujos de hechos}
	\label{tab:cu035_flujo}
	\begin{center}
		\resizebox{15cm}{!}{
		\begin{tabular}{|p{1.5cm}|p{6cm}|p{6.5cm}|}
			\hline
			\multicolumn{3}{|c|}{Detalle de flujo de hechos de caso de uso} \\
			\hline
			Nombre & \multicolumn{2}{|c|}{Nombre del flujo} \\
			\hline
			Paso & Acción del actor & Respuesta del sistema \\
			\hline
			1 & El usuario ha elegido la opción de consulta de espectadores & El sistema ha mostrado una interfaz con la lista de los espectadores existentes \\
			\hline
			2 & El usuario busca un espectador en específico & El sistema muestra el resultado de la búsqueda \\
			\hline
			3 & El usuario pulsa el botón para retirar el espectador & El sistema retira al espectador \\
			\hline
			4 & & El sistema muestra un elemento emergente informando del éxito o fracaso de la operación \\
			\hline
			5 & El usuario continua & El sistema muestra la interfaz de administración de eventos deportivos \\
			\hline
		\end{tabular}
		} \\
		\textbf{Fuente}: Autores
	\end{center}
\end{table}